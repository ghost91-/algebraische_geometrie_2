%!TEX root = algebraische_geometrie_2.tex

\chapter{Einleitung}
\begin{itemize}
	\item Die klassische algebraische Geometrie ist für Varietäten über algebraisch abgeschlossenen Körpern. Die Koordinatenringe sind dann immer reduzierte Algebren. In der algebraischen Schnitttheorie muss man aber nicht-reduzierte Algebren betrachten (\enquote{Multiplizitäten}).
	\item In der Zahlentheorie wird man gezwungen, über Zahlenkörpern zu arbeiten. Dies sind endliche Körpererweiterungen von $\Q$, also nicht algebraisch abgeschlossen.
	\item Viele Klassifikationsprobleme führen auf Modulräume, die keine Varietäten sind.
\end{itemize}
Um diese Probleme zu lösen, hat Alexander Grothendieck zu Beginn der 60er Jahre die Theorie der Schemata eingeführt (EGA I--IV). Dies ist eine relative Theorie, das heißt es wird kein Grundkörpervorausgesetzt und die Koordinatenringe sind beliebige kommutative Ringe. Das heißt, man kann (beziehungsweise muss) die Methoden der kommutativen Algebra für die Beweise nutzen.

\textbf{Erfolge:} Weil-Vermutung (Deligne 70er), Fields-Medaillen, Schemata haben sich als Standard in der algebraischen und arithmetischen Geometrie durchgesetzt.

In dieser Vorlesung seien Ringe und Algebren immer kommutativ und mit Einselement, falls nichts anderes gesagt wird.
