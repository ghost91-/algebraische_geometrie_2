\chapter{Lokal geringte Räume}

Mit dem Konzept der lokal geringten Räume kann man die Mannigfaltigkeiten aus der Analysis und Differentialgeometrie, die algebraische Varietäten aus der algebraischen Geometrie I und die Schemata aus der algebraischen Geometrie II zusammenfassen.

\begin{bem}
	Im Folgenden soll, wenn nichts anderes gesagt wird, Folgendes gelten:
	\begin{itemize}
	 	\item Alle Ringe sind kommutativ mit Eins.
	 	\item Ringhomomorphismen bilden die Eins auf die Eins ab.
	 \end{itemize}
\end{bem}

\begin{bem*}
	Ein Ring $R$ heißt \textbf{lokal}, wenn es in $R$ genau ein Maximalideal $\mathfrak{m}$ gibt.
\end{bem*}

\begin{prop}
\label{prop:2.2}
	Sei $R$ ein Ring. $R$ ist genau dann ein lokaler Ring, wenn es ein Ideal $I \neq R$ mit $R\setminus I = R^\times$ gibt.
	\begin{proof}
		Dies ist eine einfache Eigenschaft aus der kommutativen Algebra.
	\end{proof}
\end{prop}

\begin{defn}
	Ein Homomorphismus $\varphi\colon R_1 \to R_2$ von Ringen heißt \textbf{lokal}, wenn
	\[
		\varphi^{-1}(\mathfrak{m}_2) = \mathfrak{m}_1,
	\]
	wobei $\mathfrak{m}_i$ das Maximalideal von $R_i$ ist.
\end{defn}

\begin{bsp*}
	Der Homomorphismus $\Z_{\langle p \rangle} \hookrightarrow \Q$ ist kein lokaler Homomorphismus, da
	\[
		\varphi^{-1}(\{0\}) = \{0\} \subsetneq p\Z_{\langle p \rangle}
	\]
	gilt.
\end{bsp*}

\begin{bsp}
\label{bsp:2.4}
	Sei $\mathcal{F}$ die Garbe der stetigen reellwertigen Funktionen auf dem topologischen Raum $X$, das heißt $\mathcal{F}(U) \coloneqq \mathcal{C}(U)$ für alle $U$ offen in $X$ (siehe Beispiel~\ref{bsp:1.5}). Dann ist der Halm $\mathcal{F}_x$ in jedem $x \in X$ ein lokaler Ring.
	\begin{proof}
		Nach Beispiel~\ref{bsp:1.5} ist $\mathcal{F}_x$ ein Ring und sogar eine $\R$-Algebra. Die Menge
		\[
			\mathfrak{m}_x\coloneqq \{ [(U,f)] \in \mathcal{F}_x \mid U \text{ offene Umgebung von } x,\; f \in \mathcal{C}(U), \; f(x) = 0 \}
		\]
		ist ein Ideal in $\mathcal{F}_x$. Es gilt
		\begin{align*}
			\mathcal{F}_x \setminus \mathfrak{m}_x = \{ &[(U,f)] \in \mathcal{F}_x \mid f(x) \neq 0\} = \{[(U,f)] \in \mathcal{F}_x \mid\\
			&f \text{ invertierbar als stetige Funktion in einer Umgebung von } x \} = \mathcal{F}_x^\times.
		\end{align*}
		Aus Proposition~\ref{prop:2.2} folgt, dass $\mathcal{F}_x$ ein lokaler Ring ist.
	\end{proof}
\end{bsp}

Wir benötigen einen wichtigen Begriff aus der Garbentheorie:

\begin{defn}
\label{defn:2.5}
	Sei $f\colon X \to Y$ eine stetige Abbildung topologischer Räume und sei $\mathcal{F}$ eine Garbe auf $X$. Die \textbf{direkt Bildgarbe} $f_*\mathcal{F}$ ist definiert durch
	\[
		(f_*\mathcal{F})(U) \coloneqq \mathcal{F}(f^{-1}(U))
	\]
	für alle $U$ offen in $Y$. Weiter seien die Restriktionsabbildungen von $f_*\mathcal{F}$ gegeben durch 
	\[
		\rho_{UV}^{f_*\mathcal{F}} = \rho_{f^{-1}(U)f^{-1}(V)}^{\mathcal{F}}.
	\]
\end{defn}

\begin{prop}
\label{prop:2.6}
	$f_*\mathcal{F}$ wie in Definition~\ref{defn:2.5} ist eine Garbe.
	\begin{proof}
		Es ist klar, dass $f_*\mathcal{F}$ eine Prägarbe ist. Sei $U$ offen in $Y$ und $U = \bigcup_{i\in I}V_i$ eine offene Überdeckung von $U$. Sei weiter $s \in (f_*\mathcal{F})(U)$ mit $\rho_{UV_i}^{f_*\mathcal{F}}(s) = 0$ für alle $i \in I$. Es gilt $f^{-1}(U) = \bigcup_{i\in I} f^{-1}(V_i)$. Nach Definition gilt weiter $s \in \mathcal{F}(f^{-1}(U))$ und
		\[
			0 = \rho_{UV_i}^{f_*\mathcal{F}}(s) = \rho_{f^{-1}(U)f^{-1}(V_i)}^{\mathcal{F}}(s)
		\]
		und damit folgt mit f) angewendet auf $\mathcal{F}$ schon $s=0 \in \mathcal{F}(f^{-1}(U)) = f_*\mathcal{F}(U)$. Also gilt f) auch für $f_*\mathcal{F}$. Analog beweist man, dass auch g) für $f_*\mathcal{F}$ gilt.
	\end{proof}
\end{prop}

Im Folgenden betrachten wir Garben von Ringen. Alles aus Kaptiel~\ref{chap:1} und auch Proposition~\ref{prop:2.6} gelten auch für diese Garben.

\begin{defn}
\label{defn:2.7}
	\begin{itemize}
		\item Ein \textbf{geringter Raum} $(X, \mathcal{O}_X)$ besteht aus einem topologischen Raum $X$ und einer Garbe von Ringen $\mathcal{O}_X$ auf $X$.
		\item Ein \textbf{Morphismus von geringten Räumen} $(X, \mathcal{O}_X) \to (Y, \mathcal{O}_Y)$ ist ein Paar $(f, f^{\#})$, wobei $f\colon X \to Y$ eine stetige Abbildung und $f^{\#}\colon \mathcal{O}_Y \to f_*\mathcal{O}_X$ ein Homomorphismus von Garben ist.
	\end{itemize}
\end{defn}

\begin{bsp}
	Seien $\mathcal{O}_X$ (beziehungsweise $\mathcal{O}_Y$) die Garbe der stetigen rellwertigen Funktionen auf $X$ (beziehungsweise $Y$). Wir haben in Beispiel~\ref{bsp:2.4} gesehen, dass die eine Garbe von Ringen ist. Dann induziert jede stetige Abbildung $f\colon X \to Y$ einen kanonischen Morphismus $(f,f^{\#})$ von geringten Räumen durch
	\[
		f^{\#}_U\colon \mathcal{O}_Y(U) \to (f_*\mathcal{O}_X)(U) = \mathcal{O}_X(f^{-1}(U)),\; g \mapsto g \circ f.
	\]
\end{bsp}

\begin{bem}
\label{bem:2.9}
	Sei $(f,f^{\#})$ ein Morphismus von geringten Räumen wie in Definition~\ref{defn:2.7}. Sei $x \in X$ und $y \coloneqq f(x)\in Y$. Dann haben wir einen kanonischen Homomorphismus
	\[
		f_x^{\#}\colon \mathcal{O}_{Y,y} \to \mathcal{O}_{X,x},\; [(U,g)] \mapsto [(f^{-1}(U),f^{\#}(g))]
	\]
	von Ringen.
\end{bem}

\begin{defn}
	\begin{itemize}
		\item Ein \textbf{lokal geringter Raum} ist ein geringter Raum $(X,\mathcal{O}_X)$, bei dem die Halme $\mathcal{O}_{X,x}$ lokale Ringe sind.
		\item Ein \textbf{Morphismus von lokal geringten Räumen} ist ein Morphismus von geringten Räumen, für den die Homomorphismen $f_x^{\#}$ aus Bemerkung~\ref{bem:2.9} für alle $x \in X$ lokal sind.
	\end{itemize}
\end{defn}

\begin{bsp}[Fortsetzung von Beispiel 8]
	$(X,\mathcal{O}_X)$ ist ein lokal geringter Raum (siehe Beispiel~\ref{bsp:2.4}). Weiter ist $(f,f^{\#})$ ein Morphismus lokal geringter Räume.
	\begin{proof}
		Sei $x \in X$, $y = f(x)$ und $f^{\#}\colon \mathcal{O}_{Y,y} \to \mathcal{O}_{X,x},\;[(U,g)]\mapsto [(f^{-1}U,g\circ f)]$.\\
		\textbf{Zu zeigen:} $(f^{\#}_x)^{-1}(\mathfrak{m}_x) = \mathfrak{m}_y$.\\
		\enquote{$\subseteq$}: Das Bild eines ivertierbaren Elementes ist wieder invertierbar, also gilt
		\[
			f^{\#}_x(\mathcal{O}_{Y,y}\setminus \mathfrak{m}_y) \subseteq \mathcal{O}_{X,x}\setminus \mathfrak{m}_y
		\]
		und damit
		\[
			(f_x^{\#})^{-1}(\mathfrak{m}_x) \subseteq \mathfrak{m}_y
		\]
		(dies gilt für alle Morphismen von geringten Räumen).\\
		\enquote{$\supseteq$}: Sei $[(U,g)]\in \mathfrak{m}_y$, das heißt $g$ ist eine stetige Funktion auf $U$ mit $g(y)=0$. Es folgt
		\[
			g \circ f \in \mathcal{O}_X(f^{-1}U) \text{ und } (g \circ f)(x)=0
		\]
		und damit
		\[
			f_x^{\#}([(U,g)]) = [(f^{-1}U,g\circ f)] \in \mathfrak{m}_x.
		\]
	\end{proof}
\end{bsp}

\begin{bem}
	Wenn man Morphismen definiert, sollte man sie verknüpfen können (das heißt, man erhält eine Kategorie): Seien $(f,f^{\#})\colon (X,\mathcal{O}_X) \to (Y,\mathcal{O}_Y)$ und $(g,g^{\#})\colon (y,\mathcal{O}_Y) \to (Z,\mathcal{O}_Z)$ Morphismen von geringten Räumen. Dann ist
	\[
		(g,g^{\#}) \circ (f,f^{\#}) \coloneqq (g\circ f,(g\circ f)^{\#})
	\]
	mit
	\[
		(g\circ f)^{\#}_U \coloneqq f^{\#}_{g^{-1}U} \circ g^{\#}
	\]
	ein Morphismus von geringten Räumen. Falls $(f,f^{\#})$ und $(g,g^{\#})$ Morphismen von lokal geringten Räumen sind, dann ist auch $(g,g^{\#}) \circ (f,f^{\#})$ ein Morphismus von lokal geringten Räumen.
	\begin{proof}
		Sei $x \in X$, $y = f(x)$ und $z = g(y)$. Dann gilt
		\[
			((g\circ f)^{\#}_x)^{-1}(\mathfrak{m}_x) = (f^{\#}_x \circ g^{\#}_y)^{-1}(\mathfrak{m}_x) \overset{f^{\#}_x \text{ lokal}}{=} (g^{\#}_y)^{-1}(\mathfrak{m}_y) \overset{g^{\#}_y \text{ lokal}}{=} \mathfrak{m}_z.
		\]
	\end{proof}
\end{bem}