%!TEX root = algebraische_geometrie_2.tex
% vim: tw=0 noet sts=8 sw=8

\chapter{Cartier-Divisoren}

Cartier-Divisoren sind lokal durch eine Gleichung gegeben. Dabei steht die FUnktion im Vordergrund und nicht die Nullstellen. Geht man zu den Nullstellen und Polstellen über, so erhält man einen Weildivisor. 

\begin{defn}
\label{defn:14.1}
	Eine \textbf{invertierbare Garbe} $\mcf$ auf einem geringten Raum $(X,\mco_X)$ ist ein lokal freier $\mco_X$-Modul vom Rang $1$, das heißt für alle $x \in X$ gibt es eine offene Umgebung $U$ von $x$ mit $\mcf\vert_U \cong \mco_{U}$.
\end{defn}


\begin{prop}
\label{prop:14.2}
	Sein $\mce$ ein lokal freier $\mco_X$-Modul von endlichem Rang ind sei $\mce^V \coloneqq \underline{\Hom}_{\mco_X}(\mce,\mco_X)$ der \textbf{duale $\mco_X$-Modul}. Dann gilt
	\begin{enumerate}[a)]
		\item Es gilt $\mce \cong (\mce^v)^v$.
		\item Für alle $\mco_X$-Moduln gibt es einen kanonischen Isomorphismus $\underline{\Hom}_{\mco_X}(\mce,\mcf) \simto \mce^v\otimes_{\mco_X} \mcf$.
	\end{enumerate}
\end{prop}

\begin{kor}
\label{kor:14.3}
	Seien $\mcl, \mcm$ invertierbare Garben. Dann dann sind
	\[
		\mcl \otimes \mcm \coloneqq \mcl \otimes_{\mco_X} \mcm
	\]
	und
	\[
		\mcl^{-1} \coloneqq \mcl^v.
	\]
	wieder invertierbare Garben.
	\begin{proof}
		Dies folgt sofort aus Proposition~\ref{prop:14.2}.
	\end{proof}
\end{kor}

\begin{defn}
\label{defn:14.4}
	Nach Korollar~\ref{kor:14.3} bilden die Isomorphisklassen invertierbarer Garben auf $X$ eine abelsche Gruppe bezüglich $\otimes$. Sie heit die \textbf{Picard-Gruppe} von $X$ und wird mit $\Pic(X)$ bezeichnet.
\end{defn}

\begin{bem}
\label{bem:14.5}
	Sei $X$ ein geringter Raum und $U$ offen in $X$. Die Menge
	\[
		S(U) \coloneqq \{ f \in \Gamma(U,\mco_X)\mid \text{ für alle }p \in U\text{ ist }f_p \text{ kein Nullteiler in }\mco_{X,p}\}
	\]
	ist multiplikativ abgeschlossen.
\end{bem}

\begin{defn}
\label{defn:14.6}
	Sie zur Prägarbe $U \mapsto S(U)^{-1}\Gamma(U,\mco_X)$ assoziierte Garbe $\mcm_X$ heißt die \textbf{Garbe der meromorphen Funktionen} auf $X$.
\end{defn}

\begin{warn*}
	Die Definitionen in \cite[{}II.6]{hartshorne1977algebraic} und \cite[§21]{grothendieck1967elements} sind falsch. Siehe [Kleinman: Misconceptions about KX].
\end{warn*}

\begin{bem}
\label{bem:14.7}
	\begin{enumerate}[i)]
		\item Der kanonische Homomorphismus $\mco_X \to \mcm_X$ ist injektiv.
		\begin{proof}
			Betrachte die Halme, dann ist $\mco_{X,p}\to \mcm_{X,p}$ injektiv, weil wir nicht in nullteiler lokalisieren.
		\end{proof}
		\item Seien $\mco_X^\times$ und $\mcm_X^\times$ die Garben der Einheiten in den entsprechenden Garben, dann erhalten wir einen injektiven Homomorphismus $\mco_X^\times \to \mcm_X^\times$.
	\end{enumerate}
\end{bem}

\begin{bem}
\label{bem:14.8}
	\begin{enumerate}[i)]
		\item\label{bem:14.8:i} Sei $X$ ein Schema und $U=\Spec(A)$ ein offenes affines Unterschema. Dann gilt
		\[
			S(U) = \{f \in A \mid f \text{ ist kein Nullteiler in }A\}.
		\]
		\item\label{bem:14.8:ii} Ist $X$ ein integres Schema, $\eta$ der generische Punkt von $X$ und $K(X) \coloneqq \kappa(\eta)$ der Funktionenköroer, so ist $\mcm_X$ die konstante Garbe, die zu $K(X)$ assoziiert ist, das heißt für alle nicht leeren offenen $U$ in $X$ gilt $\mcm_X(U) = K(X)$ und die Restriktionen sind durch die Identität gegeben.
	\end{enumerate}
	\begin{proof}
		\begin{enumerate}[i)]
			\item Sei $f\in \Gamma(U,\mco_X) = A$. Falls $f\in S(U)$ mit $fg = 0)$ für ein $g\in \Gamma(U,\mco_X) = A$, dann gilt für alle $p \in X$ schon $f_p g_p = 0 \in \mco_{X,p}$. Also ist $g_p = 0 \in \mco_{X,p}$ für alle $p \in X$ und es folgt $g = 0\in \Gamma(U,\mco_X) = A$. Sei umgekehrt $f$ kein Nullteiler in $A$. Dann ist $f_p$ für alle $p \in X$ kein Nullteiler in $\mco_{X,p} = A_p$.
			\item Dies folgt mit \cite[{}3.29]{goertz2010algebraic}.
		\end{enumerate}
	\end{proof}
\end{bem}

\begin{defn}
\label{defn:14.9}
	\begin{enumerate}[i)]
		\item ein \textbf{Cartier-Divisor} auf $X$ ist ein Element der abelschen Gruppe
		\[
			\CaDiv(X) \coloneqq \Gamma(X,\mcm_X^\times/\mco_X^\times).
		\]
		Wir schreiben Cartier-Divisoren additiv, obwohl $\mcm_X^\times/\mco_X^\times$ multiplikativ geschrieben wird.
		\item Ein Cartier-Divisor heißt \textbf{Haupt-Cartier-Divisor (Hauptdivisor)}, falls er im Bild der kanonischen Abbildung
		\[\label{eq:14.9.1}\tag{$\star$}
			\div\colon \Gamma(X,\mcm_X^\times) \to \Gamma(X,\mcm_X^\times/\mco_X^\times)
		\]
		liegt. Zwei Cartier-Divisoren $D_1$ und $D_2$ heißen \textbf{linear äquivalent}, wenn $D_1 - D_2$ ein Hauptdivisor ist. Wir schreiben $D_1 \sim D_2$.
		\item Der Kokern von \eqref{eq:14.9.1} heißt \textbf{Cartier-Divisorenklassengruppe} und wird mit $\CaCl(X)$ bezeichnet, das heißt $\CaCl(X) = \CaDiv(X)/\{\text{Hauptdivisoren}\}$.
	\end{enumerate}
\end{defn}

\begin{bem}
\label{bem:14.10}
	\begin{enumerate}[i)]
		\item\label{bem14.10:i} Seien $U$ und $V$ offen in $X$, $f \in \Gamma(U,\mcm_X^\times$ und $g \in \Gamma(V, \mcm_X^\times)$. Dann ergibt $f \cdot g$ Sinn auf $U \cap V$, das heißt $f \cdot g\coloneqq f\vert_{U\cap V} \cdot g \vert_{U \cap V} \in \Gamma(U\cap V, \mcm_X^\times)$.
		\item\label{bem14.10:ii} Cartier-Divisoren sind Elemente in $\Gamma(X,\mcm_X^\times/\mco_X^\times)$ und werden damit surch \textbf{zulässige Familien} $(U_i,f_i)_{i\in I}$ repräsentiert, wobei Folgendes gilt:
		\begin{itemize}
			\item $(U_i)_{i\in I}$ ist eine offene Überdeckung von $X$.
			\item Es gilt $f_i \in (U_i \mco_X^\times)$.
			\item Für alle $i,j \in I$ gilt $\frac{f_i}{f_j}\in \Gamma(U_i\cap U_j,\mco_X^\times)$, das heißt es gibt ein $g_{ij} \in \mco_X^\times(U_i \cap U_j)$ mit $f_i = g_{ij} f_j$.
		\end{itemize}
		\item\label{bem14.10:iii} Zwei zulässige Familien $(U_i,f_i)_{i \in I}$ und $(U_j',f_j')_{j\in J}$ definieren genau dann denselben Cartier-Divisor, wenn $f_i(f_j')^{-1} \in \Gamma(U_i\cap U_j',\mco_X^\times)$ für alle $i\in I$ und $j \in J$ gilt.
		\item\label{bem14.10:iv} Nach \ref{bem14.10:ii} können zwei Cartier-Divisoren nach geeigneter Verfeinerung der Überdeckungen stets durch zulässige Funktionen \textbf{derselben} Überdeckung beschrieben werden.
		\item\label{bem14.10:v} Für Cartier-Divisoren $D = [(U_i,f_i)_{i\in I}]$, $E = [(U_i,g_i)_{i \in I}]$ und $f \in \mcm_X^\times(X)$ gilt:
		\begin{align*}
			D+E &= [(U_i,f_ig_i)_{i \in I}]\\
			-D &= [(U_i,f_i^{-1})_{i \in I}]\\
			\div(f) &= [(X,f)]\\
			0 &= \div(1) = [(X,1)]
		\end{align*}
	\end{enumerate}
\end{bem}

\begin{defn}
\label{defn:14.11}
	Ein Cartier-Divisor $D$ heißt \textbf{effektiv}, wenn $D = [(U_i,f_i)_{i \in I}]$ mit $f_i \in \Gamma(U_i,\mco_X)\cap \Gamma(U_i,\mcm_X^\times)$. Wir schreiben $D \ge 0$ und erhalten eine partielle Ordnung auf $\CaDiv(X)$.
\end{defn}

\begin{kons}
\label{kons:14.12}
	\begin{enumerate}[i)]
		\item Sei $D \in \CaDiv(X)$ repräsentiert durch die zulässige Familie $(U_i,f_i)_{i \in I}$. Nach Konstruktion ist $\mcm_X$ ein torsionsfreier $\mco_X$-Modul. Wir konstruieren nun einen invertierbaren $\mco_X$-Untermodul $\mco_X(D)$ von $\mcm_X$. Für $i \in I$ sei $\mcl_i\coloneqq f_i^{-1}\mco_{U_i}$ der von $f_i^{-1}$ erzeugte $\mco_{U_i}$-Untermodul von $\mcm_X\vert_{U_i} = M_{U_i}$. Für $i,j \in I$ gilt
		\[
			\mcl_i\vert_{U_i\cap U_j} = \mcl_j\vert_{U_i\cap U_j} \subseteq \mcm_X\vert_{U_i \cap U_j},
		\]
		also gibt es genau einen $\mco_X$-Untermodul $\mco_X(D)$ von $\mcm_X$ mit $\mco_X(D)\vert_{U_i} = \mcl_i$ für alle $i \in I$.
		\item Man prüft leicht, dass $\mco_X(D)$ unabhängig von der Wahl der zulässigen Familie $(U_i,f_i)_{i\in I}$ ist.
		\item Nach Konstruktion ist $\mcl_i=\mco_X(D)\vert_{U_i}$ ein lokal freier $\mco_{U_i}$-Modul vom Rang $1$. Also ist $\mco_X(D)$ ein invertierbarer $\mco_X$-Untermodul von $\mcm_X$. Wir nennen $\mco_X(D)$ die \textbf{dem Cartier-Divisor $D$ zugeordnete invertierbare Garbe}.
	\end{enumerate}
\end{kons}

\begin{prop}
\label{prop:14.13}
	\begin{enumerate}[i)]
		\item Die Abbildung
		\[
			\CaDiv(X) \to \{\text{ivertierbare }\mco_X-\text{Untermoduln von }\mcm_X\},\; D \mapsto \mco_X(D)
		\]
		ist ein bijektiver Gruppenhomomorphismus.
		\item Die Abbildung
		\[
			\CaDiv(X)\to \Pic(X),\; D \mapsto [\mco_X(D)]
		\]
		ist ein Gruppenhomomorphismus und induziert einen injektiven Gruppenhomomorphismus
		\[\label{eq:14.13.1}\tag{$\triangle$}
			\CaCl(X) \hookrightarrow \Pic(X),\; [D] \mapsto [\mco_X(D)].
		\]
		\item Falls $X$ integer ist, dann ist \eqref{eq:14.13.1} ein Isomorphismus.
	\end{enumerate}
	\begin{proof}
		\cite[Propositionen~II.6.13-II.6.15]{hartshorne1977algebraic}.
	\end{proof}
\end{prop}

\begin{bem}
\label{bem:14.14}
	Sei jetzt $X$ ein integres noethersches separiertes Schema, das regulär in Kodimension $1$ ist. Sei $D$ ein Cartier-Divisor auf $X$, gegeben durch die zulässige Familie $(U_i,f_i)_{i \in I}$. Weil $X$ integer ist gilt nach Bemerkung~\ref{bem:14.8}~\ref{bem:14.8:ii}
	\[\label{eq:14.14.1}\tag{$\square$}
		f_i \in K(X)^\times \text{ und } f_i/f_j \in \mco_{U_i\cap U_j}^\times.
	\]
	Wir definieren den \textbf{assoziierten Weildivisor} $\cyc(D)$ druch Verkleben der Hauptdivisoren $\div(f\vert_{U_i})$ auf $U_i$. Dies ist wohldefiniert wegen \eqref{eq:14.14.1}. Damit erhalten wir einen Gruppenhomomorphismus
	\[
		\cyc\colon \CaDiv(X) \to \{\text{Wildivisoren}\},\; D \mapsto \cyc(D).
	\]
\end{bem}

\begin{thm}
\label{thm:14.15}
	Falls $X$ ein integres \textbf{normales} noethersches separiertes Schema ist, dann ist dieser Gruppenhomomrphismus $\cyc$ injektiv. Er ist genau dann ein Isomorphismus, wenn $X$ loakl faktoriell ist.
	\begin{proof}
		\cite[Théorème~21.6.9]{grothendieck1967elements}.
	\end{proof}
\end{thm}

\begin{kor}
\label{kor:14.16}
	Sei $X$ wie in Theorem~\ref{thm:14.15}. Dann haben wir einen Monomorphismus
	\[
		\CH^1(X) \hookrightarrow \Pic(X),\; [D] \mapsto [\cyc(D)]. 
	\]
	Falls $X$ noch lokal faktoriell ist, dann ist dies ein Isomorphismus.
	\begin{proof}
		Nach Definition ist das Bild der Gruppe der Hauptdivisoren gleich der Gruppe der Haupt-Cartier-Divisoren. Damit folgt die Behauptung aus Theorem~\ref{thm:14.15}.
	\end{proof}
\end{kor}

\begin{bsp}
\label{bsp:14.17}
	Es gilt
	\[
		\Pic(\P^n_K) = \{\mco_{\P^n_k}(m)\mid m \in \Z\} \cong \Z
	\]
	nach Korollar~\ref{kor:14.16} und Proposition~\ref{prop:12.17}.
\end{bsp}

%%% Local Variables: 
%%% mode: latex
%%% TeX-master: "algebraische_geometrie_2"
%%% End: 
