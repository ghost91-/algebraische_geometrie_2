%!TEX root = algebraische_geometrie_2.tex
% vim: tw=0 noet sts=8 sw=8

\chapter{Schemata}

Schemata sind die Hauptobjekte der algebraischen Geometrie. Sie verallgemeinern die (quasiprojektiven) Varietäten aus der algebraischen Geometrie I.

\nextmark{Schema, Morphismus von Schemata}
\begin{defn}
\label{defn:5.1}
	\begin{enumerate}[i)]
		\item Ein \textbf{Schema} ist ein lokal geringter Raum $(X,\mco_X)$, wobei $X$ eine offene Überdeckung $(U_i)_{i\in I}$ hat, die die Eigneschaft besitzt, dass $(U_i,\mco_{X}\vert_{U_i})$ für alle $i \in I$ ein affines Schema ist. Wir nennen $X$ den \textbf{unterliegenden topologischen Raum} und $\mco_X$ die \textbf{Strukturgarbe}. Oft wird das Schema $(X,\mco_X)$ einfach mit $X$ abgekürzt, obwohl der topologische Raum das Schema nicht bestimmt.
		\item Ein \textbf{Morphismus} von Schemata ist ein Morphismus von lokal geringten Räumen.
	\end{enumerate}
	Wir erhalten die Katergorie $\Sch$ der Schemata.
\end{defn}

\nextmark{offenes Unterschema, offene Unterschemata bilden Basis}
\begin{prop}
	Sei $X = (X,\mco_X)$ ein Schema.
	\begin{enumerate}[i)]
		\item Für $U$ offen in $X$ ist $(U,(\mco_U \coloneqq \mco_X\vert_U))$ ein Schema. Weiter hat man einen kanonischen Morphismus
		\[
			j\colon (U,\mco_U) \to (X,\mco_X).
		\]
		Wir nennen dies die \textbf{induzierte Schemastruktur} auf $U$ und sagen, dass $(U,\mco_U)$ ein \textbf{offenes Unterschema} von $X$ ist.
		\item Die affinen offenen Unterschemata von $X$ bilden eine Basis der Topologie von $X$.
	\end{enumerate}
	\begin{proof}
		Dies folgt aus Proposition~\ref{prop:4.8} und Lemma~\ref{lem:4.16}.
	\end{proof}
\end{prop}

\begin{bem}
	Ein offenes Unterschema eines affinen Schemas muss im Allgemeinen nicht wieder affin sein.
\end{bem}

\nextmark{Trennungsaxiome für top. Räume}
\begin{bem*}
	Sei $X$ ein topologischer Raum.
	\begin{enumerate}[i)]
		\item $X$ erfüllt das Axiom $T_0$, wenn es zu je zwei verschiedenen Punkten $x,y \in X$ eine offene Menge $U$ mit $x \in U$, $y\notin U$ oder $x \notin U$, $y \in U$ gibt. $X$ heißt dann $T_0$-Raum.
		\item $X$ erfüllt das Axiom $T_1$, wenn es zu je zwei verschiedenen Punkten $x,y \in X$ offene Mengen $U_x,U_y$ gibt mit $x \in U_x$, $y \notin U_x$ und $x \notin U_y$, $y \in U_y$. $X$ heißt dann $T_1$-Raum.
		\item $X$ erfüllt das Axiom $T_2$, wenn es zu je zwei verschiedenen Punkten $x,y \in X$ offene Mengen $U_x,U_y$ gibt mit $x \in U_x$, $y \in U_y$ und $U_x \cap U_y = \emptyset$. $X$ heißt dann $T_2$-Raum oder \textbf{Hausdorffraum}.
	\end{enumerate}
	Man sieht leicht, dass das Axiom $T_2$ das Axiom $T_1$ impliziert und dass das Axiom $T_1$ das Axiom $T_0$ impliziert.
\end{bem*}

\nextmark{Schemata sind T0, Existenz und Eindeutigkeit generischer Punkte}
\begin{prop}
\label{prop:5.4}
	Sei $(X, \mco_X)$ ein Schema.
	\begin{enumerate}[i)]
		\item Der topologische Raum $X$ ist ein $T_0$-Raum.
		\item Jede abgeschlossene irreduzible Teilmenge des topologischen Raums $X$ besitzt genau einen generischen Punkt.
	\end{enumerate}
	\begin{proof}
		\begin{enumerate}[i)]
			\item Seien $x\neq y \in X$. Nach Definition~\ref{defn:5.1} gibt es eine affine offene Umgebung $U \cong \Spec(A)$ von $x$. Falls $y \notin U$, dann sind wir fertig, sei also ohne Beschränkung der Allgemeinheit $y \in U$. Die Punkte $x,y$ sind durch Primideale $\mfp,\mfq$ von $A$ gegeben. Es gilt $\mfp \neq \mfq$, sei also ohne Beschränkung der Allgemeinheit $\mfp \not\subseteq \mfq$. Dann gibt es ein $f \in \mfp \setminus \mfq$, und somit ist $D(f)$ offen in $X$ mit $x \notin D(f)$, aber $y\in D(f)$.
			\item Wir zeigen zunächst die Existenz: Sei $V\subseteq X$ irreduzibel und abgeschlossen. Nach Definition~\ref{defn:5.1} gibt es ein offenes affines Unterschema $U \cong \Spec(A)$ mit $U \cap V \neq \emptyset$. Da $V$ abgeschlossen in $X$ ist, ist $U \cap V$ abgeschlossen in $U$ und weil $U$ offen in $X$ ist, ist $U \cap V$ offen in $V$. In Aufgabe~1.5 zur algebraischen Geometrie I haben wir gesehen, dass jede nicht leere, offene Teilmenge eines irreduziblen topologischen Raumes irreduzibel und dicht ist, also ist $U \cap V$ irreduzibel und dicht in $V$. Sei $\mfp \coloneqq I(U \cap V) \in \Spec(A)$. Nach Proposition~\ref{prop:4.7} iii) ist $\mfp$ ein generischer Punkt von $\Spec(A)$. Dieser entspricht einem Punkt $\eta$ in $U \subseteq X$. Es gilt $\overline{\eta}=U\cap V$ und damit ist $\eta$ dicht in $V$, also ein generischer Punkt von $V$.

			Nun zeigen wir die Eindeutigkeit: Seien $\eta_1,\eta_2$ generische Punkte von $V$. Wir nehmen an, dass $\eta_1 \neq \eta_1$. Dann gibt es ohne Beschränkung der Allgemeinheit ein offenes $U$ mit $\eta_1 \in U$ und $\eta_2 \neq U$. Dann gilt $\eta_2 \in V\setminus U \neq V$ und damit folgt $\overline{\eta_2} \subseteq V\setminus U \neq V$ im Widerspruch dazu, dass $\eta_2$ ein generischer Punkt von $V$ ist.
		\end{enumerate}
	\end{proof}
\end{prop}

\begin{bem*}
	Wir haben inbesondere gezeigt, dass jedes irreduzible affine Schema ein $T_0$-Raum ist. Weiter haben wir gesehen, dass jedes irreduzible affine Schema $X$ genau einen generischen Punkt $\eta$ hat.

	Ist $(X,\mco_X)$ ein Schema, das mindestens zwei Punkte enthält. Dann ist $X$ jedoch kein $T_1$-Raum.
	\begin{proof}
		Wir wählen $y = \eta,\; x \neq y$. Falls $U$ eine offene Umgebung von $x$ ist, so gilt $y \in U$, da $y$ ein generischer Punkt ist.		
	\end{proof}
\end{bem*}

\nextmark{Verkleben von Schema-Morphismen}
\begin{lem}
\label{lem:5.5}
	Seien $X,Y$ Schemata.
	\begin{enumerate}[i)]
		\item Sei $(U_i)_{i\in I}$ eine offene Überdeckung von $X$ und seien $f_i\colon U_i\to Y$ Morphismen mit
		\[
			f_i\vert_{U_i\cap U_j} = f_j\vert_{U_i\cap U_j} \quad \forall\; i,j \in I.
		\]
		Dann gibt es genau einen Morphismus $f \colon X \to Y$ mit $f\vert_{U_i} = f_i$ für alle $i \in I$.
		\item Die durch $U \mapsto \Hom_{\Sch}(U,Y)$ mit den Restriktionsabbildungen
		\[
			\Hom_{\Sch}(U,Y) \to \Hom_{\Sch}(V,Y)
		\]
		definierte Prägarbe von Mengen auf $X$ ist eine Garbe.
	\end{enumerate}
	\begin{proof}
		\begin{enumerate}[i)]
			\item Genauer ist ein Morphismus $(f,f^{\#})\colon (X,\mco_X) \to (Y,\mco_Y)$ von lokal geringten Räumen zu konstruieren und die Eindeutigkeit zu zeigen. Gegeben sind Morphismen
			\[
				(f_i,f_i^{\#})\colon (U_i,\mco_{U_i}) \to (Y,\mco_Y),
			\]
			die auf den Überlappungen übereinstimmen. Für alle $x \in X$ gibt es ein $i \in I$ mit $x \in U_i$. Nun definieren wir die stetige Abbildung
			\[
				f\colon X \to Y,\; x \mapsto f(x) \coloneqq f_i(x).
			\]
			Dies ist wohldefiniert, da die $f_i$ auf den Überlappungen der $U_i$ übereinstimmen. Wir konstruieren nun $f^{\#}\colon \mco_Y \to f_*\mco_X$. Für $V$ offen in $Y$ müssen wir einen Homomorphismus
			\[
				f^{\#}_V\colon \mco_Y(V) \to ((f_*\mco_X)(V) = \mco_X(f^{-1}V))
			\]
			konstruieren. Sei $s \in \mco_Y(V)$. Beachte, dass $f^{-1}V$ von den offenen Teilmegenen $U_i \cap f^{-1}V$ überdeckt wird. Wir betrachten die Abbildungen
			\begin{align*}
				f_{i,V}^{\#}\colon \mco_Y(V)&\to (({f_i}_*\mco_{U_i})(V) = \mco_X(U_i \cap f^{-1}V))\\
				s &\mapsto f^{\#}_{i,V}(s).
			\end{align*}
			Wegen der Vorraussetzung stimmen die $f_{i,V}^{\#}(s)$ auf den Überlappungen der Mengen $U_i\cap f^{-1}V$ überein. Wegen der Garbeneigenschaft gibt es ein eindeutiges $t \in \mco_X(f^{-1}V)$ mit $t\vert_{U_i\cap f^{-1}V} = f_{i,V}^{\#}(s)$ für alle $i \in I$. Wir definieren $f^{\#}_V(s) \coloneqq t$. Man sieht sofort, dass $f_V^{\#}$ das Gewünschte liefert und eindeutig ist.
			\item Dies folgt aus i), indem man $X \coloneqq U$ setzt.
		\end{enumerate}
	\end{proof}
\end{lem}

\nextmark{Morphismen X --> Spec(A) vs. Morphismen A --> OX(X)}
\begin{prop}
\label{prop:5.6}
	Sei $X$ ein Schema und $A$ ein Ring. Dann ist die Abbildung
	\begin{align*}
		\Hom_{\Sch}(X,\Spec(A)) &\to \Hom_{\Ring}(A,\mco_X(X))\\
		(f,f^{\#}) & \mapsto f^{\#}_{\Spec(A)}
	\end{align*}
	eine kanonische Bijektion.
	\begin{proof}
		Sei $\varphi\in \Hom_{\Ring}(A,\mco_X(X))$. Wir müssen nun zeigen, dass es genau ein $(f,f^{\#}) \in \Hom_{\Sch}(X,\Spec(A))$ mit $f^{\#}_{\Spec(A)} = \varphi$ gibt. Wir wählen eine offene Überdeckung $(U_i)_{i\in I}$ von $X$, wobei $U_i$ für alle $i \in I$ ein offenes Unterschema von $X$ ist. Nach Proposition~\ref{prop:4.14} gibt es genau ein $(f_i,f_i^{\#}) \in \Hom_{\Sch}(U_i,\Spec(A))$ mit $f^{\#}_{i,\Spec(A)} = \varphi_i$. Für jedes affine offene Unterschema $V$ von $U_i \cap U_j$ gilt $f_i\vert_V = f_j\vert_V$ für alle $i,j \in I$. Dies folgt aus der Eindeutigkeit in Proposition~\ref{prop:4.14}. Weil die Prägarbe der Morphismen in der Kategorie $\Sch$ schon eine Garbe ist, folgt
		\[
			f_i\vert_{U_i\cap U_j} = f_j\vert_{U_i\cap U_j},
		\]
		da es nach Lemma~\ref{lem:5.5} genau ein $(f,f^{\#}) \in \Hom_{\Sch}(X,\Spec(A))$ mit $f\vert_{U_i} = f_i$ für alle $i \in I$ gibt. Dies ergibt sofort die Existenz von $(f,f^{\#})$ mit Hilfe der Garbeneigenschaft der Morphismen. Die Eindeutigkeit ergibt sich aus der Konstruktion.
	\end{proof}
\end{prop}

\nextmark{S-Schema, Morphismen von S-Schemata}
\begin{defn}
\label{defn:5.7}
	\begin{enumerate}[i)]
		\item Sei $S$ ein Schema. Ein $S$\textbf{-Schema} ist ein Morphismus $f\colon X \to S$ von Schemata. Sei $g\colon X \to S$ ein weiteres $S$-Schema, dann setzen wir
		\[
			\Hom_{S}(X,Y)=\{h\colon X\to Y\mid h \in \Hom_{\Sch}(X,Y) \text{ mit } g \circ h = f\},
		\]
		das heißt $\Hom_{S}(X,Y)$ ist die Menge aller Morphismen $h \in \Hom_{\Sch}(X,Y)$, für die folgendes Diagramm kommutiert:
		\begin{center}
			\begin{tikzcd}
				X\arrow{rr}{h}\arrow{dr}[swap]{f} && Y\arrow{dl}{g}\\
				& S
			\end{tikzcd}
		\end{center}
		\item Sei $S=\Spec(A)$ für einen Ring $A$. Dann sagen wir $A$\textbf{-Schema} statt $S$-Schema.
	\end{enumerate}
\end{defn}

\begin{bem*}
    Alternativ zur Notation in Definition~\ref{defn:5.7} verwenden wir manchmal auch folgende Sprechweisen:
    Für ein $S$-Schema~$X$ sagen wir auch: $X$ ist ein \textbf{Schema über}~$S$. Ist $S=\Spec(A)$, so sagen
    wir für ein $A$-Schema~$X$ auch: $X$ ist ein \textbf{Schema über}~$A$.
\end{bem*}

\begin{bem*}
    Nach Proposition~\ref{prop:5.6} haben wir ein $A$-Schema $f \colon X \to \Spec(A)$ genau dann, wenn wir einen Ringhomomorphismus $f^{\#}_{\Spec(A)}\colon A \to \mco_X(X)$ haben, also wenn es eine $A$-Alge\-brastruktur auf $\mco_X(X)$ gibt.
\end{bem*}

\begin{bsp}
\label{bsp:5.8}
	\begin{enumerate}[i)]
		\item Setzte $A \coloneqq \Z$. Dann existiert für jedes Schema $X$ genau ein Morphismus $X \to \Spec(\Z)$ und somit ist $X$ ein kanonisches $\Z$-Schema, da genau ein Ringhomomorphismus $\Z \to \mco_X(X)$ existiert. Damit ist $\Spec(\Z)$ ein finales Objekt in der Kategorie der Schemata.
		\item Sei $X$ ein Schema, dann setzen wir $A \coloneqq \mco_X(X)$. Dann gibt es einen kanonischen Homomorphismus $A \to \mco_X(X)$, nämlich die Identität. Damit existiert nach Proposition~\ref{prop:5.6} ein kanonischer Morphismus $X \to \Spec(A)$, das heißt $X$ ist kanonisch ein $(A=\mco_X(X))$-Schema.
	\end{enumerate}
\end{bsp}

\nextmark{Verklebedatum}
\begin{defn}
\label{defn:5.9}
	Ein \textbf{Verklebedatum} von Schemata ist eine Familie von Tripeln der Form $(X_i,U_{ij},\varphi_{ij})_{i,j\in I}$, wobei $X_i$ für alle $i \in I$ ein Schema ist, $U_{ij}$ für alle $i,j \in I$ eine offene Teilmenge von $X_i$ ist und $\varphi_{ij}$ für alle $i,j \in I$ ein Isomorphismus
	\[
		\varphi_{ij} \colon (U_{ij}=(U_{ij},\mco_{X_i}\vert_{U_{ij}})) \simto (U_{ji}=(U_{ji},\mco_{X_j}\vert_{U_{ji}}))
	\]
	mit folgenden Eigenschaften ist:
	\begin{enumerate}[i)]
		\item Für alle $i,j \in I$ gilt $\varphi_{ij} = \varphi^{-1}_{ji}$, $U_{ii} = X_i$ und $\varphi_{ii}= \id$.
		\item Für alle $i,j,k \in I$ gilt $\varphi_{ij}(U_{ij}\cap U_{ik}) = U_{ji} \cap U_{jk}$.
		\item Für alle $i,j,k \in I$ gilt $\varphi_{ik} = \varphi_{jk}\circ \varphi_{ij}$ auf $U_{ij} \cap U_{ik}$.
	\end{enumerate}
\end{defn}

\begin{lem}[Verkleben von Schemata]
\label{lem:5.10}
	Sei ein Verklebedatum $(X_i,U_{ij},\varphi_{ij})_{i,j\in I}$ gegeben. Dann gibt es ein bis auf Isomorphie eindeutig bestimmtes Schema $X$ zusammen mit Morphismen $\psi_i \colon X_i \to X$, die für alle $i,j \in I$ folgende Eigenschaften erfüllen:
	\begin{enumerate}[i)]
		\item $\psi_i$ induziert einen Isomorphismus $\psi_i \colon X_i \to \psi_i(X_i)$ von $X_i$ auf das offene Unterschema $\psi_i(X_i)$ von $X$.
		\item Es gilt $\bigcup_{i \in I} \psi_i(X_i) = X$.
		\item Es gilt $\psi_i(U_{ij}) = \psi_i(X_i) \cap \psi_j(X_j)$.
		\item Es gilt $\psi_i = \psi_j \circ \varphi_{ij}$ auf $U_{ij}$.
	\end{enumerate}
	\begin{proof}
		Sei $\widetilde{X} \coloneqq \coprod_{i\in I}X_i$ versehen mit der disjunkten Vereinigungstopologie, das heißt die offenen Mengen in $\widetilde{X}$ haben die Form $\coprod_{i\in I} U_i$, wobei $U_i$ für alle $i \in I$ offen in $X_i$ ist. Wir definieren eine Äquivalenzrelation auf $\widetilde{X}$ wie folgt:
		\begin{align*}
			a \sim b \quad:\Longleftrightarrow\quad \exists \;i,j \in I \text{ mit } a \in U_i,\, b \in U_j \text{ und } \varphi_{ij}(a) = b .
		\end{align*}
                Sei $X \coloneqq \widetilde{X}/{\sim}$ der Raum der Äquivalenzklassen. Dann erhalten wir die Klassenabbildung
		\[
			\pi\colon \widetilde{X} \twoheadrightarrow X,\; a \mapsto [a].
		\]
		Wir versehen $X$ mit der \textbf{Quotiententopologie}, das heißt $U$ ist genau dann offen in $X$, wenn $\pi^{-1}(U)$ offen in $\widetilde{X}$ ist. $\pi$ wird damit eine stetige und offene Abbildung. Nach Definition ist somit $U$ genau dann offen in $X$, wenn $\psi_i^{-1}(U)$ für alle $i \in I$ offen in $X_i$ ist, wobei $\psi_i$ durch
                % FIXME/TODO: Wieso ist $\pi$ offen? (Nicht jede Quotientenabbildung ist offen ...)
		\[
			\psi_i\colon X_i \to X,\; a \mapsto [a]
		\]
		gegeben ist. Beachte, dass $\psi_i$ für alle $i \in I$ stetig und injektiv ist. Nach Konstruktion gelten nun ii), iii), iv). Ebenso folgt, dass $\psi_i$ einen Homöomorphismus $X_i \to \psi_i(X_i)$ induziert. Weiter sehen wir aufgrund der Charakterisierung offener Mengen in $X$ ein, dass $U_i\coloneqq \psi_i(X_i)$ offen in $X$ ist. Wir definieren die Garbe
		\[
			\mco_{U_i} \coloneqq {\psi_i}_*(\mco_{X_i})
		\]
		auf $U_i$. Wir erhalten einen Isomorphismus
		\[
			\psi_i\colon(X_i,\mco_{X_i})\simto (U_i,\mco_{U_i})	
		\]
		von lokal geringten Räumen. Also ist $(U_i,\mco_{U_i})$ ein Schema. Die Garbenisomorphismen
		\[
			\varphi^{\#}_{ij}\colon \mco_{X_j}\vert_{U_{ji}} \simto (\varphi_{ij})_*(\mco_{X_i}\vert_{U_{ij}})
		\]
		auf dem Verklebedatum liefern Garbenisomorphismen 
		\[
			\widetilde{\varphi}^{\#}_{ij}\colon \mco_{U_i}\vert_{U_i \cap U_j} \simto \mco_{U_j}\vert_{U_i \cap U_j}
		\]
		mit $\widetilde{\varphi}^{\#}_{ii} = \id$, die die sogenannte \textbf{Kozykelbedingung}
		\[
			\widetilde{\varphi}_{ik} = \widetilde{\varphi}_{jk} \circ \widetilde{\varphi}_{ij} \text{ auf } U_i \cap U_j \cap U_k
		\]
		erfüllen. Durch Verkleben der Garben $(U_i,\mco_{U_i})$ entlang der $\widetilde{\varphi}_{ij}$ erhält man eine Garbe $\mco_X$ auf~$X$ (siehe Lemma~\ref{lem:5.11}). Nach Konstruktion ist $(X,\mco_X)$ ein lokal geringter Raum. Da $(X,\mco_X)$ eine offene Überdeckung durch die affinen Schemata
		\[
			(U_i,\mco_{U_i}) \simto (X_i, \mco_{X_i})
		\]
		bestitzt, folgt die Existenz. Die Eindeutigkeit ergibt sich aus der Konstruktion.
	\end{proof}
\end{lem}

\begin{lem}[Verkleben von Garben]
\label{lem:5.11}
	Sein $X$ ein topologischer Raum mit einer offenen Überdeckung $(U_i)_{i\in I}$. Für alle $i \in I$ sei $\mcf_i$ eine Garbe auf $U_i$ und für alle $i,j \in I$ sei
	\[
		\varphi_{ij}\colon \mcf_i\vert_{U_i\cap U_j} \simto \mcf_j\vert_{U_i\cap U_j}
	\]
	ein Isomorphismus von Garben mit folgenden Eigenschaften:
	\begin{enumerate}[i)]
		\item für alle $i \in I$ gilt $\varphi_{ii} = \id_{\mcf_i}$.
		\item Für alle $i,j,k \in I$ gilt $\varphi_{ik} = \varphi_{jk} \circ \varphi_{ij}$ auf $U_i \cap U_j \cap U_k$.
	\end{enumerate}
	Dann existiert eine bis auf Isomorphie eindeutige Garbe $\mcf$ auf $X$ zusammen mit Isomorphismen $\psi_i \colon \mcf\vert_{U_i} \to \mcf_i$ derart, dass für alle $i,j \in I$
	\[
                \psi_j\vert_{U_i \cap U_j} = \varphi_{ij} \circ \psi_i\vert_{U_i\cap U_j}
	\]
	gilt. Wir sagen, dass $\mcf$ durch \textbf{Verkleben der $\mcf_i$ entlang der Isomorphismen $\varphi_{ij}$} entsteht.
\end{lem}

\nextmark{disjunkte Vereinigung von Schemata}
\begin{bsp}
\label{bsp:5.12}
	Sei $(X_i)_{i\in I}$ eine Familie von Schemata. Wir setzen $U_{ij}\coloneqq \emptyset$ und wählen als $\varphi_{ij}$ die trivialen Abbildungen. Damit erhalten wir ein Verklebedatum $(X_i,U_{ij},\varphi_{ij})_{i,j\in I}$ und mit Verkleben nach Lemma~\ref{lem:5.10} damit ein Schema $X$. Es ist $X$ die \textbf{disjunkte Vereinigung} $\coprod_{i\in I}X_i$ der $X_i$ als Schemata.
\end{bsp}

\begin{bsp}
\label{bsp:5.13}
	Sei $K$ ein Körper. Für $i=1,2$ sei $X_i\coloneqq \A^1_K = \Spec(K[T])$. Wir setzen $U_{12} \coloneqq X_1 \setminus V(T)$ und $U_{21}\coloneqq X_2 \setminus V(T)$. Als $\varphi_{12}$ und $\varphi_{21}$ wählen wir die identische Abbildung auf $U_{12} = U_{21} = U = \Spec(K[T]) \setminus V(T)$. Verkleben nach Lemma~\ref{lem:5.10} liefert ein Schema $X=$\enquote{$\A^1_K$ mit verdoppeltem Nullpunkt}. Es gilt $X= X_1 \cup X_2$ und $X_1 \cap X_2 = U$. Nach dem Garbenaxiom gilt:
	\begin{align*}
		\mco_X(X) &= \{(f,g) \in \mco_X(X_1)\times \mco_X(X_2) \mid f\vert_U = g\vert_U\}\\
		&=\{(f,g) \in K[T] \times K[T] \mid \frac{f}{1} = \frac{g}{1} \in K[T]_T\}\\
		&= K[T]\text{, wobei wir } \mco(U) = \mco(D(T)) = K[T]_T \text{ benutzt haben.}
	\end{align*}
	\textbf{Behauptung:} $X$ ist kein affines Schema.
	\begin{proof}
		Wäre $X$ ein affines Schema, so wäre
		\[
			X = \Spec(\mco_X(X)) = \Spec(K[T])
		\]
		und
		\[
			(\Gamma(X,\mco_X) = K[T]) \to (\Gamma(X_1,\mco_{X_1}) = K[T])
		\]
		wäre die Identität. Also erhalten wir den Widerspruch $X = X_1$.
	\end{proof}
\end{bsp}

Wir konstruieren jetzt projektive Schemata. Dabei gehen wir analog wie bei affinen Schemata vor, aber \enquote{homogenisieren alles} (vgl. auch Algebraische Geometrie I, projektive Varietäten).

\nextmark{Proj(S), projektives \glqq Nullstellengebilde\grqq}
\begin{defn}
\label{defn:5.14}
	Sei $S = \bigoplus_{d\in \N} S_d$ ein graduierter Ring. Sei $S_+ \coloneqq \bigoplus_{d > 0}S_d$ das homogene Maximalideal, $S^{\text{hom}}\coloneqq \bigcup_{d\in \N}S_d$ die Menge der homogenen Elemente von $S$ und
	\[
		\Proj(S) \coloneqq \{\mfp \mid \mfp \text{ homogenes Primideal in } S \text{ mit } S_+ \not \subseteq \mfp\}.
	\]
	Für ein homogenes Ideal $\mfa$ von $S$ setzen wir
	\[
		V_+(\mfa) = \{\mfp \in \Proj(S) \mid \mfa \subseteq \mfp\}.
	\]
\end{defn}

\begin{lem}
\label{lem:5.15}
	\begin{enumerate}[i)]
		\item Es gilt $V_+(S_+) = \emptyset$ und $V_+(\langle 0 \rangle) = \Proj(S)$.
		\item Für homogene Ideale $\mfa,\mfb$ in $S$ gilt $V_+(\mfa) \cup V_+(\mfb) = V_+(\mfa \cdot \mfb)$.
		\item Für homogene Ideale $(\mfa_i)_{i\in I}$ in $S$ gilt $\bigcap_{i\in I}V_+(\mfa_i) = V_+\left(\sum_{i\in I}\mfa_i\right)$.
	\end{enumerate}
	\begin{proof}
		i) ist trivial. Die Eigenschaften ii) und iii) folgen aus
		\[
			V_+(\mfa) = V(\mfa) \cap \Proj(S) \subseteq \Spec(S)
		\]
		und den entsprechenden Aussagen für $V(\mfa)$ in $\Spec(S)$.
	\end{proof}
\end{lem}

\nextmark{projektives Spektrum als Schema}
\begin{kons}[des Schemas $(\Proj(S),\mco)$]
\label{kons:5.16}
	Das Lemma~\ref{lem:5.15} liefert sofort, dass $\Proj(S)$ ein topologischer Raum ist, bei dem abgeschlossenen Teilmengen gerade Mengen der Form $V_+(\mfa)$ für ein homogenes Ideal $\mfa$ von $S$ sind. Damit erhalten wir die Zariski-Topologie auf $\Proj(S)$.

	Wir definieren die Garbe $\mco$ auf $\Proj(S)$ folgendermaßen: Sei $U$ offen in $\Proj(S)$. Dann ist $\mco(U)$ die Menge der Funktionen $s\colon U \to \coprod_{\mfp \in U}S_{(\mfp)}$, die folgende Eigenschaften erfüllen:
	\begin{enumerate}[a)]
		\item Für alle $\mfp \in U$ gilt $s(\mfp) \in S_{(\mfp)}$.
		\item Für alle $\mfp \in U$ gibt es eine offene Umgebung $V$ von $\mfp$ in $U$ und $a,f\in S^{\text{hom}}$ mit folgenden Eigenschaften:
                    $\deg(a)=\deg(f)$ und für alle $\mfq \in V$ gilt $f \notin \mfq$ und $s(\mfq) =  \frac{a}{f} \in S_{(\mfq)}$.
	\end{enumerate}
	Hier ist
	\[
		S_{(\mfp)} \coloneqq \left\lbrace\text{Elemente vom Grad }0 \text{ in } S_{S^{\text{hom}}\setminus\mfp} \right\rbrace = \left\lbrace \frac{a}{s} \mid a \in S^{\text{hom}},\, s \in S^{\text{hom}}\setminus \mfp,\, \deg(a) = \deg(s)\right\rbrace.
	\]
	Analog zum affinen Fall ist $\mco$ eine Garbe auf $\Proj(S)$. Wir nennen $(\Proj(S),\mco)$ das \textbf{projektive Spektrum} von $S$.
\end{kons}

\nextmark{Eig. der Strukturgarbe auf Proj(S)}
\begin{prop}
\label{prop:5.17}
	\begin{enumerate}[i)]
		\item Für alle $\mfp \in \Proj(S)$ gibt es einen kanonischen Isomorphismus
		\[
			\mco_\mfp \simto S_{(\mfp)}.
		\]
		\item Für $f \in S^{\text{hom}}$ ist $D_+(f) \coloneqq \{\mfp \in \Proj(S) \mid f \notin \mfp\}$ offen in $\Proj(S)$. Für eine Familie $(f_i)_{i\in I}$ in $S^{\text{hom}}$ mit $\langle\{f_i\mid i\in I\}\rangle = S_+$ gilt $\Proj(S) = \bigcup_{i\in I} D_+(f_i)$.
		\item Für $d \in \N$, $d>0$, $f \in S_d$ und $S_{(f)} \coloneqq \left\lbrace \frac{a}{f^m}\mid a \in S^{\text{hom}},\, \deg(a) = md\right\rbrace$ gibt es einen kanonischen Isomorphismus
		\[
			(\varphi, \varphi^{\#})\colon\Big(D_+(f),\mco\vert_{D_+(f)}\Big)\simto \Big(\Spec(S_{(f)}),\mco_{\Spec(S_{(f)})}\Big).
		\]
		\item $(\Proj(S),\mco)$ ist in kanonischer Weise ein $S_0$-Schema.
	\end{enumerate}
	\begin{proof}
		Dies wird analog zum affinen Fall in Aufgabe 6.2 gezeigt.
	\end{proof}
\end{prop}

\nextmark{affiner/projektiver Raum über einem Ring}
\begin{defn}
\label{defn:5.18}
	Sei $A$ ein Ring und $n \in \N$. Dann definieren wir
	\[
		\A^n_A \coloneqq \Spec(A[T_1,\ldots,T_n])
	\]
	als den $n$-dimensionalen \textbf{affinen Raum} über $A$ und
	\[
		\P^n_A \coloneqq \Proj(A[T_0,\ldots,T_n])
	\]
	als den $n$-dimensionalen \textbf{projektiven Raum} über $A$.
\end{defn}
\begin{bsp}
\label{bsp:5.19}
	Sei $S \coloneqq A[T_0,\ldots,T_n]$. Für $i \in \{0,\ldots,n\}$ ergibt sich ein Isomorphismus
	\begin{align*}
		A[T_0,\ldots,T_{i-1},\widehat{T_i},T_{i+1},\ldots,T_n] & \simto S_{(T_i)}\\
		f(T_0,\ldots,T_{i-1},\widehat{T_i},T_{i+1},\ldots,T_n) & \longmapsto f\left(\frac{T_0}{T_i},\ldots,\frac{T_n}{T_i}\right),
	\end{align*}
	wobei mit $\widehat{T_i}$ gemeint ist, dass wir $T_i$ weglassen. Damit folgt $D_+(T_i) \cong \A^n_A$ und wir erhalten aus Proposition~\ref{prop:5.17}, dass $\P^n_A = \bigcup_{i=0}^n D_+(T_i)$ eine offene Überdeckung durch affine Räume ist (vergleiche Standardüberdeckung von $\P^n_k$ in der algebraischen Geometrie I).
\end{bsp}

%%% Local Variables: 
%%% mode: latex
%%% TeX-master: "algebraische_geometrie_2"
%%% End: 
