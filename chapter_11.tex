%!TEX root = algebraische_geometrie_2.tex
% vim: tw=0 noet sts=8 sw=8

\chapter{Projektive Morphismen}

Wir werden projektive Schemata oder allgemeiner projektive Morphismen studieren. Sie sind ein wichtiger Spezialfall von eigentlichen Morphismen. Im Chow-Lemma % TODO: ref
werden wir eine gewisse Umkehrung sehen und zeigen, dass jeder eigentliche Morphismus unter gewissen Umständen durch einen eigentlichen Morphismus dominiert wird.

\begin{defn}
	\label{defn:11.1}
	\begin{enumerate}[i)]
	\item Ein Morphismus $f \colon X \to Y$ von Schemata heißt \textbf{projektiv}, wenn eine abgeschlossene Immersion $i$ existiert, die das Diagramm
		\begin{center}
			\begin{tikzcd}
				X \arrow[hook]{r}{i} \ar{dr}[swap]{f} & \P_Y^n \;\mathrlap{\coloneqq \P_\Z^n \times_{\Spec(\Z)} Y} \ar{d}{p_2} \\
				& Y
			\end{tikzcd}
			\hspace*{1.2cm}
		\end{center}
		kommutativ macht.
	\item Ein Morphismus $f \colon X \to Y$ von Schemata heißt \textbf{quasi-projektiv}, wenn es eine offene Immersion $o \colon X \to X'$ in ein Schema~$X'$ und einen projektiven Morphismus $f' \colon X' \to Y$ gibt, so dass das Diagramm
		\begin{center}
			\begin{tikzcd}
				X \ar[hook]{r}{o} \ar{dr}[swap]{f} & X' \ar{d}{f'}\\
				& Y
			\end{tikzcd}
		\end{center}
		kommutiert.
	\end{enumerate}
\end{defn}

\begin{warn}
	\label{warn:11.2}
	Dies ist nicht die allgemeine Definition projektiver Morphismen aus \cite[5.5]{grothendieck1971elements}. Letztere ist im Gegensatz zu Definition~\ref{defn:11.1} lokal im Bild definiert.
	Beide Definitionen sind äquivalent, falls $Y$ eine quasi-projektive Varietät über einem affinen Schema ist.
\end{warn}

\begin{prop}
	\label{prop:11.3}
	\begin{enumerate}[i)]
		\item\label{prop:11.3:i} Sei $\varphi \colon S \to S'$ ein Homomorphismus graduierter Ringe. Dann wird ein Morphismus
		\[
			\bigl(f = \Proj(\varphi)\bigr) \colon \Proj(S') \setminus V_+\Bigl(\big\langle\varphi(S_+)\big\rangle\Bigr) \to \Proj(S)
		\]
		mit folgenden Eigenschaften induziert:
		\begin{alignat*}{2}
			f(\mfp) &\coloneqq \varphi^{-1}(\mfp) \quad & \forall \;\mfp &\in \Proj(S') \setminus V_+\Bigl(\big\langle\varphi(S_+)\big\rangle\Bigr) \\
			f^{-1} (D_+(h)) &= D_+(\varphi(h))\quad & \forall \;h &\in S_+^{\text{hom}} \\
			f|_{D_+(\varphi(h))} &= \Spec(\varphi_{(h)})\quad & \forall\; h &\in S_+^{\text{hom}}
		\end{alignat*}
		Wir verwenden dabei die folgenden kanonischen Abbildungen:
		\begin{align*}
			\Bigl(D_+(\varphi(h)) = \Spec(S_{(\varphi(h))}')\Bigr) & \to \Bigl(D_+(h) = \Spec(S_{(h)})\Bigr)\\
			\varphi_{(h)} \coloneqq \Bigl(S_{(h)} & \to S_{(\varphi(h))}'\Bigr)
		\end{align*}
		\item\label{prop:11.3:ii} Ist $\varphi$ in \ref{prop:11.3:i} surjektiv, dann ist $V_+\Bigl(\big\langle\varphi(S_+)\big\rangle\Bigr) = \emptyset$ und $\Proj(\varphi)\colon\Proj(S') \to \Proj(S)$ ist eine abgeschlossene Immersion.
		\item\label{prop:11.3:iii} Sei $A$ ein Ring, $S$ eine graduierte $A$-Algebra, $B$ eine $A$-Algebra. Betrachte die Graduierung
		\[
			S \otimes_A B = \bigoplus_{d\in \N} \left( S_d \otimes B \right).
		\]
		Dann existiert ein kanonischer Isomorphismus
		\[
			\Proj(S) \times_{\Spec(A)} \Spec(B) \simto \Proj(S \otimes_A B).
		\]
		\item\label{prop:11.3:iv} Seien $S,S'$ graduierte $A$-Algebren. Für die graduierte $A$-Algebra
		\[
			S \times_A S' \coloneqq \bigoplus_{d\in\N} \left( S_d \otimes_A S_d' \right) \subseteq S \otimes_A S'
		\]
		existert eine kanonischer Isomorphismus
		\[
			\Proj(S) \times_{\Spec(A)} \Proj(S') \simto \Proj(S \times_A S').
		\]
	\end{enumerate}
	
	\begin{proof}
		Dies folgt leicht aus den Definitionen.
	\end{proof}
\end{prop}

\begin{kor}
	\label{kor:11.4}
	\begin{enumerate}[i)]
		\item\label{kor:11.4:i} Sei $S$ ein graduierter Ring, der als $S_0$-Algebra durch endlich viele Elemente aus $S_1$ erzeugt wird. Dann ist der kanonische Morphismus
			\[
				\Proj(S) \to \Spec(S_0)
			\]
			projektiv.
		\item\label{kor:11.4:ii} Seien $Y$ ein Schema und $N \coloneqq mn + m + n = (m+1)(n+1)-1$ für $m,n \in \N$ gegeben. Dann existiert eine kanonische abgeschlossene Immersion
			\[
				\psi_{m,n} \colon \P_Y^m \times \P_Y^n \to \P_Y^N,
			\]
			die \textbf{Segre-Einbettung} genannt wird.
	\end{enumerate}
	\begin{bsp*}[Ein klassisches Beispiel aus der Algebraischen Geometrie I] Es sei $k$ ein algebraisch abgeschlossener Körper, dann erhält man
		\begin{align*}
			\psi_{m,n} \colon \P_k^m \times \P_k^n &\to \P_k^N \\
			((x_0 : \dotsm : x_m),(y_1:\dotsm:y_n)) &\mapsto (x_0 y_0 : x_0 y_1 : \dotsm : x_0 y_n: x_1 y_0 : \dotsm : x_n y_m)
		\end{align*}    
	\end{bsp*}
	\begin{proof}
		\begin{enumerate}[i)]
		\item Sei $A =S_0$. Da $S$ durch endlich viele Elemente aus $S_1$ als $A$-Algebra erzeugt wird existiert eine Surjektion
			\[
				S' \coloneqq A[T_0,\dotsc,T_n] \to S
			\]
			von graduierten Ringen. Nach Proposition~\ref{prop:11.3} \ref{prop:11.3:ii} ist $\Proj(S) \to \Proj(S')$ eine abgeschlossene Immersion, und man erhält ein kommutatives Diagramm
			\begin{center}
				\begin{tikzcd}
					\Proj(S) \ar[r] \ar[d] & \Proj(S') = \P_A^n \ar[ld] \\
					\Spec(A)
				\end{tikzcd}
			\end{center}
			und damit die Tatsache, dass $\Proj(S) \to \Spec(A)$ projektiv ist.
		\item
			Setze 
			\begin{align*}
				S &\coloneqq \Z[T_0,\dotsc,T_m]\\
				S' &\coloneqq \Z[T_0,\dotsc, T_n]\\
				S'' &\coloneqq Z_0 [ T_0,\dotsc,T_N ]\\
				&\phantom{:}= \Z[T_{ij} | 0 \leq i \leq m, 0 \leq j \leq n ]
			\end{align*}
			mit lexikographischer Ordnung auf den $T_{ij}$. Dann haben wir einen surjektiven Homomorphismus
			\begin{align*}
				S'' &\to S \times_A S'\\
				T_{ij} & \mapsto T_i \otimes T_j .
			\end{align*}
			Nach Proposition~\ref{prop:11.3} \ref{prop:11.3:ii} und \ref{prop:11.3:iv} erhalten wir eine abgeschlossene Immersion $\P_\Z^m \times_\Z \P_\Z^n \to \P_\Z^N$. Durch Basiswechsel erhalten wir die Aussage.
		\end{enumerate}
	\end{proof}
\end{kor}

\begin{prop}
	\label{prop:11.5}
	\begin{enumerate}[i)]
	\item\label{prop:11.5:i} Abgeschlossene Immersionen sind projektiv.
	\item\label{prop:11.5:ii} Projektivität ist stabil unter Basiswechsel.
	\item\label{prop:11.5:iii} Eine Komposition von projektiven Morphismen ist projektiv.
	\item\label{prop:11.5:iv} Für projektive Morphismen $X\to S$ und $Y \to S$ ist $X \times_S Y \to S$ projektiv.
	\item\label{prop:11.5:v} Sind $f \colon X \to Y$ und $g \colon Y \to Z$ Morphismen mit der Eingeschaft, dass $g$ und $g \circ f$ projektiv sind, dann ist auch $f$ projektiv.
	\end{enumerate}
	\begin{proof}[Beweisidee für \ref{prop:11.5:iii}] Seien $f \colon X \to Y$ und $g \colon Y \to Z$ projektive Morphismen. Dann erhalten wir das folgende kommutative Diagramm:
		\begin{center}
			\begin{tikzcd}
				X \ar[hook]{r}{i_X} \ar{dr}{f}
				& \P_Y^n \ar[r] \ar[d] \ar[phantom]{dr}{\square}
				& \P_Z^m \times \P_Z^n \ar{d} \ar{r}{\psi_{n,m}}[swap]{\text{Segre}}
				& \P_Z^N \ar[bend left]{ddl}\\ 
				& Y \ar[hook]{r}{i_Y} \ar{dr}{g}
				& \P_Z^m \ar{d}\\
				& & Z
			\end{tikzcd}
		\end{center}
		Man kann zeigen, dass das Rechteck in der Mitte des Diagramms karthesisch ist. Damit ist $\P_Y^n \to P_Z^m \times \P_Z^n$ nach \ref{prop:11.5:iii} eine abgeschlossene Immersion und $\psi_{n,m}$ nach Korollar~\ref{kor:11.4} \ref{kor:11.4:ii} ebenfalls. Da die Komposition abgeschlossener Immersionen wieder eine abgeschlossene Immersion ist, ist $f \circ g$ projektiv.
	\end{proof}
\end{prop}

%%% Local Variables: 
%%% mode: latex
%%% TeX-master: "algebraische_geometrie_2"
%%% End: 
