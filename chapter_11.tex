%!TEX root = algebraische_geometrie_2.tex
% vim: tw=0 noet sts=8 sw=8

\chapter{Projektive Morphismen}

Wir werden projektive Schemata oder allgemeiner projektive Morphismen studieren. Sie sind ein wichtiger Spezialfall von eigentlichen Morphismen. Im Chow-Lemma % TODO: ref
werden wir eine gewisse Umkehrung sehen und zeigen, dass jeder eigentliche Morphismus unter gewissen Umständen durch einen projektiven Morphismus dominiert wird.

\begin{defn}
	\label{defn:11.1}
	\begin{enumerate}[i)]
	\item Ein Morphismus $f \colon X \to Y$ von Schemata heißt \textbf{projektiv}, wenn eine abgeschlossene Immersion $i$ existiert, die das Diagramm
		\begin{center}
			\begin{tikzcd}
				X \arrow[hook]{r}{i} \ar{dr}[swap]{f} & \P_Y^n \;\mathrlap{\coloneqq \P_\Z^n \times_{\Spec(\Z)} Y} \ar{d}{p_2} \\
				& Y
			\end{tikzcd}
			\hspace*{1.2cm}
		\end{center}
		kommutativ macht.
	\item Ein Morphismus $f \colon X \to Y$ von Schemata heißt \textbf{quasi-projektiv}, wenn es eine offene Immersion $o \colon X \to X'$ in ein Schema~$X'$ und einen projektiven Morphismus $f' \colon X' \to Y$ gibt, so dass das Diagramm
		\begin{center}
			\begin{tikzcd}
				X \ar[hook]{r}{o} \ar{dr}[swap]{f} & X' \ar{d}{f'}\\
				& Y
			\end{tikzcd}
		\end{center}
		kommutiert.
	\end{enumerate}
\end{defn}

\begin{warn}
	\label{warn:11.2}
	Dies ist nicht die allgemeine Definition projektiver Morphismen aus \cite[{}5.5]{grothendieck1961elements}. Letztere ist im Gegensatz zu Definition~\ref{defn:11.1} lokal im Bild definiert.
	Beide Definitionen sind äquivalent, falls $Y$ eine quasi-projektive Varietät über einem affinen Schema ist.
\end{warn}

\begin{prop}
	\label{prop:11.3}
	\begin{enumerate}[i)]
		\item\label{prop:11.3:i} Sei $\varphi \colon S \to S'$ ein Homomorphismus graduierter Ringe. Dann wird ein Morphismus
		\[
			\bigl(f = \Proj(\varphi)\bigr) \colon \Proj(S') \setminus V_+\Bigl(\big\langle\varphi(S_+)\big\rangle\Bigr) \to \Proj(S)
		\]
		mit folgenden Eigenschaften induziert:
		\begin{alignat*}{2}
			f(\mfp) &\coloneqq \varphi^{-1}(\mfp) \quad & \forall \;\mfp &\in \Proj(S') \setminus V_+\Bigl(\big\langle\varphi(S_+)\big\rangle\Bigr) \\
			f^{-1} (D_+(h)) &= D_+(\varphi(h))\quad & \forall \;h &\in S_+^{\text{hom}} \\
			f|_{D_+(\varphi(h))} &= \Spec(\varphi_{(h)})\quad & \forall\; h &\in S_+^{\text{hom}}
		\end{alignat*}
		Wir verwenden dabei die folgenden kanonischen Abbildungen:
		\begin{align*}
			\Bigl(D_+(\varphi(h)) = \Spec(S_{(\varphi(h))}')\Bigr) & \to \Bigl(D_+(h) = \Spec(S_{(h)})\Bigr)\\
			\varphi_{(h)} \coloneqq \Bigl(S_{(h)} & \to S_{(\varphi(h))}'\Bigr)
		\end{align*}
		\item\label{prop:11.3:ii} Ist $\varphi$ in \ref{prop:11.3:i} surjektiv, dann ist $V_+\Bigl(\big\langle\varphi(S_+)\big\rangle\Bigr) = \emptyset$ und $\Proj(\varphi)\colon\Proj(S') \to \Proj(S)$ ist eine abgeschlossene Immersion.
		\item\label{prop:11.3:iii} Sei $A$ ein Ring, $S$ eine graduierte $A$-Algebra, $B$ eine $A$-Algebra. Betrachte die Graduierung
		\[
			S \otimes_A B = \bigoplus_{d\in \N} \left( S_d \otimes B \right).
		\]
		Dann existiert ein kanonischer Isomorphismus
		\[
			\Proj(S) \times_{\Spec(A)} \Spec(B) \simto \Proj(S \otimes_A B).
		\]
		\item\label{prop:11.3:iv} Seien $S,S'$ graduierte $A$-Algebren. Für die graduierte $A$-Algebra
		\[
			S \times_A S' \coloneqq \bigoplus_{d\in\N} \left( S_d \otimes_A S_d' \right) \subseteq S \otimes_A S'
		\]
		existert eine kanonischer Isomorphismus
		\[
			\Proj(S) \times_{\Spec(A)} \Proj(S') \simto \Proj(S \times_A S').
		\]
	\end{enumerate}
	
	\begin{proof}
		Dies folgt leicht aus den Definitionen.
	\end{proof}
\end{prop}

\begin{kor}
	\label{kor:11.4}
	\begin{enumerate}[i)]
		\item\label{kor:11.4:i} Sei $S$ ein graduierter Ring, der als $S_0$-Algebra durch endlich viele Elemente aus $S_1$ erzeugt wird. Dann ist der kanonische Morphismus
			\[
				\Proj(S) \to \Spec(S_0)
			\]
			projektiv.
		\item\label{kor:11.4:ii} Seien $Y$ ein Schema und $N \coloneqq mn + m + n = (m+1)(n+1)-1$ für $m,n \in \N$ gegeben. Dann existiert eine kanonische abgeschlossene Immersion
			\[
				\psi_{m,n} \colon \P_Y^m \times \P_Y^n \to \P_Y^N,
			\]
			die \textbf{Segre-Einbettung} genannt wird.
	\end{enumerate}
	\begin{bsp*}[Ein klassisches Beispiel aus der Algebraischen Geometrie I] Es sei $k$ ein algebraisch abgeschlossener Körper, dann erhält man
		\begin{align*}
			\psi_{m,n} \colon \P_k^m \times \P_k^n &\to \P_k^N \\
			((x_0 : \dotsm : x_m),(y_1:\dotsm:y_n)) &\mapsto (x_0 y_0 : x_0 y_1 : \dotsm : x_0 y_n: x_1 y_0 : \dotsm : x_n y_m)
		\end{align*}    
	\end{bsp*}
	\begin{proof}
		\begin{enumerate}[i)]
		\item Sei $A =S_0$. Da $S$ durch endlich viele Elemente aus $S_1$ als $A$-Algebra erzeugt wird existiert eine Surjektion
			\[
				S' \coloneqq A[T_0,\dotsc,T_n] \to S
			\]
			von graduierten Ringen. Nach Proposition~\ref{prop:11.3} \ref{prop:11.3:ii} ist $\Proj(S) \to \Proj(S')$ eine abgeschlossene Immersion, und man erhält ein kommutatives Diagramm
			\begin{center}
				\begin{tikzcd}
					\Proj(S) \ar[r] \ar[d] & \Proj(S') = \P_A^n \ar[ld] \\
					\Spec(A)
				\end{tikzcd}
			\end{center}
			und damit die Tatsache, dass $\Proj(S) \to \Spec(A)$ projektiv ist.
		\item
			Setze 
			\begin{align*}
				S &\coloneqq \Z[T_0,\dotsc,T_m]\\
				S' &\coloneqq \Z[T_0,\dotsc, T_n]\\
				S'' &\coloneqq Z_0 [ T_0,\dotsc,T_N ]\\
				&\phantom{:}= \Z[T_{ij} | 0 \leq i \leq m, 0 \leq j \leq n ]
			\end{align*}
			mit lexikographischer Ordnung auf den $T_{ij}$. Dann haben wir einen surjektiven Homomorphismus
			\begin{align*}
				S'' &\to S \times_A S'\\
				T_{ij} & \mapsto T_i \otimes T_j .
			\end{align*}
			Nach Proposition~\ref{prop:11.3} \ref{prop:11.3:ii} und \ref{prop:11.3:iv} erhalten wir eine abgeschlossene Immersion $\P_\Z^m \times_\Z \P_\Z^n \to \P_\Z^N$. Durch Basiswechsel erhalten wir die Aussage.
		\end{enumerate}
	\end{proof}
\end{kor}

\begin{prop}
	\label{prop:11.5}
	\begin{enumerate}[i)]
	\item\label{prop:11.5:i} Abgeschlossene Immersionen sind projektiv.
	\item\label{prop:11.5:ii} Projektivität ist stabil unter Basiswechsel.
	\item\label{prop:11.5:iii} Eine Komposition von projektiven Morphismen ist projektiv.
	\item\label{prop:11.5:iv} Für projektive Morphismen $X\to S$ und $Y \to S$ ist $X \times_S Y \to S$ projektiv.
	\item\label{prop:11.5:v} Sind $f \colon X \to Y$ und $g \colon Y \to Z$ Morphismen mit der Eingeschaft, dass $g$ und $g \circ f$ projektiv sind, dann ist auch $f$ projektiv.
	\end{enumerate}
	\begin{proof}[Beweisidee für \ref{prop:11.5:iii}] Seien $f \colon X \to Y$ und $g \colon Y \to Z$ projektive Morphismen. Dann erhalten wir das folgende kommutative Diagramm:
		\begin{center}
			\begin{tikzcd}
				X \ar[hook]{r}{i_X} \ar{dr}{f}
				& \P_Y^n \ar[r] \ar[d] \ar[phantom]{dr}{\square}
				& \P_Z^m \times \P_Z^n \ar{d} \ar{r}{\psi_{n,m}}[swap]{\text{Segre}}
				& \P_Z^N \ar[bend left]{ddl}\\ 
				& Y \ar[hook]{r}{i_Y} \ar{dr}{g}
				& \P_Z^m \ar{d}\\
				& & Z
			\end{tikzcd}
		\end{center}
		Man kann zeigen, dass das Rechteck in der Mitte des Diagramms karthesisch ist. Damit ist $\P_Y^n \to P_Z^m \times \P_Z^n$ nach \ref{prop:11.5:iii} eine abgeschlossene Immersion und $\psi_{n,m}$ nach Korollar~\ref{kor:11.4} \ref{kor:11.4:ii} ebenfalls. Da die Komposition abgeschlossener Immersionen wieder eine abgeschlossene Immersion ist, ist $f \circ g$ projektiv.
	\end{proof}
\end{prop}

\begin{lem}
\label{lem:11.6}
	\begin{enumerate}[i)]
		\item\label{lem:11.6:i} Für ein homogenes Ideal $I$ im graduierten Ring $S$ mit $I \subseteq S_+$ gilt
		\[
			S_+ \cap \sqrt{I} = S_+ \cap \bigcap_{\substack{\mfp \in \Proj(S)\\I\subseteq\mfp}}\mfp.
		\]
		\item\label{lem:11.6:ii} Es gilt $\Proj(S) = \emptyset$ genau dann, wenn $S_+\subseteq \nil(S)$.
	\end{enumerate}
	\begin{proof}
		\begin{enumerate}[i)]
			\item \cite[Proposition~13.2~(3)]{goertz2010algebraic}.
			\item \begin{description}
				\item[\enquote{$\Leftarrow$}:] Sei $S_+\subseteq \nil(S)$. Dann gilt
				\[
					S_+ \subseteq \nil(S) = \sqrt{\{0\}} = \bigcap_{\mfp\in \Spec(S)}\mfp
				\]
				und damit folgt $\Proj(S) = \emptyset$.
				\item[\enquote{$\Rightarrow$}:] Sei $\Proj(S) = \emptyset$. Wir verwenden \ref{lem:11.6:i} für $I=\{0\}$. Dann folgt
				\[
					S_+\cap \sqrt{\{0\}} = S_+ \cap \bigcap_{\substack{\mfp \in \Proj(S)\\U\subseteq\mfp}}\mfp = S_+
				\]
				und daraus folgt wiederum
				\[
					S_+ \subseteq \nil(S).
				\]
			\end{description}
		\end{enumerate}
	\end{proof}
\end{lem}

\begin{thm}
\label{thm:11.7}
	Ein projektiver Morphismus von Schemata ist eigentlich.
	\begin{proof}
		Sei $f\colon X \to Y$ ein projektiver Morphismus, das heißt $f$ faktorisiert auf folgende Weise:
		\begin{center}
			\begin{tikzcd}
				X\arrow{r}{i} \arrow{dr}[swap]{f}& \P^n_Y\arrow{d}{p_2}\\
				&Y
			\end{tikzcd}
		\end{center}
		Eine abgeschlossene Immersion ist nach Aufgabe~11.2 eigentlich. Weil die Eigenschaft, eigentlich zu sein, abgeschlossen unter Komposition ist, genügt es zu zeigen, dass $\P_Y^n\to Y$ eigentlich ist. Weil die Eigenschaft, eigentlich zu sein, nach Aufgabe~11.2 stabil unter Basiswechsel ist, genügt es zu zeigen, dass $\P^n_\Z\to \Spec(\Z)$ eigentlich ist. Nach Aufgabe~10.3 ist $p_2\colon \P^n_\Z \to \Spec(\Z)$ separiert und nach Beispiel~\ref{bsp:6.12} vom endlichem Typ. Es bleibt zu zeigen, dass $p_2$ universell abgeschlossen ist. Sei also $X$ ein Schema, dann wollen wir zeigen, dass $\pi\colon \P^n_X\to X$ abgeschlossen ist. Da wir dies lokal auf $X$ prüfen können, können wir ohne Beschränkung der Allgemeinheit $X = \Spec(A)$ für einen Ring $A$ annehmen. Für $B\coloneqq A[T_0,\ldots,T_n]$ gilt $\P^n_X=\Proj(B)$. Sei $V$ abgeschlossen in $\P^n_X$. Wir müssen zeigen, dass $\pi(V)$ abgeschlossen in $X$ ist. Nach Definition der Topologie auf $\Proj(B)$ folgt
		\[
			V = V_+(\mfa)
		\]
		für ein homogenes Ideal $\mfa$ von $B$. Wir werden zeigen, dass $X \setminus\pi(V)$ offen in $X$ ist. Sei $C\coloneqq B/\mfa$, dann ist $C$ ein graduierter Ring. Da der Quotientenhomomorphismus $B\to C$ surjektiv ist, erhalten wir eine abgeschlossene Immersion $(V=\Proj(C))\to (\Proj(B) =\P^n_X = \P^n_A)$ und damit $V$ als abgeschlosssenes Unterschema von $\P^n_X$. Für $\mfp \in X \setminus \pi(V)$ gilt $V \cap \pi^{-1}(\mfp) = \emptyset$. Beachte, dass $V\cap \pi^{-1}(\mfp)$ Die Faser der Abbildung $V\overset{\pi}{\longrightarrow} X$ über $\mfp$ ist. Nach Proposition~\ref{prop:11.3} gilt
		\[
			V \cap \pi^{-1}(\mfp) = V \times_X\Spec(\kappa(\mfp)) \overset{\text{Proposition~\ref{prop:11.3}~\ref{prop:11.3:iii}}}{=} \Proj(C\otimes_A\kappa(\mfp))
		\]
		Aus $\emptyset = \Proj(C\otimes_A\kappa(\mfp))$ folgt aber mit Lemma~\ref{lem:11.6}~\ref{lem:11.6:ii}
		\[\label{eq:11.7.1}\tag{$\star$}
			(C\otimes_A\kappa(\mfp))_+ \subseteq \nil(C\otimes_A\kappa(\mfp)).
		\]
		Wegen $B=A[T_0,\ldots,T_n]$ ist $B$ als $A$-Algebra durch endlich viele Elemente aus $B_1$ erzeugt (zum Beispiel $T_0,\ldots,T_n$). Deswegen ist auch $C=B/\mfa$ als $C_0$-Algebra durch endlich viele Elemente aus $C_1$ erzeugt und damit ist auch $C\otimes_A\kappa(\mfp)$ als $C_0\otimes_A\kappa(\mfp)$-ALgebra erzeugt durch endlich viele Elemente aus $C_1\otimes_A\kappa(\mfp)$. Anders geasgt kann man jedes $c \in C\otimes_A\kappa(\mfp)$ als Polynom mit Koeffizienten in $C_0\otimes\kappa(\mfp)$ un in den \enquote{Variablen} $\overline{c_0},\ldots,\overline{c_n} \in C_1 \otimes_A\kappa(\mfp)$ schreieben. Nach \eqref{eq:11.7.1} sind  die Elemente $\overline{c_0},\ldots,\overline{c_n}$ alle nilpotent. Es gibt also ein $d \in \N$ mit der Eigenschaft, dass $C_{d'}\otimes_A \kappa(\mfp) = \{0\}$ für alle $d'\ge d$ gilt. Betrachte jetzt
		\[
			(C_d)_\mfp = C_d\otimes_A A_\mfp.
		\]
		Da $C_d$ also $A$-Modul endlich erzeugt ist, ist $(C_d)_\mfp$ endlich erzeugt über dem lokalen Ring $A_\mfp$. Wegen
		\[
			(C_d)_\mfp \otimes_A \kappa(\mfp) = C_d \otimes_A \kappa(\mfp) = \{0\}
		\]
		folgt mit dem Lemma von Nakayama
		\begin{equation*}\label{eq:11.7.2}
			(C_d)_\mfp = \{0\}.
		\end{equation*}
		Nach Lemma~\ref{lem:9.13} ist $\Supp(C_d)$ abgeschlossen. Es ist aber $\mfp \notin \Supp(C_d)$ wegen \eqref{eq:11.7.2}. Also gibt es ein $h \in A$ mit
		\[
			\mfp \in D(h) \subseteq \Spec(A) \setminus \Supp(C_d).
		\]
		Daraus folgt
		\[
			\underbrace{C_d \otimes_A A_h}_{=(C_d)_h} = \{0\},
		\]
		es gibt also ein $N \in \N$ mit
		\[
			h^N C_d = \{0\}.
		\]
		Wir dürfen $h$ durch $h^N$ ersetzen, dann folgt $hC_d = \{0\}$. Mit $C_d = B_d/\mfa_d$ folgt $hB_d \subseteq \mfa_d$. Inbesondere gilt $hT_i^d \in \mfa_d$ für alle $i \in \{0,\ldots,n\}$.

		Um nun zu zeigen, dass $X \setminus\pi(V)$ offen ist, genügt es zu zeigen, dass $D(h)$ als offene Umgebung von $\mfp$ und $\pi(V)$ disjunkt sind. Dabei gehen wir indirekt vor. Wir nehmen also an, dass es ein $\mfq \in V = V_+(\mfa)$ mit $\pi(\mfq) \in D(h)$ gibt. Es gilt $\pi(\mfq) = A \cap \mfq$ nach Definition von $\pi$. Aus $\pi(\mfq) \in D(h)$ folgt $h \notin A \cap \mfq$. Wegen $h \in A$ gilt insbesondere $h \notin \mfq$. Andererseits gilt $h T_i^d \in \mfa \subseteq \mfq$ und daraus folgt $T_i^d \in \mfq$, denn $\mfq$ ist ein Primideal. Es gilt also auch $T_i\in \mfq$ für alle $i \in \{0,\ldots,n\}$ und damit $B_+ \subseteq \mfq$ im WIderspruch zu $\mfq \in \Proj(B)$.
	\end{proof}
\end{thm}

\begin{kor}
\label{kor:11.8}
	Ein quasiprojektiver Morphismus ist separiert und von endlichem Typ.
	\begin{proof}
		Nach Aufgabe~10.2 ist eine offene Immersion separiert. Offenbar ist eine offene Immersion auch von endlichem Typ. Sei $f\colon X \to Y$ ein quasiprojektiver Morphismus, das heißt $f$ hat eine Faktorisierung $f=g\circ i$, wobei $i$ eine offene Immersion und $g$ ein projektiver Morphismus ist. Da $i$ von endlichem Typ und separiert ist, und weil $g$ nach Theorem~\ref{thm:11.7} von endlichem Typ und separirt ist, ist auch $f=g\circ i$ von endlichem Typ und separiert, da diese Eigneschaften abgeschlossen unter Komposition sind.
	\end{proof}
\end{kor}

Es gibt eigentliche Morphismen, die nicht projektiv sind. In der Theorie der torischen Varietäten gibt man sogar Beispiele von eigentlichen torischen Varietäten über Körpern an, die nicht projektiv sind. Das folgende Lemma ist die bestmögliche \enquote{Umkehrung} von Theorem~\ref{thm:11.7}.

\begin{lem}[Lemma von Chow]
\label{lem:11.9}
	Sei $Y$ ein noethersches Schema und $g \colon X \to Y$ ein eigentlicher Morphismus. Dann existiert ein Schema $X'$, ein Morphismus $h\colon X' \to X$ und eine offene dichte Teilmenge $U$ von $X$ mit der eigenschaft, dass $f \coloneqq g \circ h$ projektiv ist und $h$ einen Morphismus $g^{-1}(U)\to U$ induziert.
	\begin{center}
		\begin{tikzcd}
			h^{-1}(U) \arrow{r}{\sim} & U\\
			X' \arrow{r}{h}\arrow{dr}{f} & X\arrow{d}{g}\\
			{}& Y
		\end{tikzcd}
	\end{center}
	\begin{proof}
		\cite[{}5.6]{grothendieck1961elements}
	\end{proof}
\end{lem}

\begin{bem}
\label{bem:11.10}
	Aus Proposition~\ref{prop:9.3}~\ref{lem:9.13:ii} und Proposition~\ref{prop:9.8} folgt, dass für quasikohärente (beziehungsweise kohärente) $\mco_X$-Moduln $\mcf, \mcg$ auf dem (noetherschen) Schema $X$ auch $\mcf\otimes_{\mco_X}\mcg$ ein quasikohärenter (beziehungsweise kohärenter) $\mco_X$-Modul ist.
\end{bem}

\begin{defn}
\label{defn:11.11}
	Sei $S$ ein graduierter Ring und $X = \Proj(S)$.
	\begin{enumerate}[i)]
		\item Ein \textbf{graduierter $S$-Modul} $M$ ist ein $S$-Modul $M$ mit einer Graduierung $M=\bigoplus_{n\in \Z}M_n$ mit der Eigenschaft, dass $S_dM_n \subseteq M_{d+n}$ für alle $d,n  \in \Z$ gilt. Das Tensorprodukt $M\otimes_S N$ versehen wir mit der Graduierung, die die Eigenschaft hat, dass $\deg(m\otimes n) = \deg(m)+\deg(n)$ für alle $m \in M^\text{hom}$ und $n \in N^\text{hom}$ gilt.
		\item Sei $M$ ein graduierter $S$-Modul und $n \in \Z$. Dann definieren wir den graduierten $S$-Modul $M(n)$ als den Modul $M$ mit neuer Graduierung
			\[
				M(n)_m \coloneqq M_{m+n} \quad \forall\, m \in \Z.
			\]
			\enquote{Wir verschieben die alte Graduierung um $n$ Schritte nach links.} Insbesondere bezeichnet $S(n)$ den $S$-Modul $S$ mit der Graduierung
			\[
				S(n)_m\coloneqq S_{m+n}.
			\]
		\item Ist $n \in \Z$, so bezeichnen wir mit $\mco_X(n)$ den quasikohärenten $\mco_X$-Modul $\widetilde{S(n)}$ aus Proposition~\ref{prop:9.6}. Der $\mco_X$-Modul $\mco_X(1) = \widetilde{S(1)}$ heißt die \textbf{getwistete Garbe von Serre}.
		\item Ist $\mcf$ ein $\mco_X$-Modul, dann haben wir einen quasikohärenten $\mco_X$-Modul
			\[ \mcf(n)\coloneqq \mcf \otimes_{\mco_X}\mco_X(n)  . \]
	\end{enumerate}
\end{defn}

\begin{prop}
\label{prop:11.12}
	Sei $S$ ein graduierter Ring, der von $S_1$ also $S_0$-Algebra erzeugt wird. Sei $X\coloneqq \Proj(S)$. Dann gilt:
	\begin{enumerate}[i)]
		\item $\mco_X(n)$ ist eine invertierbare Garbe auf $X$, das heißt lokal frei vom Rang $1$.
		\item Für graduierte $S$-Moduln $M,N$ gibt es kanonische Isomorphismen
		\begin{align*}
			\widetilde{M\otimes_S N} & \cong \widetilde{M}\otimes_{\mco_X} \widetilde{N}\\
			\widetilde{M}(n) & \cong \widetilde{M(n)}\\
			\mco_X(m)\otimes_{\mco_X} \mco_X(n) &\cong \mco_X(m+n).
		\end{align*}
	\end{enumerate}
\end{prop}

\begin{defn}
\label{defn:11.13}
	Sei $S$ ein graduierter Ring, $X=\Proj(S)$ und $\mcf$ ein $\mco_X$-Modul.
	\begin{enumerate}[i)]
		\item Ein Element $s \in S_d=(S(d))_0$ definiert für jedes $f \in S_+^\text{hom}$ ein Element $\frac{s}{1}$ in $(S(d))_{(f)} = \Gamma(D_+(f),\mco_X(d))$. Diese Elemente verkleben zu einem Element in $\Gamma(X,\mco_X(d))$, das wir (ebenfalls) mit $s$ bezeichnen.
		\item Der \textbf{graduierte $S$-Modul assoziiert zu $\mcf$} ist die graduierte abelsche Gruppe
			\[ \Gamma_*(\mcf)\coloneqq \bigoplus_{n\in \Z}\Gamma(X,\mcf(n)) \]
			mit folgender Struktur: Sind $s \in S_d$ und $t \in \Gamma(X,\mcf(n))$, so ist $st \in \Gamma_*(\mcf)$ das Bild von $s \otimes t$ unter
		\[
		 	\Gamma(X,\mco_X(d))\otimes_S \Gamma(X,\mcf(n)) \to \Bigl(\Gamma(X,\,\mco_X(d)\otimes_{\mco_X}\!\mcf(n)) = \Gamma(X,\mcf(n+d))\Bigr).
		\] 
	\end{enumerate}
\end{defn}

\begin{prop}
\label{prop:11.14}
	Sei $A$ ein Ring, $S \coloneqq A[T_0,\ldots,T_r]$ und $X\coloneqq \Proj(S) = \P^r_A$. Dann gibt es einen kanonischen Isomorphismus $S \simto \Gamma_*(\mco_X)$ graduierter Ringe. Es gilt also $D_d \cong \Gamma(X,\mco_X(d))$ für alle $d \in \Z$. Insbesondere gilt $\Gamma(X,\mco_X(d)) = 0$.
	\begin{proof}
	 	Dies wird in Aufgabe~11.4 gezeigt.
	 \end{proof} 
\end{prop}

\begin{prop}
\label{prop:11.15}
	Sei $S$ ein graduierter Ring, der als $S_0$-Algebra durch endlich viele Elemente aus $S_1$ erzeugt wird. Dann gibt es für jeden quasikohärenten $\mco_X$-Modul $\mcf$ einen kanonischen Isomorphismus $\widetilde{\Gamma_*(\mcf)}\simto \mcf$.
	\begin{proof}
		\cite[Proposition~II.5.15]{hartshorne1977algebraic}.
	\end{proof}
\end{prop}

\begin{defn}
\label{defn:11.16}
	\begin{enumerate}[i)]
		\item Sei $Y$ ein Schema und $n\in \N$. Wir erhalten ein kartesisches Diagramm
		\begin{center}
			\begin{tikzcd}[column sep=large, row sep=large]
				\P^n_Y\arrow{r}{g}[swap]{\text{Basiswechsel}}\arrow{d}[swap]{\pi_2}\arrow[phantom]{dr}{\square}& \P^n_{\Z}\arrow{d}\\
				Y \arrow{r} & \Spec(\Z)\mathclap{,}
			\end{tikzcd}
		\end{center}
		wobei die Morphismen $\colon Y \to \Spec(\Z)$ und $\P^n_\Z\to \Spec(\Z)$ jeweils die kanonischen Morphismen sind. Auf $\P^n_\Z = \Proj(\Z[T_0,\ldots,T_n])$ ist $\mco_{\P^n_\Z}(m)$ für alle $m \in \Z$ definiert. Nun heißt
		\[
			\mco(m) \coloneqq \mco_{\P^n_Y}(m) \coloneqq g^*\left(\mco_{\P^n_\Z}(m)\right)
		\]
		die \textbf{getwistete Garbe von Serre}.
		\item Sei $X$ ein Schema über $Y$. Eine invertierbare Garbe $\mcf$ heißt \textbf{sehr ampel relativ zu $Y$}, wenn es ein $n \in \N$, eine offene Immersion $i \colon X \hookrightarrow \P^n_Y$ und einen Isomorphismus $\mcf \cong i^*\left(\mco_{\P^n_Y}(1)\right)$ gibt. 
	\end{enumerate}
\end{defn}

\begin{bem*}
	Analog wie bei den projektiven Morphismen weicht die Definition von der in \cite[éfinition~4.4.2]{grothendieck1961elements} ab. Es gilt dasselbe wie in Proposition~\ref{prop:11.15}.
\end{bem*}

\begin{thm}[Theorem von Serre]
\label{thm:11.17}
	Sei $X$ ein projektives Schema über $S\coloneqq \Spec(A)$ für einen noetherschen Ring $A$. Weiter sei $\mco(1)$ eine sehr ample invertierbare Garbe relativ zu $S$ auf $X$. Für einen kohärenten $\mco_X$-Modul $\mcf$ definieren wir für alle $n \in \N$
	\[
		\mcf(n) \coloneqq \mcf \otimes_{\mco_X}\mco(n),
	\]
	mit $\mco(n) = \mco(1)^{\otimes n}$ für $n > 0$, $\mco(0)=\mco_X$ und $\mco(n) = \underline{\Hom}(\mco(-n),\mco_X))$ für $n < 0$. Dann gibt es ein $n_0\in \N$ mit der Eigenschaft, dass $\mcf(n)$ für alle $n \ge n_0$ von endlich vielen globalen Schnitten erzeugt wird.
	\begin{proof}
		\cite[Theorem~5.17]{hartshorne1977algebraic}.
	\end{proof}
\end{thm}

\begin{kor}
\label{kor:11.18}
	Sei $X$ ein projektives Schema über einem noetherschen Ring $A$ und sei $\mcf$ eine kohärente Garbe auf $X$. Dann gibt es $n_1,\ldots,n_k \in \Z$ und einen surjektiven Homomorphismus $\bigoplus_{i=1}^k\mco(n_i)\\twoheadrightarrow \mcf$.
	\begin{proof}
		Nach Theorem~\ref{thm:11.17} gibt es ein $n \in \N$ und $s_1,\ldots,s_k\in \Gamma(X,\mcf(n))$, die $\mcf(n)$ erzeugen. Wir erhalten eine Surjektion
		\[
			\bigoplus_{i=1}^k\mco_X \twoheadrightarrow \mcf(n), \; (a_1,\ldots,a_k)\mapsto \sum_{i=1}^ka_is_i.
		\]
		Da $\mco_X(-n)$ eine invertierbare Garbe ist, ist tensorieren mit $\mco_X(-n)$ exakt. Wir erhalten also wie gewünscht folgende Surjektion:
		\begin{center}
			\begin{tikzcd}
				\left(\bigoplus_{i=1}^k\mco_X\right)\otimes_{\mco_X}\mco_X(-n) \arrow[two heads]{r}\arrow[equal]{d} & \mcf(n)\otimes_{\mco_X}\mco_X(-n)\arrow[equal]{d}\\
				\bigoplus_{i=1}^k\mco_X\otimes_{\mco_X}\mco_X(-n) \arrow[equal]{d} & \mcf \otimes_{\mco_X}\mco_X(n) \otimes_{\mco_X}\mco_X(-n) \arrow[equal]{d}\\
				\bigoplus_{i=1}^k\mco_X(-n) & \mcf \otimes_{\mco_X}\mco_X \cong \mcf
			\end{tikzcd}
		\end{center}
	\end{proof}
\end{kor}

\begin{thm}
\label{thm:11.19}
	Sei $K$ ein Körper, $A$ eine endlich erzeugte $K$-Algebra, $X$ ein projektives Schema über $\Spec(A)$ und $\mcf$ ein kohärenter $\mco_X$-Modul. Dann ist $\Gamma(X,\mcf)$ ein endlich erzeugter $A$-Modul.
	\begin{proof}
		\cite[Theorem~II.5.19]{hartshorne1977algebraic}.
	\end{proof}
\end{thm}

\begin{satz}[Endlichkeitssatz]
\label{satz:11.20}
	Seien $X$ und $Y$ Schemata von endlichem Typ über dem Körper $K$. Weiter sei $f \colon X \to Y$ ein projektiver Morphismus und $\mcf$ ein kohärenter $\mco_X$-Modul. Dann ist $f_*(\mcf)$ ein kohärenter $\mco_X$-Modul.
	\begin{proof}
		Nach Proposition~\ref{prop:9.8} ist $f_*(\mcf)$ genau dann kohärent, wenn $f_*(\mcf)\vert_U$ für alle offenen $U = \Spec(A) \subseteq Y$ kohärent ist. Sei also $U = \Spec(A)$ ein offenes affines Unterschema von $Y$. Wir betrachten folgendes Diagramm:
		\begin{center}
			\begin{tikzcd}
				f^{-1}(U)\arrow{r}{g}\arrow[hook]{d} & \Spec(A) = U\arrow[hook]{d}\\
				X \arrow{r}[swap]{f} & Y
			\end{tikzcd}
		\end{center}
		Damit folgt aus den Definitionen $f_*(\mcf)\vert_U = g_*\left(\mcf\vert_{f^{-1}(U)}\right)$, also kann man ohne Beschränking der Allgemeinheit annehmen, dass $Y=U=\Spec(A)$ gilt. Da $X$ und $Y$ von endlichem Typ über $K$ sind, sind sie noethersch. Damit folgt aus Proposition~\ref{prop:9.10}, dass $f_*(\mcf)$ quasikohärent ist und damit nach Proposition~\ref{prop:9.8}
		\[
			f_*(\mcf) = \widetilde{\Gamma(Y,f_*(\mcf))} = \widetilde{\Gamma(X,\mcf)}.
		\]
		Da $Y$ von endlichem Typ über $K$ ist, ist $A$ eine endlich erzeigte $K$-Algebra. Nach Theorem~\ref{thm:11.19} ist $\Gamma(X,\mcf)$ ein endlich erzeugter $A$-Modul und damit ist $f_*(\mcf)$ nach Proposition~\ref{prop:9.8} kohärent.
	\end{proof}
\end{satz}

\begin{bem}
\label{bem:11.21}
	Der Endlichkeitssatz~\ref{satz:11.20} gilt allgemeiner für eigentliche Morphismen zwischen noetherschen Schemata.
	\begin{proof}
		\cite[Théorème~3.2.1]{grothendieck1961elementsiii}.
	\end{proof}
\end{bem}

\begin{thm}
\label{thm:11.22}
	Sei $k$ ein algebraisch abgeschlosseren Körper. Dann gibt es einen kanonischen volltreuen Funktor $t$ von der Kategorie der irreduziblen quasi-projektiven (beziehungsweise projektiven) Varietäten über $k$ im Sinne der Algebraischen Geometrie~I in die Kategorie der Schemata über $k$, dessen Bild die Klasse der integren quasi-projektiven (beziehungsweise reduzierten projektiven) Schemata über $k$ ist.
	\begin{proof}
		Nach Aufgabe~11.3 gibt es einen volltreuen Funktor
		\[
			t\colon \{\text{affine Varietäten über }k\} \to \Sh,\; V \mapsto \Spec(k[V]),
		\]
		dessen Bild die Kategorie der affinen reduzierten Schemata von endlichem Typ über $k$ ist. Dabei entsprechen die irreduziblen affinen Varietäten den integren affinen Schemata von endlichem Typ über $k$. Weiter haben wir gesehen, dass die offenen, beziehungsweise abgeschlossenen Teilmengen einer Varietät $V$ den offenen, beziehungsweise abgeschlossenen Teilmengen von $t(V)$ entsprechen. Weiter ist $t$ verträglich mit Vereinigungen und Schnitten offener Mengen. Damit erhält man durch Verkleben den gewünschten Funktor $t$.

		Sei $V$ eine irreduzible quasi-projektive (beziehungsweise projektive) Varietät über $k$. Dann ist $V$ eine offene Teilmenge in einer abgeschlossenen Menge $\overline{V}$ in $\P^n_k$. Wegen $t(\P^n_k) = \P^n_k$ und da die Entsprechung offener, beziehungsweise abgeschlossener Teilmengen unter $t$ nach Verklebung weiterhing gilt, ist $t(V)$ offen in der abgeschlossenen Menge $\overline{t(V)} \subseteq \P^n_k$. also ist $t(V)$ als Schema quasi-projektiv (beziehungsweise projektiv). Da der Funktor $t$ auf affinen Varietäten volltreu ist, ist $t$ volltreu (nutze affine offene Übereckungen).

		Sei umgekehrt $X$ ein projektives reduziertes Schema über $k$. Dann ist $X$ ein abgeschlossenes Unterschema von $\P^n_k$. Betrachte die Menge $V$ der abgeschlossenen Punkte in $X$. Nach \cite[Exercise~II.3.14]{hartshorne1977algebraic} ist $V$ dicht in $X$ und nach \cite[Corrolary~II.5.16]{hartshorne1977algebraic} ist $X$ die Nullstellenmenge eines homogenen Ideals $\mfa \subseteq k[X_0,\ldots,X_n]$. Damit folgt, dass $V$ die Nullstellenmenge von $\mfa$ in $\P^n_k$ (im Sinne der Varietäten) und damit eine projektive Varietät ist. Dasselbe Argument angewandt auf abgeschlossene Teilmengen von $X$ zeigt, dass $t(V)$ und $X$ als topologische Räume gleich sind. Da weiter $t(V)$ und $X$ beide reduzierte abgeschlossene Unterschemata von $\P^n_k$ sind, folgt nach Aufgabe~8.3 $t(V) = X$.

		Der quasi-projektive Fall folgt durch Übergang zu offenen Teilmengen.
	\end{proof}
\end{thm}

\begin{defn}
\label{defn:11.23}
	Eine Varietät über einem beliebigen Körper $K$ ist ein separiertes reduziertes Schema von endlichem Typ über $K$.
\end{defn}

\begin{bem}
\label{bem:11.24}
	Theorem~\ref{thm:11.22} zeigt, dass die alten quasi-projektiven Varietäten auch Varietäten im Sinn von Definition~\ref{defn:11.23} sind. Es gibt aber auch Varietäten, die nicht quasi-projektiv sind.
\end{bem}

%%% Local Variables: 
%%% mode: latex
%%% TeX-master: "algebraische_geometrie_2"
%%% End: 
