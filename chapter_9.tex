%!TEX root = algebraische_geometrie_2.tex
% vim: tw=0 noet sts=8 sw=8

\chapter{Quasikohärente Modulgarben}
\label{chap:9}

Wenn man in der Konstruktion der Strukturgarbe $\mco_X$ auf $X = \Spec(A)$ den Ring $A$ durch einen $A$-Modul $M$ ersetzt, erhält man eine $\mco_X$-Modulgarbe. Auf einem beliebigen Schema $X$ heißt eine $\mco_X$-Modulgarbe quasikohärent, wenn wenn sie lokal von obiger Gestalt ist. Die quasikohärenten Garben erfüllen wichtige Eigenschaften in der algebraischen Geometrie, wie wir später sehen werden. Die kohärenten Garben spielen auf noetherschen Schemata die Rolle der endlich erzeugten Moduln in der Algebra.

\nextmark{Zu einem A-Modul assoziierte Garbe auf Spec(A)}
\begin{kons}
\label{kons:9.1}
	Sei $A$ ein Ring und $M$ ein $A$-Modul. Wir definieren eine Garbe $\widetilde{M}$ auf $\Spec(A)$ auf folgende Weise. Für $\mfp \in \Spec(A)$ haben wir die Lokalisierung
	\[
		M_{\mfp} = M \otimes_A A_\mfp.
	\]
	Sei also $U$ eine offene Teilmenge von $\Spec(A)$. Wir definieren $\widetilde{M}(U)$ als die Menge der Funktionen $s \colon U \to \coprod_{\mfp \in U} M_{\mfp}$, die folgende Eigenschaften erfüllen:
	\begin{enumerate}[a)]
		\item Für alle $\mfp \in U$ gilt $s(\mfp) \in M_{\mfp}$
		\item Für alle $\mfp \in U$ gibt es eine offene Umgebung $V$ von $\mfp$ in $U$ und $m\in M,\;f\in A$ mit $V \subseteq D(f)$ und $s(\mfq) = \frac{m}{f} \in M_\mfq$ für alle $\mfq \in V$.
	\end{enumerate}
	Offensichtlich bildet $\widetilde{M}$ mit \enquote{Einschränken von Funktionen} als Restriktionsabbildungen eine Garbe.
\end{kons}

\nextmark{Eig. der assoziierten OX-Moduln, I}
\begin{prop}
\label{prop:9.2}
	Für die Garbe $\widetilde{M}$ auf $X=\Spec(A)$ gelten folgende Eigenschaften:
	\begin{enumerate}[i)]
		\item\label{prop:9.2:i} $\widetilde{M}$ ist kanonisch ein $\mco_X$-Modul.
		\item\label{prop:9.2:ii} Für alle $\mfp \in X$ gilt $\widetilde{M}_{\mfp} = M_{\mfp}$.
		\item\label{prop:9.2:iii} Für alle $f \in A$ ist der $(A_f = \mco_X(D(f)))$-Modul $\widetilde{M}(D(f))$ kanonisch Isomorph zur Lokalisierung $M_f = M \otimes_A A_f$.
		\item\label{prop:9.2:iv} Es gilt $\Gamma(X,\widetilde{M}) = M$.
		\item\label{prop:9.2:v} Für alle $\mco_X$-Moduln $\mcf$ und alle $A$-Moduln $M$ existieren kanonische Isomorphismen:
		\begin{center}
			\begin{tikzcd}
				\Hom_{\mco_X}(\widetilde{M},\mcf) \arrow[shift left=1.1ex]{r}{\overset{\alpha}{\sim}} & \Hom_A(M,\Gamma(X,\mcf)) \arrow[shift left=1.1ex]{l}{\underset{\beta}{\sim}}
			\end{tikzcd}
		\end{center}
	\end{enumerate}
	\begin{proof}
		\ref{prop:9.2:i} folgt aus der Konstruktion von $\widetilde{M}$ in Konstruktion~\ref{kons:9.1} (beziehungsweise $\mco_X$ in Konstruktion~\ref{kons:4.11}). \ref{prop:9.2:ii}--\ref{prop:9.2:iv} werden analog zu den entsprechenden Aussagen in Proposition~\ref{prop:4.12} für $\mco_X$ bewiesen.

		Wir beweisen v): Um $\alpha$ zu definieren, nehmen wir einfach die globalen Schnitte des Homomorphismus $\widetilde{M}\to \mcf$ und erhalten einen $A$-Modulhomomorphismus $\Gamma(X,\widetilde{M}) \to \Gamma(X,\mcf)$. Nach \ref{prop:9.2:iv} erhalten wir $M = \Gamma(X,\widetilde{M})$ und damit einen $A$-Modulhomomorphismus $M \to \Gamma(X,\mcf)$ wie gewünscht. Dies definiert $\alpha$. Sei nun ein $A$-Modulhomomorphismus $h\colon M \to \Gamma(X,\mcf)$ gegeben. Wir definieren einen kanonischen $\mco_X$-Modulhomomorphismus $\beta(h)\colon \widetilde{M}\to \mcf$ wie folgt: Für $f \in A$ gilt $\mco_X(D(f)) = A_f$ und $\Gamma(D(f),\mcf)$ ist damit ein $A_f$-Modul. Nach \ref{prop:9.2:iii} gilt
		\[
			\widetilde{M}(D(f)) = M_f = M \otimes_A A_f.
		\]
		Da die Mengen $D(f)$ eine Basis der Topologie von $X=\Spec(A)$ bilden, folgt aus den Garbenaxiomen, dass ein Morphismus $\widetilde{M}\to \mcf$ von $\mco_X$-Moduln durch ein auf Durchschnitten $D(f)\cap D(g) = D(fg)$ kompatibeles System von $A_f$-linearen Abbildungen $\widetilde{M}(D(f)) \to \mcf(D(f))$ eindeutig gegeben ist.
		Damit das Diagramm
		\begin{center}
			\begin{tikzcd}[column sep=1.3cm]
				(M=\widetilde{M}(X)) \arrow{r}{h = \alpha(\beta(h))} \arrow{d} & \Gamma(X,\mcf)\arrow{d}\\
				\widetilde{M}(D(f)) \arrow{r}{\beta(h)_{D(f)}} & \Gamma(D(f),\mcf)
			\end{tikzcd}
		\end{center}
		kommutiert, müssen wir die Abbildung
		\begin{alignat*}{2}
			(M \otimes_A A_f = \widetilde{M}(D(f))) &\to \Gamma(X,\mcf)\otimes_A A_f &&\to \Gamma(D(f),\mcf)\\
			m \otimes a &\mapsto  h(m)\otimes a \\
			& \phantom{\mathbin\mapsto h(m)} \mathllap{s} \otimes a &&\mapsto a \cdot s\vert_{D(f)}
		\end{alignat*}
		wählen. Diese liefern ein System kompatibler Abbildungen und definieren damit ein kanonisches $\beta(h)\in \Hom_{\mco_X}(\widetilde{M},\mcf)$. Wählt man speziell $f=1$ mit $D(1) = X$, $M_1=M$ und $A_1 = A$, dann folgt $\alpha \circ \beta = \id$. Da sich
		\[
			(\Gamma(X,\widetilde{M}) = M) \to \Gamma(X,\mcf)
		\]
		wie oben gesehen zu einem Homomorphismus $\widetilde{M} \to \mcf$ fortsetzt, folgt auch $\beta \circ \alpha = \id$.
	\end{proof}
\end{prop}

\nextmark{Eig. der assoziierten OX-Moduln, II}
\begin{prop}
\label{prop:9.3}
Sei $\varphi\colon A \to B$ ein Ringhomomorphismus und $(f\coloneqq\Spec(\varphi)) \colon \Spec(B) \to (X \coloneqq \Spec(A))$ der zugehörige Morphismus von Schemata. Dann gilt:
	\begin{enumerate}[i)]
		\item Für jede exakte Sequenz
		\[
			0 \to M' \to M \to M'' \to 0
		\]
		von $A$-Moduln ist auch
		\[
			0 \to \widetilde{M'} \to \widetilde{M} \to \widetilde{M''} \to 0
		\]
		eine exakte Sequenz von $\mco_X$-Moduln und für alle $A$-Moduln $M,N$ existiert eine kanonische Bijektion
		\[
			\Hom_A(M,N) \simto \Hom_{\mco_X}(\widetilde{M},\widetilde{N}).
		\]
		\item Für alle $A$-Moduln $M,N$ gilt $\widetilde{(M\otimes_A N)} = \widetilde{M} \otimes_{\mco_X} \widetilde{N}$.
		\item Für eine Familie $(M_i)_{i\in I}$ von $A$-Moduln gilt $\widetilde{(\bigoplus_{i\in I}M_i)} = \bigoplus_{i\in I} \widetilde{M_i}$
		\item Sei $N$ ein $B$-Modul. Da $B$ eine $A$-Algebra ist, kann man $N$ als $A$-Modul betrachten. Wir bezeichnen diese $A$-Modulstruktur auf $N$ mit $_{A}N$. Es gilt $f_*(\widetilde{N}) = \widetilde{(_{A}N)}$.
		\item Für einen $A$-Modul $M$ gilt
		\[
			f^*(\widetilde{M}) = \widetilde{(M \otimes_A B)}.
		\]
		\item Für einen $A$-Modul~$M$ und $\mfp \in \Spec(A)$, $\mfq \in \Spec(B)$ mit $f(\mfq) = \mfp$ gilt
		\[
			f^*(\widetilde{M})_\mfq = \widetilde{(M_\mfp \otimes_{A_{\mfp}} B_\mfq)}
		\]
	\end{enumerate}
	\begin{proof}
		Dies wird in den Übungen gezeigt.
	\end{proof}
\end{prop}

\begin{bem}
\label{bem:9.4}
	Sei $(X,\mco_X)$ ein geringter Raum.
	\begin{enumerate}[i)]
		\item\label{bem:9.4:i} Wir betrachten einen $\mco_X$-Modul $\mcf$. Es gibt einen kanonischen Isomorphismus
		\[
			\Hom_{\mco_X}(\mco_X,\mcf) \simto \Gamma(X,\mcf),\; h \mapsto h_X(1)
		\]
		von $\Gamma(X,\mco_X)$-Moduln. Die Umkerhabbildung ist folgendermaßen gegeben: Der globale Schnitt $s \in \Gamma(X,\mcf)$ wird auf den Homomorphismus $h_s\colon \mco_X \to \mcf$, der auf einer offenen Menge $U$ durch $h_s(U)\colon \mco_X(U) \to \mcf(U),\; f \mapsto f \circ s\vert_U$ definiert ist, geschickt.
		\item Für einen $A$-Modul $M$ definieren wir $M^I \coloneqq \prod_{i\in I}M$ und den Teilmodul $M^{(I)} \coloneqq \bigoplus_{i\in I} M$. Wir verwenden die analoge Notation für $\mco_X$-Moduln. Für eine beliebige Menge $I$ gilt aufgrund der universellen Eigenschaft der direkten Summe:
		\begin{align*}
			\Hom_{\mco_X}(\mco_X^{(I)},\mcf) \qquad &\overset{\mathclap{\text{Definition}}}{=} \qquad \Hom_{\mco_X}\Bigl(\bigoplus_{i\in I}\mco_X,\mcf\Bigr)\\
			&= \qquad \prod_{i\in I} \Hom_{\mco_X}(\mco_X,\mcf)\\
			&\overset{\mathclap{\text{Definition}}}{=} \qquad \Hom_{\mco_X}(\mco_X,\mcf)^I\\
			&\overset{\text{\ref{bem:9.4:i}}}{=} \qquad \Gamma(X,\mcf)^I
		\end{align*}	 
	\end{enumerate}
\end{bem}

\nextmark{(quasi-)kohärenter OX-Modul, Erzeugendensystem eines OX-Moduls}
\begin{defn}
\label{defn:9.5}
	\begin{enumerate}[i)]
		\item Seien $(X,\mco_X)$ ein geringter Raum und $\mcf$ ein $\mco_X$-Modul wie oben. Weiter sei $(s_i)_{i\in I}\in \Gamma(X,\mcf)^I$ eine Familie von globalen Schnitten von $\mcf$. Wir sagen, dass \textbf{$\mcf$ von $(s_i)_{i\in I}$ erzeugt wird}, falls der zugehörige Morphismus $\mco_X^{(I)}\to \mcf$ aus Bemerkung~\ref{bem:9.4} ii) surjektiv ist.
		\item $\mcf$ heißt \textbf{quasikohärenter} $\mco_X$-Modul, falls jeder Punkt aus $X$ eine offene Umgebung $U$ hat mit der Eigenschaft, dass $\mcf\vert_U$ isomorph zum Kokern eines $\mco_X\vert_U$-Modulhomomorphismus von freien $\mco_X\vert_U$-Moduln ist, das heißt, dass es eine exakte Sequenz
		\begin{equation*}
		\label{eq:9.5.1}\tag{$\star$}
			\mco_X\vert_U^{(I)} \to \mco_X\vert_U^{(J)} \to \mcf\vert_U \to 0
		\end{equation*}
		von $\mco_X\vert_U$-Moduln gibt.
		\item Sei jetzt $(X,\mco_X)$ ein Schema. Dann heißt ein quasikohärenter $\mco_X$-Modul \textbf{kohärent}, wenn man in \eqref{eq:9.5.1} die Indexmengen $I$ und $J$ endlich wählen kann.
	\end{enumerate}
\end{defn}

\nextmark{Äquivalente Formulierung (quasi-)kohärenter OX-Moduln}
\begin{prop}
\label{prop:9.6}
	Sei $(X,\mco_X)$ ein Schema und $\mcf$ ein $\mco_X$-Modul.
	\begin{enumerate}[i)]
		\item\label{prop:9.6:i} $\mcf$ ist genau dann quasikohärent, wenn es für alle $p\in X$ eine offene affine Umgebung $U=\Spec(A)$ und einen $A$-Modul $M$ mit $\mcf\vert_U \cong \widetilde{M}$ gibt.
		\item\label{prop:9.6:ii} Wenn $X$ ein noethersches Schema ist, ist $\mcf$ genau dann hohärent, wenn es für alle $p\in X$ eine offene affine Umgebung $U=\Spec(A)$ und einen endlich erzeugten $A$-Modul~$M$ mit $\mcf\vert_U \cong \widetilde{M}$ gibt.
	\end{enumerate}
	\begin{proof}
		\enquote{$\Longrightarrow$} bei \ref{prop:9.6:i} (beziehungsweise \ref{prop:9.6:ii}). Sei also $\mcf$ quasikohärent (beziehungsweise kohärent) und $p\in X$. Wir wählen eine offene Umgebung $U$ von $p$, für die \eqref{eq:9.5.1} gilt. Durch Verkleinern können wir ohne Beschränkung der Allgemeinheit annehmen, dass $U$ offen und affin ist, also $U= \Spec(A)$ für einen gewissen Ring $A$. Dann gilt
		\[
			\mco_X\vert_U = \mco_U = \mco_{\Spec(A)} = \widetilde{A}
		\]
		und damit folgt
		\begin{equation*}
		\label{eq:9.6.1}\tag{$\star\star$}
			(\mco_X\vert_U)^{(I)} = \mco_U^{(I)} \overset{\text{Proposition~\ref{prop:9.3}}}{=} \widetilde{(A^{(I)})}.
		\end{equation*}
		Sei $M$ der Kokern des Homomorphismus $A^{(I)}\to A^{(J)}$, der durch Betrachtung der globalen Schnitte von $(\mco_X\vert_U)^{(I)}\to (\mco_X\vert_U)^{(J)}$ abgeleitet wird. Wir erhalten wegen der Definition von Kokern und der Exaktheit des Funktors $\widetilde{\;\cdot\;}$ ein kommutatives Diagramm mit exaten Zeilen:
		\begin{center}
			\begin{tikzcd}
				(\mco_X\vert_U)^{(I)}\arrow{r}\arrow{d}{\cong} & (\mco_X\vert_U)^{(J)}\arrow{r}\arrow{d}{\cong} & \mcf\vert_U \arrow{r}\arrow{d} & 0\\
				\widetilde{A^{(I)}} \arrow{r} & \widetilde{A^{(J)}} \arrow{r} & \widetilde{M} \arrow{r} & 0
			\end{tikzcd}
		\end{center}
		Also wird (Fünferlemma!) ein Isomorphismus $\mcf\vert_U \simto \widetilde{M}$ von $(\mco_X\vert_U = \mco_U)$-Moduln induziert.

		\enquote{$\Longleftarrow$}: Offenbar ist jeder $A$-Modul $M$ der Kokern von einem Homomorphismus freier Moduln, das heißt
		\begin{equation*}
		\label{eq:9.6.2}\tag{$\star{\star}\star$}
			A^{(I)} \to A^{(J)} \to M \to 0,
		\end{equation*}
		wähle zum Beispiel $J=M$ und $I = \ker(A^{(J)}\to M)$. Falls $A$ noethersch und $M$ endlich erzeugt ist, dann kann man $I$ und $J$ endlich wählen (wähle $J=$~endliches Erzeugendensystem von $M$ und $I=$~endliches Erzeugendensystem von $\ker(A^{(J)}\to M)$). Sei nun $U=\Spec(A)$ und $\widetilde{M}\cong \mcf\vert_U$. Dann ergibt \eqref{eq:9.6.2} und Proposition~\ref{prop:9.3} eine exakte Folge
		\[
			\underbrace{\widetilde{A^{(I)}}}_{\cong \mco_X\vert_U^{(I)}} \to \underbrace{\widetilde{A^{(J)}}}_{\cong \mco_X\vert_U^{(J)}}\to \underbrace{\widetilde{M}}_{\cong \mcf\vert_U} \to 0
		\]
		wie in \eqref{eq:9.5.1}.
	\end{proof}
\end{prop}

\begin{bsp}
\label{bsp:9.7}
	\begin{enumerate}[i)]
		\item Für jedes Schema $X$ ist $\mco_X$ ein kohärenter $\mco_X$-Modul.
		\item\label{bsp:9.7:ii} Seien $A$ ein Ring und $X=\Spec(A)$, $\mfa$ ein Ideal von $A$ und $i\colon (Y=\Spec(A/\mfa))\to X$ das zugehörige abgeschlossene Unterschema. Dann gibt es einen kanonischen Isomorphismus
		\begin{equation*}
		\label{eq:9.7.1}\tag{$\triangle$}
			\widetilde{A/\mfa} \simto i_*\mco_Y.
		\end{equation*}
		Ist $A$ noethersch, so folgt insbesondere, dass $i_*\mco_Y$ ein kohärenter $\mco_X$-Modul ist.
	\end{enumerate}
	\begin{proof}[Beweis von \ref{bsp:9.7:ii}]
		Es genügt einen Isomorphismus \eqref{eq:9.7.1} zu konstruieren. Es gilt
		\[
			\Gamma(X,i_*\mco_Y)=\mco_Y(Y) =A/\mfa
		\]
		und somit induziert der Isomorphismus $A/\mfa \to \Gamma(X,i_*\mco_Y)$ nach Proposition~\ref{prop:9.2} einen Homomorphismus
		\[
			\widetilde{A/\mfa} \to i_*\mco_Y.
		\]
		Für $f \in A$ mit Bild $\widetilde{f}\in A/\mfa$ gilt
		\[
			(i_*\mco_Y)(D(f)) = \mco_Y(i^{-1}D(f)) = \mco_Y(D(\widetilde{f})) = (A/\mfa)_{\widetilde{f}} = (A/\mfa)_f = \widetilde{A/\mfa}(D(f)).
		\]
		Da die $D(f)$ eine Basis bilden folgt, dass
		\[
			\widetilde{A/\mfa} \simto i_*\mco_Y
		\]
		ein Isomorphismus ist.
	\end{proof}
\end{bsp}

\pagebreak[2]
\begin{prop}
\label{prop:9.8}
	Seien $X$ ein Schema und $\mcf$ ein $\mco_X$-Modul
	\begin{enumerate}[i)]
		\item $\mcf$ ist genau dann quasikohärent, wenn es für alle offenen affinen Unterschemata von $X$ einen $A$-Modul $M$ mit $\mcf\vert_U \cong \widetilde{M}$ gibt.
		\item Falls $X$ noethersch ist, dann ist $\mcf$ genau dann kohärent, wenn es für alle offenen affinen Unterschemata von $X$ einen endlich erzeugten $A$-Modul $M$ mit $\mcf\vert_U \cong \widetilde{M}$ gibt.
	\end{enumerate}
	\begin{proof}
		\cite[Proposition~II.5.9]{hartshorne1977algebraic}.
	\end{proof}
\end{prop}

\nextmark{Exaktheit des Globale-Schnitte-Funktors im Fall quasikohärenter OX-Moduln}
\begin{prop}
\label{prop:9.9}
	Sei $X=\Spec(A)$ und
	\[
		0 \to \mcf' \to \mcf \to \mcf'' \to 0
	\]
	eine exakte Folge von $\mco_X$-Moduln. Falls $\mcf'$ quasikohärent ist, so ist die Sequenz
	\[
		0 \to \Gamma(X,\mcf') \to \Gamma(X,\mcf) \to \Gamma(X,\mcf'') \to 0	
	\]
	exakt.
	\begin{proof}
		\cite[Proposition~II.5.6]{hartshorne1977algebraic}.
	\end{proof}
\end{prop}

\nextmark{Vererbungseigenschaften quasikohärenter OX-Moduln}
\begin{prop}
\label{prop:9.10}
	Sei $f\colon X \to Y$ ein Morphismus von Schemata und sei $\mcf$ beziehungsweise~$\mcg$ ein quasikohärenter $\mco_X$- beziehungsweise $\mco_Y$-Modul.
	\begin{enumerate}[i)]
		\item\label{prop:9.10:i}   $f^*\mcg$ ist quasikohärent.
		\item\label{prop:9.10:ii}  Falls $X$ und $Y$ noethersch sind und falls $\mcg$ kohärent ist, so ist auch $f^*\mcg$ kohärent.
		\item\label{prop:9.10:iii} Falls $X$ noethersch ist, so ist auch $f_*\mcf$ quasikohärent.
	\end{enumerate}
	\begin{proof} \ref{prop:9.10:i} und \ref{prop:9.10:ii}: Die Behauptung ist lokal auf $X$ und $Y$, also sei ohne Beschränkung der Allgemeinheit $X=\Spec(B)$ und $Y=\Spec(A)$. Dann folgt die Behauptung aus Proposition~\ref{prop:9.3}\,v).
		
		Es bleib \ref{prop:9.10:iii} zu zeigen: Sei $X$ noethersch. Es ist zu zeigen, dass $f_*\mcf$ quasikohärent ist. Die Behauptung ist lokal auf $Y$, sei also ohne Beschränkung der Allgemeinheit $Y=\Spec(A)$. Da $X$ ein noethersches Schema ist, ist $X$ ein noetherscher topologischer Raum. Jede offene Teilmenge eines noetherschen topologischen Raumes ist quasikompakt. Also besitzt $X$ eine endliche offene Überdeckung $(U_i)_{i\in I}$ durch offene affine Unterschemata $U_i$. Weiter besitzt $U_{ij}\coloneqq U_i \cap U_j$ eine endliche offene Überdeckung $(U_{ijk})_{k\in I_{ij}}$ durch offene affine Unterschemata $U_{ijk}$. Wir bezeichnen die kanonischen Morphismen
			\[
			 	X \to Y,\; U_i\to Y,\;U_{ijk}\to Y
			\]
			alle mit $f$. Dann besagen die Garbenaxiome, dass die Folge
			\begin{equation*}
			\label{eq:9.10.1}\tag{$\bigcirc$}
				\begin{alignedat}{3}
					0 \to f_*\mcf &\to \quad\prod_{i\in I}f_*(\mcf\vert_{U_i}) &&\to \prod_{i,j\in I}\prod_{k \in I_{ij}}f_*(\mcf\vert_{U_{ijk}})\\
					s &\mapsto (s\vert_{U_i})_{i\in I},\; (s_i)_{i\in I} &&\mapsto (s_i\vert_{U_{ijk}}-s_j\vert_{U_{ijk}})_{i,j\in I,\, k \in I_{ij}}
				\end{alignedat}
			\end{equation*}
			exakt ist. Nach Proposition~\ref{prop:9.3} iv) sind $f_*(\mcf\vert_{U_i})$ und $f_*(\mcf\vert_{U_{ijk}})$ quasikohärente $\mco_Y$-Moduln. Aus Proposition~\ref{prop:9.3} iii) folgt, dass auch die obigen Produktgarben quasikohärent sind (denn über endlichen Indexmengen stimmen Produkte und direkte Summen überein). In Übung~9.4 sehen wir, dass der Kern von einem Homomorphismus von quasikohärenten $\mco_Y$-Moduln wieder quasikohärent ist. Also folgt aus \eqref{eq:9.10.1}, dass $f_*(\mcf)$ quasikohärent ist.
	\end{proof}
\end{prop}

\begin{bem}
\label{bem:9.11}
	Für einen Morphismus $f\colon X \to Y$ noetherscher Schemata muss die direkte Bildgarbe eines kohärenten $\mco_X$-Moduls nicht kohärent sein.

	\textbf{Gegenbeispiel:} Sei $X=\A^1_K$ und $Y =\Spec(K)$ für einen Körper $K$. Weiter sei $\mcf = \mco_X$. Dann gilt
	\[
		f_*(\mcf)(Y) = \mcf(X) = K[x]
	\]
	bezüglich des kanonischen Morphismus $f\colon X \to Y$. Aber $K[x]$ ist nicht endlich erzeugt als Modul über $K=\mco_Y(Y)$ und damit ist $f_*(\mcf)$ nicht kohärent.
\end{bem}

\nextmark{Träger}
\begin{defn}
\label{defn:9.12}
	Sei $\mcf$ eine beliebige Garbe auf einem topologischen Raum $X$. \begin{enumerate}[i)]
		\item\label{defn:9.12:i} Sei $U$ offen in $X$ und $s \in \mcf(U)$. $\Supp(s)\coloneqq \{p\in U \mid s_p\ne 0 \text{ im Halm } \mcf_p\}$ heißt \textbf{Träger} des Schnitts $s$.
		\item\label{defn:9.12:ii} $\Supp(\mcf)\coloneqq \{ p \in X \mid \mcf_p \neq 0\}$ heißt \textbf{Träger} von $\mcf$.	
	\end{enumerate}
\end{defn}

\begin{lem}
\label{lem:9.13}
	\begin{enumerate}[i)]
		\item\label{lem:9.13:i} Sei $\mcf$ eine Garbe auf $X$ und $U$ offen in $X$. Ist $s \in \mcf(U)$, so ist $\Supp(s)$ abgeschlossen in $U$.
		\item\label{lem:9.13:ii} Sei $X=\Spec(A)$ für einen Ring $A$ und sei $\mcf=\widetilde{M}$ für einen $A$-Modul $M$. Für $m \in (M = \Gamma(X,\mcf))$ gilt $\Supp(m) = V(\Ann_A(m))$ für das Annihilatorideal
		\[
			\Ann_A(m) \coloneqq \{a \in A \mid a \cdot m = 0\}.
		\]
		\item\label{lem:9.13:iii} Sei $\mcf$ wie in \ref{lem:9.13:ii} für einen endlich erzeugten $A$-Modul~$M$. Dann gilt $\Supp(\mcf)=V(\Ann_A(M))$ für
		\[
			\Ann_A(M) \coloneqq \{a \in A \mid a \cdot m = 0 \text{ für alle } m \in M\}.
		\]
		\item\label{lem:9.13:iv} Für ein noethersches Schema $X$ und einen kohärenten $\mco_X$-Modul $\mcf$ ist $\Supp(\mcf)$ abgeschlossen in $X$.
	\end{enumerate}
	\begin{proof}
		Dies wird in Übung~10.1 bewiesen. 
	\end{proof}
\end{lem}

\begin{defn}
\label{defn:9.14}
	Sei $Y$ ein abgeschlossenes Unterschema des Schemas $X$ gegeben durch die abgeschlossene Immersion $i \colon Y \to X$, das heißt $i$ ist eine Homöomorphismus auf die abgeschlossene Teilmenge $i(Y)$ und $i^{\#}\colon \mco_X \to i_*\mco_Y$ ist surjektiv. Sei $J_Y\coloneqq \ker(i^{\#})$ die Idealgarbe zu $Y$. Es gilt $\mco_X/J_Y \cong i_*\mco_Y$.
\end{defn}

\begin{prop}
\label{prop:9.15}
	Sei $X$ ein Schema.
	\begin{enumerate}[i)]
		\item\label{prop:9.15:i} Für ein abgeschlossenes Unterschema $i \colon Y \to X$ ist $J_Y$ ein quasikohärenter $\mco_X$-Modul.
		\item\label{prop:9.15:ii} Ist $X$ ein noethersches Schema, so ist $J_Y$ kohärent.
		\item\label{prop:9.15:iii} Die Abbildung
		\begin{align*}
			\{\text{abgeschlossene Unterschemata in } X\} & \to \{\text{quasikohärente }\mco_X\text{-Idealgarben}\}\\
			Y &\mapsto J_Y
		\end{align*}
		ist bijektiv.
 	\end{enumerate}
 	\begin{proof}
 		\begin{enumerate}[i)]
 			\item Die Behauptung ist lokal auf $X$, sei also ohne Beschränkung der Allgemeinheit $X=\Spec(A)$ für einen Ring $X$. Da $Y$ ein abgeschlossenes Unterschema von $\Spec(A)$ ist, ist $Y$ nach Theorem~\ref{thm:7.5} durch
 			\[
 				i\colon \Spec(A/\mfa) \to \Spec(A)
 			\]
 			für ein Ideal $\mfa$ in $A$ gegeben. Trivialerweise ist $\mfa = \ker(A \to A/\mfa)$ und mit Proposition~\ref{prop:9.3}\ref{lem:9.13:i} folgt
 			\[
 				\widetilde{\mfa} = \ker(\widetilde{A} \to \widetilde{A/\mfa}) \overset{\text{Beispiel~\ref{bsp:9.7}~\ref{bsp:9.7:ii}}}{=} \ker(\mco_X \to i_* \mco_Y) = J_Y.
 			\]
 			Also ist $J_Y$ quasikohärent wie gewünscht.
 			\item Die Behauptung ist wieder lokal, sei also ohne Beschränkung der Allgemeinheit $X = \Spec(A)$ für einen Ring $A$. Wie oben gilt $J_Y \cong \widetilde{\mfa}$. Da $A$ noethersch ist, folgt, dass $\mfa$ endlich erzeugt und damit $J_Y$ kohärent ist.
 			\item Wir definieren die Umkehrabbildung: Sei $J$ eine quasikohärente $\mco_X$-Idealgarbe auf $X$. Dann ist $\mco_X/J$ quasikohärent. Wir betrachten den globalen Schnitt $1 \in \Gamma(X,\mco_X/J)$. Er erzeugt die Garbe $\mco_X/J$ in jedem Halm, das heißt es gilt $1_p \ne 0 \in (\mco_X/J)_p$ genau dann, wenn $(\mco_X/J)_p \ne 0$. Nach Lemma~\ref{lem:9.13}\ref{lem:9.13:i} ist
 			\[
 				Y \coloneqq \Supp(\mco_X/J) = \Supp(1)
 			\]
 			abgeschlossen in $X$. Wir versehen $Y$ mit der von $X$ induzierten Topologie und betrachten die Inklusion $i\colon Y \hookrightarrow X$. Wir betrachten den Morphismus
 			\[
 			\label{eq:9.15.1}\tag{$\star$}
 				(Y,i^{-1}(\mco_X/J)) \to (X,\mco_X)
 			\]
 			von lokal geringten Räumen. Wir müssen zeigen, dass $(Y,i^{-1}(\mco_X/J))$ ein Schema ist und dass \eqref{eq:9.15.1} eine abgeschlossene Immersion mit Idealgarbe $J$ ist. Da wir beide Behauptungen lokal auf $X$ prüfen können sei ohne Beschränung der Allgemeinheit $X = \Spec(A)$ für einen Ring $A$. Aus Proposition~\ref{prop:9.8} erhalten wir $J = \widetilde{\mfa}$ für ein Ideal $\mfa$ in $A$, da $J$ quasikohärent ist. Mit Proposition~\ref{prop:9.3} folgt $\mco_X/J = \widetilde{A/\mfa}$ und damit folgt
 			\[
 				\Supp(\widetilde{A/\mfa}) = \{\mfp \in \Spec(A) \mid (A/\mfa)_\mfp \neq 0\} = \{\mfp \in \Spec(A) \mid \mfp \supseteq \mfa\} = V(\mfa).
 			\]
 			Weiter folgt
 			\[
 				(Y,i^{-1}(\mco_X/J)) = (V(\mfa),\widetilde{A/\mfa}) = \Spec(A/\mfa)
 			\]
 			da die beiden Garben die gleichen Halem haben:
 			\[
 				(i^{-1}(\mco_X/J))_\mfp = (\mco_X/J)_\mfp = (\widetilde{A/\mfa)})_\mfp\quad \forall\, \mfp \in V(\mfa)
 			\]
 			Damit ist $(Y,i^{-1}(\mco_X/J))$ ein Schema und \eqref{eq:9.15.1} eine abgeschlossene Immersion mit Idealgarbe $J$. Wir müssen noch zeigen, dass die Abbildungen invers zueinander sind. Das kann man lokal auf $U=\Spec(A)$ für einen Ring $A$ prüfen. Dann folgt es aus Theorem~\ref{thm:7.5}.
 		\end{enumerate}
 	\end{proof}
\end{prop}

\begin{defn}
\label{defn:9.16}
	Sei $S$ ein graduierter Ring und $M$ ein graduierter $S$-Modul. Dann definieren wir analog zum affinen Fall eine Garbe $\widetilde{M}$ auf $\Proj(S)$. Für ein homogenes Primideal $\mfp \in \Proj(S)$ sei $M_{(\mfp)}\coloneqq$ Gruppe der Elemente vom Grad $0$ in $T^{-1}M$, mit $T\coloneqq \{s \in S^\text{hom}\mid s \in S \setminus \mfp\}$. Also ist
	\[
		M_{(\mfp)} = \Bigg\lbrace \frac{m}{s} \;\Bigg\vert\; m \in M^{\text{hom}},\, s \in S^{\text{hom}},\, \deg(m) = \deg(s) \Bigg\rbrace
	\]
	ein $S_{(\mfp)}$-Modul. Sei $U$ offen in $\Proj(S)$. Definiere $\widetilde{M}(U)$ als die Menge der Funktionen $s\colon U \to \coprod_{\mfp \in U}M_{(\mfp)}$, die folgende Eigenschaften erfüllen:
	\begin{enumerate}[a)]
		\item Für alle $\mfp \in U$ gilt $s(\mfp) \in M_{(\mfp)}$.
		\item Für alle $\mfp \in U$ gibt es eine offene Umgebung $V$ von $\mfp$ in $U$ und $m \in M^{\text{hom}}$, $f\in S^{\text{hom}}$ mit den folgenden Eigenschaften: $\deg(m)=\deg(f)$ und für alle $\mfq \in V$ gilt $f \notin \mfq$ und $s(\mfq) =  \frac{m}{f} \in M_{(\mfq)}$.
	\end{enumerate}
	Mit \enquote{Einschränken von Funktionen} als Restriktionsabbildungen wird $\widetilde{M}$ zu einer Garbe auf $\Proj(S)$.
\end{defn}

\begin{prop}
\label{prop:9.17}
	\begin{enumerate}[i)]
		\item\label{prop:9.17:i} Für alle $\mfp \in \Proj(S)$ gilt $(\widetilde{M})_\mfp = M_{(\mfp)}$.
		\item\label{prop:9.17:ii} Für alle $f\in S_+^{\text{hom}}$ gilt $\widetilde{M}\vert_{D^+(f)} = \widetilde{M_{(f)}}$
		\item\label{prop:9.17:iii} $\widetilde{M}$ ist ein quasikohärenter $\mco_{\Proj(S)}$-Modul. Falls $S$ noethersch ist und $M$ ein endlich erzeugter $S$-Modul ist, so ist $\widetilde{M}$ kohärent.
	\end{enumerate}
	\begin{proof}
		i) und ii) folgen analog zum Beweis von Proposition~\ref{prop:5.17}, beziehungsweise Übung~\ref{defn:6.2} mit $M$ statt $S$. iii) folgt aus ii).
	\end{proof}
\end{prop}
