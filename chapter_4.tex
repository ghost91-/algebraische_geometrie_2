%!TEX root = algebraische_geometrie_2.tex
% vim: tw=0 noet sts=8 sw=8

\chapter{Affine Schemata}

Zu jedem Ring $A$ (kommutativ und mit Eins) betrachten wir das Spektrum $\Spec(A)$ der Primideale, das in natürlicher Weise eine Topologie besitzt. Durch Lokalisierung von $A$ erhalten wir eine Garbe $\mco_{\Spec(A)}$ auf $\Spec(A)$ und damit einen lokal gringten Raum $(\Spec(A),\mco_{\Spec(A)})$. Im folgenden Kapitel~5 werden dies die Bausteine für Schemata sein. Affine Schemata sind ähnlich wie affine Varietäten aus der Algebraischen Geometrie I, mit dem Unterschied, dass Primideale statt Maximalideale als Punkte und beliebige Ringe zugelassen werden.

\nextmark{\glqq Nullstellengebilde\grqq\ bei Spec}
\begin{defn}
	Für $M\subseteq A$ sei $V(M)\coloneqq\{\mfp \in \Spec(A) \mid \mfp \supseteq M \}$. Dies entspricht der Nullstellenmenge aus der Algebraischen Gerometrie I.
\end{defn}

\begin{lem}
\label{lem:4.2}
	\begin{enumerate}[i)]
		\item Sei $\mfa\coloneqq \langle M \rangle$ das von $M\subseteq A$ erzeugte Ideal. Dann gilt $V(\mfa) = V(M)$.
		\item Es gilt $V(\{0\}) = \Spec(A)$ und $V(A) = \emptyset$.
		\item Für Ideale $\mfa,\mfb$ von $A$ gilt $V(\mfa) \cup V(\mfb) = V(\mfa \cdot \mfb)$.
		\item Für eine Familie $(\mfa_i)_{i\in I}$ von Idealen in $A$ gilt $\bigcap_{i\in I}V(\mfa_i) = V\left(\sum_{i \in I} \mfa_i\right)$.
	\end{enumerate}
	\begin{proof}
		i) und ii) sind trivial.
		\begin{enumerate}[i)]
		\setcounter{enumi}{2}
		\item Es gilt
		\[
			\mfa \cdot \mfb = \langle \{a \cdot b \mid a \in \mfa,\; b \in \mfb\} \rangle .
		\]
		\enquote{$\subseteq$}: Sei $\mfp \in V(\mfa)$. Dann gilt $\mfa \subseteq \mfp$ und damit $\mfa \cdot \mfb \subseteq \mfa \subseteq \mfp$, also $\mfp \in V(\mfa \cdot \mfb)$.

		\enquote{$\supseteq$}: Sei $\mfp \in V(\mfa \cdot \mfb)$. Dann gilt $\mfa \cdot \mfb \subseteq \mfp$. Falls $\mfb \subseteq \mfp$ ist, dann gilt $\mfp \in V(\mfb)$ und wir sind fertig. Sei also ohne Beschränkung der Allgemeinheit $\mfb \not\subseteq \mfp$. Dann gibt es ein $b \in \mfb \setminus \mfp$. Für jedes $a \in \mfa$ gilt dann $a \cdot b \in \mfa \cdot \mfb \subseteq \mfp$. Da $\mfp$ prim ist, gilt also schon $a \in \mfp$ und damit $\mfa \subseteq \mfp$, also $\mfp \in V(\mfa)$.
		\item Dies ist einfach nachzurechnen.
		\end{enumerate}
	\end{proof}
\end{lem}

\begin{defn}[Zariski-Topologie auf \texorpdfstring{$\Spec(A)$}{Spec(A)}]
	Wir definieren eine Teilmenge von $\Spec(A)$ als abgeschlossen, wenn sie die Form $V(\mfa)$ für ein Ideal $\mfa$ von $A$ hat. Eine Teilmenge $U$ von $\Spec(A)$ heißt dann offen, wenn $\Spec(A)\setminus U$ abgeschlossen ist. Nach Lemma~\ref{lem:4.2} definiert dies eine Topologie auf $\Spec(A)$, die wir \textbf{Zariski-Topologie} nennen.
\end{defn}

\nextmark{Def. Verschwindeideal, Ideale in A vs. Topologie auf Spec(A)}
\begin{prop}
\label{prop:4.4}
	Für $Y\subseteq \Spec(A)$ definieren wir das Verschwindungsideal $I(Y) \coloneqq \bigcap_{\mfp \in Y} \mfp$.
	\begin{enumerate}[i)]
		\item Für ein Ideal $\mfa$ von $A$ gilt $I(V(\mfa)) = \sqrt{\mfa}$.
		\item Für $Y \subseteq \Spec(A)$ gilt $\sqrt{I(Y)}=I(Y)$ und $\overline{Y} = V(I(Y))$.
		\item Die Abbildungen
		\begin{center}
			\begin{tikzcd}
				\{\text{abgeschlossene Teilmengen in } \Spec(A)\} \arrow[shift left=1.1ex]{r}{I} & \{\text{Ideale }\mfa \text{ in } A \text{ mit } \sqrt{\mfa} = \mfa\} \arrow[shift left=1.1ex]{l}{V}
			\end{tikzcd}
		\end{center}
		sind bijektiv, zueinander invers und inklusionsumkehrend.
		\item $Y \subseteq \Spec(A)$ ist genau dann irreduzibel, wenn $I(Y)$ ein Primideal ist.
		\item Die Korrespondenz aus iii) induziert eine Bijektion
		\begin{center}
			\begin{tikzcd}
				\{\text{irreduzible abgeschlossene Teilmengen in } \Spec(A)\} \arrow[shift left=1.1ex]{r}{I} & \Spec(A) \arrow[shift left=1.1ex]{l}{V}
			\end{tikzcd}
		\end{center}
	\end{enumerate}
	\begin{proof}
		Wir benutzen
		\[
			I(V(\mfa)) = \bigcap_{\mfp \in V(\mfa)}\mfp = \bigcap_{\substack{\mfp \supseteq \mfa\\\mfp \in \Spec(A)}}\mfp = \sqrt{\mfa}.
		\]
		Dann folgen die Behauptungen analog wie bei affinen Varietäten.
	\end{proof}
\end{prop}

\nextmark{abgeschlossener/generischer Punkt, Spezialisierung}
\begin{defn}
	Sei $X$ ein topologischer Raum.
	\begin{enumerate}[i)]
		\item Ein Punkt $p \in X$ heißt \textbf{abgeschlossen}, wenn $\{p\}$ abgeschlossen ist.
		\item Ein Punkt $p \in X$ heißt \textbf{generischer Punkt}, wenn $\overline{\{p\}} = X$ gilt.
		\item Ein Punkt $q \in X$ heißt \textbf{Spezialisierung} von $p \in X$, wenn $q \in \overline{\{p\}}$ ist.
	\end{enumerate}
\end{defn}

\begin{lem}
\label{lem:4.6}
	Sei $X$ ein topologischer Raum.
	\begin{enumerate}[i)]
		\item Ist $X$ hausdorffsch, so ist jeder Punkt abgeschlossen.
		\item Existiert ein generischer Punkt in $X$, dann ist $X$ irreduzibel.
	\end{enumerate}
	\begin{proof}
		\begin{enumerate}[i)]
			\item Dies ist einfach zu zeigen.
			\item Sei $p$ ein generischer Punkt von $X$ und $X = X_1 \cup X_2$, wobei $X_1$ und $X_2$ abgeschlossen sind. Wir müssen zeigen, dass $X_1=X$ oder $X_2=X$ gilt. Es gilt $p \in X_i$ für ein $i \in \{1,2\}$ und damit
			\[
				X = \overline{\{p\}} \subseteq X_i.
			\]
		\end{enumerate}
	\end{proof}
\end{lem}

\begin{prop}
\label{prop:4.7}
	Sei $X = \Spec(A)$.
	\begin{enumerate}[i)]
		\item Für $\mfp \in X$ gilt $V(\mfp) = \overline{\{\mfp\}}$. Für $\mfp, \mfq \in X$ gilt insbesondere $\mfp \subseteq \mfq$ genau dann, wenn $\mfq \in \overline{\{\mfp\}}$ gilt.
		\item Unter der Korrespondenz aus Proposition~\ref{prop:4.4} iii) entsprechen die abgeschlossenen Punkte von $\Spec(A)$ genau den Maximalideal von $A$.
		\item Für $Y \subseteq X$ irreduzibel und abgeschlossen gilt $Y = \overline{\{\mfp\}}$ für $\mfp = I(Y)$. Damit ist $\mfp$ der nach i) eindeutig bestimmte generische Punkt von $Y$.
		\item Ist $A$ ein noetherscher Ring, so ist $X$ ein noetherscher topologischer Raum.
		\item Sei $A$ ein noetherscher Ring. Dann entsprechen die irreduziblen Komponenten von $\Spec(A)$ unter der Korrespondenz aus Proposition~\ref{prop:4.4} iii) genau den minimalen Primidealen von $A$.
		\item In einem noetherschen Ring existieren nur endliche viele minimale Primideale.
	\end{enumerate}
	\begin{proof}
		Dies wird in Übung~4.1 gezeigt.
	\end{proof}
\end{prop}

\nextmark{Def./Eig. standard-offener Mengen D( . )}
\begin{prop}
\label{prop:4.8}
	Für $f \in A$ sei $V(f) \coloneqq V(\langle f \rangle)$ und $D(f)\coloneqq \Spec(A) \setminus V(f)$.
	\begin{enumerate}[i)]
		\item Für eine Familie $(f_i)_{i\in I}$ in $A$ und $g \in A$ gilt:
		\[
			D(g) \subseteq \bigcup_{i\in I}D(f_i) \Leftrightarrow g \in \sqrt{\langle \{f_i \mid i \in I\}\rangle}
		\]
		\item Die Mengen $(D(f))_{f\in A}$ bilden eine Basis der Zariski-Topologie.
		\item Versehen wir $D(f)$ mit der induzierten Topologie, so ist $D(f)$ quasikompakt. Insbesondere ist $\Spec(A)=D(1)$ quasikompakt.
	\end{enumerate}
	\begin{proof}
		\begin{enumerate}[i)]
			\item Es gilt:
			\begin{align*}
				&D(g) \subseteq \bigcup_{i\in I} D(f_i)\\
				\Longleftrightarrow \quad & V(g) \supseteq \bigcap_{i \in I} V(f_i) = V(\langle\{f_i\mid i \in I\}\rangle)\\
				\Longleftrightarrow \quad & \sqrt{\langle g \rangle} \subseteq \sqrt{\langle\{f_i\mid i \in I\}\rangle}\\
				\Longleftrightarrow \quad & g \in \sqrt{\langle\{f_i\mid i \in I\}\rangle}
			\end{align*}
			\item Sei $\mfa$ ein Ideal und $\mfp \in U \coloneqq \Spec(A) \setminus V(\mfa)$. Zu zeigen ist, dass es ein $f \in A$ mit $\mfp \in D(f) \subseteq U$ gibt (Basiseigenschaft). Wegen $\mfp \notin V(\mfa)$ gibt es ein $f \in \mfa \setminus \mfp$ und man sieht leicht, dass dieses $f$ das Gewünschte liefert.
			\item Quasikompakt heißt, dass jede offene Überdeckung $(V_i)_{i\in I}$ von $D(f)$ eine endliche Teilüberdeckung hat, das heißt es gibt ein endliches $I_0 \subseteq I$ mit $\bigcup_{i\in I_0} V_i \supseteq D(f)$. Nach ii) bilden die Mengen $D(f)$ eine Basis, also sei ohne Beschränkung der Allgemeinheit $V_i = D(f_i)$ für ein $f_i \in A$. Nun gilt:
			\begin{align*}
				&D(f) \subseteq \bigcup_{i \in I} D(f_i)\\
				\overset{\text{i)}}{\Longleftrightarrow} \quad & f \in \sqrt{\langle \{f_i \mid i \in I\}\rangle}\\
				\Longleftrightarrow \quad & \exists\; m \ge 1,\; f^m = \sum_{i\in I_0} a_i f_i \in \langle \{f_i \mid i \in I\}\rangle \text{ für ein endliches }I_0 \subseteq I, a_i \in A\\
				\Longleftrightarrow \quad & \exists \text{ endliches } I_0 \subseteq I,\; f \in \sqrt{\langle\{f_i\mid i \in I_0\}\rangle}\\
				\Longleftrightarrow \quad & D(f) \subseteq \bigcup_{i \in I_0} D(f_i)
			\end{align*}
			Also ist $D(f)$ quasikompakt. Insbesondere ist $D(1) = \Spec(A)$ quasikompakt.
		\end{enumerate}
	\end{proof}
\end{prop}

\nextmark{Def./Eig. von Spec(phi)}
\begin{prop}
\label{prop:4.9}
	Sei $\varphi\colon A \to B$ ein Ringhomomorphismus. Dann gilt:
	\begin{enumerate}[i)]
		\item Die Abbildung $f\colon X = \Spec(B) \to Y = \Spec(A),\; \mfp \mapsto \varphi^{-1}(\mfp)$ ist stetig. Oft bezeichnen wir $f$ auch mit $\Spec(\varphi) = f$. Weiter gelten folgende Identitäten:
		\begin{alignat*}{2}
			f^{-1}(V(M)) &= V(\varphi(M)) \quad &&\forall \;M \subseteq A\\
			f^{-1}(D(a)) &= D(\varphi(a)) \quad &&\forall \;a \in A\\
			\overline{f(V(\mfb))} &= V(\varphi^{-1}(\mfb)) \quad &&\forall\; \mfb \text{ Ideal in } B
		\end{alignat*}
		\item Ist $\varphi$ surjektiv und $\mfa \coloneqq \ker(\varphi)$, dann definiert $f$ einen Homöomorphismus
		\[
			\Spec(B) \overset{\sim}{\longto} V(\mfa).
		\]
	\end{enumerate}
	
	\begin{proof}
		Dies wird in Aufgabe~4.2 gezeigt.
	\end{proof}
\end{prop}

\nextmark{Lokalisierung}
\begin{eri}
	Sei $S \subseteq A$ abgeschlossen unter Multiplikation und $1 \in S$. Dann ist die Lokalisierung $A_S$ definiert als die Menge der Äquivalenzklassen unter folgender Äquivalenzrelation auf $A \times S$:
	\[
		(a,s)\sim(a',s') \Longleftrightarrow \exists\; t \in S \text{ mit } t(s'a-sa') = 0
	\]
	Die Äquivalenzklasse von $(a,s)$ wird mit $\frac{a}{s}$ bezeichnet. $A_S$ ist ein Ring.
	\begin{enumerate}[i)]
		\item Für $\mfp \in \Spec(A)$ sei $A_{\mfp}\coloneqq A_S$ mit $S=A\setminus \mfp$.
		\item Für $f \in A$ sei $A_f \coloneqq A_S$ mit $S = \{f^m \mid m \in \N \}$.
	\end{enumerate}
\end{eri}

\begin{kons}[Lokal geringter Raum auf \texorpdfstring{$\Spec(A)$}{Spec(A)}]
\label{kons:4.11}
	\begin{enumerate}[i)]
		\item Sei $U$ offen in $\Spec(A)$. Dann bezeichnen wir mit $\mco(U)$ die Menge der Funktionen $s\colon U \to \coprod_{\mfp \in U}A_{\mfp}$, die folgende Eigenschaften erfüllen:
		\begin{enumerate}[a)]
			\item\label{kons:4.11:a} Für alle $\mfp \in U$ gilt $s(\mfp) \in A_{\mfp}$
			\item\label{kons:4.11:b} Für alle $\mfp \in U$ gibt es eine offene Umgebung $V$ von $\mfp$ in $U$ und $a,f\in A$ mit $V \subseteq D(f)$ und $s(\mfq) =  \frac{a}{f} \in A_\mfq$ für alle $\mfq \in V$.
		\end{enumerate}
		\item Aufgrund der lokalen Natur der Axiome a) und b) sieht man, dass $U \mapsto \mco(U)$ eine Garbe von Ringen auf $\Spec(A)$ definiert. Dabei sind die Addition und Multiplikation von solchen Funktionen punktweise unter Benutzung der entsprechenden Operation in den $A_{\mfp}$ definiert.
	\end{enumerate}
\end{kons}

\nextmark{Eig. der Strukturgarbe auf Spec(A)}
\begin{prop}
\label{prop:4.12}
	\begin{enumerate}[i)]
		\item Für alle $\mfp \in \Spec(A)$ gibt es einen kanonischen Isomorphismus
		\[
			\mco_{\mfp} \overset{\sim}{\longto} A_{\mfp}
		\]
		von Ringen, wobei $\mco_\mfp$ der Halm der Garbe $\mco$ in $\mfp$ ist. Insbesondere ist $\mco_\mfp$ ein lokaler Ring und damit ist $(\Spec(A),\mco)$ ein lokal geringter Raum.
		\item Für alle $f \in A$ gibt es einen kanonischen Isomorphismus
		\[
			A_f \overset{\sim}{\longto}\mco(D(f))
		\]
		von Ringen.
		\item Es gilt $\mco(\Spec(A)) = A$.
	\end{enumerate}
	\begin{proof}
		\begin{enumerate}[i)]
			\item Es gilt
			\[
				\mco_\mfp = \varinjlim_{\substackclap{\mfp \in U\\U \text{ offen}}} \, \mco(U),
			\]
			das heißt ein Element von $\mco_\mfp$ ist repräsentiert durch $(U,s)$, wobei $U$ offen mit $\mfp \in U$ und $s \in \mco(U)$ ist. Weiter gilt $[(U,s)] = [(U',s')] \in \mco_\mfp$ genau dann, wenn es ein offenes $V \subseteq U \cap U'$ gibt mit $\mfp \in V$ und $s\vert_V = s'\vert_V$. Sei $\mfp \in \Spec(A)$ und $U$ eine offene Umgebung von $\mfp$. Für $U$ offen in $\Spec(A)$ betrachten wir
			\[
			 	\mco(U) \to A_\mfp,\; s \mapsto s(\mfp).
			\]
			Nach \ref{kons:4.11:a} sind dies wohldefinierte Ringhomomorphismen. Weiter sind diese Homomorphismen verträglich mit Einschränkungen auf kleinere Umgebungen $V$. Also wird ein Ringhomomorphismus $\varphi\colon\mco_\mfp \to A_\mfp,\; [(U,s)] \mapsto s(\mfp)$ induziert.
			Sei $\frac{a}{f}\in A_\mfp$, also $f \notin \mfp,\; a \in A$. Dies definiert einen Schnitt
			\[
				s = \frac{a}{f}\colon D(f) \to \coprod_{\mfq \in D(f)} A_\mfq,\; \mfq \mapsto s(\mfq) \coloneqq \frac{a}{f} \in A_\mfq.
			\]
			Offenbar gilt $\varphi(s) = \frac{a}{f}\in A_\mfp$, also ist $\varphi$ surjektiv.

			Sei $U$ eine Umgebung von $\mfp$ und seien $s,t \in \mco(U)$ mit $s(\mfp) = t(\mfp)$, das heißt
			\[
				\varphi([(U,s)]) = \varphi([(U,t)]).
			\]
			Wir wollen nun $s=t \in \mco_\mfp$ zeigen. Indem wir die Umgebung von $\mfp$ gegebenenfalls verkleinern, dürfen wir nach \ref{kons:4.11:b} annehmen, dass es $a,b,f,g \in A$ mit $f,g \notin \mfp$ gibt, so dass $s = \frac{a}{f} \in \mco(U)$ und $t=\frac{b}{g}\in \mco(U)$ gilt. Dann gilt:
			\begin{align*}
				& s(\mfp) = \varphi(s) = \varphi(t) = t(\mfp)\\
				\Longrightarrow \quad & \frac{a}{f} = \frac{b}{g} \in A_\mfp\\
				\Longrightarrow \quad & \exists\; h \in A \setminus \mfp \text{ mit } h(ga-fb)=0
			\end{align*}
			Weiter gilt $\frac{a}{f}=\frac{b}{g} \in A_\mfq$ für alle $\mfq \in \Spec(A)$ mit $f,g,h \notin \mfq$. Die Menge dieser $\mfq$ ist gleich der offenen Menge $W \coloneqq D(f)\cap D(g) \cap D(h) \ni \mfp$. Also gilt $s\vert_{W\cap U} = t\vert_{W\cap U}$ und damit $s=t \in \mco_\mfp$ nach der Definition von $\mco_\mfp$. Also ist $\varphi$ injektiv.

			Damit ist $\varphi$ ein kanonischer Isomorphismus.

			\item Wir definieren
			\begin{align*}
				\psi_f\colon A_f &\to \mco(D(f))\\
				\frac{a}{f^n} & \mapsto \left(s\colon D(f) \to \coprod_{\mfp \in D(f)}A_\mfp,\; \mfp \mapsto s(\mfp)\coloneqq \frac{a}{f^n}\in A_\mfp\right).
			\end{align*}
			Offenbar ist $\psi_f$ ein wohldefinierter Ringhomomorphismus.

			Sei $\psi_f\left(\frac{a}{f^n}\right) = \psi_f \left(\frac{b}{f^m}\right)$. Dann gilt $\frac{a}{f^n}=\frac{b}{f^m} \in A_\mfp$ für alle $\mfp \in D(f)$. Sei
			\[
				\mfa\coloneqq \Ann_A(f^m a-f^n b) = \{g \in A \mid g \cdot (f^m a - f^n b) = 0\}.
			\]
			Beachte, dass $\mfa$ ein Ideal in $A$ ist. Nun gibt es ein $h \in A \setminus \mfp$ mit $h(f^m a -f^n b) = 0$, also $h \in \mfa \setminus \mfp$. Also gilt $\mfa \not \subseteq \mfp$, das heißt $\mfp \notin V(\mfa)$. Dies gilt für alle $\mfp \in D(f)$, also gilt $D(f) \cap V(\mfa) = \emptyset$. Damit gilt $V(\mfa) \subseteq V(f)$. Mit Proposition~\ref{prop:4.4} folgt wegen $f \in \sqrt{\mfa}$, dass es ein $k \ge 1$ mit $f^k \in \mfa$ gibt. Dann gilt
			$f^k(f^m a -f^n b) = 0$ und damit $\frac{a}{f^n} = \frac{b}{g^m} \in A_f$. Also ist $\psi_f$ injektiv.

			Sei $s \in \mco(D(f))$. Nach Proposition~\ref{prop:4.8} ii) und der Definition von $\mco$ in Konstruktion~\ref{kons:4.11} gibt es eine Familie $(h_i)_{i\in I}$ in $A$ mit
			\begin{equation*}
			\label{eq:4.12.1}
				D(f) = \bigcup_{i\in I} D(h_i) \tag{$\star$}
			\end{equation*}
			und $a_i,g_i \in A$ mit $D(h_i) \subseteq D(g_i)$, sodass für alle $i \in I$
			\[
				s\vert_{D(h_i)} = \frac{a_i}{g_i} \in \mco(D(h_i))
			\]
			gilt. Aus $D(h_i) \subseteq D(g_i)$ folgt mit Proposition~\ref{prop:4.8}, dass es $n_i \ge 1$ und $c_i \in A$ mit $h_i^{n_i} = c_i g_i$ gibt. Dann gilt
			\[
				\frac{a_i}{g_i} = \frac{c_i a_i}{h_i^{n_i}} \in \mco(D(h_i)) = \mco(D(h_i^{n_i})).
			\]
			Wir ersetzen $h_i$ durch $h_i^{n_i}$ und $a_i$ durch $c_i a_i$ und erhalten
			\[
				s\vert_{D(h_i)} = \frac{a_i}{h_i} \in \mco(D(h_i)).
			\]
			Da $D(f)$ nach Proposition~\ref{prop:4.8} quasikompakt ist, gibt es $h_1,\ldots,h_r$ mit
			\[
				D(f) \overset{\eqref{eq:4.12.1}}{=}  D(h_1)\cup \cdots \cup D(h_r).
			\]
			Wir fassen $\frac{a_i}{h_i}$ und $\frac{a_j}{h_j}$ als Elemente von $A_{h_ih_j}$ auf. Wegen $D(h_i)\cap D(h_j) = D(h_ih_j)$ gilt 
			\[
				\psi_{h_ih_j}\left(\frac{a_i}{h_i}\right) = \psi_{h_ih_j}\left(\frac{a_j}{h_j}\right) = s\vert_{D(h_ih_j)} \in \mco(D(h_ih_j)).
			\]
			Aus der Injetivität von $\psi_{h_ih_j}$ folgt $\frac{a_i}{h_i} = \frac{a_j}{h_j} \in A_{h_ih_j}$. Deswegen gibt es $n_{ij}\ge 1$ mit
			\begin{equation*}
			\label{eq:4.12.2}
				(h_ih_j)^{n_{ij}}(h_ja_i-h_ia_j) = 0 \quad \forall \;1 \le i,\; j \le r. \tag{$\star\star$}
			\end{equation*}
			Sei $n \coloneqq \max\{n_{ij}\mid 1\; \le i, j \le r\}$, $\widetilde{a}_i\coloneqq h_i^na_i$, $\widetilde{h}_i \coloneqq h_i^{n+1}$. Wieder gilt $D(\widetilde{h}_i) = D(h_i)$. Aus \eqref{eq:4.12.2} folgt
			\[
				h_j^{n+1}(h_ia_i) - h_i^{n+1}(h_j^na_j) = 0 \quad \forall \;1 \le i,\; j\le r.
			\]
			Deswegen gilt
			\[
				s\vert_{D(\widetilde{h}_i)} = \frac{\widetilde{a}_i}{\widetilde{h}_i} \text{ und } \widetilde{h}_i\widetilde{a}_j = \widetilde{a}_i\widetilde{h}_j \quad \forall \;1 \leq i,\; j \le r.
			\]
			Aus \eqref{eq:4.12.1} folgt mit Proposition~\ref{prop:4.8} i), dass es ein $m \ge 1$ mit $f^m = \sum_{i=1}^r b_i \widetilde{h}_i$ für gewisse $b_i \in A$ gibt. Wir setzen $a \coloneqq \sum_{i=1}^r b_i \widetilde{a}_i$. Für $1 \le j \le r$ gilt
			\[
				\widetilde{h}_j a = \sum_{i=1}^r\widetilde{h}_jb_i\widetilde{a}_i = \sum_{i=1}^r\widetilde{h}_ib_i \widetilde{a}_j = f^m \widetilde{a}_j
			\]
			und damit folgt
			\[
				\frac{\widetilde{a}_j}{\widetilde{h}_j} = \frac{a}{f^m} \in A_{\widetilde{h}_j} \overset{\psi_{\widetilde{h}_j}}{\hookrightarrow} \mco(D(\widetilde{h}_j))
			\]
			und damit $\psi_f\left(\frac{a}{f}\right) = s$ auf jedem $D(\widetilde{h}_j)$. Da $\mco$ eine Garbe ist, gilt dies auch auf $D(f)$. Damit ist $\psi_f$ surjektiv.

			Insgesamt ist $\psi_f$ ein Isomorphismus.
			\item Dies folgt aus ii) mit $f=1$.
		\end{enumerate}
	\end{proof}
\end{prop}

\begin{bem*}
	Wir nennen $\mco$ die Garbe der regulären Funktionen auf $\Spec(A)$.
\end{bem*}

\nextmark{Spektrum eines Rings (als lgR)}
\begin{defn}
	Der lokal geringte Raum $(\Spec(A), \mco)$ heißt \textbf{Spektrum} von $A$.
\end{defn}

\pagebreak[2]
\nextmark{Ringhomomorphismen vs. Morphismen zwischen Spektren}
\begin{prop}
\label{prop:4.14}
	Seien $A, B$ Ringe, $X \coloneqq \Spec(B)$ und $Y \coloneqq \Spec(A)$.
	\begin{enumerate}[i)]
		\item Ein Ringhomomorphismus $\varphi\colon A \to B$ induziert eine stetige Abbildung
		\[
			(f = \Spec(\varphi))\colon X \to Y,\; \mfp \mapsto \varphi^{-1}(\mfp),
		\]
		einen Garbenhomomorphismus
		\[
			f^{\#}\colon \mco_X \to f_*\mco_Y
		\]
		und einen Morphismus
		\[
			(f,f^{\#})\colon (X,\mco_X)\to (Y,\mco_Y)
		\]
		lokal geringter Räume.
		\item Ein Morphismus lokal geringter Räume $(f,f^{\#})\colon (X,\mco_X)\to(Y,\mco_Y)$ induziert einen Ringhomomorphismus
		\[
			(\varphi = \Gamma(Y,f^{\#}))\colon (A = \mco_Y(Y)) \to (B=\mco_X(X)).
		\]
		\item Es gibt eine bijektive Korrespondenz
		\begin{align*}
			\phi\colon \Hom_{\Ring}(A,B) &\to \Hom_{\lgR}((\Spec(B),\mco_{\Spec(B)}),(\Spec(A),\mco_{\Spec(A)}))\\
			\varphi & \mapsto ((f=\Spec(\varphi)),f^{\#})
		\end{align*}
		und die Umkehrabbildung $\Psi$ ist gegeben durch
		\[
			(f,f^{\#})\mapsto \Gamma(\Spec(A),f^{\#}).
		\]
	\end{enumerate}
	\begin{proof}
		Dies wird in Aufgabe~5.1 gezeigt.
	\end{proof}
\end{prop}

\nextmark{affines Schema}
\begin{defn}
	Ein \textbf{affines Schema} ist ein lokal geringter Raum $(X,\mco_X)$, welcher für einen Ring~$A$ zu $(\Spec(A),\mco_{\Spec(A)})$ isomorph ist.
\end{defn}

\nextmark{standard-offene Mengen sind affin}
\begin{lem}
\label{lem:4.16}
	Sei $f \in A$. Dann existiert ein kanonischer Isomorphismus
	\[
		(\Spec(A_f),\mco_{\Spec(A_f)}) \overset{\sim}{\longto} (D(f),\mco_{\Spec(A)}\vert_{D(f)})
	\]
	von lokal geringten Räumen. Insbesondere ist $(D(f),\mco_{\Spec(A)}\vert_{D(f)})$ ein affines Schema.
	\begin{proof}
		Dies folgt leicht aus Proposition~\ref{prop:4.12} und wird in Aufgabe~5.2 gezeigt.
	\end{proof}
\end{lem}

\nextmark{Restklassenkörper}
\begin{defn}
	Sei $(X,\mco_X)$ ein lokal geringter Raum und $p \in X$. Dann ist $\mco_{X,p}$ ein lokaler Ring mit eindeutigem Maximalideal $\mfm_p$. Also ist
	\[
		\kappa(p) \coloneqq \mco_{X,p}/\mfm_p
	\]
	ein Körper, den wir \textbf{Restklassenkörper} von $(X,\mco_X)$ in $p$ nennen.
\end{defn}
