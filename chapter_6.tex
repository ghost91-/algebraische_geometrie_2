%!TEX root = algebraische_geometrie_2.tex
% vim: tw=0

\chapter{Erste Eigenschaften von Schemata}
\label{chapter:6}

Nachdem wir Schemata definiert haben untersuchen wir hier erste Eigenschaften, wie \enquote{Endlichkeit} von Morphismen.

\begin{eri}
\label{eri:6.1}
	\begin{enumerate}[i)]
		\item Ein topologischer Raum heißt \textbf{quasikompakt}, wenn jede offene Überdeckung eine endliche Teilüberdeckung hat.
		\item Ein Ring $A$ heißt \textbf{reduziert}, wenn $\nil(A) = \{0\}$ gilt, wobei $\nil(A) \coloneqq \sqrt{\{0\}} = \bigcap_{\mfp \in \Spec(A)} \mfp$ das \textbf{Nilradikal} von $A$ ist.
	\end{enumerate}
\end{eri}

\begin{defn}
\label{defn:6.2}
	\begin{enumerate}[i)]
                \newcommand{\multitext}[1]{%
                $\left\{\begin{tabular}{@{}l@{}}%
                    #1
                \end{tabular} \right\}$%
                }
                %
		\item Ein Schema heißt
                \multitext{%
			zusammenhängend\\
			irreduzibel\\
			quasikompakt},
		wenn der unterliegene topologische Raum
                \multitext{%
			zusammenhängend\\
			irreduzibel\\
                        quasikompakt}
                ist.
		\item Ein Schema $(X, \mco_X)$ heißt
                \multitext{%
			integer\\
			reduziert},
		falls für jedes offene $U$ in $X$ schon $\mco_X(U)$
                \multitext{%
			integer\\
			reduziert}
		ist.
	\end{enumerate}
\end{defn}

\begin{lem}
\label{lem:6.3}
	$X=\Spec(A)$ ist genau dann irreduzibel, wenn $\nil(A)$ ein Primideal ist. In diesem Fall ist $\nil(A)$ der generische Punkt von $X$.
	\begin{proof}
		Es gilt
		\begin{align*}
			&X \text{ irreduzibel}\\
                        \underset{\mathclap{\text{und Lemma~\ref{lem:4.6} ii)}}}{\overset{\mathclap{\text{Proposition~\ref{prop:5.4} ii)}}}{\Longleftrightarrow}} \qquad\quad & X \text{ hat generischen Punkt}\\
			 \Longleftrightarrow \qquad\quad & \exists\; \mfp \in \Spec(A) \text{ mit } X = \overline{\{\mfp\}} = V(\mfp)\\
			 \Longleftrightarrow \qquad\quad & \exists \text{ kleinstes Primideal in } A\\
			 \Longleftrightarrow \qquad\quad &  \bigcap_{\mfp \in X} \mfp \text{ ist Primideal} .
		\end{align*}
	\end{proof}
\end{lem}

\begin{lem}
\label{lem:6.4}
	Sei $X$ ein Schema und $f \in \mco_X(X)$. Wir betrachten die zu $f$ assoziierte Funktion
	\[
		F\colon X \to \coprod_{p \in X}\kappa(p),\; p \mapsto F(p)\coloneqq f_p+\mfm_{X,p}\in \kappa(p).
	\]
	Die Menge $V(F) \coloneqq \{p\in X\mid F(p) = 0\}$ ist abgeschlossen in $X$ und $X_f\coloneqq X \setminus V(F)$ ist offen in $X$.
	\begin{proof}
		$X$ wird durch offene affine Unterschemata $U=\Spec(A)$ überdeckt. Um zu zeigen, dass $V(F)$ abgeschlossen ist, genügt es zu zeigen, dass für eine fixierte solche Überdeckung für alle Unterschemata~$U$ der Überdeckung schon $V(F)\cap U$ abgeschlossen in $U$ ist. Setze
		\[
			\widetilde{f}\coloneqq f\vert_U\in \mco_X(U) \cong A.
		\]
		Für $\mfp\in U = \Spec(A)$ gilt:
		\begin{align*}
			&\mfp \in V(F) \cap U\\
			\Longleftrightarrow \quad & F(\mfp)=0\\
                        \Longleftrightarrow \quad & f_{\mfp}+\mfm_{X,\mfp} = 0 \in \kappa(\mfp) \cong A_\mfp/\mfp A_\mfp\\
			\Longleftrightarrow \quad & \widetilde{f}_\mfp \in \mfp A_\mfp\\
			\Longleftrightarrow \quad & \widetilde{f} \in \mfp\\
			\Longleftrightarrow \quad & \mfp \in V(\langle \widetilde{f} \rangle) \subset \Spec(A) .
		\end{align*}
		Somit ist $V(F)\cap U$ abgeschlossen in $U = \Spec(A)$.
	\end{proof}
\end{lem}

