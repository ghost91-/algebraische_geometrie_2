%!TEX root = algebraische_geometrie_2.tex
% vim: tw=0

\chapter{Erste Eigenschaften von Schemata}
\label{chapter:6}

Nachdem wir Schemata definiert haben untersuchen wir hier erste Eigenschaften, wie \enquote{Endlichkeit} von Morphismen.

\begin{eri}
\label{eri:6.1}
	\begin{enumerate}[i)]
		\item Ein topologischer Raum heißt \textbf{quasikompakt}, wenn jede offene Überdeckung eine endliche Teilüberdeckung hat.
		\item Ein Ring $A$ heißt \textbf{reduziert}, wenn $\nil(A) = \{0\}$ gilt, wobei $\nil(A) \coloneqq \sqrt{\{0\}} = \bigcap_{\mfp \in \Spec(A)} \mfp$ das \textbf{Nilradikal} von $A$ ist.
	\end{enumerate}
\end{eri}

\begin{defn}
\label{defn:6.2}
	\begin{enumerate}[i)]
		\item Ein Schema heißt
		\multitext{%
		zusammenhängend\\
		irreduzibel\\
		quasikompakt},
		wenn der unterliegene topologische Raum
		\multitext{%
		zusammenhängend\\
		irreduzibel\\
		quasikompakt}
		ist.
		\item Ein Schema $(X, \mco_X)$ heißt
		\multitext{%
		integer\\
		reduziert},
		falls für jedes offene $U$ in $X$ schon $\mco_X(U)$
		\multitext{%
		integer\\
		reduziert}
		ist.
	\end{enumerate}
\end{defn}

\begin{lem}
\label{lem:6.3}
	$X=\Spec(A)$ ist genau dann irreduzibel, wenn $\nil(A)$ ein Primideal ist. In diesem Fall ist $\nil(A)$ der generische Punkt von $X$.
	\begin{proof}
		Es gilt
		\begin{align*}
			&X \text{ irreduzibel}\\
			\underset{\mathclap{\text{und Lemma~\ref{lem:4.6} ii)}}}{\overset{\mathclap{\text{Proposition~\ref{prop:5.4} ii)}}}{\Longleftrightarrow}} \qquad\quad & X \text{ hat generischen Punkt}\\
			 \Longleftrightarrow \qquad\quad & \exists\; \mfp \in \Spec(A) \text{ mit } X = \overline{\{\mfp\}} = V(\mfp)\\
			 \Longleftrightarrow \qquad\quad & \exists \text{ kleinstes Primideal in } A\\
			 \Longleftrightarrow \qquad\quad &  \bigcap_{\mfp \in X} \mfp \text{ ist Primideal} .
		\end{align*}
	\end{proof}
\end{lem}

\begin{lem}
\label{lem:6.4}
	Sei $X$ ein Schema und $f \in \mco_X(X)$. Wir betrachten die zu $f$ assoziierte Funktion
	\[
		F\colon X \to \coprod_{p \in X}\kappa(p),\; p \mapsto F(p)\coloneqq f_p+\mfm_{X,p}\in \kappa(p).
	\]
	Die Menge $V(f) \coloneqq \{p\in X\mid F(p) = 0\}$ ist abgeschlossen in $X$ und $X_f\coloneqq X \setminus V(f)$ ist offen in $X$.
	\begin{proof}
		$X$ wird durch offene affine Unterschemata $U=\Spec(A)$ überdeckt. Um zu zeigen, dass $V(f)$ abgeschlossen ist, genügt es zu zeigen, dass für eine fixierte solche Überdeckung für alle Unterschemata~$U$ der Überdeckung schon $V(f)\cap U$ abgeschlossen in $U$ ist. Setze
		\[
			\widetilde{f}\coloneqq f\vert_U\in \mco_X(U) \cong A.
		\]
		Für $\mfp\in U = \Spec(A)$ gilt:
		\begin{align*}
			&\mfp \in V(f) \cap U\\
			\Longleftrightarrow \quad & F(\mfp)=0\\
			\Longleftrightarrow \quad & f_{\mfp}+\mfm_{X,\mfp} = 0 \in \kappa(\mfp) \cong A_\mfp/\mfp A_\mfp\\
			\Longleftrightarrow \quad & \widetilde{f}_\mfp \in \mfp A_\mfp\\
			\Longleftrightarrow \quad & \widetilde{f} \in \mfp\\
			\Longleftrightarrow \quad & \mfp \in V(\langle \widetilde{f} \rangle) \subset \Spec(A) .
		\end{align*}
		Somit ist $V(f)\cap U$ abgeschlossen in $U = \Spec(A)$. Dies zeigt auch, dass diese Definition von $V(f)$ konsitent mit der Defininition von $V(f)$ ist, im Fall, dass $X$ ein affines Schema ist.
	\end{proof}
\end{lem}

\begin{prop}\label{prop:6.5}
	Sei $X$ ein Schema.
	\begin{enumerate}[i)]
		\item $X$ ist genau dann reduziert, wenn $\mco_{X,p}$ für alle $p \in X$ ein reduzierter Ring ist.
		\item Ist $X$ integer, so ist $\mco_{X,p}$ für alle $p \in X$ ein Integritätsbereich.
		\item $X$ ist integer genau dann, wenn $X$ reduziert und irreduzibel ist.
	\end{enumerate}
	\begin{proof}
		\begin{enumerate}[i)]
			\item Nach Definition ist $X$ genau dann reduziert, wenn $\mco_X(U)$ für alle $U$ offen in $X$ ein reduzierter Ring ist.
			\item \enquote{$\Longrightarrow$}: Sei $X$ also reduziert und $p \in X$. Falls es ein $f \in \mco_{X,p}\setminus\{0\}$ mit $f^n=0 \in \mco_{X,p}$ gibt für ein $n \in \N$, so gibt es ein $\widetilde{f}\in \mco_{X}(U)$ für eine offene Umgebung $U$ von $p$ mit $\widetilde{f}_p=f \in \mco_{X,p}$ mit $\widetilde{f}^n = 0$. Dies ist jedoch ein Widerspruch zu der Tastsache, dass $\mco_{X}(U)$ ein reduzierter Ring ist.\\
			\enquote{$\Longleftarrow$}: Sei $\mco_X(U)$ ein reduzierter Ring
			offen in $X$. Falls $\mco_X(U)$ nicht reduziert ist, dann gibt es ein $f \in \mco_X(U)\setminus\{0\}$ mit $f^n = 0$ für ein $n \in \N$. Dann gibt es ein $p \in U$ mit $f_p \neq 0 \in \mco_{X,p}$. Mit
			\[
				(f^n)_p = (f_p)^n=0 \in \mco_{X,p}
			\]
			folgt dann aber ein Widerspruch.
			\item Sei $X$ integer, das heißt $\mco_X(U)$ ist für alle $U$ offen in $X$ integer. Betrachte $p \in X$. Wir wählen eine affine offene Umgebung $U = \Spec(A)$ von $p$ in $X$. Dann ist $p$ durch $\mfp \in \Spec(A)$ gegeben. Weil $X$ integer ist, ist $A = \mco_X(U)$ ein Integritätsbereich. Es gilt $\mco_{X,p} = A_\mfp$. Da die Lokalisierung eines Integritätsbereiches wieder integer ist, folgt die Behauptung.
			\item \enquote{$\Longrightarrow$}: Sei $X$ ein integres Schema. Dann ist $X$ trivialer Weise reduziert. Wir zeigen wieder indirekt, dass $X$ irreduzibel ist. Ist $X$ nicht irreduzibel, so gibt es nicht-leere disjunkte offene Teilmengen $U_1,U_2$ von $X$. Da $U_1$ und $U_2$ disjunkt sind, gilt
			\[
				\mco_X{U_1 \dcup U_2} = \mco_{X}(U_1) \times \mco_{X}(U_2).
			\]
			Solch ein Produkt hat jedoch immer Nullteiler der Form $(f,0) \cdot (0,g) = (0,0)$.\\
			\enquote{$\Longleftarrow$}: Sei nun $X$ reduziert und irreduzibel. Sei weiter $U$ offen in $X$. Wir müssen zeigen, dass $\mco_X(U)$ ein Integritätsbereich ist. Seien $f,g \in \mco_X(U)$ mit $f\cdot g = 0$. Es gilt
			\[
				X = V(0) = V(fg) = V(f) \cup V(g).
			\]
			Da $U$ offen im irreduziblem Raum $X$ ist, ist auch $U$ irreduzibel. Deswegen können wir ohne Beschränkung der Allgemeinheit $V(f) = U$ setzen. Beachte, dass $V(f)$ wie in Lemma~\ref{lem:6.4} und damit abgeschlossen ist. Wir zeigen nun für jedes affine offene Unterschema $V=\Spec(A)$ von $U$, dass $f\vert_V = 0$ gilt, woraus mit der Garbeneigenschaft $f=0$ folgt, dass $U$ integer ist. Für $V=\Spec(A) \subset U = V(f)$ offen gilt nach Definition $f\vert_V \in \mfp$ für alle $\mfp \in \Spec(A)$. Also gilt
			\[
				f\vert_V \in \bigcap_{\mfp \in \Spec(A)} \mfp = \nil(A) = \{0\},
			\]
			da $A = \mco_X(V)$ reduziert ist.
		\end{enumerate}
	\end{proof}
\end{prop}

\begin{defn}
\label{defn:6.6}
	\begin{enumerate}[i)]
		\item Ein Schema $X$ heißt \textbf{lokal noethersch}, wenn $\mco_{X}(U)$ für alle affinen offenen Unterschemata $U = \Spec(A)$ noethersch ist.
		\item Ein Schema $X$ heißt \textbf{noethersch}, wenn $X$ lokal noethersch und quasikompakt ist.
	\end{enumerate}
\end{defn}


\begin{prop}
\label{prop:6.7}
	Sei $X$ ein Schema.
	\begin{enumerate}[i)]
		\item $X$ ist genau dann (lokal) noethersch, wenn es eine (endliche) Überdeckung $(U_i)_{i\in I}$ durch affine offene Unterschemata $U_i = \Spec(A_i)$ mit noetherschen Ringen $A_i$ gibt.
		\item Ein affines Schema $\Spec(A)$ ist genau dann noethersch, wenn $A$ ein noetherscher Ring ist.
		\item Ist $X$ ein noethersches Schema, so ist der unterliegende topologische Raum nothersch und hab damit eine Zerlegung in endlich viele irreduzible Komponenten.
		\item Ist $X$ ein noethersches Schema, so ist $\mco_{X,p}$ für alle $p \in X$ ein noetherscher lokaler Ring.
	\end{enumerate}
	\begin{proof}
		\cite[Proposition II.3.2]{hartshorne1977algebraic}.
	\end{proof}
\end{prop}

\begin{defn}
\label{defn:6.8}
	Sei $f\colon X \to Y$ ein Morphismus von Schemata.
	\begin{enumerate}[i)]
		\item $f$ heißt \textbf{injektiv (beziehungsweise surjektiv, beziehungsweise bijektiv)}, wenn die unterliegende Abbildung topologischer Räume injektiv (beziehungsweise surjektiv, beziehungsweise bijektiv) ist.
		\item $f$ heißt \textbf{offen (beziehungsweise abgeschlossen, beziehungsweise Homöomorphismus)}, wenn die unterliegende Abbildung topologischer Räume  offen (beziehungsweise abgeschlossen, beziehungsweise ein Homöomorphismus) ist.
		\item $f$ heißt \textbf{quasikompakt}, wenn $f^{-1}(V)$ für alle $V$ quasikompakt und offen in $Y$ quasikompakt ist.
		\item $f$ heißt \textbf{lokal von endlichem Typ}, wenn für alle affinen offenen Unterschemata $V=\Spec(A)$ von $Y$ und jedes affine offene Unterschema $U = \Spec(B)$ in $f^{-1}(V)$ schon $B$ eine endliche $A$-Algebra ist vermöge der Abbildung
		\[
			(A = \mco_{Y}(V))\overset{f^{\#}}{\longto}(\Gamma(V,f_*\mco_X)=\Gamma(f^{-1}V,\mco_X))\overset{\text{Einschränkung}}{\longto}(\Gamma(U,\mco_X) = B).
		\]
		\item $f$ heißt \textbf{von endlichem Typ}, wenn $f$ lokal von endlichem Typ und quasikompakt ist.
		\item $f$ heißt \textbf{endlich}, wenn für alle affinen offenen Unterschemata $V = \Spec(A)$ von $Y$ schon $f^{-1}(V)$ ein affines offenes Unterschema von $X$ und $B = \Gamma(f^{-1}(V),\mco_X)$ ein endlich erzeugter $A$-Modul ist vermöge der selben Abbildung, wie in iv).
	\end{enumerate}
\end{defn}

\begin{bem}
\label{bem:6.9}
	Nach Proposition~\ref{prop:4.7} ist jedes affine Schema quasikompakt. Also ist ein Schema $X$ genau dann quasikompakt, wenn es eine endliche Vereinigung von affinen offenen Unterschemata ist.
\end{bem}

\begin{prop}
\label{prop:6.10}
	Sei $f\colon X \to Y$ ein Morphismus von Schemata.
	\begin{enumerate}[i)]
		\item $f$ ist genau dann quasikompakt, wenn es eine Überdeckung $(V_i)_{i\in I}$ von $Y$ durch affine offene Unterschemata $V_i = \Spec(B_i)$ mit quasikompakten Urbildern $f^{-1}(V_i)$ gibt.
		\item $f$ ist genau dann lokal von endlichem Typ, wenn es eine Überdeckung $(V_i)_{i \in I}$ von $Y$ durch affine offene Unterschemata $V_i = \Spec(B_i)$ mit
		\[
			f^{-1}(V_i) = \bigcup_{j \in J_i}U_{ij}
		\]
		gibt für gewisse offene affine Unterschemata $U_{ij} = \Spec(B_{ij})$, wobei $B_{ij}$ für alle $i,j \in I$  eine endlich erzeugte $B_i$-Algebra ist.
		\item $f$ ist genau dann endlich, wenn es eine Überdeckung $(V_i)_{i \in I}$ von $Y$ durch affine offene Unterschemata $V_i = \Spec(A_i)$ gibt, mit der Eigenschaft, dass $f^{-1}(V_i) = \Spec(B_i)$ für alle $i \in I$ ein affines offenes Unterschema von $X$ ist und $B_i$ für alle $i \in I$ ein endlich erzeugter $A_i$-Modul ist.
	\end{enumerate}
	\begin{proof}
		Dies wird in den Aufgaben 7.1 bis 7.3 bewiesen.
	\end{proof}
\end{prop}

\begin{prop}
\label{prop:6.11}
	\begin{enumerate}[i)]
		\item Sei $Y$ ein lokal noethersches Schema und sei $f\colon X \to Y$ ein Morphismus lokal von endlichem Typ. Dann ist auch $X$ ein lokal noethersches Schema.
		\item Sei $Y$ ein noethersches Schema und sei $f \colon X \to Y$ ein Morphismus von endlichem Typ. Dann ist auch $X$ ein noethersches Schema.
	\end{enumerate}
	\begin{proof}
		Nach dem hilbertschen Basissatz ist eine endlich erzeugte Algebra über einem noetherschen Ring wieder noethersch.
		\begin{enumerate}[i)]
			\item Es gibt eine Überdeckung $(V_i)_{i \in I}$ von $Y$ durch offene affine Unterschemata $V_i=\Spec(B_i)$ mit der Eigenschaft, dass $f^{-1}(V_i) = \bigcup_{j \in J_i} U_{ij}$ gilt, wobei $U_{ij} = \Spec(B_{ij})$ ist und $B_{ij}$ für alle $i,j \in I$ eine endlich erzeugte $B_i$-Algebra ist. Da $Y$ lokal noethersch ist, ist $B_i$ nothersch und damit nach dem hilbertschen Basissatz auch $B_{ij}$. Nach Proposition~\ref{prop:6.7} ist $X$ lokal noethersch.
			\item Dies folgt aus i), weil $f$ quasikompakt ist.
		\end{enumerate}
	\end{proof}
\end{prop}

\begin{bsp}
\label{bsp:6.12}
	Sei $A$ ein Ring. Dann sind die kanonischen Morphismen $\A^n_A \to \Spec(A)$ und $\P^n_A\to\Spec(A)$ quasikompakt (benutze die endliche Standadüberdeckung durch affine Räume). Sie sind auch von endlichem Typ. Sie sind genau dann endlich, wenn wenn $n = 0$ ist. Falls $A$ ein noetherscher Ring ist, dann sind $\A_A^n$ und $\P_A^n$ noethersche Schemata.
\end{bsp}

\begin{defn}
\label{defn:6.13}
	Sei $X$ ein Schema.
	\begin{enumerate}[i)]
		\item Die \textbf{Dimension} von $X$ ist definiert als die Dimension des unterliegenden topologischen Raumes und wird mit $\dim(X)$ bezeichnet. Es gilt also
		\begin{align*}
			\dim(X) = \sup\{&n \in \N \mid \text{ es gibt eine Kette }Z_0 \subsetneq Z_1 \subsetneq \ldots \subsetneq Z_n\\
			&\text{von abgeschlossenen irreduziblen Teilmengen in }X\}.
		\end{align*}
		\item Für $Z \subseteq X$ abgeschlossen und irreduzibel heißt
		\begin{align*}
			\codim(Z,X) \coloneqq \sup\{&n \in \N \mid \text{ es gibt eine Kette } Z = Z_0 \subsetneq Z_1 \subsetneq \ldots \subsetneq Z_n\\
			&\text{von abgeschlossenen irreduziblen Teilmengen in }X\}.
		\end{align*}
		die \textbf{Kodimension} von $Z$ in $X$.
		\item Für $Y$ abgeschlossen in $X$ definieren wir
		\[
			\codim(Y,X) \coloneqq \inf\{\codim(Z,X) \mid Z \text{ abgeschlossen und irreduzibel in } Y\}.
		\]
		Konvention: $\codim(\emptyset,X) = +\infty$.
	\end{enumerate}	
\end{defn}

\begin{bem}
\label{bem:6.14}
	\begin{enumerate}[i)]
		\item Ist $X = \Spec(A)$, so gilt $\dim(X) = \dim(A)$, wobei $\dim(A)$ die Krulldimension von $A$ ist.
		\item Ist $X$ ein affines integres Schema über dem Körper $K$ mit der Eigenschaft, dass der Morphismus $X \to \Spec(K)$ von endlichem Typ ist, dann gilt
		\[
			\dim(Y) + \codim(Y,X) = \dim(X)
		\]
		für alle abgeschlossenen irreduziblen Teilmengen von $X$. Dies folgt aus dem entsprechenden Satz für die Krulldimension (siehe auch \cite[Chapter 5, §14]{matsumura1970commutative}). Für beliebige Schemata ohne gewisse Endlichkeitsvorraussetzungen ist dies im Allgemeinen falsch.
	\end{enumerate}
\end{bem}



