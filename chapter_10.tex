%!TEX root = algebraische_geometrie_2.tex
% vim: tw=0 noet sts=8 sw=8

\chapter{Separierte und eigentliche Morphismen}
\label{chap:10}
Ein topologischer Raum $X$ ist genau dann hausdoffsch, wenn die Diagonale $\Delta \coloneqq \{(x,x) \mid x \in X\}$ abgeschlossen in $X \times X$ ist. Nun ist bekanntlich ein Schema $X$ nicht hausdorffsch, (außer $\dim(X) = 0$) und $X\times X$ ist auch nicht mit der Produkttopologie versehen. Deshalb ist es sinnvoll, die Diagonale zu benutzen, um den Begriff des \textbf{separierten} Morphismus einzuführen. Das ersetzt hausdorffsch in der algebraischen Geometrie.

Weiter werden wir \textbf{eigentliche} Morphismen definieren, die eine ähnliche Rolle spielen, wie die Kompaktheit in der Analysis. Das präzise Analogon in der topologie ist eine stetige Abbildung mit der Eigenschaft, dass die Urbilder kompakter Mengen wieder kompakt sind.

\begin{defn}
\label{defn:10.1} Sei $f \colon X \to Y$ ein Morphismus von Schemata.
	\begin{enumerate}[i)]
		\item\label{defn:10.1:i} Wir benutzen das gefaserte Produkt um den \textbf{Diagonalmorphismus}
		\[
			\Delta\coloneqq \Delta_{X/Y} \coloneqq \Delta_f \colon X \to X \times_Y X
		\]
		zu definieren:
		\begin{center}
			\begin{tikzcd}
				X \arrow[bend left]{drr}{\id_X}\arrow[bend right]{ddr}[swap]{\id_X}\arrow[dashed]{dr}{\exists!\;\Delta}\\
				{} & X \times_Y X \arrow{d}[swap]{p_1}\arrow{r}{p_2} & X \arrow{d}{f}\\
				{} & X \arrow{r}[swap]{f} & Y
			\end{tikzcd}
		\end{center}
		\item\label{defn:10.1:ii} $f$ heißt \textbf{separiert}, wenn $\Delta_f$ eine abgeschlossene Immersion ist.
		\item\label{defn:10.1:iii} $X$ heißt \textbf{separiertes Schema}, wenn der kanonische Morphismus $X \to \Spec(\Z)$ aus Beispiel~\ref{bsp:5.8} separiert ist.
	\end{enumerate}
\end{defn}

\begin{lem}
\label{lem:10.2}
	Sei $f \colon X \to Y$ ein Morphismus von Schemata. Dann ist $f$ genau dann separiert, wenn es eine offene Überdeckung $(V_i)_{i\in I}$ von $Y$ gibt, mit der Eigenschaft, dass $f^{-1}(V_i) \to V_i$ für alle $i \in I$ separiert ist.
	\begin{proof}
		mit der Universellen Eigenschaft def gefaserten Produkts folgt
		\[
			W_i \coloneqq p_1^{-1}(f^{-1}(V_i)) \cap p_2^{-1}(f^{-1}(V_i)) = f^{-1}(V_i) \times_{V_i} f^{-1}(V_i) \subseteq X \times_Y X.
		\]
		Die $W_i$ bilden eine offene Überdeckung der Diagonalen $\Delta(X)$ und es gilt $\Delta_{X/Y}\vert_{W_i} = \Delta_{f^{-1}(V_i)/V_i}$, das heißt, es gilt
		\[
			\Bigg(\Delta^{-1}_{X/Y}(W_i) \overset{\Delta_{X/Y}}{\longto} W_i\Bigg) = \Bigg(f^{-1}(V_i) \overset{\Delta_{f^{-1}(V_i)/V_i}}{\longto} W_i\Bigg).
		\]
		Da man eine abgschlossene Immersion lokal auf dem Bild prüfen kann, folgt die Behauptung. 
	\end{proof}
\end{lem}

\begin{prop}
\label{prop:10.3}
	Jeder Morphismus $f\colon X \to Y$ von affinen Schemata ist separiert.
	\begin{proof}
		Seien $X=\Spec(B)$ und $Y=\Spec(A)$ für gewisse Ring $A$ und $B$. Dann wird $B$ durch $f$ zu einer $A$-Algebra. Die Morphismen $g_1\colon B \to B \otimes_A B,\; b \mapsto b \otimes 1$ und $g_2\colon B \to B \otimes_A B,\; b \mapsto 1 \otimes b$ induzieren die Projektionen $p_1,p_2 \colon X\times_Y X\to X$. Dann induziert $\varphi\colon B\otimes_A B \to B, b \otimes b' \mapsto bb'$ den Diagonalmorphismus $\Delta_f\colon X \to X \times_Y X$. $\varphi$ ist surjektiv, also ist $\Delta_f$ nach Konstruktion~\ref{kons:7.3} eine abgeschlossene Immersion gegeben durch das Ideal $\ker(\varphi)$.
	\end{proof}
\end{prop}
