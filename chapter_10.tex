%!TEX root = algebraische_geometrie_2.tex
% vim: tw=0 noet sts=8 sw=8

%%% Local Variables: 
%%% mode: latex
%%% TeX-master: "algebraische_geometrie_2"
%%% End: 

\chapter{Separierte und eigentliche Morphismen}
\label{chap:10}
Ein top. Raum $X$ ist genau dann hausdoffsch, wenn die Diagonale $\Delta \coloneqq \{(x,x) \mid x \in X\}$ abgeschlossen in $X \times X$ ist. Nun ist bekanntlich ein Schema $X$ nicht hausdorffsch (außer $\dim(X) = 0$) und $X\times X$ ist auch nicht mit der Produkttopologie versehen. Deshalb ist es sinnvoll, die Diagonale zu benutzen, um den Begriff des \textbf{separierten} Morphismus einzuführen. Das ersetzt hausdorffsch in der algebraischen Geometrie.

Weiter werden wir \textbf{eigentliche} Morphismen definieren, die eine ähnliche Rolle spielen, wie die Kompaktheit in der Analysis. Das präzise Analogon in der Topologie ist eine stetige Abbildung mit der Eigenschaft, dass die Urbilder kompakter Mengen wieder kompakt sind.

\nextmark{Diagonalmorphismus, separierter Morphismus, separiertes Schema}
\begin{defn}
\label{defn:10.1} Sei $f \colon X \to Y$ ein Morphismus von Schemata.
	\begin{enumerate}[i)]
		\item\label{defn:10.1:i} Wir benutzen das gefaserte Produkt, um den \textbf{Diagonalmorphismus}
		\[
			\Delta\coloneqq \Delta_{X/Y} \coloneqq \Delta_f \colon X \to X \times_Y X
		\]
		zu definieren:
		\begin{center}
			\begin{tikzcd}
				X \arrow[bend left]{drr}{\id_X}\arrow[bend right]{ddr}[swap]{\id_X}\arrow[dashed]{dr}{\exists!\;\Delta}\\
				{} & X \times_Y X \arrow{d}[swap]{p_1}\arrow{r}{p_2} & X \arrow{d}{f}\\
				{} & X \arrow{r}[swap]{f} & Y
			\end{tikzcd}
		\end{center}
		\item\label{defn:10.1:ii} $f$ heißt \textbf{separiert}, wenn $\Delta_f$ eine abgeschlossene Immersion ist.
		\item\label{defn:10.1:iii} $X$ heißt \textbf{separiertes Schema}, wenn der kanonische Morphismus $X \to \Spec(\Z)$ aus Beispiel~\ref{bsp:5.8} separiert ist.
	\end{enumerate}
\end{defn}

\begin{lem}
\label{lem:10.2}
	Sei $f \colon X \to Y$ ein Morphismus von Schemata. Dann ist $f$ genau dann separiert, wenn es eine offene Überdeckung $(V_i)_{i\in I}$ von $Y$ gibt, mit der Eigenschaft, dass $f^{-1}(V_i) \to V_i$ für alle $i \in I$ separiert ist.
	\begin{proof}
		Mit der universellen Eigenschaft des gefaserten Produkts folgt
		\[
			W_i \coloneqq p_1^{-1}(f^{-1}(V_i)) \cap p_2^{-1}(f^{-1}(V_i)) = f^{-1}(V_i) \times_{V_i} f^{-1}(V_i) \subseteq X \times_Y X.
		\]
		Die $W_i$ bilden eine offene Überdeckung der Diagonalen $\Delta(X)$ und es gilt $\Delta_{X/Y}\vert_{W_i} = \Delta_{f^{-1}(V_i)/V_i}$, das heißt, es gilt
		\[
			\Bigg(\Delta^{-1}_{X/Y}(W_i) \xrightarrow{\Delta_{X/Y}} W_i\Bigg) = \Bigg(f^{-1}(V_i) \xrightarrow{\Delta_{f^{-1}(V_i)/V_i}} W_i\Bigg).
		\]
		Da man die Eigenschaft \enquote{abgschlossene Immersion} lokal auf dem Bild prüfen kann, folgt die Behauptung. 
	\end{proof}
\end{lem}

\pagebreak[2]
\begin{prop}
\label{prop:10.3}
	Jeder Morphismus $f\colon X \to Y$ von affinen Schemata ist separiert.
	\begin{proof}
		Seien $X=\Spec(B)$ und $Y=\Spec(A)$ für gewisse Ring $A$ und $B$. Dann wird $B$ durch $f$ zu einer $A$-Algebra. Die Morphismen $g_1\colon B \to B \otimes_A B,\; b \mapsto b \otimes 1$ und $g_2\colon B \to B \otimes_A B,\; b \mapsto 1 \otimes b$ induzieren die Projektionen $p_1,p_2 \colon X\times_Y X\to X$. Dann induziert $\varphi\colon B\otimes_A B \to B, b \otimes b' \mapsto bb'$ den Diagonalmorphismus $\Delta_f\colon X \to X \times_Y X$. $\varphi$ ist surjektiv, also ist $\Delta_f$ nach Konstruktion~\ref{kons:7.3} eine abgeschlossene Immersion gegeben durch das Ideal $\ker(\varphi)$.
	\end{proof}
\end{prop}

\begin{prop}
  \label{prop:10.4}
  Ein Morphismus $f \colon X \to Y$ ist genau dann separiert, wenn $\Delta(X)$ abgeschlossen ist in $X \times_Y X$.
  \begin{proof}
    \begin{description}
    \item{\glqq$\boldsymbol{\Rightarrow}$\grqq} $f$ separiert $\Rightarrow$ $\Delta$ abgeschlossene Immersion $\Rightarrow$ $\Delta(X)$ abgeschlossen in $X \times_Y X$
    \item{\glqq$\boldsymbol{\Leftarrow}$\grqq} Sei $\Delta(X)$ abgeschlossen in $X \times_Y X$. Wir versehen $\Delta(X) \subseteq X \times_Y X$ mit der induzierten Topologie. Beachte, dass der Morphismus $\Delta$ ein Schnitt von $p_1$ (bzw. $p_2$) ist, dass heißt $p_i \circ \Delta = \id_X$, also ist $\Delta \colon X \to X \times_Y X$ injektiv und surjektiv weil abgeschlossene Immersion.$\Delta$ ist also bijektiv und stetig.
      
      Für $U\subseteq X$ offen gilt $\Delta(U) = \Delta(X) \cap (U \times_Y U)$ offen in $\Delta(X)$, also ist $\Delta \colon X \to \Delta(X)$ eine offene Abbildung und damit ein Homöomorphismus.

      Damit $\Delta$ eine abgeschlossene Immersion it, bleibt die Surjektivität des Garbenhomomorphismus $\Delta^\#\colon \mco_{X \times_Y X} \to \Delta_*  \mco_X$ zu Zeigen. Das kann man lokal auf $\mco_X \times_Y X$ testen. Sei $P \in X \times_Y X$ und $V$ eine affine, offenene Umgebung von $f(P)$ in $Y$, sowie $U$ offene, affine Umgebung von $P$ in $f^{-1}(V)$. $\Rightarrow$ $U \times_Y U$ ist offene affine Umgebung von $\Delta(P)$ in $X \times_Y X$. Nach \ref{prop:10.3} ist $\Delta\colon U \to U \times_Y U$ eine abgeschlossene Immersion, damit ist $\Delta^\#$ surjektiv über $U \times_Y U$ und damit in einer Umgebung von $\Delta(P)$.

      Für $Q \in (X \times_Y X) \setminus \Delta(X)$ müssen wir die Surjektivität von $\Delta^\#$ auch in einer Umgebung testen. Da $(X \times_Y X) \setminus \Delta(X)$ offen ist folgt $(\Delta_* \mco_X)_Q = 0$, dass heißt die Surjektivität ist trivial im Halm über $Q$.
    \end{description}
  \end{proof}
\end{prop}

\begin{defn}
  \label{defn:10.5}
  Sei $f \colon X \to Y$ Morphismus von Schemata.
  \begin{enumerate}[i)]
  \item $f$ heißt \textbf{abgeschlossen} genau dann, wenn für alle $Z \subseteq X$ abgeschlossen gilt, dass $f(Z)\subseteq Y$ abgeschlossen ist.
  \item $f$ heißt \textbf{universell abgeschlossen}, wenn jeder Basiswechsel $f'$ von $f$ abgeschlossen ist.
  \item $f$ heißt \textbf{eigentlich}, wenn $f$ separiert, von endlichem Typ und universell abgeschlossen ist.
  \end{enumerate}
\end{defn}

\begin{bem}
  \label{bem:10.6}
  Sei $f\colon X \to Y$ ein Morphismus von endlichem Typ zwischen noetherschen Schemata,

  Wir sagen, dass $f$ das \textbf{Bewertungskriterium für Separiertheit (bzw. Eigenltichkeit)} erfüllt, falls gilt:
    
  Für jeden diskreten Bewertungsring $R$ mit $K = \Quot(R)$, $i \colon \Spec(K) \to \Spec(R)$ der durch $R \subseteq K$ induziere Morphismus, und alle Morphismen $\Spec(K) \to X, \Spec(R) \to Y$ für die das Diagramm
  \begin{center}
    \begin{tikzcd}
      \Spec(K) \arrow{r} \arrow{d}{i} & X \arrow{d} \\
      \Spec(R) \arrow{r} & Y
    \end{tikzcd}
  \end{center}
  kommutiert, gibt es einen (bzw. genau einen) Morphismus $d \colon \Spec(R) \to X$ so, dass das Diagramm
  \begin{center}
    \begin{tikzcd}
      \Spec(K) \ar{r} \ar{d}{i} & X \ar{d} \\
      \Spec(R) \ar{r} \ar{ru}{d}& Y
    \end{tikzcd}
  \end{center}
  in beiden Dreiecken kommutiert.
\end{bem}

\begin{thm}
  \label{thm:10.7}
  Ein Morphismus $f \colon X \to Y$ von Schemata ist genau dann separiert (bzw. eigentlich), wenn $f$ das Kriterium für Separiertheit (bzw. Eigentlichkeit) erfüllt.
  \begin{proof}
    Siehe z.B. \cite[7.2.3]{grothendieck1971elements}, oder \cite[Thm. 15.9]{goertz2010algebraic} (bzw. \cite[Thm 7.3.8]{grothendieck1971elements})
  \end{proof}
\end{thm}
