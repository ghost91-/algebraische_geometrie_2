\documentclass[a4paper,parskip=half,11pt,ngerman]{scrbook}
\usepackage[utf8]{inputenc}
\usepackage[T1]{fontenc}
\usepackage{babel}
\usepackage[left=2.5cm,right=2.5cm,top=2.5cm,bottom=2.5cm]{geometry}
\usepackage{amsmath}
\usepackage{amsthm}
\usepackage{amssymb}
\usepackage{lmodern}
\usepackage{scrpage2}
\usepackage{hyperref}
\usepackage{mathtools}
\usepackage{tikz-cd}
\usepackage{enumerate}
\usepackage{csquotes}
\hypersetup{pdfinfo={Title={Algebraische Geometrie II}, Author={Johannes Loher}}}
\setkomafont{disposition}{\normalfont\bfseries}

\swapnumbers
\newtheorem{thm}{Theorem}[chapter]
\newtheorem{satz}[thm]{Satz}
\newtheorem{prop}[thm]{Proposition}
\newtheorem{lem}[thm]{Lemma}
\newtheorem{kor}[thm]{Korollar}
\theoremstyle{definition}
\newtheorem{defn}[thm]{Definition}
\newtheorem{bsp}[thm]{Beispiel}
\newtheorem{bem}[thm]{Bemerkung}
\newtheorem{warn}[thm]{Warnung}
\newtheorem*{bem*}{Bemerkung}
\newtheorem*{bsp*}{Beispiel}

\newcommand{\N}{\mathbb {N}}
\newcommand{\Z}{\mathbb {Z}}
\newcommand{\Q}{\mathbb {Q}}
\newcommand{\R}{\mathbb {R}}
\newcommand{\C}{\mathbb {C}}
\newcommand{\K}{\mathbb {K}}
\newcommand{\A}{\mathbb {A}}
\newcommand{\F}{\mathbb {F}}

\newcommand{\Vect}{\operatorname{Vect}}
\newcommand{\Set}{\operatorname{Set}}
\newcommand{\Mod}{\operatorname{Mod}}
\newcommand{\Grp}{\operatorname{Grp}}
\newcommand{\Top}{\operatorname{Top}}
\newcommand{\Ab}{\operatorname{Ab}}
\newcommand{\Nat}{\operatorname{Nat}}
\newcommand{\Op}{\operatorname{Op}}
\newcommand{\PMod}{\operatorname{PMod}}
\newcommand{\Sh}{\operatorname{Sh}}
\newcommand{\PSh}{\operatorname{PSh}}

\renewcommand{\div}{\operatorname{div}}
\newcommand{\laplace}{\Delta}
\newcommand{\grad}{\operatorname{grad}}
\newcommand{\covariant}[1][dt]{\frac{\nabla}{#1}}
\newcommand{\ggT}{\operatorname{ggT}}
\newcommand{\sign}{\operatorname{sign}}
\newcommand{\id}{\operatorname{id}}
\newcommand{\im}{\operatorname{im}}
\newcommand{\coker}{\operatorname{coker}}
\newcommand{\dom}{\operatorname{dom}}
\newcommand{\codom}{\operatorname{codom}}
\newcommand{\rg}{\operatorname{rg}}
\newcommand{\EW}{\operatorname{E}}
\newcommand{\var}{\operatorname{var}}
\newcommand{\cov}{\operatorname{cov}}
\newcommand{\Hom}{\operatorname{Hom}}
\newcommand{\End}{\operatorname{End}}
\newcommand{\Aut}{\operatorname{Aut}}
\newcommand{\Gal}{\operatorname{Gal}}
\newcommand{\Quot}{\operatorname{Quot}}
\newcommand{\Pot}{\mathcal{P}}
\newcommand{\Mat}{\operatorname{M}}
\newcommand{\Hess}{\operatorname{Hess}}

\newcommand{\Tgroup}{\operatorname{T}}
\newcommand{\Ogroup}{\operatorname{O}}
\newcommand{\SOgroup}{\operatorname{SO}}
\newcommand{\Ggroup}{\operatorname{G}}
\newcommand{\SGgroup}{\operatorname{SG}}
\newcommand{\Agroup}{\operatorname{A}}
\newcommand{\SAgroup}{\operatorname{SA}}
\newcommand{\GLgroup}{\operatorname{GL}}
\newcommand{\SLgroup}{\operatorname{SL}}


\newcommand{\abs}[1]{\left \lvert #1 \right \rvert}
\newcommand{\norm}[1]{\left \lVert #1 \right \rVert}
\newcommand{\Char}{\operatorname{char}}
\newcommand{\Mipo}{\operatorname{Mipo}}
\newcommand{\ord}{\operatorname{ord}}
\renewcommand{\Im}{\operatorname{Im}}
\renewcommand{\Re}{\operatorname{Re}}
\newcommand{\dcup}{\mathbin{\dot{\cup}}}

% matrices will take an optional argument for format (like arrays)
\makeatletter
\renewcommand*\env@matrix[1][*\c@MaxMatrixCols c]{%
  \hskip -\arraycolsep
  \let\@ifnextchar\new@ifnextchar
  \array{#1}
}
\makeatother

\newenvironment{smallpmatrix}{\left( \begin{smallmatrix}}{\end{smallmatrix} \right)}
\newenvironment{smallbmatrix}{\left[ \begin{smallmatrix}}{\end{smallmatrix} \right]}
\newenvironment{smallBmatrix}{\left\lbrace \begin{smallmatrix}}{\end{smallmatrix} \right\rbrace}
\newenvironment{smallvmatrix}{\left\vert \begin{smallmatrix}}{\end{smallmatrix} \right\vert}
\newenvironment{smallVmatrix}{\left\Vert \begin{smallmatrix}}{\end{smallmatrix} \right\Vert}

% provide commands to create verical lines, that take up no space
\newcommand{\lvline}[1]{\multicolumn{1}{|c}{#1}}
\newcommand{\rvline}[1]{\multicolumn{1}{c|}{#1}}

\begin{document}
\author{Johannes Loher}
\title{Algebraische Geometrie II}
\subtitle{Inoffizielles Vorlesungsskript zur Vorlesung Algebraische Geometrie II von Prof. Walter Gubler im Sommersemester 2014 an der Universität Regensburg}
\date{\today}
\maketitle
\thispagestyle{empty}
\newpage
\tableofcontents
\newpage

\chapter{Garben}
Garben sind abstrakte Verallgemeinerungen von Funktionenräumen. Sie sind fundamental für das Studium von Mannigfaltigkeiten und Schemata.

$X$ sei ein topologischer Raum.

\begin{defn}
\label{defn:1.1}
	Eine \textbf{Prägarbe} $\mathcal{F}$ (von abelschen Gruppen) auf $X$ besteht aus folgenden Daten:
	\begin{enumerate}[a)]
		\item Für alle $U$ offen in $X$ sein $\mathcal{F}(U)$ eine abelsche Gruppe.
		\item Für alle $V \subseteq U$ offen in $X$ sei $\rho_{UV}:\mathcal{F}(U)\to\mathcal{F}(V)$ ein Homomorphismus.
		\item Es sei $\mathcal{F}(\emptyset) = 0$.
		\item Für alle $U$ offen in $X$ sei $\rho_{UU} = \id_{\mathcal{F}(U)}$.
		\item Für alle $W \subseteq V \subseteq U$ offen in $X$ sei $\rho_{UW} = \rho_{VW} \circ \rho_{UV}$.
	\end{enumerate}
	Die Elemente von $\mathcal{F}(U)$ heißen \textbf{Schnitte} von $\mathcal{F}$ über $U$. Der Homomorphismus $\rho_{UV}:\mathcal{F}(U)\to\mathcal{F}(V)$ heißt \textbf{Restriktionsabbildung} von $U$ auf die offene Teilmenge $V$ von $U$
\end{defn}

\begin{bem}
	Analog definiert man Prägarben von Ringen, Algebren oder Mengen, ...
\end{bem}

\begin{defn}
	Eine Prägarbe $\mathcal{F}$ auf $X$ heißt \textbf{Garbe}, falls zusätzlich für jede offene Menge $U$ in $X$ und jede offene Überdeckung $U=\bigcup_{i \in I}V_i$ von $U$ folgendes gilt:
	\begin{enumerate}[a)]
		\setcounter{enumi}{5}
		\item Ist $s \in \mathcal{F}(U)$ und $\rho_{UV_i}(s) = 0$ für alle $i \in I$, so gilt bereits $s=0$.
		\item Sind $s_i \in \mathcal{F}(V_i)$ für alle $i \in I$ mit $\rho_{V_i V_i \cap V_J}(s_i) = \rho_{V_j V_i \cap V_J}(s_j)$ für alle $i,j \in I$, so gibt es ein $s \in \mathcal{F}(U)$ mit $\rho_{UV_i}(s) = s_i$ für alle $i \in I$.
	\end{enumerate}
	Eine Garbe ist also duch lokale Informationen vollständig bestimmt.
\end{defn}

\begin{bem}
	Nach f) ist der Schnitt $s$ in g) eindeutig bestimmt.
	\begin{proof}
		Sind $s, s'$ zwei solche Schnitte in g), so gilt:
		\[
			\rho_{UV_i}(s-s') = \rho_{UV_i}(s) - \rho_{UV_i}(s') \overset{\text{g)}}{=}s_i-s_i = 0 
		\]
		Also gilt nach f) schon $s-s'=0$ und damit $s=s'$.
	\end{proof}
\end{bem}

\begin{bsp}
\label{bsp:1.5}
	Sei $x$ ein topologischer Raum. Für $U$ offen in $X$ sei
	\[
		\mathcal{F}(U)\coloneqq \{f:U\to \R \mid f \text{ ist Funktion}\}.
	\]
	Dies ist eine Garbe (abelscher Gruppen und sogar $\R$-Algebren) und die Restirktionsabbildungen sind gegeben durch:
	\[
		\rho_{UV}:\mathcal{F}(U) \to \mathcal{F}(V),\; f \mapsto f\vert_V
	\]
	Die Menge der stetigen Funktionen $C(U)$ liefert eine \textbf{Untergarbe} $\mathcal{F}'$ von $\mathcal{F}$ ($\mathcal{F}'(U)\coloneqq C(U)$), das heißt $\mathcal{F}'$ ist auch eine Garbe, es gilt $\mathcal{F}'(U)\subseteq \mathcal{F}(U)$ für alle $U$ offen und für die Restriktionen ist folgendes Diagramm kommutativ:
	\begin{center}
		\begin{tikzcd}
			\mathcal{F}'(U) \arrow[hook]{r} \arrow{d}[swap]{\rho_{UV}} & \mathcal{F}(U)\arrow{d}{\rho_{UV}}\\
			\mathcal{F}'(V) \arrow[hook]{r} & \mathcal{F}(V)
		\end{tikzcd}
	\end{center}

\end{bsp}

\begin{bsp}
	Sein $A$ eine fixierte abelsche Gruppe. Die zugehörige \textbf{konstante Prägarbe} $\mathcal{F}$ ist definiert durch:
	\begin{itemize}
		\item Es sei $\mathcal{F}(U) \coloneqq \left\lbrace \begin{array}{c c}A & \text{falls } U \neq \emptyset\\0& \text{falls } U = \emptyset\end{array}\right.$.
		\item Es sei $\rho_{UV} \coloneqq \left\lbrace \begin{array}{c c}\id_A & \text{falls } V \neq \emptyset\\0& \text{falls } V = \emptyset\end{array}\right.$.
	\end{itemize}
	Falls $X$ nicht zusammenhängend und $A \neq 0$ ist, dann ist $\mathcal{F}$ keine Garbe. Sei zum Beispiel $X=\{p,q\}$ mit der diskreten Topologie. Seien außerdem $U=\{p,q\}$, $V_1=\{p\}$, $V_2=\{q\}$. Dann gilt $\mathcal{F}(U)=\mathcal{F}(V_1)=\mathcal{F}(V_2) = A$ und $\mathcal{F}(\emptyset)=0$. Seien nun $s_1 \neq 0 \in \mathcal{F}(V_1)$ und $s_2 = 0 \in \mathcal{F}(V_2)$. Wäre $\mathcal{F}$ eine Garbe, dann gäbe es ein $s \in \mathcal{F}(U) = A$ mit $\rho_{UV_i}(s) = s_i$. Dies ist aber offenbar nicht der Fall.
\end{bsp}

\begin{bsp}
\label{bsp:1.7}
	Seien $X,S$ topologische Räume und $\pi:S\to X$ eine Abbildung mit folgenden Eigenschaften:
	\begin{enumerate}[a)]
		\setcounter{enumi}{8}
		\item $\pi$ ist surjektiv und ein lokaler Homöomorphismus.
		\item $\pi^{-1}(x)$ ist für alle $x\in X$ eine abelsche Gruppe.
		\item Sei $S\times_X S\coloneqq \{(s_1,s_2) \in S \times S \mid \pi(s_1)=\pi(s_2)\}$ das \textbf{Faserprodukt} über $X$ mit der von $S \times S$ induzierten Topologie. Dann induzieren die Addition und Invertierung aus j) stetige Abbildungen $S\times_X S \to S,\; (s_1,s_2) \mapsto s_1+s_2$, beziehungsweise $S \to S,\; s \mapsto -s$.
	\end{enumerate}
	Die Abbildung $\pi$ heißt \textbf{Projektion} und $\pi^{-1}(x)$ heißt \textbf{Faser} von $x$.
	\begin{itemize}
		\item Sei $U \subseteq X$ offen. Eine stetige Funktion $f:U \to S$ heißt \textbf{Schnitt}, wenn $\pi \circ f = \id_U$, das heißt für alle $x \in U$ gilt $f(x) \in \pi^{-1}(x)$.
		\item Es gibt einen kanonischen globalen Schnitt $X \to S, \; x \mapsto 0_x \in f^{-1}(x)$, den wir \textbf{Nullschnitt} nennen.
		\item Mit $\Gamma(U,S)$ bezeichnen wir den Raum der Schnitte von $S$ über $U$, wir setzen also
		\[
			\Gamma(U,S) \coloneqq \{d:U\to S \mid f \text{ stetig},\; \pi\circ f = \id_U\}.
		\]
	\end{itemize}
	\textbf{Behauptung:} Die Abbildung $U \mapsto \Gamma(U,S)$ zusammen mit der Restriktion $\rho_{UV}(f) \coloneqq f\vert_V$ ist eine Garbe.
	\begin{proof}
		Dies ist klar.
	\end{proof}
\end{bsp}

\begin{bem}
	Weil $\pi$ ein lokaler Homöomorphismus ist, muss jeder Schnitt eine offene Abbildung sein (das heißt das Bild einer offenen Teilmenge ist offen). Falls zwei Schnitte in $x \in X$ übereinstimmen, dann stimmen sie auch auf einer Umgebung von $x$ überein.
\end{bem}

\begin{bsp*}
	Die Abbildung $\R \to \{z\in \C \mid \abs{z} = 1\},\; t \mapsto e^{2 \pi i t}$ ist eine Abbildung wie in Beispiel~\ref{bsp:1.7}.
\end{bsp*}

\begin{bem}
	Das Beispiel~\ref{bsp:1.7} erklärt die abstrakten Begriffe aus Definition~\ref{defn:1.1}. Wir werden sehen, dass jede Garbe durch einen topologischen Raum $S$ und eine Abbildung $\pi:S \to X$ wie in Beispiel~\ref{bsp:1.7} dargestellt werden kann.
\end{bem}

\begin{bsp}
	In Beispiel~\ref{bsp:1.5} wählen wir ein $x \in X$. Wir definieren $f\sim g$, falls es eine Umgebung $U$ von $x$ gibt mit $f\vert_U = g\vert_U$. Dies liefert eine Äquivalenzrelation auf der Menge der reellwertigen Funktionen, die auf einer Umgebung von $x$ definiert sind. Der \textbf{Halm} $\mathcal{F}_x$ ist definiert als Raum der Äquivalenzklassen bezüglich dieser Äquivalenzrelation.
\end{bsp}

\begin{defn}
	Sei $\mathcal{F}$ eine Prägarbe auf $X$, dann verallgemeinern wir die obige Konstruktion. Sei $x \in X$. Wir betrachten die Menge $\{(U,s)\mid s \in \mathcal{F}(U),\; U \text{ offene Umgebung von } x\}$. Wir definieren auf dieser Menge eine Relation auf folgende Weise: Es gelte	$(U,s) \sim (V,t)$ genau dann, wenn es eine offene Umgebung $W\subseteq U \cap V$ von $x$ mit $\rho_{UW}(s) = \rho_{VW}(t)$ gibt. Man zeigt leicht, dass dies eine Äquivalenzrelation ist. Der \textbf{Halm} $\mathcal{F}_x$ ist definiert als der Raum der Äquivalenzklassen bezüglich dieser Äquivalenzrelation. Dies ist eine abelsche Gruppe:
	\[
		[(U_1,s_1)] + [(U_2,s_2)] = [(U_1 \cap U_2, \rho_{U_1U_1\cap U_2}(s_1) + \rho_{U_2U_1\cap U_2}(s_2)].
	\]
	Wir schreiben auch $s_x$ anstatt von $[(U,s)] \in \mathcal{F}_x$.
\end{defn}

\begin{defn}
	Ein \textbf{Homomorphismus} $\varphi: \mathcal{F} \to \mathcal{G}$ \textbf{von (Prä-)Garben} auf $x$ ist eine Familie von Homomorphismen $\varphi_U:\mathcal{F}(U) \to \mathcal{G}(U)$ abelscher Gruppen für alle $U$ offen in $X$, sodass
	\begin{center}
		\begin{tikzcd}
			\mathcal{F}(U) \arrow{r}{\varphi_U} \arrow{d}[swap]{\rho_{UV}} & \mathcal{G}(U)\arrow{d}{\rho_{UV}}\\
			\mathcal{F}(V) \arrow{r}[swap]{\varphi_V} & \mathcal{G}(V)
		\end{tikzcd}
	\end{center}
	für alle $V \subseteq U$ offen in $X$ kommutiert.

	Wir können Homomorphismen $\varphi:\mathcal{F}\to \mathcal{G}$ und $\psi:\mathcal{G} \to \mathcal{H}$ von (Prä-)Garben zu einem Homomorphismus $\psi\circ \varphi:\mathcal{F} \to \mathcal{H}$ verknüpfen. Damit können wir auch Isomorphismen von (Prä-)Garben definieren.

	Die (Prä-)Garben bilden eine Kategorie.
\end{defn}

\begin{prop}
	Sei $\varphi: \mathcal{F} \to \mathcal{G}$ ein Homomorphismus von Garben auf $X$. Dann ist $\varphi$ genau dann ein Isomorphismus von Garben, wenn $\varphi_x:\mathcal{F}_x \to \mathcal{G}_x,\; s_x \mapsto \varphi_x(s_x) \coloneqq [(U,\varphi\vert_U(s))]$ für alle $x \in X$ ein Isomorphismus abelscher Gruppen ist.
	\begin{proof}
		Dies ist eine einfache Übung.
	\end{proof}
	Beachte, dass diese Aussage nicht für Prägarben gilt.
\end{prop}

\end{document}