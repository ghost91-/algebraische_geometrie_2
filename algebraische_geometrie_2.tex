\documentclass[a4paper,parskip=half,11pt,ngerman,oneside]{scrbook}
\usepackage[utf8]{inputenc}
\usepackage[T1]{fontenc}
\usepackage{babel}
\usepackage[left=2.5cm,right=2.5cm,top=2.5cm,bottom=2.5cm]{geometry}
\usepackage{amsmath}
\usepackage{amsthm}
\usepackage{amssymb}
\usepackage{lmodern}
\usepackage{scrpage2}
\usepackage{hyperref}
\usepackage{mathtools}
\usepackage{tikz-cd}
\usepackage{enumerate}
\usepackage{csquotes}
\hypersetup{pdfinfo={Title={Algebraische Geometrie II}, Author={Johannes Loher}},colorlinks=false, pdfborder={0 0 0}}
\setkomafont{disposition}{\normalfont\bfseries}

\swapnumbers
\newtheorem{thm}{Theorem}[chapter]
\newtheorem{satz}[thm]{Satz}
\newtheorem{prop}[thm]{Proposition}
\newtheorem{lem}[thm]{Lemma}
\newtheorem{kor}[thm]{Korollar}
\theoremstyle{definition}
\newtheorem{defn}[thm]{Definition}
\newtheorem{bsp}[thm]{Beispiel}
\newtheorem{bem}[thm]{Bemerkung}
\newtheorem{kons}[thm]{Konstruktion}
\newtheorem{warn}[thm]{Warnung}
\newtheorem{eri}[thm]{Erinnerung}
\newtheorem*{bem*}{Bemerkung}
\newtheorem*{bsp*}{Beispiel}

\newcommand{\N}{\mathbb {N}}
\newcommand{\Z}{\mathbb {Z}}
\newcommand{\Q}{\mathbb {Q}}
\newcommand{\R}{\mathbb {R}}
\newcommand{\C}{\mathbb {C}}
\newcommand{\K}{\mathbb {K}}
\newcommand{\A}{\mathbb {A}}
\newcommand{\F}{\mathbb {F}}

\newcommand{\mcf}{\mathcal{F}}
\newcommand{\mcg}{\mathcal{G}}
\newcommand{\mcH}{\mathcal{H}}
\newcommand{\mco}{\mathcal{O}}
\newcommand{\mfa}{\mathfrak{a}}
\newcommand{\mfb}{\mathfrak{b}}
\newcommand{\mfm}{\mathfrak{m}}
\newcommand{\mfp}{\mathfrak{p}}
\newcommand{\mfq}{\mathfrak{q}}

\newcommand{\Vect}{\operatorname{Vect}}
\newcommand{\Set}{\operatorname{Set}}
\newcommand{\Mod}{\operatorname{Mod}}
\newcommand{\Grp}{\operatorname{Grp}}
\newcommand{\Top}{\operatorname{Top}}
\newcommand{\Ab}{\operatorname{Ab}}
\newcommand{\Nat}{\operatorname{Nat}}
\newcommand{\Op}{\operatorname{Op}}
\newcommand{\PMod}{\operatorname{PMod}}
\newcommand{\Sh}{\operatorname{Sh}}
\newcommand{\PSh}{\operatorname{PSh}}

\renewcommand{\div}{\operatorname{div}}
\newcommand{\laplace}{\Delta}
\newcommand{\grad}{\operatorname{grad}}
\newcommand{\covariant}[1][dt]{\frac{\nabla}{#1}}
\newcommand{\ggT}{\operatorname{ggT}}
\newcommand{\sign}{\operatorname{sign}}
\newcommand{\id}{\operatorname{id}}
\newcommand{\im}{\operatorname{im}}
\newcommand{\coker}{\operatorname{coker}}
\newcommand{\dom}{\operatorname{dom}}
\newcommand{\codom}{\operatorname{codom}}
\newcommand{\rg}{\operatorname{rg}}
\newcommand{\EW}{\operatorname{E}}
\newcommand{\var}{\operatorname{var}}
\newcommand{\cov}{\operatorname{cov}}
\newcommand{\Hom}{\operatorname{Hom}}
\newcommand{\End}{\operatorname{End}}
\newcommand{\Aut}{\operatorname{Aut}}
\newcommand{\Gal}{\operatorname{Gal}}
\newcommand{\Quot}{\operatorname{Quot}}
\newcommand{\Pot}{\mathcal{P}}
\newcommand{\Mat}{\operatorname{M}}
\newcommand{\Hess}{\operatorname{Hess}}
\newcommand{\Tor}{\operatorname{Tor}}
\newcommand{\Spec}{\operatorname{Spec}}
\newcommand{\Ann}{\operatorname{Ann}}

\newcommand{\Tgroup}{\operatorname{T}}
\newcommand{\Ogroup}{\operatorname{O}}
\newcommand{\SOgroup}{\operatorname{SO}}
\newcommand{\Ggroup}{\operatorname{G}}
\newcommand{\SGgroup}{\operatorname{SG}}
\newcommand{\Agroup}{\operatorname{A}}
\newcommand{\SAgroup}{\operatorname{SA}}
\newcommand{\GLgroup}{\operatorname{GL}}
\newcommand{\SLgroup}{\operatorname{SL}}


\newcommand{\abs}[1]{\left \lvert #1 \right \rvert}
\newcommand{\norm}[1]{\left \lVert #1 \right \rVert}
\newcommand{\Char}{\operatorname{char}}
\newcommand{\Mipo}{\operatorname{Mipo}}
\newcommand{\ord}{\operatorname{ord}}
\renewcommand{\Im}{\operatorname{Im}}
\renewcommand{\Re}{\operatorname{Re}}
\newcommand{\dcup}{\mathbin{\dot{\cup}}}
\newcommand{\longto}{\longrightarrow}
\newcommand{\substackclap}[1]{\mathclap{\substack{#1}}}

% matrices will take an optional argument for format (like arrays)
\makeatletter
\renewcommand*\env@matrix[1][*\c@MaxMatrixCols c]{%
  \hskip -\arraycolsep
  \let\@ifnextchar\new@ifnextchar
  \array{#1}
}
\makeatother

\newenvironment{smallpmatrix}{\left( \begin{smallmatrix}}{\end{smallmatrix} \right)}
\newenvironment{smallbmatrix}{\left[ \begin{smallmatrix}}{\end{smallmatrix} \right]}
\newenvironment{smallBmatrix}{\left\lbrace \begin{smallmatrix}}{\end{smallmatrix} \right\rbrace}
\newenvironment{smallvmatrix}{\left\vert \begin{smallmatrix}}{\end{smallmatrix} \right\vert}
\newenvironment{smallVmatrix}{\left\Vert \begin{smallmatrix}}{\end{smallmatrix} \right\Vert}

% provide commands to create verical lines, that take up no space
\newcommand{\lvline}[1]{\multicolumn{1}{|c}{#1}}
\newcommand{\rvline}[1]{\multicolumn{1}{c|}{#1}}


\begin{document}
\author{Johannes Loher}
\title{Algebraische Geometrie II}
\subtitle{Inoffizielles Vorlesungsskript zur Vorlesung Algebraische Geometrie II von Prof. Walter Gubler im Sommersemester 2014 an der Universität Regensburg}
\date{\today}
\maketitle
\thispagestyle{empty}
\tableofcontents
\thispagestyle{empty}

\frontmatter
%!TEX root = algebraische_geometrie_2.tex
% vim: tw=0

\chapter{Einleitung}
\begin{itemize}
	\item Die klassische algebraische Geometrie ist für Varietäten über algebraisch abgeschlossenen Körpern. Die Koordinatenringe sind dann immer reduzierte Algebren. In der algebraischen Schnitttheorie muss man aber nicht-reduzierte Algebren betrachten (\enquote{Multiplizitäten}).
	\item In der Zahlentheorie wird man gezwungen, über Zahlenkörpern zu arbeiten. Dies sind endliche Körpererweiterungen von $\Q$, also nicht algebraisch abgeschlossen.
	\item Viele Klassifikationsprobleme führen auf Modulräume, die keine Varietäten sind.
\end{itemize}
Um diese Probleme zu lösen, hat Alexander Grothendieck zu Beginn der 60er Jahre die Theorie der Schemata eingeführt (EGA I--IV). Dies ist eine relative Theorie, das heißt es wird kein Grundkörpervorausgesetzt und die Koordinatenringe sind beliebige kommutative Ringe. Das heißt, man kann (beziehungsweise muss) die Methoden der kommutativen Algebra für die Beweise nutzen.

\textbf{Erfolge:} Weil-Vermutung (Deligne 70er), Fields-Medaillen, Schemata haben sich als Standard in der algebraischen und arithmetischen Geometrie durchgesetzt.

In dieser Vorlesung seien Ringe und Algebren immer kommutativ und mit Einselement, falls nichts anderes gesagt wird.

\mainmatter
\chapter{Garben}
Garben sind abstrakte Verallgemeinerungen von Funktionenräumen. Sie sind fundamental für das Studium von Mannigfaltigkeiten und Schemata.

$X$ sei ein topologischer Raum.

\begin{defn}
\label{defn:1.1}
	Eine \textbf{Prägarbe} $\mathcal{F}$ (von abelschen Gruppen) auf $X$ besteht aus folgenden Daten:
	\begin{enumerate}[a)]
		\item Für alle $U$ offen in $X$ sein $\mathcal{F}(U)$ eine abelsche Gruppe.
		\item Für alle $V \subseteq U$ offen in $X$ sei $\rho_{UV}:\mathcal{F}(U)\to\mathcal{F}(V)$ ein Homomorphismus.
		\item Es sei $\mathcal{F}(\emptyset) = 0$.
		\item Für alle $U$ offen in $X$ sei $\rho_{UU} = \id_{\mathcal{F}(U)}$.
		\item Für alle $W \subseteq V \subseteq U$ offen in $X$ sei $\rho_{UW} = \rho_{VW} \circ \rho_{UV}$.
	\end{enumerate}
	Die Elemente von $\mathcal{F}(U)$ heißen \textbf{Schnitte} von $\mathcal{F}$ über $U$. Der Homomorphismus $\rho_{UV}:\mathcal{F}(U)\to\mathcal{F}(V)$ heißt \textbf{Restriktionsabbildung} von $U$ auf die offene Teilmenge $V$ von $U$
\end{defn}

\begin{bem}
	Analog definiert man Prägarben von Ringen, Algebren oder Mengen, ...
\end{bem}

\begin{defn}
	Eine Prägarbe $\mathcal{F}$ auf $X$ heißt \textbf{Garbe}, falls zusätzlich für jede offene Menge $U$ in $X$ und jede offene Überdeckung $U=\bigcup_{i \in I}V_i$ von $U$ folgendes gilt:
	\begin{enumerate}[a)]
		\setcounter{enumi}{5}
		\item Ist $s \in \mathcal{F}(U)$ und $\rho_{UV_i}(s) = 0$ für alle $i \in I$, so gilt bereits $s=0$.
		\item Sind $s_i \in \mathcal{F}(V_i)$ für alle $i \in I$ mit $\rho_{V_i V_i \cap V_J}(s_i) = \rho_{V_j V_i \cap V_J}(s_j)$ für alle $i,j \in I$, so gibt es ein $s \in \mathcal{F}(U)$ mit $\rho_{UV_i}(s) = s_i$ für alle $i \in I$.
	\end{enumerate}
	Eine Garbe ist also duch lokale Informationen vollständig bestimmt.
\end{defn}

\begin{bem}
	Nach f) ist der Schnitt $s$ in g) eindeutig bestimmt.
	\begin{proof}
		Sind $s, s'$ zwei solche Schnitte in g), so gilt:
		\[
			\rho_{UV_i}(s-s') = \rho_{UV_i}(s) - \rho_{UV_i}(s') \overset{\text{g)}}{=}s_i-s_i = 0 
		\]
		Also gilt nach f) schon $s-s'=0$ und damit $s=s'$.
	\end{proof}
\end{bem}

\begin{bsp}
\label{bsp:1.5}
	Sei $x$ ein topologischer Raum. Für $U$ offen in $X$ sei
	\[
		\mathcal{F}(U)\coloneqq \{f:U\to \R \mid f \text{ ist Funktion}\}.
	\]
	Dies ist eine Garbe (abelscher Gruppen und sogar $\R$-Algebren) und die Restirktionsabbildungen sind gegeben durch:
	\[
		\rho_{UV}:\mathcal{F}(U) \to \mathcal{F}(V),\; f \mapsto f\vert_V
	\]
	Die Menge der stetigen Funktionen $C(U)$ liefert eine \textbf{Untergarbe} $\mathcal{F}'$ von $\mathcal{F}$ ($\mathcal{F}'(U)\coloneqq C(U)$), das heißt $\mathcal{F}'$ ist auch eine Garbe, es gilt $\mathcal{F}'(U)\subseteq \mathcal{F}(U)$ für alle $U$ offen und für die Restriktionen ist folgendes Diagramm kommutativ:
	\begin{center}
		\begin{tikzcd}
			\mathcal{F}'(U) \arrow[hook]{r} \arrow{d}[swap]{\rho_{UV}} & \mathcal{F}(U)\arrow{d}{\rho_{UV}}\\
			\mathcal{F}'(V) \arrow[hook]{r} & \mathcal{F}(V)
		\end{tikzcd}
	\end{center}

\end{bsp}

\begin{bsp}
	Sein $A$ eine fixierte abelsche Gruppe. Die zugehörige \textbf{konstante Prägarbe} $\mathcal{F}$ ist definiert durch:
	\begin{itemize}
		\item Es sei $\mathcal{F}(U) \coloneqq \left\lbrace \begin{array}{c c}A & \text{falls } U \neq \emptyset\\0& \text{falls } U = \emptyset\end{array}\right.$.
		\item Es sei $\rho_{UV} \coloneqq \left\lbrace \begin{array}{c c}\id_A & \text{falls } V \neq \emptyset\\0& \text{falls } V = \emptyset\end{array}\right.$.
	\end{itemize}
	Falls $X$ nicht zusammenhängend und $A \neq 0$ ist, dann ist $\mathcal{F}$ keine Garbe. Sei zum Beispiel $X=\{p,q\}$ mit der diskreten Topologie. Seien außerdem $U=\{p,q\}$, $V_1=\{p\}$, $V_2=\{q\}$. Dann gilt $\mathcal{F}(U)=\mathcal{F}(V_1)=\mathcal{F}(V_2) = A$ und $\mathcal{F}(\emptyset)=0$. Seien nun $s_1 \neq 0 \in \mathcal{F}(V_1)$ und $s_2 = 0 \in \mathcal{F}(V_2)$. Wäre $\mathcal{F}$ eine Garbe, dann gäbe es ein $s \in \mathcal{F}(U) = A$ mit $\rho_{UV_i}(s) = s_i$. Dies ist aber offenbar nicht der Fall.
\end{bsp}

\begin{bsp}
\label{bsp:1.7}
	Seien $X,S$ topologische Räume und $\pi:S\to X$ eine Abbildung mit folgenden Eigenschaften:
	\begin{enumerate}[a)]
		\setcounter{enumi}{8}
		\item $\pi$ ist surjektiv und ein lokaler Homöomorphismus.
		\item $\pi^{-1}(x)$ ist für alle $x\in X$ eine abelsche Gruppe.
		\item Sei $S\times_X S\coloneqq \{(s_1,s_2) \in S \times S \mid \pi(s_1)=\pi(s_2)\}$ das \textbf{Faserprodukt} über $X$ mit der von $S \times S$ induzierten Topologie. Dann induzieren die Addition und Invertierung aus j) stetige Abbildungen $S\times_X S \to S,\; (s_1,s_2) \mapsto s_1+s_2$, beziehungsweise $S \to S,\; s \mapsto -s$.
	\end{enumerate}
	Die Abbildung $\pi$ heißt \textbf{Projektion} und $\pi^{-1}(x)$ heißt \textbf{Faser} von $x$.
	\begin{itemize}
		\item Sei $U \subseteq X$ offen. Eine stetige Funktion $f:U \to S$ heißt \textbf{Schnitt}, wenn $\pi \circ f = \id_U$, das heißt für alle $x \in U$ gilt $f(x) \in \pi^{-1}(x)$.
		\item Es gibt einen kanonischen globalen Schnitt $X \to S, \; x \mapsto 0_x \in f^{-1}(x)$, den wir \textbf{Nullschnitt} nennen.
		\item Mit $\Gamma(U,S)$ bezeichnen wir den Raum der Schnitte von $S$ über $U$, wir setzen also
		\[
			\Gamma(U,S) \coloneqq \{d:U\to S \mid f \text{ stetig},\; \pi\circ f = \id_U\}.
		\]
	\end{itemize}
	\textbf{Behauptung:} Die Abbildung $U \mapsto \Gamma(U,S)$ zusammen mit der Restriktion $\rho_{UV}(f) \coloneqq f\vert_V$ ist eine Garbe.
	\begin{proof}
		Dies ist klar.
	\end{proof}
\end{bsp}

\begin{bem}
	Weil $\pi$ ein lokaler Homöomorphismus ist, muss jeder Schnitt eine offene Abbildung sein (das heißt das Bild einer offenen Teilmenge ist offen). Falls zwei Schnitte in $x \in X$ übereinstimmen, dann stimmen sie auch auf einer Umgebung von $x$ überein.
\end{bem}

\begin{bsp*}
	Die Abbildung $\R \to \{z\in \C \mid \abs{z} = 1\},\; t \mapsto e^{2 \pi i t}$ ist eine Abbildung wie in Beispiel~\ref{bsp:1.7}.
\end{bsp*}

\begin{bem}
	Das Beispiel~\ref{bsp:1.7} erklärt die abstrakten Begriffe aus Definition~\ref{defn:1.1}. Wir werden sehen, dass jede Garbe durch einen topologischen Raum $S$ und eine Abbildung $\pi:S \to X$ wie in Beispiel~\ref{bsp:1.7} dargestellt werden kann.
\end{bem}

\begin{bsp}
	In Beispiel~\ref{bsp:1.5} wählen wir ein $x \in X$. Wir definieren $f\sim g$, falls es eine Umgebung $U$ von $x$ gibt mit $f\vert_U = g\vert_U$. Dies liefert eine Äquivalenzrelation auf der Menge der reellwertigen Funktionen, die auf einer Umgebung von $x$ definiert sind. Der \textbf{Halm} $\mathcal{F}_x$ ist definiert als Raum der Äquivalenzklassen bezüglich dieser Äquivalenzrelation.
\end{bsp}

\begin{defn}
	Sei $\mathcal{F}$ eine Prägarbe auf $X$, dann verallgemeinern wir die obige Konstruktion. Sei $x \in X$. Wir betrachten die Menge $\{(U,s)\mid s \in \mathcal{F}(U),\; U \text{ offene Umgebung von } x\}$. Wir definieren auf dieser Menge eine Relation auf folgende Weise: Es gelte	$(U,s) \sim (V,t)$ genau dann, wenn es eine offene Umgebung $W\subseteq U \cap V$ von $x$ mit $\rho_{UW}(s) = \rho_{VW}(t)$ gibt. Man zeigt leicht, dass dies eine Äquivalenzrelation ist. Der \textbf{Halm} $\mathcal{F}_x$ ist definiert als der Raum der Äquivalenzklassen bezüglich dieser Äquivalenzrelation. Dies ist eine abelsche Gruppe:
	\[
		[(U_1,s_1)] + [(U_2,s_2)] = [(U_1 \cap U_2, \rho_{U_1U_1\cap U_2}(s_1) + \rho_{U_2U_1\cap U_2}(s_2)].
	\]
	Wir schreiben auch $s_x$ anstatt von $[(U,s)] \in \mathcal{F}_x$.
\end{defn}

\begin{defn}
	Ein \textbf{Homomorphismus} $\varphi: \mathcal{F} \to \mathcal{G}$ \textbf{von (Prä-)Garben} auf $x$ ist eine Familie von Homomorphismen $\varphi_U:\mathcal{F}(U) \to \mathcal{G}(U)$ abelscher Gruppen für alle $U$ offen in $X$, sodass
	\begin{center}
		\begin{tikzcd}
			\mathcal{F}(U) \arrow{r}{\varphi_U} \arrow{d}[swap]{\rho_{UV}} & \mathcal{G}(U)\arrow{d}{\rho_{UV}}\\
			\mathcal{F}(V) \arrow{r}[swap]{\varphi_V} & \mathcal{G}(V)
		\end{tikzcd}
	\end{center}
	für alle $V \subseteq U$ offen in $X$ kommutiert.

	Wir können Homomorphismen $\varphi:\mathcal{F}\to \mathcal{G}$ und $\psi:\mathcal{G} \to \mathcal{H}$ von (Prä-)Garben zu einem Homomorphismus $\psi\circ \varphi:\mathcal{F} \to \mathcal{H}$ verknüpfen. Damit können wir auch Isomorphismen von (Prä-)Garben definieren.

	Die (Prä-)Garben bilden eine Kategorie.
\end{defn}

\begin{prop}
	Sei $\varphi: \mathcal{F} \to \mathcal{G}$ ein Homomorphismus von Garben auf $X$. Dann ist $\varphi$ genau dann ein Isomorphismus von Garben, wenn $\varphi_x:\mathcal{F}_x \to \mathcal{G}_x,\; s_x \mapsto \varphi_x(s_x) \coloneqq [(U,\varphi\vert_U(s))]$ für alle $x \in X$ ein Isomorphismus abelscher Gruppen ist.
	\begin{proof}
		Dies ist eine einfache Übung.
	\end{proof}
	Beachte, dass diese Aussage nicht für Prägarben gilt.
\end{prop}
%!TEX root = algebraische_geometrie_2.tex
% vim: tw=0 noet sts=8 sw=8

\chapter{Lokal geringte Räume}

Mit dem Konzept der lokal geringten Räume kann man die Mannigfaltigkeiten aus der Analysis und Differentialgeometrie, die algebraische Varietäten aus der algebraischen Geometrie I und die Schemata aus der algebraischen Geometrie II zusammenfassen.

\begin{bem}
	Im Folgenden soll, wenn nichts anderes gesagt wird, Folgendes gelten:
	\begin{itemize}
	 	\item Alle Ringe sind kommutativ mit Eins.
	 	\item Ringhomomorphismen bilden die Eins auf die Eins ab.
	 \end{itemize}
\end{bem}

\begin{bem*}
	Ein Ring $R$ heißt \textbf{lokal}, wenn es in $R$ genau ein Maximalideal $\mathfrak{m}$ gibt.
\end{bem*}

\begin{prop}
\label{prop:2.2}
	Sei $R$ ein Ring. $R$ ist genau dann ein lokaler Ring, wenn es ein Ideal $I \neq R$ mit $R\setminus I = R^\times$ gibt.
	\begin{proof}
		Dies ist eine einfache Eigenschaft aus der kommutativen Algebra.
	\end{proof}
\end{prop}

\nextmark{lokaler Ringhomomorphismus}
\begin{defn}
	Ein Homomorphismus $\varphi\colon R_1 \to R_2$ von Ringen heißt \textbf{lokal}, wenn
	\[
		\varphi^{-1}(\mathfrak{m}_2) = \mathfrak{m}_1,
	\]
	wobei $\mathfrak{m}_i$ das Maximalideal von $R_i$ ist.
\end{defn}

\begin{bsp*}
	Der Homomorphismus $\Z_{\langle p \rangle} \hookrightarrow \Q$ ist kein lokaler Homomorphismus, da
	\[
		\varphi^{-1}(\{0\}) = \{0\} \subsetneq p\Z_{\langle p \rangle}
	\]
	gilt.
\end{bsp*}

\begin{bsp}
\label{bsp:2.4}
	Sei $\mathcal{F}$ die Garbe der stetigen reellwertigen Funktionen auf dem topologischen Raum $X$, das heißt $\mathcal{F}(U) \coloneqq \mathcal{C}(U)$ für alle $U$ offen in $X$ (siehe Beispiel~\ref{bsp:1.5}). Dann ist der Halm $\mathcal{F}_x$ in jedem $x \in X$ ein lokaler Ring.
	\begin{proof}
		Nach Beispiel~\ref{bsp:1.5} ist $\mathcal{F}_x$ ein Ring und sogar eine $\R$-Algebra. Die Menge
		\[
			\mathfrak{m}_x\coloneqq \{ [(U,f)] \in \mathcal{F}_x \mid U \text{ offene Umgebung von } x,\; f \in \mathcal{C}(U), \; f(x) = 0 \}
		\]
		ist ein Ideal in $\mathcal{F}_x$. Es gilt
		\begin{align*}
			\mathcal{F}_x \setminus \mathfrak{m}_x = \{ &[(U,f)] \in \mathcal{F}_x \mid f(x) \neq 0\} = \{[(U,f)] \in \mathcal{F}_x \mid\\
			&f \text{ invertierbar als stetige Funktion in einer Umgebung von } x \} = \mathcal{F}_x^\times.
		\end{align*}
		Aus Proposition~\ref{prop:2.2} folgt, dass $\mathcal{F}_x$ ein lokaler Ring ist.
	\end{proof}
\end{bsp}

Wir benötigen einen wichtigen Begriff aus der Garbentheorie:

\nextmark{direkte Bildgarbe}
\begin{defn}
\label{defn:2.5}
	Sei $f\colon X \to Y$ eine stetige Abbildung topologischer Räume und sei $\mathcal{F}$ eine Garbe auf $X$. Die \textbf{direkt Bildgarbe} $f_*\mathcal{F}$ ist definiert durch
	\[
		(f_*\mathcal{F})(U) \coloneqq \mathcal{F}(f^{-1}(U))
	\]
	für alle $U$ offen in $Y$. Weiter seien die Restriktionsabbildungen von $f_*\mathcal{F}$ gegeben durch 
	\[
		\rho_{UV}^{f_*\mathcal{F}} = \rho_{f^{-1}(U)f^{-1}(V)}^{\mathcal{F}}.
	\]
\end{defn}

\begin{prop}
\label{prop:2.6}
	$f_*\mathcal{F}$ wie in Definition~\ref{defn:2.5} ist eine Garbe.
	\begin{proof}
		Es ist klar, dass $f_*\mathcal{F}$ eine Prägarbe ist. Sei $U$ offen in $Y$ und $U = \coprod_{i\in I}V_i$ eine offene Überdeckung von $U$. Sei weiter $s \in (f_*\mathcal{F})(U)$ mit $\rho_{UV_i}^{f_*\mathcal{F}}(s) = 0$ für alle $i \in I$. Es gilt $f^{-1}(U) = \coprod_{i\in I} f^{-1}(V_i)$. Nach Definition gilt weiter $s \in \mathcal{F}(f^{-1}(U))$ und
		\[
			0 = \rho_{UV_i}^{f_*\mathcal{F}}(s) = \rho_{f^{-1}(U)f^{-1}(V_i)}^{\mathcal{F}}(s)
		\]
		und damit folgt mit f) angewendet auf $\mathcal{F}$ schon $s=0 \in \mathcal{F}(f^{-1}(U)) = f_*\mathcal{F}(U)$. Also gilt f) auch für $f_*\mathcal{F}$. Analog beweist man, dass auch g) für $f_*\mathcal{F}$ gilt.
	\end{proof}
\end{prop}

Im Folgenden betrachten wir Garben von Ringen. Alles aus Kaptiel~\ref{chap:1} und auch Proposition~\ref{prop:2.6} gelten auch für diese Garben.

\nextmark{geringter Raum, Morphismus geringter Räume}
\begin{defn}
\label{defn:2.7}
	\begin{itemize}
		\item Ein \textbf{geringter Raum} $(X, \mathcal{O}_X)$ besteht aus einem topologischen Raum~$X$ und einer Garbe von Ringen $\mathcal{O}_X$ auf $X$.
		\item Ein \textbf{Morphismus von geringten Räumen} $(X, \mathcal{O}_X) \to (Y, \mathcal{O}_Y)$ ist ein Paar $(f, f^{\#})$, wobei $f\colon X \to Y$ eine stetige Abbildung und $f^{\#}\colon \mathcal{O}_Y \to f_*\mathcal{O}_X$ ein Homomorphismus von Garben ist.
	\end{itemize}
	Wir erhalten die Kategorie $\gR$ der geringten Räume.
\end{defn}

\begin{bsp}
\label{bsp:2.8}
	Seien $\mathcal{O}_X$ (beziehungsweise $\mathcal{O}_Y$) die Garbe der stetigen rellwertigen Funktionen auf $X$ (beziehungsweise $Y$). Wir haben in Beispiel~\ref{bsp:2.4} gesehen, dass die eine Garbe von Ringen ist. Dann induziert jede stetige Abbildung $f\colon X \to Y$ einen kanonischen Morphismus $(f,f^{\#})$ von geringten Räumen durch
	\[
		f^{\#}_U\colon \mathcal{O}_Y(U) \to (f_*\mathcal{O}_X)(U) = \mathcal{O}_X(f^{-1}(U)),\; g \mapsto g \circ f.
	\]
\end{bsp}

\nextmark{induzierter Morphismus auf den Halmen}
\begin{bem}
\label{bem:2.9}
	Sei $(f,f^{\#})$ ein Morphismus von geringten Räumen wie in Definition~\ref{defn:2.7}. Sei $x \in X$ und $y \coloneqq f(x)\in Y$. Dann haben wir einen kanonischen Homomorphismus
	\[
		f_x^{\#}\colon \mathcal{O}_{Y,y} \to \mathcal{O}_{X,x},\; [(U,g)] \mapsto [(f^{-1}(U),f^{\#}_U(g))]
	\]
	von Ringen.
\end{bem}

\nextmark{lokal geringter Raum, Morphismus lokal geringter Räume}
\begin{defn}
	\begin{itemize}
		\item Ein \textbf{lokal geringter Raum} ist ein geringter Raum $(X,\mathcal{O}_X)$, bei dem die Halme $\mathcal{O}_{X,x}$ lokale Ringe sind.
		\item Ein \textbf{Morphismus von lokal geringten Räumen} ist ein Morphismus von geringten Räumen, für den die Homomorphismen $f_x^{\#}$ aus Bemerkung~\ref{bem:2.9} für alle $x \in X$ lokal sind.
	\end{itemize}
	Wir erhalten die Kategorie $\lgR$ der lokal geringten Räume.
\end{defn}

\begin{bsp}[Fortsetzung von Beispiel~\ref{bsp:2.8}]
	$(X,\mathcal{O}_X)$ ist ein lokal geringter Raum (siehe Beispiel~\ref{bsp:2.4}). Weiter ist $(f,f^{\#})$ ein Morphismus lokal geringter Räume.
	\begin{proof}
		Sei $x \in X$, $y = f(x)$ und $f^{\#}\colon \mathcal{O}_{Y,y} \to \mathcal{O}_{X,x},\;[(U,g)]\mapsto [(f^{-1}U,g\circ f)]$.\\
		\textbf{Zu zeigen:} $(f^{\#}_x)^{-1}(\mathfrak{m}_x) = \mathfrak{m}_y$.\\
		\enquote{$\subseteq$}: Das Bild eines ivertierbaren Elementes ist wieder invertierbar, also gilt
		\[
			f^{\#}_x(\mathcal{O}_{Y,y}\setminus \mathfrak{m}_y) \subseteq \mathcal{O}_{X,x}\setminus \mathfrak{m}_y
		\]
		und damit
		\[
			(f_x^{\#})^{-1}(\mathfrak{m}_x) \subseteq \mathfrak{m}_y
		\]
		(dies gilt für alle Morphismen von geringten Räumen).\\
		\enquote{$\supseteq$}: Sei $[(U,g)]\in \mathfrak{m}_y$, das heißt $g$ ist eine stetige Funktion auf $U$ mit $g(y)=0$. Es folgt
		\[
			g \circ f \in \mathcal{O}_X(f^{-1}U) \text{ und } (g \circ f)(x)=0
		\]
		und damit
		\[
			f_x^{\#}([(U,g)]) = [(f^{-1}U,g\circ f)] \in \mathfrak{m}_x.
		\]
	\end{proof}
\end{bsp}

\begin{bem}
	Wenn man Morphismen definiert, sollte man sie verknüpfen können (das heißt, man erhält eine Kategorie): Seien $(f,f^{\#})\colon (X,\mathcal{O}_X) \to (Y,\mathcal{O}_Y)$ und $(g,g^{\#})\colon (Y,\mathcal{O}_Y) \to (Z,\mathcal{O}_Z)$ Morphismen von geringten Räumen. Dann ist
	\[
		(g,g^{\#}) \circ (f,f^{\#}) \coloneqq (g\circ f,(g\circ f)^{\#})
	\]
	mit
	\[
		(g\circ f)^{\#}_U \coloneqq f^{\#}_{g^{-1}U} \circ g^{\#}_U
	\]
	ein Morphismus von geringten Räumen. Falls $(f,f^{\#})$ und $(g,g^{\#})$ Morphismen von lokal geringten Räumen sind, dann ist auch $(g,g^{\#}) \circ (f,f^{\#})$ ein Morphismus von lokal geringten Räumen.
	\begin{proof}
		Sei $x \in X$, $y = f(x)$ und $z = g(y)$. Dann gilt
		\[
			((g\circ f)^{\#}_x)^{-1}(\mathfrak{m}_x) = (f^{\#}_x \circ g^{\#}_y)^{-1}(\mathfrak{m}_x) \overset{f^{\#}_x \text{ lokal}}{=} (g^{\#}_y)^{-1}(\mathfrak{m}_y) \overset{g^{\#}_y \text{ lokal}}{=} \mathfrak{m}_z.
		\]
	\end{proof}
\end{bem}

%!TEX root = algebraische_geometrie_2.tex
% vim: tw=0 noet sts=8 sw=8

\chapter{Modulgarben auf geringten Räumen}

\nextmark{(Garbe von) OX-Modul(n), Morphismus von OX-Moduln}
\begin{defn}
	Sei $(X,\mathcal{O}_X)$ ein geringter Raum. Eine \textbf{Garbe von} $\mathcal{O}_X$\textbf{-Moduln} (oder einfach ein $\mathcal{O}_X$\textbf{-Modul}) ist eine Garbe $\mathcal{F}$ von abelschen Gruppen auf $X$, die Folgendes erfüllt:
	\begin{enumerate}[i)]
		\item Für $U \subseteq X$ offen ist $\mathcal{F}(U)$ ein $\mathcal{O}_X(U)$-Modul.
		\item Für offene $U \subseteq V \subseteq X$ ist die Restriktionsabbildung
		\[
			\rho_{VU}^{\mathcal{F}}\colon \mathcal{F}(V) \to \mathcal{F}(U)
		\]
		verträglich mit der Modulstruktur bezüglich
		\[
			\rho_{VU}\colon \mathcal{O}_X(V) \to \mathcal{O}_X(U),
		\]
		das heißt es gilt
		\[
			\rho_{VU}^\mathcal{F}(\lambda \cdot \alpha) = \rho_{VU}(\lambda) \cdot \rho_{VU}^\mathcal{F}(\alpha)
		\]
		für alle $\alpha \in \mathcal{F}(V)$ und $\lambda \in \mathcal{O}_X(V)$.
	\end{enumerate}
	Ein \textbf{Morphismus von} $\mathcal{O}_X$\textbf{-Moduln} (oder $\mathcal{O}_X$\textbf{-linearer Morphismus}) ist ein Garbenmorphismus $\mathcal{F}\to \mathcal{G}$, wobei $\mathcal{F}(U) \to \mathcal{G}(U)$ für alle $U$ offen in $X$ ein $\mathcal{O}_X(U)$-Modulhomomorphismus ist. Wir bezeichnen die Kategorie  der $\mathcal{O}_X$-Moduln mit $(\mathcal{O}_X\text{-}{\Mod})$ und setzen
	\[
		\Hom_{\mathcal{O}_X}(\mathcal{F},\mathcal{G}) \coloneqq
                \Hom_{(\mathcal{O}_X\text{-}{\Mod})}(\mathcal{F},\mathcal{G}).
	\]
\end{defn}

\nextmark{Exaktheit von Sequenzen von OX-Moduln, OX-Hom, Restklassenkörper, Faser}
\begin{bem}
	Sei $(X, \mathcal{O}_X)$ ein geringter Raum.
	\begin{enumerate}[i)]
		\item Sei $\varphi\colon \mathcal{F} \to \mathcal{G}$ ein $\mathcal{O}_X$-Modulmorphismus. In Aufgabe~1.4 wurden für den zugehörigen Garbenmorphismus die Garben $\ker(\varphi)$ und $\im(\varphi)$ auf $X$ definiert. Diese sind auf kanonische Weise $\mathcal{O}_X$-Moduln.
		\item Ist $\mathcal{F}'$ eine Untergrabe von $\mathcal{O}_X$-Moduln des $\mathcal{O}_X$-Moduls $\mathcal{F}$, so ist die Quotientengarbe $\mathcal{F}/\mathcal{F}'$ (nach Garbifizierung) ebenfalls ein $\mathcal{O}_X$-Modul.
		\item Das direkte Produkt, die direkte Summe (hier wird garbifiziert) und der direkte Limes von $\mathcal{O}_X$-Moduln haben wieder die Struktur eines $\mathcal{O}_X$-Moduls.
		\item Eine Sequenz
		\[
			\cdots \to \mathcal{F}_{i+1} \overset{\varphi_{i+1}}{\longto} \mathcal{F}_i \overset{\varphi_{i}}{\longto} \mathcal{F}_{i-1} \to \cdots
		\]
		von $\mathcal{O}_X$-Moduln heißt \textbf{exakt}, wenn die zugehörige Sequenz von Garben exakt ist, das heißt für alle $i \in \Z$ gilt
		\[
			\im(\varphi_{i+1}) =  \ker(\varphi_i).
		\]
		\item Sei $\mathcal{F}$ ein $\mathcal{O}_X$-Modul und $U \subseteq X$ offen. Dann ist $\mathcal{F}\vert_U$ ein $\mathcal{O}_X\vert_U$-Modul. Seien $\mathcal{F}$ und $\mathcal{G}$ zwei Garben abelscher Gruppen auf $X$. In Aufgabe~2.2 wurde gezeigt, dass die Prägarbe
		\[
			U \mapsto \Hom_{\Sh(X)}(\mathcal{F}\vert_U,\mathcal{G}\vert_U)
		\]
		bereits eine Garbe $\underline{\Hom}(\mathcal{F},\mathcal{G})$ ist. Sind $\mathcal{F}$ und $\mathcal{G}$ zwei $\mathcal{O}_X$-Moduln, so wird durch
		\[
			U \mapsto \Hom_{\mathcal{O}_X\vert_U}(\mathcal{F}\vert_U,\mathcal{G}\vert_U)
		\]
		eine Untergarbe $\underline{\Hom}_{\mathcal{O}_X}(\mathcal{F},\mathcal{G})$ von $\underline{\Hom}(\mathcal{F},\mathcal{G})$ definiert, welche ebenfalls die Struktur eines $\mathcal{O}_X$-Moduls trägt.
		\item Ist $(X, \mathcal{O}_X)$ lokal geringt, $\mathcal{F}$ ein $\mathcal{O}_X$-Modul und $p\in X$, so ist der Halm $\mathcal{F}_p$ ein $\mathcal{O}_{X,p}$-Modul. Sei
		\[
			\kappa(p) \coloneqq \mathcal{O}_{X,p}/\mathfrak{m}_p
		\]
		der \textbf{Restklassenkörper} von $p$. Dann heißt der $\kappa(p)$-Vektorraum
		\[
			\mathcal{F}(p) \coloneqq \mathcal{F}_p \otimes_{\mathcal{O}_{X,p}} \kappa(p)
		\]
		die \textbf{Faser} von $\mathcal{F}$ in $p$.
	\end{enumerate}
\end{bem}

\nextmark{(lokal) freier OX-Modul, Rang, invertierbare Garbe}
\begin{defn}
	Sei $(X, \mathcal{O}_X)$ ein lokal geringter Raum.
	\begin{enumerate}[i)]
		\item Ein $\mathcal{O}_X$-Modul $\mathcal{F}$ heißt \textbf{frei}, falls $\mathcal{F} \cong \bigoplus_{i\in I}\mathcal{O}_X$ gilt.
		\item Ein $\mathcal{O}_X$-Modul $\mathcal{F}$ heißt \textbf{lokal frei}, falls es eine offene Überdeckung $(U_i)_{i\in I}$ von $X$ gibt, für die jeder $\mathcal{O}_X\vert_{U_i}$-Modul $\mathcal{F}\vert_{U_i}$ frei ist. In diesem Fall definiert die lokal-konstante Funktion
		\[
			X \to \N \cup \{\infty\}, \; p \mapsto \rg_p(\mathcal{F}) \coloneqq \dim_{\kappa(p)}(\mathcal{F}(p))
		\]
		den \text{Rang} des lokal freien $\mathcal{O}_X$-Moduls $\mathcal{F}$. Falls $p \in U_i$ und $\mathcal{F}\vert_{U_i} \cong \bigoplus_{j\in J} \mathcal{O}_X\vert_{U_i}$ für ein $i \in I$, dann ist $\rg_p(\mathcal{F}) = \abs{J}$.
		Ist $\rg_p(\mathcal{F})< \infty$ für alle $p \in X$, so ist $\mathcal{F}$ \textbf{von endlichem Rang}. Ist $r = \rg_p(\mathcal{F})$ konstant für alle $p \in X$, so heißt $\mathcal{F}$ \textbf{lokal frei vom Rang} $r$. Ist $X$ zusammenhängend, so hat ein lokal freier $\mathcal{O}_X$-Modul $\mathcal{F}$ einen wohldefinierten Rang $\rg_{\mathcal{O}_X}(\mathcal{F}) \in \N \cup \{\infty\}$.
		\item Ein lokal freier $\mathcal{O}_X$-Modul von Rang $1$ heißt \textbf{invertierbare Garbe}.
	\end{enumerate}
\end{defn}

\nextmark{inverse Bildgarbe}
\begin{defn}
	Sei $f\colon X \to Y$ eine stetige Abbildung topologischer Räume. Für eine Garbe $\mathcal{G}$ auf $Y$ betrachten wir die Prägarbe
	\[
		U \mapsto \varinjlim_{\substackclap{f(U)\subseteq V\\V\subseteq Y \text{ offen}}} \, \mathcal{G}(V) = \frac{\{(V,s)\mid V \supseteq f(U),\; V \subseteq Y \text{ offen},\;s\in\mathcal{G}(V)\}}{(V,s)\sim(V',s') \Leftrightarrow \exists \;W \supseteq f(U),\; W \subseteq V \cap V' \text{ offen mit } s\vert_W = s'\vert_W}.
	\]
	Die \textbf{inverse Bildgarbe} $f^{-1}\mathcal{G}$ auf $X$ ist die dazu assoziierte Garbe. Die Konstruktion definiert einen kontravarianten Funktor
	\[
		f^{-1}\colon \Sh(Y) \to \Sh(X),\; \mathcal{F} \mapsto f^{-1}\mathcal{F}. 
	\]
\end{defn}

\nextmark{Adjunktion zwischen f\textasciicircum-1 und f\_*}
\begin{prop}
\label{prop:3.5}
	Sei $f\colon X \to Y$ stetig. Dann existiert für jede Garbe $\mathcal{F}$ auf $X$ und jede Garbe~$\mathcal{G}$ auf $Y$ eine Bijektion
	\[
            \Hom_{\Sh(X)}(f^{-1}\mathcal{G},\mathcal{F}) \overset{\cong}{\longto} \Hom_{\Sh(Y)}(\mathcal{G},f_*\mathcal{F}),
	\]
	welche natürlich in $\mathcal{F}$ und $\mathcal{G}$ ist. Man sagt $f^{-1}$ ist \textbf{linksadjungiert} zu $f_*$ und $f_*$ ist \textbf{rechtsadjungiert} zu $f^{-1}$
\end{prop}

\nextmark{Tensorprodukt von OX-Moduln, inverse Bildgarbe bei OX-Moduln}
\begin{kons}
	\begin{enumerate}[i)]
		\item Sei $(X,\mco_X)$ ein geringter Raum und seien $\mcf$ und $\mcg$ zwei $\mco_X$-Moduln auf~$X$. Dann definiert
		\[
			U \mapsto \mcf(U)\otimes_{\mco_X(U)}\mcg(U)
		\]
		eine Prägarbe auf $X$. Sei $\mcf\otimes_{\mco_X} \mcg$ die dazu assoziierte Garbe. Man sieht sofort, dass das \textbf{Tensorprodukt} $\mcf\otimes_{\mco_X} \mcg$ ein $\mco_X$-Modul ist.
		\item Sei $f\colon(X,\mco_X)\to(Y,\mco_Y)$ ein Morphismus geringter Räume und sei $\mcf$ ein $\mco_X$-Modul auf~$X$. Wir haben in \ref{defn:2.5} die direkte Bildgarbe $f_*\mcf$ definiert. Offenbar ist $f_*\mcf$ ein $f_*\mco_X$-Modul.
		\begin{proof}
			Sei $V$ offen in $Y$, dann gilt $f_*\mcf(V) = \mcf(f^{-1}V)$ und dies ist ein $f_*\mco_X(V) = \mco_X(f^{-1}V)$-Modul.
		\end{proof}
		Nach der Definition von Morphismen geringter Räume gibt es einen Homomorphismus $f^{\#}\colon \mco_Y \to f_*\mco_X$ von Ringgarben. Durch Verknüpfung sehen wir, dass die direkte Bildgarbe ein $\mco_Y$-Modul auf $Y$ ist. Beachte, dass $f_*$ ein kovarianter Funktor ist.
		\item Sei $f\colon (X,\mco_X) \to (Y,\mco_Y)$ ein Morphimus geringter Räume und $\mcg$ ein $\mco_Y$-Modul. Wir betrachten den geringten Raum $(X, f^{-1}\mco_Y)$, wobei die inverse Bildgarbe durch die zur Prägarbe
		\[
                        U \mapsto \varinjlim_{\substackclap{f(U)\subseteq V\\V\subseteq Y \text{ offen}}} \, \mco_Y(V)
		\]
		assoziierte Garbe definiert ist. Beachte, dass $f^{-1}\mco_Y$ eine Ringgarbe ist. Analog sei $f^{-1}\mcg$ die zur Prägrabe
		\[
                        U \mapsto \varinjlim_{\substackclap{f(U)\subseteq V\\V\subseteq Y \text{ offen}}} \, \mathcal{G}(V)
		\]
		assoziierte Garbe. Dann erhalten wir $f^{-1}\mcg$ als $f^{-1}\mco_Y$-Modul. Wieder haben wir einen Homomorphismus $f^{\#}\colon \mco_Y \to f_*\mco_X$ von Ringgarben. Mit Hilfe der Adjunktion aus Proposition~\ref{prop:3.5} erhalten wir einen Homomorphismus $f^{-1}\mco_Y \to \mco_X$ von Ringgarben. Wir definieren das \textbf{inverse Bild} der Modulgarbe $\mcg$ als
		\[
			f^*\mcg \coloneqq f^{-1}\mcg \otimes_{f^{-1}\mco_Y}\mco_X.
		\]
		Indem wir von rechts mit $\mco_X$ tensorieren, erhalten wir tatsächlich einen $\mco_X$-Modul.
                \begin{proof}[Beweisskizze]
			Aus der Modultheorie ist folgendes bekannt: Ist $M$ ein $A$-Modul und $\varphi\colon A \to B$ ein Ringhomomorphismus, dann wird $B$ durch
			\[
				a \cdot m \coloneqq \varphi(a) \cdot m
			\]
			zu einem $A$-Modul. Auf dem $A$-Modul $M\otimes_A B$ definieren wir eine $B$-Modulstruktur durch
			\[
				b\cdot(m\otimes a) \coloneqq m\otimes (a\cdot b).
			\]
			Dies machen wir genauso für Garben auf jeder offenen Menge
		\end{proof}
		Man beachtem dass $f^*$ ein kovarianter Funktor ist.
	\end{enumerate}
\end{kons}

\nextmark{Verträglichkeit von Tensorprodukt und inversem Bild mit Halmen}
\begin{lem}
	\begin{enumerate}[i)]
		\item Seien $(X,\mco_X)$ ein geringter Raum, $p\in X$ und $\mcf,\mcg$ zwei $\mco_X$-Moduln. Dann gilt
		\[
			(\mcf\otimes_{\mco_X}\mcg)_p = \mcf_p \otimes_{\mco_{X,p}} \mcg_p.
		\]
		\item Sei $f \colon (X, \mco_X) \to (Y, \mco_Y)$ ein Morphismus geringter Räume, $p \in X$ und $\mcg$ ein $\mco_Y$-Modul. Dann gilt
		\[
			(f^{-1}\mcg)_p = \mcg_{f(p)}
		\]
		und
		\[
			(f^*\mcg)_p = \mcg_{f(p)}\otimes_{\mco_{Y,f(p)}}\mco_{X,p},
		\]
		wobei $\mco_{Y,f(p)}\to \mco_{X,p}$ der Homomorphismus der Halme $f^{\#}_p$ aus Bemerkung~\ref{bem:2.9} ist.
		\begin{proof}
			Dies wird in den Übungen 3.2 und 3.3 gezeigt.
		\end{proof}
	\end{enumerate}
\end{lem}

\nextmark{Adjunktion zwischen f\textasciicircum* und f\_* (bei OX-Moduln)}
\begin{lem}
	Sei $f\colon (X,\mco_X) \to (Y,\mco_Y)$ ein Morphismus geringter Räume. Dann ist der Funktor $f^*$ linksdadjungiert zum Funktor $f_*$, das heißt für alle $\mco_X$-Moduln $\mcf$ und alle $\mco_Y$-Moduln~$\mcg$ gilt
	\[
		\Hom_{\mco_X}(f^*\mcg,\mcf) \cong \Hom_{\mco_Y}(\mcg,f_*\mcf)
	\]
	natürlich in $\mcf$ und $\mcg$.
	\begin{proof}[Beweisskizze]
		Dies beruht auf der Adjunktion aus Proposition~\ref{prop:3.5} und dem Tensorieren mit $\mco_X$.
	\end{proof}
\end{lem}

%%% Local Variables: 
%%% mode: latex
%%% TeX-master: "algebraische_geometrie_2"
%%% End: 
%!TEX root = algebraische_geometrie_2.tex
% vim: tw=0

\chapter{Affine Schemata}

Zu jedem Ring $A$ (kommutativ und mit Eins) betrachten wir das Spektrum $\Spec(A)$ der Primideale, das in natürlicher Weise eine Topologie besitzt. Durch Lokalisierung von $A$ erhalten wir eine Garbe $\mco_{\Spec(A)}$ auf $\Spec(A)$ und damit einen lokal gringten Raum $(\Spec(A),\mco_{\Spec(A)})$. Im folgenden Kapitel~5 werden dies die Bausteine für Schemata sein. Affine Schemata sind ähnlich wie affine Varietäten aus der Algebraischen Geometrie I, mit dem Unterschied, dass Primideale statt Maximalideale als Punkte und beliebige Ringe zugelassen werden.

\begin{defn}
	Für $M\subseteq A$ sei $V(M)\coloneqq\{\mfp \in \Spec(A) \mid \supseteq M \}$. Dies entspricht der Nullstellenmenge aus der Algebraischen Gerometrie I.
\end{defn}

\begin{lem}
\label{lem:4.2}
	\begin{enumerate}[i)]
		\item Sei $\mfa\coloneqq \langle M \rangle$ das von $M\subseteq A$ erzeugte Ideal. Dann gilt $V(\mfa) = V(M)$.
		\item Es gilt $V(\{0\}) = \Spec(A)$ und $V(A) = \emptyset$.
		\item Für Ideale $\mfa,\mfb$ von $A$ gilt $V(\mfa) \cup V(\mfb) = V(\mfa \cdot \mfb)$.
		\item Für eine Familie $(\mfa_i)_{i\in I}$ von Idealen in $A$ gilt $\bigcap_{i\in I}V(\mfa_i) = V\left(\sum_{i \in I} \mfa_i\right)$.
	\end{enumerate}
	\begin{proof}
		i) und ii) sind trivial.
		\begin{enumerate}[i)]
		\setcounter{enumi}{2}
		\item Es gilt
		\[
			\mfa \cdot \mfb = \langle \{a \cdot b \mid a \in \mfa,\; b \in \mfb\} \rangle .
		\]
		\enquote{$\subseteq$}: Sei $\mfp \in V(\mfa)$. Dann gilt $\mfa \subseteq \mfp$ und damit $\mfa \cdot \mfb \subseteq \mfa \subseteq \mfp$, also $\mfp \in V(\mfa \cdot \mfb)$.

		\enquote{$\supseteq$}: Sei $\mfp \in V(\mfa \cdot \mfp)$. Dann gilt $\mfa \cdot \mfb \subseteq \mfp$. Falls $\mfb \subseteq \mfp$ ist, dann gilt $\mfp \in V(\mfb)$ und wir sind fertig. Sei also ohne Beschränkung der Allgemeinheit $\mfb \not\subseteq \mfp$. Dann gibt es ein $b \in \mfb \setminus \mfp$. Für jedes $a \in \mfa$ gilt dann $a \cdot b \in \mfa \cdot \mfb \subseteq \mfp$. Da $\mfp$ prim ist, gilt also schon $a \in \mfp$ und damit $\mfa \subseteq \mfp$, also $\mfp \in V(\mfa)$.
		\item Dies ist einfach nachzurechnen.
		\end{enumerate}
	\end{proof}
\end{lem}

\begin{defn}[Zariski-Topologie auf $\Spec(A)$]
	Wir definieren eine Teilmenge von $\Spec(A)$ als abgeschlossen, wenn sie die Form $V(\mfa)$ für ein Ideal $\mfa$ von $A$ hat. Eine Teilmenge $U$ von $\Spec(A)$ heißt dann offen, wenn $\Spec(A)\setminus U$ abgeschlossen ist. Nach Lemma~\ref{lem:4.2} definiert dies eine Topologie auf $\Spec(A)$, die wir \textbf{Zariski-Topologie} nennen.
\end{defn}

\begin{prop}
\label{prop:4.4}
	Für $Y\subseteq \Spec(A)$ definieren wir das Verschwindungsideal $I(Y) \coloneqq \bigcap_{\mfp \in Y} \mfp$.
	\begin{enumerate}[i)]
		\item Für ein Ideal $\mfa$ von $A$ gilt $I(V(\mfa)) = \sqrt{\mfa}$.
		\item Für $Y \subseteq \Spec(A)$ gilt $\sqrt{I(Y)}=I(Y)$ und $\overline{Y} = V(I(Y))$.
		\item Die Abbildungen
		\begin{center}
			\begin{tikzcd}
				\{\text{abgeschlossene Teilmengen in } \Spec(A)\} \arrow[shift left=1.1ex]{r}{I} & \{\text{Ideale }\mfp \text{ in } A \text{ mit } \sqrt{\mfa} = \mfa\} \arrow[shift left=1.1ex]{l}{V}
			\end{tikzcd}
		\end{center}
		sind bijektiv, zueinander invers und inklusionsumkehrend.
		\item $Y \subseteq \Spec(A)$ ist genau dann irreduzibel, wenn $I(Y)$ ein Primideal ist.
		\item Die Korrespondenz aus iii) induziert eine Bijektion
		\begin{center}
			\begin{tikzcd}
				\{\text{irreduzible abgeschlossene Teilmengen in } \Spec(A)\} \arrow[shift left=1.1ex]{r}{I} & \Spec(A) \arrow[shift left=1.1ex]{l}{V}
			\end{tikzcd}
		\end{center}
	\end{enumerate}
	\begin{proof}
		Wir benutzen
		\[
			I(V(\mfa)) = \bigcap_{\mfp \in V(\mfa)}\mfp = \bigcap_{\substack{\mfp \supseteq \mfa\\\mfp \in \Spec(A)}}\mfp = \sqrt{\mfa}.
		\]
		Dann folgen die Behauptungen analog wie bei affinen Varietäten.
	\end{proof}
\end{prop}

\begin{defn}
	Sei $X$ ein topologischer Raum.
	\begin{enumerate}[i)]
		\item Ein Punkt $p \in X$ heißt \textbf{abgeschlossen}, wenn $\{p\}$ abgeschlossen ist.
		\item Ein Punkt $p \in X$ heißt \textbf{generischer Punkt}, wenn $\overline{\{p\}} = X$ gilt.
		\item Ein Punkt $q \in X$ heißt \textbf{Spezialisierung} von $p \in X$, wenn $q \in \overline{\{p\}}$ ist.
	\end{enumerate}
\end{defn}

\begin{lem}
	Sei $X$ ein topologischer Raum.
	\begin{enumerate}[i)]
		\item Ist $X$ hausdorffsch, so ist jeder Punkt abgeschlossen.
		\item Existiert ein generischer Punkt in $X$, dann ist $X$ irreduzibel.
	\end{enumerate}
	\begin{proof}
		\begin{enumerate}[i)]
			\item Dies ist einfach zu zeigen.
			\item Sei $p$ ein generischer Punkt von $X$ und $X = X_1 \cup X_2$, wobei $X_1$ und $X_2$ abgeschlossen sind. Wir müssen zeigen, dass $X_1=X$ oder $X_2=X$ gilt. Es gilt $p \in X_i$ für ein $i \in \{1,2\}$ und damit
			\[
				X = \overline{\{p\}} \subseteq X_i.
			\]
		\end{enumerate}
	\end{proof}
\end{lem}

\begin{prop}
\label{prop:4.7}
	Sei $X = \Spec(A)$.
	\begin{enumerate}[i)]
		\item Für $\mfp \in X$ gilt $V(\mfp) = \overline{\{\mfp\}}$. Für $\mfp, \mfq \in X$ gilt insbesondere $\mfp \subseteq \mfq$ genau dann, wenn $\mfq \in \overline{\{\mfp\}}$ gilt.
		\item Unter der Korrespondenz aus Proposition~\ref{prop:4.4} iii) entsprechen die abgeschlossenen Punkte von $\Spec(A)$ genau den Maximalideal von $A$.
		\item Für $Y \subseteq X$ irreduzibel und abgeschlossen gilt $Y = \overline{\{\mfp\}}$ für $\mfp = I(Y)$. Damit ist $\mfp$ der nach i) eindeutig bestimmte generische Punkt von $Y$.
		\item Ist $A$ ein noetherscher Ring, so ist $X$ ein noetherscher topologischer Raum.
		\item Sei $A$ ein noetherscher Ring. Dann entsprechen die irreduziblen Komponenten von $\Spec(A)$ unter der Korrespondenz aus Proposition~\ref{prop:4.4} iii) genau den minimalen Primidealen von $A$.
		\item In einem noetherschen Ring existieren nur endliche viele minimale Primideale.
	\end{enumerate}
	\begin{proof}
		Dies wird in Übung~4.1 gezeigt.
	\end{proof}
\end{prop}

\begin{prop}
\label{prop:4.8}
	Für $f \in A$ sei $V(f) \coloneqq V(\langle f \rangle)$ und $D(f)\coloneqq \Spec(A) \setminus V(f)$.
	\begin{enumerate}[i)]
		\item Für eine Familie $(f_i)_{i\in I}$ in $A$ und $g \in A$ gilt:
		\[
			D(g) \subseteq \bigcup_{i\in I}D(f_i) \Leftrightarrow g \in \sqrt{\langle \{f_i \mid i \in I\}\rangle}
		\]
		\item Die Mengen $(D(f))_{f\in A}$ bilden eine Basis der Zariski-Topologie.
		\item Versehen wir $D(f)$ mit der induzierten Topologie, so ist $D(f)$ quasikompakt. Insbesondere ist $\Spec(A)=D(1)$ quasikompakt.
	\end{enumerate}
	\begin{proof}
		\begin{enumerate}[i)]
			\item Es gilt:
			\begin{align*}
				&D(g) \subseteq \bigcup_{i\in I} D(f_i)\\
				\Longleftrightarrow \quad & V(g) \supseteq \bigcap_{i \in I} V(f_i) = V(\langle\{f_i\mid i \in I\}\rangle)\\
				\Longleftrightarrow \quad & \sqrt{\langle g \rangle} \subseteq \sqrt{\langle\{f_i\mid i \in I\}\rangle}\\
				\Longleftrightarrow \quad & g \in \sqrt{\langle\{f_i\mid i \in I\}\rangle}
			\end{align*}
			\item Sei $\mfa$ ein Ideal und $\mfp \in U \coloneqq \Spec(A) \setminus V(\mfa)$. Zu zeigen ist, dass es ein $f \in A$ mit $\mfp \in D(f) \subseteq U$ gibt (Basiseigenschaft). Wegen $\mfp \notin V(\mfa)$ gibt es ein $f \in \mfa \setminus \mfp$ und man sieht leicht, dass dieses $f$ das Gewünschte liefert.
			\item Quasikompakt heißt, dass jede offene Überdeckung $(V_i)_{i\in I}$ von $D(f)$ eine endliche Teilüberdeckung hat, das heißt es gibt ein endloiches $I_0 \subseteq I$ mit $\bigcup_{i\in I_0} V_i \supseteq D(f)$. Nach ii) bilden die Mengen $D(f)$ eine Basis, also sei ohne Beschränkung der Allgemeinheit $V_i = D(f_i)$ für ein $f_i \in A$. Nun gilt:
			\begin{align*}
				&D(f) \subseteq \bigcup_{i \in I} D(f_i)\\
				\overset{\text{i)}}{\Longleftrightarrow} \quad & f \in \sqrt{\langle \{f_i \mid i \in I\}\rangle}\\
				\Longleftrightarrow \quad & \exists\; m \ge 1,\; f^m = \sum_{i\in I_0} a_i f_i \in \langle \{f_i \mid i \in I\}\rangle \text{ für ein endliches }I_0 \subseteq I, a_i \in A\\
				\Longleftrightarrow \quad & \exists \text{ endliches } I_0 \subseteq I,\; f \in \sqrt{\langle\{f_i\mid i \in I_0\}\rangle}\\
				\Longleftrightarrow \quad & D(f) \subseteq \bigcup_{i \in I_0} D(f_i)
			\end{align*}
			Also ist $D(f)$ quasikompakt. Insbesondere ist $D(1) = \Spec(A)$ quasikompakt.
		\end{enumerate}
	\end{proof}
\end{prop}

\begin{prop}
	Sei $\varphi\colon A \to B$ ein Ringhomomorphismus. Dann gilt:
	\begin{enumerate}[i)]
		\item Die Abbildung $f\colon X = \Spec(B) \to Y = \Spec(A),\; \mfp \mapsto \varphi^{-1}(\mfp)$ ist stetig. Oft bezeichnen wir $f$ auch mit $\Spec(\varphi) = f$. Weiter gelten folgende Identitäten:
		\begin{alignat*}{2}
			f^{-1}(V(M)) &= V(\varphi(M)) \quad &&\forall \;M \subseteq A\\
			f^{-1}(D(a)) &= D(\varphi(a)) \quad &&\forall \;a \in A\\
			\overline{f(V(\mfb))} &= V(\varphi^{-1}(\mfb)) \quad &&\forall\; \mfb \text{ Ideal in } B
		\end{alignat*}
		\item Ist $\varphi$ surjektiv und $\mfa \coloneqq \ker(\varphi)$, dann definiert $f$ einen Homöomorphismus
		\[
			\Spec(B) \overset{\sim}{\longto} V(\mfa).
		\]
	\end{enumerate}
	
	\begin{proof}
		Dies wird in Aufgabe~4.2 gezeigt.
	\end{proof}
\end{prop}

\begin{eri}
	Sei $S \subseteq A$ abgeschlossen unter Multiplikation und $1 \in S$. Dann ist die Lokalisierung $A_S$ definiert als die Menge der Äquivalenzklassen unter folgender Äquivalenzrelation auf $A \times S$:
	\[
		(a,s)\sim(a',s') \Longleftrightarrow \exists\; t \in S \text{ mit } t(s'a-sa') = 0
	\]
	Die Äquivalenzklasse von $(a,s)$ wird mit $\frac{a}{s}$ bezeichnet. $A_S$ ist ein Ring.
	\begin{enumerate}[i)]
		\item Für $\mfp \in \Spec(A)$ sei $A_{\mfp}\coloneqq A_S$ mit $S=A\setminus \mfp$.
		\item Für $f \in A$ sei $A_f \coloneqq A_S$ mit $S = \{f^m \mid m \in \N \}$.
	\end{enumerate}
\end{eri}

\begin{kons}[Lokal geringter Raum auf $\Spec(A)$]
\label{kons:4.11}
	\begin{enumerate}[i)]
		\item Sei $U$ offen in $\Spec(A)$. Dann bezeichnen wir mit $\mco(U)$ die Menge der Funktionen $s\colon U \to \coprod_{\mfp \in U}A_{\mfp}$, die folgende Eigenschaften erfüllen:
		\begin{enumerate}[a)]
			\item Für alle $\mfp \in U$ gilt $s(p) \in A_{\mfp}$
			\item Für alle $\mfp \in U$ gibt es eine offene Umgebung $V$ von $\mfp$ in $U$ und $a,f\in A$ mit $V \subseteq D(f)$ und $s(\mfq) =  \frac{a}{f} \in A_\mfq$ für alle $\mfq \in V$.
		\end{enumerate}
		\item Aufgrund der lokalen Natur der Axiome a) und b) sieht man, dass $U \mapsto \mco(U)$ eine Garbe von Ringen auf $\Spec(A)$ definiert. Dabei sind die Addition und Multiplikation von solchen Funktionen punktweise unter Benutzung der entsprechenden Operation in den $A_{\mfp}$ definiert.
	\end{enumerate}
\end{kons}

\begin{prop}
\label{prop:4.12}
	\begin{enumerate}[i)]
		\item Für alle $\mfp \in \Spec(A)$ gibt es einen kanonischen Isomorphismus
		\[
			\mco_{\mfp} \overset{\sim}{\longto} A_{\mfp}
		\]
		von Ringen, wobei $\mco_\mfp$ der Halm der Garbe $\mco$ in $\mfp$ ist. Insbesondere ist $\mco_\mfp$ ein lokaler Ring und damit ist $(\Spec(A),\mco)$ ein lokal geringter Raum.
		\item Für alle $f \in A$ gibt es einen kanonischen Isomorphismus
		\[
			A_f \overset{\sim}{\longto}\mco_{(D(f))}
		\]
		von Ringen.
		\item Es gilt $\mco(\Spec(A)) = A$.
	\end{enumerate}
	\begin{proof}
		\begin{enumerate}[i)]
			\item Es gilt
			\[
				\mco_\mfp = \varinjlim_{\substackclap{\mfp \in U\\U \text{ offen}}} \, \mco(U),
			\]
			das heißt ein Element von $\mco_\mfp$ ist repräsentiert durch $(U,s)$, wobei $U$ offen mit $\mfp \in U$ und $s \in \mco(U)$ ist. Weiter gilt $[(U,s)] = [(U',s')] \in \mco_\mfp$ genau dann, wenn es ein offenes $V \subseteq U \cap U'$ gibt mit $\mfp \in V$ und $s\vert_V = s'\vert_V$. Sei $\mfp \in \Spec(A)$ und $U$ eine offene Umgebung von $\mfp$. Für $U$ offen in $\Spec(A)$ betrachten wir
			\[
			 	\mco(U) \to A_\mfp,\; s \mapsto s(\mfp).
			\]
			Nach a) sind dies wohldefinierte Ringhomomorphismen. Weiter sind diese Homomorphismen verträglich mit Einschränkungen auf kleinere Umgebungen $V$. Also wird ein Ringhomomorphismus $\varphi\colon\mco_\mfp \to A_\mfp,\; [(U,s)] \mapsto s(\mfp)$ induziert.
			Sei $\frac{a}{f}\in A_\mfp$, also $f \notin \mfp,\; a \in A$. Dies definiert einen Schnitt
			\[
				s = \frac{a}{f}\colon D(f) \to \coprod_{\mfq \in D(f)} A_\mfq,\; \mfq \mapsto s(\mfq) \coloneqq \frac{a}{f} \in A_\mfq.
			\]
			Offenbar gilt $\varphi(s) = \frac{a}{f}\in A_\mfp$, also ist $\varphi$ surjektiv.

			Sei $U$ eine Umgebung von $\mfp$ und seien $s,t \in \mco(U)$ mit $s(\mfp) = t(\mfp)$, das heißt
			\[
				\varphi([(U,s)]) = \varphi([(U,t)]).
			\]
			Wir wollen nun $s=t \in \mco_\mfp$ zeigen. Indem wir die Umgebung von $\mfp$ gegebenenfalls verkleinern, dürfen wir nach b) annehmen, dass es $a,b,f,g \in A$ mit $f,g \notin \mfp$ gibt mit $s = \frac{a}{f} \in \mco(U)$ und $t=\frac{b}{g}\in \mco(U)$ gibt. Dann gilt:
			\begin{align*}
				& s(\mfp) = \varphi(s) = \varphi(t) = t(\mfp)\\
				\Longrightarrow \quad & \frac{a}{f} = \frac{b}{g} \in A_\mfp\\
				\Longrightarrow \quad & \exists\; h \in A \setminus \mfp \text{ mit } h(ga-fb)=0
			\end{align*}
			Weiter gilt $\frac{a}{f}=\frac{b}{g} \in A_\mfq$ für alle $\mfq \in \Spec(A)$ mit $f,g,h \notin \mfq$. Die Menge dieser $\mfq$ ist gleich der offenen Menge $W \coloneqq D(f)\cap D(g) \cap D(h) \ni \mfp$. Also gilt $s\vert_{W\cap U} = t\vert_{W\cap U}$ und damit $s=t \in \mco_\mfp$ nach der Definition von $\mco_\mfp$. Also ist $\varphi$ injektiv.

			Damit ist $\varphi$ ein kanonischer Isomorphismus.

			\item Wir definieren
			\begin{align*}
				\psi_f\colon A_f &\to \mco(D(f))\\
				\frac{a}{f^n} & \mapsto \left(s\colon D(f) \to \coprod_{\mfp \in D(f)}A_\mfp,\; \mfp \mapsto s(\mfp)\coloneqq \frac{a}{f^n}\in A_\mfp\right).
			\end{align*}
			Offenbar ist $\psi_f$ ein wohldefinierter Ringhomomorphismus.

			Sei $\psi_f\left(\frac{a}{f^n}\right) = \psi_f \left(\frac{b}{f^m}\right)$. Dann gilt $\frac{a}{f^n}=\frac{b}{f^m} \in A_\mfp$ für alle $\mfp \in D(f)$. Sei
			\[
				\mfa\coloneqq \Ann_A(f^m a-f^n b) = \{g \in A \mid g \cdot (f^m a - f^n b) = 0\}.
			\]
			Beachte, dass $\mfa$ ein Ideal in $A$ ist. Nun gibt es ein $h \in A \setminus \mfp$ mit $h(f^m a -f^n b) = 0$, also $h \in \mfa \setminus \mfp$. Also gilt $\mfa \not \subseteq \mfp$, das heißt $\mfp \notin V(\mfa)$. Dies gilt für alle $\mfp \in D(f)$, also gilt $D(f) \cap V(\mfa) = \emptyset$. Damit gilt $V(\mfa) \subseteq V(f)$. Mit Proposition~\ref{prop:4.4} folgt wegen $f \in \sqrt{\mfa}$, dass es ein $k \ge 1$ mit $f^k \in \mfa$ gibt. Dann gilt
			$f^k(f^m a -f^n b) = 0$ und damit $\frac{a}{f^n} = \frac{b}{g^m} \in A_f$. Also ist $\psi_f$ injektiv.

			Sei $s \in \mco(D(f))$. Nach Proposition~\ref{prop:4.8} ii) und der Definition von $\mco$ in Konstruktion~\ref{kons:4.11} gibt es eine Familie $(h_i)_{i\in I}$ in $A$ mit
			\begin{equation*}
			\label{eq:4.12.1}
				D(f) = \bigcup_{i\in I} D(h_i) \tag{$\star$}
			\end{equation*}
			und $a_i,g_i \in A$ mit $D(h_i) \subseteq D(g_i)$, sodass für alle $i \in I$
			\[
				s\vert_{D(h_i)} = \frac{a_i}{g_i} \in \mco(D(h_i))
			\]
			gilt. Aus $D(h_i) \subseteq D(g_i)$ folgt mit Proposition~\ref{prop:4.8}, dass es $n_i \ge 1$ und $c_i \in A$ mit $h_i^{n_i} = c_i g_i$ gibt. Dann gilt
			\[
				\frac{a_i}{g_i} = \frac{c_i a_i}{h_i^{n_i}} \in \mco(D(h_i)) = \mco(D(h_i^{n_i})).
			\]
			Wir ersetzen $h_i$ durch $h_i^{n_i}$ und $a_i$ durch $c_i a_i$ und erhalten
			\[
				s\vert_{D(h_i)} = \frac{a_i}{h_i} \in \mco(D(h_i)).
			\]
			Da $D(f)$ nach Proposition~\ref{prop:4.8} quasikompakt ist, gibt es $h_1,\ldots,h_r$ mit
			\[
				D(f) \overset{\eqref{eq:4.12.1}}{=}  D(h_1)\cup \cdots \cup D(h_r).
			\]
			Wir fassen $\frac{a_i}{h_i}$ und $\frac{a_j}{h_j}$ als Elemente von $A_{h_ih_j}$ auf. Wegen $D(h_i)\cap D(h_j) = D(h_ih_j)$ gilt 
			\[
				\psi_{h_ih_j}\left(\frac{a_i}{h_i}\right) = \psi_{h_ih_j}\left(\frac{a_j}{h_j}\right) = s\vert_{D(h_ih_j)} \in \mco(D(h_ih_j)).
			\]
			Aus der Injetivität von $\psi_{h_ih_j}$ folgt $\frac{a_i}{h_i} = \frac{a_j}{h_j} \in A_{h_ih_j}$. Deswegen gibt es $n_{ij}\ge 1$ mit
			\begin{equation*}
			\label{eq:4.12.2}
				(h_ih_j)^{n_{ij}}(h_ja_i-h_ia_j) = 0 \quad \forall \;1 \le i,\; j \le r. \tag{$\star\star$}
			\end{equation*}
			Sei $n \coloneqq \max\{n_{ij}\mid 1\; \le i, j \le r\}$, $\tilde{a}_i\coloneqq h_i^na_i$, $\tilde{h}_i \coloneqq h_i^{n+1}$. Wieder gilt $D(\tilde{h}_i) = D(h_i)$. Aus \eqref{eq:4.12.2} folgt
			\[
				h_j^{n+1}(h_ia_i) - h_i^{n+1}(h_j^na_j) = 0 \quad \forall \;1 \le i,\; j\le r.
			\]
			Deswegen gilt
			\[
				s\vert_{D(\tilde{h}_i)} = \frac{\tilde{a}_i}{\tilde{h}_i} \text{ und } \tilde{h}_i\tilde{a}_j = \tilde{a}_i\tilde{h}_j \quad \forall \;1 \leq i,\; j \le r.
			\]
			Aus \eqref{eq:4.12.1} folgt mit Proposition~\ref{prop:4.8} i), dass es ein $m \ge 1$ mit $f^m = \sum_{i=1}^r b_i \tilde{h}_i$ für gewisse $b_i \in A$ gibt. Wir setzen $a \coloneqq \sum_{i=1}^r b_i \tilde{a}_i$. Für $1 \le j \le r$ gilt
			\[
				\tilde{h}_j a = \sum_{i=1}^r\tilde{h}_jb_i\tilde{a}_i = \sum_{i=1}^r\tilde{h}_ib_i \tilde{a}_j = f^m \tilde{a}_j
			\]
			und damit folgt
			\[
				\frac{\tilde{a}_j}{\tilde{h}_j} = \frac{a}{f^m} \in A_{\tilde{h}_j} \overset{\psi_{\tilde{h}_j}}{\hookrightarrow} \mco(D(\tilde{h}_j))
			\]
			und damit $\psi_f\left(\frac{a}{f}\right) = s$ auf jedem $D(\tilde{h}_j)$. Da $\mco$ eine Garbe ist, gilt dies auch auf $D(f)$. Damit ist $\psi_f$ surjektiv.

			Insgesamt ist $\psi_f$ ein Isomorphismus.
			\item Dies folgt aus ii) mit $f=1$.
		\end{enumerate}
	\end{proof}
\end{prop}

\begin{bem*}
	Wir nennen $\mco$ die Garbe der regulären Funktionen auf $\Spec(A)$.
\end{bem*}

\begin{defn}
	Der lokal geringte Raum $(\Spec(A), \mco)$ heißt \textbf{Spektrum} von $A$.
\end{defn}

\begin{prop}
	Seien $A, B$ Ringe, $X \coloneqq \Spec(B)$ und $Y \coloneqq \Spec(A)$.
	\begin{enumerate}[i)]
		\item Ein Ringhomomorphismus $\varphi\colon A \to B$ induziert eine stetige Abbildung
		\[
			(f = \Spec(\varphi))\colon X \to Y,\; \mfp \mapsto \varphi^{-1}(\mfp),
		\]
		einen Garbenhomomorphismus
		\[
			f^{\#}\colon \mco_X \to f_*(\mco_Y)
		\]
		und einen Morphismus
		\[
			(f,f^{\#})\colon (X,\mco_X)\to (Y,\mco_Y)
		\]
		lokal geringter Räume.
		\item Ein Morphismus lokal geringter Räume $(f,f^{\#})\colon (X,\mco_X)\to(Y,\mco_Y)$ induziert einen Ringhomomorphismus
		\[
			(\varphi = \Gamma(Y,f^{\#}))\colon (A = \mco(Y)) \to (B=\mco(X)).
		\]
		\item Es gibt eine bijektive Korrespondenz
		\begin{align*}
			\phi\colon \Hom_{\Ring}(A,B) &\to \Hom_{\lgR}((\Spec(B),\mco_{\Spec(B}),(\Spec(A),\mco_{\Spec(A}))\\
			\varphi & \mapsto ((f=\Spec(\varphi)),f^{\#})
		\end{align*}
		und die Umkehrabbildung $\Psi$ ist gegeben durch
		\[
			(f,f^{\#})\mapsto \Gamma(\Spec(A),f^{\#}).
		\]
	\end{enumerate}
	\begin{proof}
		Dies wird in Aufgabe~5.1 gezeigt.
	\end{proof}
\end{prop}

\begin{defn}
	Ein \textbf{affines Schema} ist ein lokal geringter Raum $(X,\mco_X)$, der für einen Ring $A$ zu $(\Spec(A),\mco_{\Spec(A)})$ isomorph ist.
\end{defn}

\begin{lem}
\label{lem:4.16}
	Sei $f \in A$. Dann existiert ein kanonischer Isomorphismus
	\[
		(\Spec(A_f),\mco_{\Spec(A_f)}) \overset{\sim}{\longto} (D(f),\mco_{\Spec(A)}\vert_{D(f)})
	\]
	von lokal geringten Räumen. Insbesondere ist $(D(f),\mco_{\Spec(A)}\vert_{D(f)})$ ein affines Schema.
	\begin{proof}
		Dies folgt leicht aus Proposition~\ref{prop:4.12} und wird in Aufgabe~5.2 gezeigt.
	\end{proof}
\end{lem}

\begin{defn}
	Sei $(X,\mco_X)$ ein lokal geringter Raum und $p \in X$. Dann ist $\mco_{X,p}$ ein lokaler Ring mit eindeutigem Maximalideal $\mfm_p$. Also ist
	\[
		\kappa(p) \coloneqq \mco_{X,p}/\mfm_p
	\]
	ein Körper, den wir \textbf{Restklassenkörper} von $(X,\mco_X)$ in $p$ nennen.
\end{defn}

\end{document}
