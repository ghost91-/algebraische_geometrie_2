\documentclass[a4paper,parskip=half,11pt,ngerman,oneside]{scrbook}
\usepackage[utf8]{inputenc}
\usepackage[T1]{fontenc}
\usepackage{babel}
\usepackage[left=2.5cm,right=2.5cm,top=2.5cm,bottom=2.5cm]{geometry}
\usepackage{amsmath}
\usepackage{amsthm}
\usepackage{amssymb}
\usepackage{lmodern}
\usepackage{scrpage2}
\usepackage{hyperref}
\usepackage{mathtools}
\usepackage{tikz-cd}
\usepackage[shortlabels]{enumitem}
\usepackage{csquotes}
\hypersetup{pdfinfo={Title={Algebraische Geometrie II}, Author={Johannes Loher}},colorlinks=false, pdfborder={0 0 0}}
\usepackage[backend=biber,style=alphabetic]{biblatex}
\bibliography{bibliography}
\setkomafont{disposition}{\normalfont\bfseries}

\swapnumbers
\newtheorem{thm}{Theorem}[chapter]
\newtheorem{satz}[thm]{Satz}
\newtheorem{prop}[thm]{Proposition}
\newtheorem{lem}[thm]{Lemma}
\newtheorem{kor}[thm]{Korollar}
\theoremstyle{definition}
\newtheorem{defn}[thm]{Definition}
\newtheorem{bsp}[thm]{Beispiel}
\newtheorem{bem}[thm]{Bemerkung}
\newtheorem{kons}[thm]{Konstruktion}
\newtheorem{warn}[thm]{Warnung}
\newtheorem{eri}[thm]{Erinnerung}
\newtheorem*{bem*}{Bemerkung}
\newtheorem*{bsp*}{Beispiel}

\newcommand{\N}{\mathbb {N}}
\newcommand{\Z}{\mathbb {Z}}
\newcommand{\Q}{\mathbb {Q}}
\newcommand{\R}{\mathbb {R}}
\newcommand{\C}{\mathbb {C}}
\newcommand{\K}{\mathbb {K}}
\newcommand{\A}{\mathbb {A}}
\newcommand{\F}{\mathbb {F}}
\renewcommand{\P}{\mathbb {P}}

\newcommand{\mcf}{\mathcal{F}}
\newcommand{\mcg}{\mathcal{G}}
\newcommand{\mcH}{\mathcal{H}}
\newcommand{\mco}{\mathcal{O}}
\newcommand{\mcj}{\mathcal{J}}
\newcommand{\mfa}{\mathfrak{a}}
\newcommand{\mfb}{\mathfrak{b}}
\newcommand{\mfm}{\mathfrak{m}}
\newcommand{\mfp}{\mathfrak{p}}
\newcommand{\mfq}{\mathfrak{q}}

\newcommand{\Vect}{\operatorname{Vect}}
\newcommand{\Set}{\operatorname{Set}}
\newcommand{\Mod}{\operatorname{Mod}}
\newcommand{\Grp}{\operatorname{Grp}}
\newcommand{\Top}{\operatorname{Top}}
\newcommand{\Ab}{\operatorname{Ab}}
\newcommand{\Nat}{\operatorname{Nat}}
\newcommand{\Op}{\operatorname{Op}}
\newcommand{\PMod}{\operatorname{PMod}}
\newcommand{\Sh}{\operatorname{Sh}}
\newcommand{\PSh}{\operatorname{PSh}}
\newcommand{\Ring}{\operatorname{Ring}}
\newcommand{\Sch}{\operatorname{Sch}}
\newcommand{\gR}{\operatorname{gR}}
\newcommand{\lgR}{\operatorname{lgR}}

\renewcommand{\div}{\operatorname{div}}
\newcommand{\laplace}{\Delta}
\newcommand{\grad}{\operatorname{grad}}
\newcommand{\covariant}[1][dt]{\frac{\nabla}{#1}}
\newcommand{\ggT}{\operatorname{ggT}}
\newcommand{\sign}{\operatorname{sign}}
\newcommand{\id}{\operatorname{id}}
\newcommand{\im}{\operatorname{im}}
\newcommand{\coker}{\operatorname{coker}}
\newcommand{\dom}{\operatorname{dom}}
\newcommand{\codom}{\operatorname{codom}}
\newcommand{\rg}{\operatorname{rg}}
\newcommand{\EW}{\operatorname{E}}
\newcommand{\var}{\operatorname{var}}
\newcommand{\cov}{\operatorname{cov}}
\newcommand{\Hom}{\operatorname{Hom}}
\newcommand{\End}{\operatorname{End}}
\newcommand{\Aut}{\operatorname{Aut}}
\newcommand{\Gal}{\operatorname{Gal}}
\newcommand{\Quot}{\operatorname{Quot}}
\newcommand{\Pot}{\mathcal{P}}
\newcommand{\Mat}{\operatorname{M}}
\newcommand{\Hess}{\operatorname{Hess}}
\newcommand{\Tor}{\operatorname{Tor}}
\newcommand{\Spec}{\operatorname{Spec}}
\newcommand{\Ann}{\operatorname{Ann}}
\newcommand{\Proj}{\operatorname{Proj}}
\newcommand{\nil}{\operatorname{nil}}
\newcommand{\codim}{\operatorname{codim}}

\newcommand{\Tgroup}{\operatorname{T}}
\newcommand{\Ogroup}{\operatorname{O}}
\newcommand{\SOgroup}{\operatorname{SO}}
\newcommand{\Ggroup}{\operatorname{G}}
\newcommand{\SGgroup}{\operatorname{SG}}
\newcommand{\Agroup}{\operatorname{A}}
\newcommand{\SAgroup}{\operatorname{SA}}
\newcommand{\GLgroup}{\operatorname{GL}}
\newcommand{\SLgroup}{\operatorname{SL}}


\newcommand{\abs}[1]{\left \lvert #1 \right \rvert}
\newcommand{\norm}[1]{\left \lVert #1 \right \rVert}
\newcommand{\Char}{\operatorname{char}}
\newcommand{\Mipo}{\operatorname{Mipo}}
\newcommand{\ord}{\operatorname{ord}}
\renewcommand{\Im}{\operatorname{Im}}
\renewcommand{\Re}{\operatorname{Re}}
\newcommand{\dcup}{\mathbin{\dot{\cup}}}
\newcommand{\longto}{\longrightarrow}
\newcommand{\simto}{\overset{\sim}{\longrightarrow}}
\newcommand{\substackclap}[1]{\mathclap{\substack{#1}}}
\newcommand{\multitext}[1]{%
$\left\{\begin{tabular}{@{}l@{}}%
	#1
\end{tabular} \right\}$%
}

% matrices will take an optional argument for format (like arrays)
\makeatletter
\renewcommand*\env@matrix[1][*\c@MaxMatrixCols c]{%
  \hskip -\arraycolsep
  \let\@ifnextchar\new@ifnextchar
  \array{#1}
}
\makeatother

\newenvironment{smallpmatrix}{\left( \begin{smallmatrix}}{\end{smallmatrix} \right)}
\newenvironment{smallbmatrix}{\left[ \begin{smallmatrix}}{\end{smallmatrix} \right]}
\newenvironment{smallBmatrix}{\left\lbrace \begin{smallmatrix}}{\end{smallmatrix} \right\rbrace}
\newenvironment{smallvmatrix}{\left\vert \begin{smallmatrix}}{\end{smallmatrix} \right\vert}
\newenvironment{smallVmatrix}{\left\Vert \begin{smallmatrix}}{\end{smallmatrix} \right\Vert}

% provide commands to create verical lines, that take up no space
\newcommand{\lvline}[1]{\multicolumn{1}{|c}{#1}}
\newcommand{\rvline}[1]{\multicolumn{1}{c|}{#1}}


\begin{document}
\author{Johannes Loher}
\title{Algebraische Geometrie II}
\subtitle{Inoffizielles Vorlesungsskript zur Vorlesung Algebraische Geometrie II von Prof. Walter Gubler im Sommersemester 2014 an der Universität Regensburg}
\date{\today}
\maketitle
\thispagestyle{empty}
\tableofcontents
\thispagestyle{empty}

\frontmatter
%!TEX root = algebraische_geometrie_2.tex
% vim: tw=0

\chapter{Einleitung}
\begin{itemize}
	\item Die klassische algebraische Geometrie ist für Varietäten über algebraisch abgeschlossenen Körpern. Die Koordinatenringe sind dann immer reduzierte Algebren. In der algebraischen Schnitttheorie muss man aber nicht-reduzierte Algebren betrachten (\enquote{Multiplizitäten}).
	\item In der Zahlentheorie wird man gezwungen, über Zahlenkörpern zu arbeiten. Dies sind endliche Körpererweiterungen von $\Q$, also nicht algebraisch abgeschlossen.
	\item Viele Klassifikationsprobleme führen auf Modulräume, die keine Varietäten sind.
\end{itemize}
Um diese Probleme zu lösen, hat Alexander Grothendieck zu Beginn der 60er Jahre die Theorie der Schemata eingeführt (EGA I--IV). Dies ist eine relative Theorie, das heißt es wird kein Grundkörpervorausgesetzt und die Koordinatenringe sind beliebige kommutative Ringe. Das heißt, man kann (beziehungsweise muss) die Methoden der kommutativen Algebra für die Beweise nutzen.

\textbf{Erfolge:} Weil-Vermutung (Deligne 70er), Fields-Medaillen, Schemata haben sich als Standard in der algebraischen und arithmetischen Geometrie durchgesetzt.

In dieser Vorlesung seien Ringe und Algebren immer kommutativ und mit Einselement, falls nichts anderes gesagt wird.

\mainmatter
\chapter{Garben}
Garben sind abstrakte Verallgemeinerungen von Funktionenräumen. Sie sind fundamental für das Studium von Mannigfaltigkeiten und Schemata.

$X$ sei ein topologischer Raum.

\begin{defn}
\label{defn:1.1}
	Eine \textbf{Prägarbe} $\mathcal{F}$ (von abelschen Gruppen) auf $X$ besteht aus folgenden Daten:
	\begin{enumerate}[a)]
		\item Für alle $U$ offen in $X$ sein $\mathcal{F}(U)$ eine abelsche Gruppe.
		\item Für alle $V \subseteq U$ offen in $X$ sei $\rho_{UV}:\mathcal{F}(U)\to\mathcal{F}(V)$ ein Homomorphismus.
		\item Es sei $\mathcal{F}(\emptyset) = 0$.
		\item Für alle $U$ offen in $X$ sei $\rho_{UU} = \id_{\mathcal{F}(U)}$.
		\item Für alle $W \subseteq V \subseteq U$ offen in $X$ sei $\rho_{UW} = \rho_{VW} \circ \rho_{UV}$.
	\end{enumerate}
	Die Elemente von $\mathcal{F}(U)$ heißen \textbf{Schnitte} von $\mathcal{F}$ über $U$. Der Homomorphismus $\rho_{UV}:\mathcal{F}(U)\to\mathcal{F}(V)$ heißt \textbf{Restriktionsabbildung} von $U$ auf die offene Teilmenge $V$ von $U$
\end{defn}

\begin{bem}
	Analog definiert man Prägarben von Ringen, Algebren oder Mengen, ...
\end{bem}

\begin{defn}
	Eine Prägarbe $\mathcal{F}$ auf $X$ heißt \textbf{Garbe}, falls zusätzlich für jede offene Menge $U$ in $X$ und jede offene Überdeckung $U=\bigcup_{i \in I}V_i$ von $U$ folgendes gilt:
	\begin{enumerate}[a)]
		\setcounter{enumi}{5}
		\item Ist $s \in \mathcal{F}(U)$ und $\rho_{UV_i}(s) = 0$ für alle $i \in I$, so gilt bereits $s=0$.
		\item Sind $s_i \in \mathcal{F}(V_i)$ für alle $i \in I$ mit $\rho_{V_i V_i \cap V_J}(s_i) = \rho_{V_j V_i \cap V_J}(s_j)$ für alle $i,j \in I$, so gibt es ein $s \in \mathcal{F}(U)$ mit $\rho_{UV_i}(s) = s_i$ für alle $i \in I$.
	\end{enumerate}
	Eine Garbe ist also duch lokale Informationen vollständig bestimmt.
\end{defn}

\begin{bem}
	Nach f) ist der Schnitt $s$ in g) eindeutig bestimmt.
	\begin{proof}
		Sind $s, s'$ zwei solche Schnitte in g), so gilt:
		\[
			\rho_{UV_i}(s-s') = \rho_{UV_i}(s) - \rho_{UV_i}(s') \overset{\text{g)}}{=}s_i-s_i = 0 
		\]
		Also gilt nach f) schon $s-s'=0$ und damit $s=s'$.
	\end{proof}
\end{bem}

\begin{bsp}
\label{bsp:1.5}
	Sei $x$ ein topologischer Raum. Für $U$ offen in $X$ sei
	\[
		\mathcal{F}(U)\coloneqq \{f:U\to \R \mid f \text{ ist Funktion}\}.
	\]
	Dies ist eine Garbe (abelscher Gruppen und sogar $\R$-Algebren) und die Restirktionsabbildungen sind gegeben durch:
	\[
		\rho_{UV}:\mathcal{F}(U) \to \mathcal{F}(V),\; f \mapsto f\vert_V
	\]
	Die Menge der stetigen Funktionen $C(U)$ liefert eine \textbf{Untergarbe} $\mathcal{F}'$ von $\mathcal{F}$ ($\mathcal{F}'(U)\coloneqq C(U)$), das heißt $\mathcal{F}'$ ist auch eine Garbe, es gilt $\mathcal{F}'(U)\subseteq \mathcal{F}(U)$ für alle $U$ offen und für die Restriktionen ist folgendes Diagramm kommutativ:
	\begin{center}
		\begin{tikzcd}
			\mathcal{F}'(U) \arrow[hook]{r} \arrow{d}[swap]{\rho_{UV}} & \mathcal{F}(U)\arrow{d}{\rho_{UV}}\\
			\mathcal{F}'(V) \arrow[hook]{r} & \mathcal{F}(V)
		\end{tikzcd}
	\end{center}

\end{bsp}

\begin{bsp}
	Sein $A$ eine fixierte abelsche Gruppe. Die zugehörige \textbf{konstante Prägarbe} $\mathcal{F}$ ist definiert durch:
	\begin{itemize}
		\item Es sei $\mathcal{F}(U) \coloneqq \left\lbrace \begin{array}{c c}A & \text{falls } U \neq \emptyset\\0& \text{falls } U = \emptyset\end{array}\right.$.
		\item Es sei $\rho_{UV} \coloneqq \left\lbrace \begin{array}{c c}\id_A & \text{falls } V \neq \emptyset\\0& \text{falls } V = \emptyset\end{array}\right.$.
	\end{itemize}
	Falls $X$ nicht zusammenhängend und $A \neq 0$ ist, dann ist $\mathcal{F}$ keine Garbe. Sei zum Beispiel $X=\{p,q\}$ mit der diskreten Topologie. Seien außerdem $U=\{p,q\}$, $V_1=\{p\}$, $V_2=\{q\}$. Dann gilt $\mathcal{F}(U)=\mathcal{F}(V_1)=\mathcal{F}(V_2) = A$ und $\mathcal{F}(\emptyset)=0$. Seien nun $s_1 \neq 0 \in \mathcal{F}(V_1)$ und $s_2 = 0 \in \mathcal{F}(V_2)$. Wäre $\mathcal{F}$ eine Garbe, dann gäbe es ein $s \in \mathcal{F}(U) = A$ mit $\rho_{UV_i}(s) = s_i$. Dies ist aber offenbar nicht der Fall.
\end{bsp}

\begin{bsp}
\label{bsp:1.7}
	Seien $X,S$ topologische Räume und $\pi:S\to X$ eine Abbildung mit folgenden Eigenschaften:
	\begin{enumerate}[a)]
		\setcounter{enumi}{8}
		\item $\pi$ ist surjektiv und ein lokaler Homöomorphismus.
		\item $\pi^{-1}(x)$ ist für alle $x\in X$ eine abelsche Gruppe.
		\item Sei $S\times_X S\coloneqq \{(s_1,s_2) \in S \times S \mid \pi(s_1)=\pi(s_2)\}$ das \textbf{Faserprodukt} über $X$ mit der von $S \times S$ induzierten Topologie. Dann induzieren die Addition und Invertierung aus j) stetige Abbildungen $S\times_X S \to S,\; (s_1,s_2) \mapsto s_1+s_2$, beziehungsweise $S \to S,\; s \mapsto -s$.
	\end{enumerate}
	Die Abbildung $\pi$ heißt \textbf{Projektion} und $\pi^{-1}(x)$ heißt \textbf{Faser} von $x$.
	\begin{itemize}
		\item Sei $U \subseteq X$ offen. Eine stetige Funktion $f:U \to S$ heißt \textbf{Schnitt}, wenn $\pi \circ f = \id_U$, das heißt für alle $x \in U$ gilt $f(x) \in \pi^{-1}(x)$.
		\item Es gibt einen kanonischen globalen Schnitt $X \to S, \; x \mapsto 0_x \in f^{-1}(x)$, den wir \textbf{Nullschnitt} nennen.
		\item Mit $\Gamma(U,S)$ bezeichnen wir den Raum der Schnitte von $S$ über $U$, wir setzen also
		\[
			\Gamma(U,S) \coloneqq \{d:U\to S \mid f \text{ stetig},\; \pi\circ f = \id_U\}.
		\]
	\end{itemize}
	\textbf{Behauptung:} Die Abbildung $U \mapsto \Gamma(U,S)$ zusammen mit der Restriktion $\rho_{UV}(f) \coloneqq f\vert_V$ ist eine Garbe.
	\begin{proof}
		Dies ist klar.
	\end{proof}
\end{bsp}

\begin{bem}
	Weil $\pi$ ein lokaler Homöomorphismus ist, muss jeder Schnitt eine offene Abbildung sein (das heißt das Bild einer offenen Teilmenge ist offen). Falls zwei Schnitte in $x \in X$ übereinstimmen, dann stimmen sie auch auf einer Umgebung von $x$ überein.
\end{bem}

\begin{bsp*}
	Die Abbildung $\R \to \{z\in \C \mid \abs{z} = 1\},\; t \mapsto e^{2 \pi i t}$ ist eine Abbildung wie in Beispiel~\ref{bsp:1.7}.
\end{bsp*}

\begin{bem}
	Das Beispiel~\ref{bsp:1.7} erklärt die abstrakten Begriffe aus Definition~\ref{defn:1.1}. Wir werden sehen, dass jede Garbe durch einen topologischen Raum $S$ und eine Abbildung $\pi:S \to X$ wie in Beispiel~\ref{bsp:1.7} dargestellt werden kann.
\end{bem}

\begin{bsp}
	In Beispiel~\ref{bsp:1.5} wählen wir ein $x \in X$. Wir definieren $f\sim g$, falls es eine Umgebung $U$ von $x$ gibt mit $f\vert_U = g\vert_U$. Dies liefert eine Äquivalenzrelation auf der Menge der reellwertigen Funktionen, die auf einer Umgebung von $x$ definiert sind. Der \textbf{Halm} $\mathcal{F}_x$ ist definiert als Raum der Äquivalenzklassen bezüglich dieser Äquivalenzrelation.
\end{bsp}

\begin{defn}
	Sei $\mathcal{F}$ eine Prägarbe auf $X$, dann verallgemeinern wir die obige Konstruktion. Sei $x \in X$. Wir betrachten die Menge $\{(U,s)\mid s \in \mathcal{F}(U),\; U \text{ offene Umgebung von } x\}$. Wir definieren auf dieser Menge eine Relation auf folgende Weise: Es gelte	$(U,s) \sim (V,t)$ genau dann, wenn es eine offene Umgebung $W\subseteq U \cap V$ von $x$ mit $\rho_{UW}(s) = \rho_{VW}(t)$ gibt. Man zeigt leicht, dass dies eine Äquivalenzrelation ist. Der \textbf{Halm} $\mathcal{F}_x$ ist definiert als der Raum der Äquivalenzklassen bezüglich dieser Äquivalenzrelation. Dies ist eine abelsche Gruppe:
	\[
		[(U_1,s_1)] + [(U_2,s_2)] = [(U_1 \cap U_2, \rho_{U_1U_1\cap U_2}(s_1) + \rho_{U_2U_1\cap U_2}(s_2)].
	\]
	Wir schreiben auch $s_x$ anstatt von $[(U,s)] \in \mathcal{F}_x$.
\end{defn}

\begin{defn}
	Ein \textbf{Homomorphismus} $\varphi: \mathcal{F} \to \mathcal{G}$ \textbf{von (Prä-)Garben} auf $x$ ist eine Familie von Homomorphismen $\varphi_U:\mathcal{F}(U) \to \mathcal{G}(U)$ abelscher Gruppen für alle $U$ offen in $X$, sodass
	\begin{center}
		\begin{tikzcd}
			\mathcal{F}(U) \arrow{r}{\varphi_U} \arrow{d}[swap]{\rho_{UV}} & \mathcal{G}(U)\arrow{d}{\rho_{UV}}\\
			\mathcal{F}(V) \arrow{r}[swap]{\varphi_V} & \mathcal{G}(V)
		\end{tikzcd}
	\end{center}
	für alle $V \subseteq U$ offen in $X$ kommutiert.

	Wir können Homomorphismen $\varphi:\mathcal{F}\to \mathcal{G}$ und $\psi:\mathcal{G} \to \mathcal{H}$ von (Prä-)Garben zu einem Homomorphismus $\psi\circ \varphi:\mathcal{F} \to \mathcal{H}$ verknüpfen. Damit können wir auch Isomorphismen von (Prä-)Garben definieren.

	Die (Prä-)Garben bilden eine Kategorie.
\end{defn}

\begin{prop}
	Sei $\varphi: \mathcal{F} \to \mathcal{G}$ ein Homomorphismus von Garben auf $X$. Dann ist $\varphi$ genau dann ein Isomorphismus von Garben, wenn $\varphi_x:\mathcal{F}_x \to \mathcal{G}_x,\; s_x \mapsto \varphi_x(s_x) \coloneqq [(U,\varphi\vert_U(s))]$ für alle $x \in X$ ein Isomorphismus abelscher Gruppen ist.
	\begin{proof}
		Dies ist eine einfache Übung.
	\end{proof}
	Beachte, dass diese Aussage nicht für Prägarben gilt.
\end{prop}
%!TEX root = algebraische_geometrie_2.tex
% vim: tw=0 noet sts=8 sw=8

\chapter{Lokal geringte Räume}

Mit dem Konzept der lokal geringten Räume kann man die Mannigfaltigkeiten aus der Analysis und Differentialgeometrie, die algebraische Varietäten aus der algebraischen Geometrie I und die Schemata aus der algebraischen Geometrie II zusammenfassen.

\begin{bem}
	Im Folgenden soll, wenn nichts anderes gesagt wird, Folgendes gelten:
	\begin{itemize}
	 	\item Alle Ringe sind kommutativ mit Eins.
	 	\item Ringhomomorphismen bilden die Eins auf die Eins ab.
	 \end{itemize}
\end{bem}

\begin{bem*}
	Ein Ring $R$ heißt \textbf{lokal}, wenn es in $R$ genau ein Maximalideal $\mathfrak{m}$ gibt.
\end{bem*}

\begin{prop}
\label{prop:2.2}
	Sei $R$ ein Ring. $R$ ist genau dann ein lokaler Ring, wenn es ein Ideal $I \neq R$ mit $R\setminus I = R^\times$ gibt.
	\begin{proof}
		Dies ist eine einfache Eigenschaft aus der kommutativen Algebra.
	\end{proof}
\end{prop}

\nextmark{lokaler Ringhomomorphismus}
\begin{defn}
	Ein Homomorphismus $\varphi\colon R_1 \to R_2$ von Ringen heißt \textbf{lokal}, wenn
	\[
		\varphi^{-1}(\mathfrak{m}_2) = \mathfrak{m}_1,
	\]
	wobei $\mathfrak{m}_i$ das Maximalideal von $R_i$ ist.
\end{defn}

\begin{bsp*}
	Der Homomorphismus $\Z_{\langle p \rangle} \hookrightarrow \Q$ ist kein lokaler Homomorphismus, da
	\[
		\varphi^{-1}(\{0\}) = \{0\} \subsetneq p\Z_{\langle p \rangle}
	\]
	gilt.
\end{bsp*}

\begin{bsp}
\label{bsp:2.4}
	Sei $\mathcal{F}$ die Garbe der stetigen reellwertigen Funktionen auf dem topologischen Raum $X$, das heißt $\mathcal{F}(U) \coloneqq \mathcal{C}(U)$ für alle $U$ offen in $X$ (siehe Beispiel~\ref{bsp:1.5}). Dann ist der Halm $\mathcal{F}_x$ in jedem $x \in X$ ein lokaler Ring.
	\begin{proof}
		Nach Beispiel~\ref{bsp:1.5} ist $\mathcal{F}_x$ ein Ring und sogar eine $\R$-Algebra. Die Menge
		\[
			\mathfrak{m}_x\coloneqq \{ [(U,f)] \in \mathcal{F}_x \mid U \text{ offene Umgebung von } x,\; f \in \mathcal{C}(U), \; f(x) = 0 \}
		\]
		ist ein Ideal in $\mathcal{F}_x$. Es gilt
		\begin{align*}
			\mathcal{F}_x \setminus \mathfrak{m}_x = \{ &[(U,f)] \in \mathcal{F}_x \mid f(x) \neq 0\} = \{[(U,f)] \in \mathcal{F}_x \mid\\
			&f \text{ invertierbar als stetige Funktion in einer Umgebung von } x \} = \mathcal{F}_x^\times.
		\end{align*}
		Aus Proposition~\ref{prop:2.2} folgt, dass $\mathcal{F}_x$ ein lokaler Ring ist.
	\end{proof}
\end{bsp}

Wir benötigen einen wichtigen Begriff aus der Garbentheorie:

\nextmark{direkte Bildgarbe}
\begin{defn}
\label{defn:2.5}
	Sei $f\colon X \to Y$ eine stetige Abbildung topologischer Räume und sei $\mathcal{F}$ eine Garbe auf $X$. Die \textbf{direkt Bildgarbe} $f_*\mathcal{F}$ ist definiert durch
	\[
		(f_*\mathcal{F})(U) \coloneqq \mathcal{F}(f^{-1}(U))
	\]
	für alle $U$ offen in $Y$. Weiter seien die Restriktionsabbildungen von $f_*\mathcal{F}$ gegeben durch 
	\[
		\rho_{UV}^{f_*\mathcal{F}} = \rho_{f^{-1}(U)f^{-1}(V)}^{\mathcal{F}}.
	\]
\end{defn}

\begin{prop}
\label{prop:2.6}
	$f_*\mathcal{F}$ wie in Definition~\ref{defn:2.5} ist eine Garbe.
	\begin{proof}
		Es ist klar, dass $f_*\mathcal{F}$ eine Prägarbe ist. Sei $U$ offen in $Y$ und $U = \coprod_{i\in I}V_i$ eine offene Überdeckung von $U$. Sei weiter $s \in (f_*\mathcal{F})(U)$ mit $\rho_{UV_i}^{f_*\mathcal{F}}(s) = 0$ für alle $i \in I$. Es gilt $f^{-1}(U) = \coprod_{i\in I} f^{-1}(V_i)$. Nach Definition gilt weiter $s \in \mathcal{F}(f^{-1}(U))$ und
		\[
			0 = \rho_{UV_i}^{f_*\mathcal{F}}(s) = \rho_{f^{-1}(U)f^{-1}(V_i)}^{\mathcal{F}}(s)
		\]
		und damit folgt mit f) angewendet auf $\mathcal{F}$ schon $s=0 \in \mathcal{F}(f^{-1}(U)) = f_*\mathcal{F}(U)$. Also gilt f) auch für $f_*\mathcal{F}$. Analog beweist man, dass auch g) für $f_*\mathcal{F}$ gilt.
	\end{proof}
\end{prop}

Im Folgenden betrachten wir Garben von Ringen. Alles aus Kaptiel~\ref{chap:1} und auch Proposition~\ref{prop:2.6} gelten auch für diese Garben.

\nextmark{geringter Raum, Morphismus geringter Räume}
\begin{defn}
\label{defn:2.7}
	\begin{itemize}
		\item Ein \textbf{geringter Raum} $(X, \mathcal{O}_X)$ besteht aus einem topologischen Raum~$X$ und einer Garbe von Ringen $\mathcal{O}_X$ auf $X$.
		\item Ein \textbf{Morphismus von geringten Räumen} $(X, \mathcal{O}_X) \to (Y, \mathcal{O}_Y)$ ist ein Paar $(f, f^{\#})$, wobei $f\colon X \to Y$ eine stetige Abbildung und $f^{\#}\colon \mathcal{O}_Y \to f_*\mathcal{O}_X$ ein Homomorphismus von Garben ist.
	\end{itemize}
	Wir erhalten die Kategorie $\gR$ der geringten Räume.
\end{defn}

\begin{bsp}
\label{bsp:2.8}
	Seien $\mathcal{O}_X$ (beziehungsweise $\mathcal{O}_Y$) die Garbe der stetigen rellwertigen Funktionen auf $X$ (beziehungsweise $Y$). Wir haben in Beispiel~\ref{bsp:2.4} gesehen, dass die eine Garbe von Ringen ist. Dann induziert jede stetige Abbildung $f\colon X \to Y$ einen kanonischen Morphismus $(f,f^{\#})$ von geringten Räumen durch
	\[
		f^{\#}_U\colon \mathcal{O}_Y(U) \to (f_*\mathcal{O}_X)(U) = \mathcal{O}_X(f^{-1}(U)),\; g \mapsto g \circ f.
	\]
\end{bsp}

\nextmark{induzierter Morphismus auf den Halmen}
\begin{bem}
\label{bem:2.9}
	Sei $(f,f^{\#})$ ein Morphismus von geringten Räumen wie in Definition~\ref{defn:2.7}. Sei $x \in X$ und $y \coloneqq f(x)\in Y$. Dann haben wir einen kanonischen Homomorphismus
	\[
		f_x^{\#}\colon \mathcal{O}_{Y,y} \to \mathcal{O}_{X,x},\; [(U,g)] \mapsto [(f^{-1}(U),f^{\#}_U(g))]
	\]
	von Ringen.
\end{bem}

\nextmark{lokal geringter Raum, Morphismus lokal geringter Räume}
\begin{defn}
	\begin{itemize}
		\item Ein \textbf{lokal geringter Raum} ist ein geringter Raum $(X,\mathcal{O}_X)$, bei dem die Halme $\mathcal{O}_{X,x}$ lokale Ringe sind.
		\item Ein \textbf{Morphismus von lokal geringten Räumen} ist ein Morphismus von geringten Räumen, für den die Homomorphismen $f_x^{\#}$ aus Bemerkung~\ref{bem:2.9} für alle $x \in X$ lokal sind.
	\end{itemize}
	Wir erhalten die Kategorie $\lgR$ der lokal geringten Räume.
\end{defn}

\begin{bsp}[Fortsetzung von Beispiel~\ref{bsp:2.8}]
	$(X,\mathcal{O}_X)$ ist ein lokal geringter Raum (siehe Beispiel~\ref{bsp:2.4}). Weiter ist $(f,f^{\#})$ ein Morphismus lokal geringter Räume.
	\begin{proof}
		Sei $x \in X$, $y = f(x)$ und $f^{\#}\colon \mathcal{O}_{Y,y} \to \mathcal{O}_{X,x},\;[(U,g)]\mapsto [(f^{-1}U,g\circ f)]$.\\
		\textbf{Zu zeigen:} $(f^{\#}_x)^{-1}(\mathfrak{m}_x) = \mathfrak{m}_y$.\\
		\enquote{$\subseteq$}: Das Bild eines ivertierbaren Elementes ist wieder invertierbar, also gilt
		\[
			f^{\#}_x(\mathcal{O}_{Y,y}\setminus \mathfrak{m}_y) \subseteq \mathcal{O}_{X,x}\setminus \mathfrak{m}_y
		\]
		und damit
		\[
			(f_x^{\#})^{-1}(\mathfrak{m}_x) \subseteq \mathfrak{m}_y
		\]
		(dies gilt für alle Morphismen von geringten Räumen).\\
		\enquote{$\supseteq$}: Sei $[(U,g)]\in \mathfrak{m}_y$, das heißt $g$ ist eine stetige Funktion auf $U$ mit $g(y)=0$. Es folgt
		\[
			g \circ f \in \mathcal{O}_X(f^{-1}U) \text{ und } (g \circ f)(x)=0
		\]
		und damit
		\[
			f_x^{\#}([(U,g)]) = [(f^{-1}U,g\circ f)] \in \mathfrak{m}_x.
		\]
	\end{proof}
\end{bsp}

\begin{bem}
	Wenn man Morphismen definiert, sollte man sie verknüpfen können (das heißt, man erhält eine Kategorie): Seien $(f,f^{\#})\colon (X,\mathcal{O}_X) \to (Y,\mathcal{O}_Y)$ und $(g,g^{\#})\colon (Y,\mathcal{O}_Y) \to (Z,\mathcal{O}_Z)$ Morphismen von geringten Räumen. Dann ist
	\[
		(g,g^{\#}) \circ (f,f^{\#}) \coloneqq (g\circ f,(g\circ f)^{\#})
	\]
	mit
	\[
		(g\circ f)^{\#}_U \coloneqq f^{\#}_{g^{-1}U} \circ g^{\#}_U
	\]
	ein Morphismus von geringten Räumen. Falls $(f,f^{\#})$ und $(g,g^{\#})$ Morphismen von lokal geringten Räumen sind, dann ist auch $(g,g^{\#}) \circ (f,f^{\#})$ ein Morphismus von lokal geringten Räumen.
	\begin{proof}
		Sei $x \in X$, $y = f(x)$ und $z = g(y)$. Dann gilt
		\[
			((g\circ f)^{\#}_x)^{-1}(\mathfrak{m}_x) = (f^{\#}_x \circ g^{\#}_y)^{-1}(\mathfrak{m}_x) \overset{f^{\#}_x \text{ lokal}}{=} (g^{\#}_y)^{-1}(\mathfrak{m}_y) \overset{g^{\#}_y \text{ lokal}}{=} \mathfrak{m}_z.
		\]
	\end{proof}
\end{bem}

%!TEX root = algebraische_geometrie_2.tex
% vim: tw=0 noet sts=8 sw=8

\chapter{Modulgarben auf geringten Räumen}

\nextmark{(Garbe von) OX-Modul(n), Morphismus von OX-Moduln}
\begin{defn}
	Sei $(X,\mathcal{O}_X)$ ein geringter Raum. Eine \textbf{Garbe von} $\mathcal{O}_X$\textbf{-Moduln} (oder einfach ein $\mathcal{O}_X$\textbf{-Modul}) ist eine Garbe $\mathcal{F}$ von abelschen Gruppen auf $X$, die Folgendes erfüllt:
	\begin{enumerate}[i)]
		\item Für $U \subseteq X$ offen ist $\mathcal{F}(U)$ ein $\mathcal{O}_X(U)$-Modul.
		\item Für offene $U \subseteq V \subseteq X$ ist die Restriktionsabbildung
		\[
			\rho_{VU}^{\mathcal{F}}\colon \mathcal{F}(V) \to \mathcal{F}(U)
		\]
		verträglich mit der Modulstruktur bezüglich
		\[
			\rho_{VU}\colon \mathcal{O}_X(V) \to \mathcal{O}_X(U),
		\]
		das heißt es gilt
		\[
			\rho_{VU}^\mathcal{F}(\lambda \cdot \alpha) = \rho_{VU}(\lambda) \cdot \rho_{VU}^\mathcal{F}(\alpha)
		\]
		für alle $\alpha \in \mathcal{F}(V)$ und $\lambda \in \mathcal{O}_X(V)$.
	\end{enumerate}
	Ein \textbf{Morphismus von} $\mathcal{O}_X$\textbf{-Moduln} (oder $\mathcal{O}_X$\textbf{-linearer Morphismus}) ist ein Garbenmorphismus $\mathcal{F}\to \mathcal{G}$, wobei $\mathcal{F}(U) \to \mathcal{G}(U)$ für alle $U$ offen in $X$ ein $\mathcal{O}_X(U)$-Modulhomomorphismus ist. Wir bezeichnen die Kategorie  der $\mathcal{O}_X$-Moduln mit $(\mathcal{O}_X\text{-}{\Mod})$ und setzen
	\[
		\Hom_{\mathcal{O}_X}(\mathcal{F},\mathcal{G}) \coloneqq
                \Hom_{(\mathcal{O}_X\text{-}{\Mod})}(\mathcal{F},\mathcal{G}).
	\]
\end{defn}

\nextmark{Exaktheit von Sequenzen von OX-Moduln, OX-Hom, Restklassenkörper, Faser}
\begin{bem}
	Sei $(X, \mathcal{O}_X)$ ein geringter Raum.
	\begin{enumerate}[i)]
		\item Sei $\varphi\colon \mathcal{F} \to \mathcal{G}$ ein $\mathcal{O}_X$-Modulmorphismus. In Aufgabe~1.4 wurden für den zugehörigen Garbenmorphismus die Garben $\ker(\varphi)$ und $\im(\varphi)$ auf $X$ definiert. Diese sind auf kanonische Weise $\mathcal{O}_X$-Moduln.
		\item Ist $\mathcal{F}'$ eine Untergrabe von $\mathcal{O}_X$-Moduln des $\mathcal{O}_X$-Moduls $\mathcal{F}$, so ist die Quotientengarbe $\mathcal{F}/\mathcal{F}'$ (nach Garbifizierung) ebenfalls ein $\mathcal{O}_X$-Modul.
		\item Das direkte Produkt, die direkte Summe (hier wird garbifiziert) und der direkte Limes von $\mathcal{O}_X$-Moduln haben wieder die Struktur eines $\mathcal{O}_X$-Moduls.
		\item Eine Sequenz
		\[
			\cdots \to \mathcal{F}_{i+1} \overset{\varphi_{i+1}}{\longto} \mathcal{F}_i \overset{\varphi_{i}}{\longto} \mathcal{F}_{i-1} \to \cdots
		\]
		von $\mathcal{O}_X$-Moduln heißt \textbf{exakt}, wenn die zugehörige Sequenz von Garben exakt ist, das heißt für alle $i \in \Z$ gilt
		\[
			\im(\varphi_{i+1}) =  \ker(\varphi_i).
		\]
		\item Sei $\mathcal{F}$ ein $\mathcal{O}_X$-Modul und $U \subseteq X$ offen. Dann ist $\mathcal{F}\vert_U$ ein $\mathcal{O}_X\vert_U$-Modul. Seien $\mathcal{F}$ und $\mathcal{G}$ zwei Garben abelscher Gruppen auf $X$. In Aufgabe~2.2 wurde gezeigt, dass die Prägarbe
		\[
			U \mapsto \Hom_{\Sh(X)}(\mathcal{F}\vert_U,\mathcal{G}\vert_U)
		\]
		bereits eine Garbe $\underline{\Hom}(\mathcal{F},\mathcal{G})$ ist. Sind $\mathcal{F}$ und $\mathcal{G}$ zwei $\mathcal{O}_X$-Moduln, so wird durch
		\[
			U \mapsto \Hom_{\mathcal{O}_X\vert_U}(\mathcal{F}\vert_U,\mathcal{G}\vert_U)
		\]
		eine Untergarbe $\underline{\Hom}_{\mathcal{O}_X}(\mathcal{F},\mathcal{G})$ von $\underline{\Hom}(\mathcal{F},\mathcal{G})$ definiert, welche ebenfalls die Struktur eines $\mathcal{O}_X$-Moduls trägt.
		\item Ist $(X, \mathcal{O}_X)$ lokal geringt, $\mathcal{F}$ ein $\mathcal{O}_X$-Modul und $p\in X$, so ist der Halm $\mathcal{F}_p$ ein $\mathcal{O}_{X,p}$-Modul. Sei
		\[
			\kappa(p) \coloneqq \mathcal{O}_{X,p}/\mathfrak{m}_p
		\]
		der \textbf{Restklassenkörper} von $p$. Dann heißt der $\kappa(p)$-Vektorraum
		\[
			\mathcal{F}(p) \coloneqq \mathcal{F}_p \otimes_{\mathcal{O}_{X,p}} \kappa(p)
		\]
		die \textbf{Faser} von $\mathcal{F}$ in $p$.
	\end{enumerate}
\end{bem}

\nextmark{(lokal) freier OX-Modul, Rang, invertierbare Garbe}
\begin{defn}
	Sei $(X, \mathcal{O}_X)$ ein lokal geringter Raum.
	\begin{enumerate}[i)]
		\item Ein $\mathcal{O}_X$-Modul $\mathcal{F}$ heißt \textbf{frei}, falls $\mathcal{F} \cong \bigoplus_{i\in I}\mathcal{O}_X$ gilt.
		\item Ein $\mathcal{O}_X$-Modul $\mathcal{F}$ heißt \textbf{lokal frei}, falls es eine offene Überdeckung $(U_i)_{i\in I}$ von $X$ gibt, für die jeder $\mathcal{O}_X\vert_{U_i}$-Modul $\mathcal{F}\vert_{U_i}$ frei ist. In diesem Fall definiert die lokal-konstante Funktion
		\[
			X \to \N \cup \{\infty\}, \; p \mapsto \rg_p(\mathcal{F}) \coloneqq \dim_{\kappa(p)}(\mathcal{F}(p))
		\]
		den \text{Rang} des lokal freien $\mathcal{O}_X$-Moduls $\mathcal{F}$. Falls $p \in U_i$ und $\mathcal{F}\vert_{U_i} \cong \bigoplus_{j\in J} \mathcal{O}_X\vert_{U_i}$ für ein $i \in I$, dann ist $\rg_p(\mathcal{F}) = \abs{J}$.
		Ist $\rg_p(\mathcal{F})< \infty$ für alle $p \in X$, so ist $\mathcal{F}$ \textbf{von endlichem Rang}. Ist $r = \rg_p(\mathcal{F})$ konstant für alle $p \in X$, so heißt $\mathcal{F}$ \textbf{lokal frei vom Rang} $r$. Ist $X$ zusammenhängend, so hat ein lokal freier $\mathcal{O}_X$-Modul $\mathcal{F}$ einen wohldefinierten Rang $\rg_{\mathcal{O}_X}(\mathcal{F}) \in \N \cup \{\infty\}$.
		\item Ein lokal freier $\mathcal{O}_X$-Modul von Rang $1$ heißt \textbf{invertierbare Garbe}.
	\end{enumerate}
\end{defn}

\nextmark{inverse Bildgarbe}
\begin{defn}
	Sei $f\colon X \to Y$ eine stetige Abbildung topologischer Räume. Für eine Garbe $\mathcal{G}$ auf $Y$ betrachten wir die Prägarbe
	\[
		U \mapsto \varinjlim_{\substackclap{f(U)\subseteq V\\V\subseteq Y \text{ offen}}} \, \mathcal{G}(V) = \frac{\{(V,s)\mid V \supseteq f(U),\; V \subseteq Y \text{ offen},\;s\in\mathcal{G}(V)\}}{(V,s)\sim(V',s') \Leftrightarrow \exists \;W \supseteq f(U),\; W \subseteq V \cap V' \text{ offen mit } s\vert_W = s'\vert_W}.
	\]
	Die \textbf{inverse Bildgarbe} $f^{-1}\mathcal{G}$ auf $X$ ist die dazu assoziierte Garbe. Die Konstruktion definiert einen kontravarianten Funktor
	\[
		f^{-1}\colon \Sh(Y) \to \Sh(X),\; \mathcal{F} \mapsto f^{-1}\mathcal{F}. 
	\]
\end{defn}

\nextmark{Adjunktion zwischen f\textasciicircum-1 und f\_*}
\begin{prop}
\label{prop:3.5}
	Sei $f\colon X \to Y$ stetig. Dann existiert für jede Garbe $\mathcal{F}$ auf $X$ und jede Garbe~$\mathcal{G}$ auf $Y$ eine Bijektion
	\[
            \Hom_{\Sh(X)}(f^{-1}\mathcal{G},\mathcal{F}) \overset{\cong}{\longto} \Hom_{\Sh(Y)}(\mathcal{G},f_*\mathcal{F}),
	\]
	welche natürlich in $\mathcal{F}$ und $\mathcal{G}$ ist. Man sagt $f^{-1}$ ist \textbf{linksadjungiert} zu $f_*$ und $f_*$ ist \textbf{rechtsadjungiert} zu $f^{-1}$
\end{prop}

\nextmark{Tensorprodukt von OX-Moduln, inverse Bildgarbe bei OX-Moduln}
\begin{kons}
	\begin{enumerate}[i)]
		\item Sei $(X,\mco_X)$ ein geringter Raum und seien $\mcf$ und $\mcg$ zwei $\mco_X$-Moduln auf~$X$. Dann definiert
		\[
			U \mapsto \mcf(U)\otimes_{\mco_X(U)}\mcg(U)
		\]
		eine Prägarbe auf $X$. Sei $\mcf\otimes_{\mco_X} \mcg$ die dazu assoziierte Garbe. Man sieht sofort, dass das \textbf{Tensorprodukt} $\mcf\otimes_{\mco_X} \mcg$ ein $\mco_X$-Modul ist.
		\item Sei $f\colon(X,\mco_X)\to(Y,\mco_Y)$ ein Morphismus geringter Räume und sei $\mcf$ ein $\mco_X$-Modul auf~$X$. Wir haben in \ref{defn:2.5} die direkte Bildgarbe $f_*\mcf$ definiert. Offenbar ist $f_*\mcf$ ein $f_*\mco_X$-Modul.
		\begin{proof}
			Sei $V$ offen in $Y$, dann gilt $f_*\mcf(V) = \mcf(f^{-1}V)$ und dies ist ein $f_*\mco_X(V) = \mco_X(f^{-1}V)$-Modul.
		\end{proof}
		Nach der Definition von Morphismen geringter Räume gibt es einen Homomorphismus $f^{\#}\colon \mco_Y \to f_*\mco_X$ von Ringgarben. Durch Verknüpfung sehen wir, dass die direkte Bildgarbe ein $\mco_Y$-Modul auf $Y$ ist. Beachte, dass $f_*$ ein kovarianter Funktor ist.
		\item Sei $f\colon (X,\mco_X) \to (Y,\mco_Y)$ ein Morphimus geringter Räume und $\mcg$ ein $\mco_Y$-Modul. Wir betrachten den geringten Raum $(X, f^{-1}\mco_Y)$, wobei die inverse Bildgarbe durch die zur Prägarbe
		\[
                        U \mapsto \varinjlim_{\substackclap{f(U)\subseteq V\\V\subseteq Y \text{ offen}}} \, \mco_Y(V)
		\]
		assoziierte Garbe definiert ist. Beachte, dass $f^{-1}\mco_Y$ eine Ringgarbe ist. Analog sei $f^{-1}\mcg$ die zur Prägrabe
		\[
                        U \mapsto \varinjlim_{\substackclap{f(U)\subseteq V\\V\subseteq Y \text{ offen}}} \, \mathcal{G}(V)
		\]
		assoziierte Garbe. Dann erhalten wir $f^{-1}\mcg$ als $f^{-1}\mco_Y$-Modul. Wieder haben wir einen Homomorphismus $f^{\#}\colon \mco_Y \to f_*\mco_X$ von Ringgarben. Mit Hilfe der Adjunktion aus Proposition~\ref{prop:3.5} erhalten wir einen Homomorphismus $f^{-1}\mco_Y \to \mco_X$ von Ringgarben. Wir definieren das \textbf{inverse Bild} der Modulgarbe $\mcg$ als
		\[
			f^*\mcg \coloneqq f^{-1}\mcg \otimes_{f^{-1}\mco_Y}\mco_X.
		\]
		Indem wir von rechts mit $\mco_X$ tensorieren, erhalten wir tatsächlich einen $\mco_X$-Modul.
                \begin{proof}[Beweisskizze]
			Aus der Modultheorie ist folgendes bekannt: Ist $M$ ein $A$-Modul und $\varphi\colon A \to B$ ein Ringhomomorphismus, dann wird $B$ durch
			\[
				a \cdot m \coloneqq \varphi(a) \cdot m
			\]
			zu einem $A$-Modul. Auf dem $A$-Modul $M\otimes_A B$ definieren wir eine $B$-Modulstruktur durch
			\[
				b\cdot(m\otimes a) \coloneqq m\otimes (a\cdot b).
			\]
			Dies machen wir genauso für Garben auf jeder offenen Menge
		\end{proof}
		Man beachtem dass $f^*$ ein kovarianter Funktor ist.
	\end{enumerate}
\end{kons}

\nextmark{Verträglichkeit von Tensorprodukt und inversem Bild mit Halmen}
\begin{lem}
	\begin{enumerate}[i)]
		\item Seien $(X,\mco_X)$ ein geringter Raum, $p\in X$ und $\mcf,\mcg$ zwei $\mco_X$-Moduln. Dann gilt
		\[
			(\mcf\otimes_{\mco_X}\mcg)_p = \mcf_p \otimes_{\mco_{X,p}} \mcg_p.
		\]
		\item Sei $f \colon (X, \mco_X) \to (Y, \mco_Y)$ ein Morphismus geringter Räume, $p \in X$ und $\mcg$ ein $\mco_Y$-Modul. Dann gilt
		\[
			(f^{-1}\mcg)_p = \mcg_{f(p)}
		\]
		und
		\[
			(f^*\mcg)_p = \mcg_{f(p)}\otimes_{\mco_{Y,f(p)}}\mco_{X,p},
		\]
		wobei $\mco_{Y,f(p)}\to \mco_{X,p}$ der Homomorphismus der Halme $f^{\#}_p$ aus Bemerkung~\ref{bem:2.9} ist.
		\begin{proof}
			Dies wird in den Übungen 3.2 und 3.3 gezeigt.
		\end{proof}
	\end{enumerate}
\end{lem}

\nextmark{Adjunktion zwischen f\textasciicircum* und f\_* (bei OX-Moduln)}
\begin{lem}
	Sei $f\colon (X,\mco_X) \to (Y,\mco_Y)$ ein Morphismus geringter Räume. Dann ist der Funktor $f^*$ linksdadjungiert zum Funktor $f_*$, das heißt für alle $\mco_X$-Moduln $\mcf$ und alle $\mco_Y$-Moduln~$\mcg$ gilt
	\[
		\Hom_{\mco_X}(f^*\mcg,\mcf) \cong \Hom_{\mco_Y}(\mcg,f_*\mcf)
	\]
	natürlich in $\mcf$ und $\mcg$.
	\begin{proof}[Beweisskizze]
		Dies beruht auf der Adjunktion aus Proposition~\ref{prop:3.5} und dem Tensorieren mit $\mco_X$.
	\end{proof}
\end{lem}

%%% Local Variables: 
%%% mode: latex
%%% TeX-master: "algebraische_geometrie_2"
%%% End: 
%!TEX root = algebraische_geometrie_2.tex
% vim: tw=0

\chapter{Affine Schemata}

Zu jedem Ring $A$ (kommutativ und mit Eins) betrachten wir das Spektrum $\Spec(A)$ der Primideale, das in natürlicher Weise eine Topologie besitzt. Durch Lokalisierung von $A$ erhalten wir eine Garbe $\mco_{\Spec(A)}$ auf $\Spec(A)$ und damit einen lokal gringten Raum $(\Spec(A),\mco_{\Spec(A)})$. Im folgenden Kapitel~5 werden dies die Bausteine für Schemata sein. Affine Schemata sind ähnlich wie affine Varietäten aus der Algebraischen Geometrie I, mit dem Unterschied, dass Primideale statt Maximalideale als Punkte und beliebige Ringe zugelassen werden.

\begin{defn}
	Für $M\subseteq A$ sei $V(M)\coloneqq\{\mfp \in \Spec(A) \mid \supseteq M \}$. Dies entspricht der Nullstellenmenge aus der Algebraischen Gerometrie I.
\end{defn}

\begin{lem}
\label{lem:4.2}
	\begin{enumerate}[i)]
		\item Sei $\mfa\coloneqq \langle M \rangle$ das von $M\subseteq A$ erzeugte Ideal. Dann gilt $V(\mfa) = V(M)$.
		\item Es gilt $V(\{0\}) = \Spec(A)$ und $V(A) = \emptyset$.
		\item Für Ideale $\mfa,\mfb$ von $A$ gilt $V(\mfa) \cup V(\mfb) = V(\mfa \cdot \mfb)$.
		\item Für eine Familie $(\mfa_i)_{i\in I}$ von Idealen in $A$ gilt $\bigcap_{i\in I}V(\mfa_i) = V\left(\sum_{i \in I} \mfa_i\right)$.
	\end{enumerate}
	\begin{proof}
		i) und ii) sind trivial.
		\begin{enumerate}[i)]
		\setcounter{enumi}{2}
		\item Es gilt
		\[
			\mfa \cdot \mfb = \langle \{a \cdot b \mid a \in \mfa,\; b \in \mfb\} \rangle .
		\]
		\enquote{$\subseteq$}: Sei $\mfp \in V(\mfa)$. Dann gilt $\mfa \subseteq \mfp$ und damit $\mfa \cdot \mfb \subseteq \mfa \subseteq \mfp$, also $\mfp \in V(\mfa \cdot \mfb)$.

		\enquote{$\supseteq$}: Sei $\mfp \in V(\mfa \cdot \mfp)$. Dann gilt $\mfa \cdot \mfb \subseteq \mfp$. Falls $\mfb \subseteq \mfp$ ist, dann gilt $\mfp \in V(\mfb)$ und wir sind fertig. Sei also ohne Beschränkung der Allgemeinheit $\mfb \not\subseteq \mfp$. Dann gibt es ein $b \in \mfb \setminus \mfp$. Für jedes $a \in \mfa$ gilt dann $a \cdot b \in \mfa \cdot \mfb \subseteq \mfp$. Da $\mfp$ prim ist, gilt also schon $a \in \mfp$ und damit $\mfa \subseteq \mfp$, also $\mfp \in V(\mfa)$.
		\item Dies ist einfach nachzurechnen.
		\end{enumerate}
	\end{proof}
\end{lem}

\begin{defn}[Zariski-Topologie auf $\Spec(A)$]
	Wir definieren eine Teilmenge von $\Spec(A)$ als abgeschlossen, wenn sie die Form $V(\mfa)$ für ein Ideal $\mfa$ von $A$ hat. Eine Teilmenge $U$ von $\Spec(A)$ heißt dann offen, wenn $\Spec(A)\setminus U$ abgeschlossen ist. Nach Lemma~\ref{lem:4.2} definiert dies eine Topologie auf $\Spec(A)$, die wir \textbf{Zariski-Topologie} nennen.
\end{defn}

\begin{prop}
\label{prop:4.4}
	Für $Y\subseteq \Spec(A)$ definieren wir das Verschwindungsideal $I(Y) \coloneqq \bigcap_{\mfp \in Y} \mfp$.
	\begin{enumerate}[i)]
		\item Für ein Ideal $\mfa$ von $A$ gilt $I(V(\mfa)) = \sqrt{\mfa}$.
		\item Für $Y \subseteq \Spec(A)$ gilt $\sqrt{I(Y)}=I(Y)$ und $\overline{Y} = V(I(Y))$.
		\item Die Abbildungen
		\begin{center}
			\begin{tikzcd}
				\{\text{abgeschlossene Teilmengen in } \Spec(A)\} \arrow[shift left=1.1ex]{r}{I} & \{\text{Ideale }\mfp \text{ in } A \text{ mit } \sqrt{\mfa} = \mfa\} \arrow[shift left=1.1ex]{l}{V}
			\end{tikzcd}
		\end{center}
		sind bijektiv, zueinander invers und inklusionsumkehrend.
		\item $Y \subseteq \Spec(A)$ ist genau dann irreduzibel, wenn $I(Y)$ ein Primideal ist.
		\item Die Korrespondenz aus iii) induziert eine Bijektion
		\begin{center}
			\begin{tikzcd}
				\{\text{irreduzible abgeschlossene Teilmengen in } \Spec(A)\} \arrow[shift left=1.1ex]{r}{I} & \Spec(A) \arrow[shift left=1.1ex]{l}{V}
			\end{tikzcd}
		\end{center}
	\end{enumerate}
	\begin{proof}
		Wir benutzen
		\[
			I(V(\mfa)) = \bigcap_{\mfp \in V(\mfa)}\mfp = \bigcap_{\substack{\mfp \supseteq \mfa\\\mfp \in \Spec(A)}}\mfp = \sqrt{\mfa}.
		\]
		Dann folgen die Behauptungen analog wie bei affinen Varietäten.
	\end{proof}
\end{prop}

\begin{defn}
	Sei $X$ ein topologischer Raum.
	\begin{enumerate}[i)]
		\item Ein Punkt $p \in X$ heißt \textbf{abgeschlossen}, wenn $\{p\}$ abgeschlossen ist.
		\item Ein Punkt $p \in X$ heißt \textbf{generischer Punkt}, wenn $\overline{\{p\}} = X$ gilt.
		\item Ein Punkt $q \in X$ heißt \textbf{Spezialisierung} von $p \in X$, wenn $q \in \overline{\{p\}}$ ist.
	\end{enumerate}
\end{defn}

\begin{lem}
	Sei $X$ ein topologischer Raum.
	\begin{enumerate}[i)]
		\item Ist $X$ hausdorffsch, so ist jeder Punkt abgeschlossen.
		\item Existiert ein generischer Punkt in $X$, dann ist $X$ irreduzibel.
	\end{enumerate}
	\begin{proof}
		\begin{enumerate}[i)]
			\item Dies ist einfach zu zeigen.
			\item Sei $p$ ein generischer Punkt von $X$ und $X = X_1 \cup X_2$, wobei $X_1$ und $X_2$ abgeschlossen sind. Wir müssen zeigen, dass $X_1=X$ oder $X_2=X$ gilt. Es gilt $p \in X_i$ für ein $i \in \{1,2\}$ und damit
			\[
				X = \overline{\{p\}} \subseteq X_i.
			\]
		\end{enumerate}
	\end{proof}
\end{lem}

\begin{prop}
\label{prop:4.7}
	Sei $X = \Spec(A)$.
	\begin{enumerate}[i)]
		\item Für $\mfp \in X$ gilt $V(\mfp) = \overline{\{\mfp\}}$. Für $\mfp, \mfq \in X$ gilt insbesondere $\mfp \subseteq \mfq$ genau dann, wenn $\mfq \in \overline{\{\mfp\}}$ gilt.
		\item Unter der Korrespondenz aus Proposition~\ref{prop:4.4} iii) entsprechen die abgeschlossenen Punkte von $\Spec(A)$ genau den Maximalideal von $A$.
		\item Für $Y \subseteq X$ irreduzibel und abgeschlossen gilt $Y = \overline{\{\mfp\}}$ für $\mfp = I(Y)$. Damit ist $\mfp$ der nach i) eindeutig bestimmte generische Punkt von $Y$.
		\item Ist $A$ ein noetherscher Ring, so ist $X$ ein noetherscher topologischer Raum.
		\item Sei $A$ ein noetherscher Ring. Dann entsprechen die irreduziblen Komponenten von $\Spec(A)$ unter der Korrespondenz aus Proposition~\ref{prop:4.4} iii) genau den minimalen Primidealen von $A$.
		\item In einem noetherschen Ring existieren nur endliche viele minimale Primideale.
	\end{enumerate}
	\begin{proof}
		Dies wird in Übung~4.1 gezeigt.
	\end{proof}
\end{prop}

\begin{prop}
\label{prop:4.8}
	Für $f \in A$ sei $V(f) \coloneqq V(\langle f \rangle)$ und $D(f)\coloneqq \Spec(A) \setminus V(f)$.
	\begin{enumerate}[i)]
		\item Für eine Familie $(f_i)_{i\in I}$ in $A$ und $g \in A$ gilt:
		\[
			D(g) \subseteq \bigcup_{i\in I}D(f_i) \Leftrightarrow g \in \sqrt{\langle \{f_i \mid i \in I\}\rangle}
		\]
		\item Die Mengen $(D(f))_{f\in A}$ bilden eine Basis der Zariski-Topologie.
		\item Versehen wir $D(f)$ mit der induzierten Topologie, so ist $D(f)$ quasikompakt. Insbesondere ist $\Spec(A)=D(1)$ quasikompakt.
	\end{enumerate}
	\begin{proof}
		\begin{enumerate}[i)]
			\item Es gilt:
			\begin{align*}
				&D(g) \subseteq \bigcup_{i\in I} D(f_i)\\
				\Longleftrightarrow \quad & V(g) \supseteq \bigcap_{i \in I} V(f_i) = V(\langle\{f_i\mid i \in I\}\rangle)\\
				\Longleftrightarrow \quad & \sqrt{\langle g \rangle} \subseteq \sqrt{\langle\{f_i\mid i \in I\}\rangle}\\
				\Longleftrightarrow \quad & g \in \sqrt{\langle\{f_i\mid i \in I\}\rangle}
			\end{align*}
			\item Sei $\mfa$ ein Ideal und $\mfp \in U \coloneqq \Spec(A) \setminus V(\mfa)$. Zu zeigen ist, dass es ein $f \in A$ mit $\mfp \in D(f) \subseteq U$ gibt (Basiseigenschaft). Wegen $\mfp \notin V(\mfa)$ gibt es ein $f \in \mfa \setminus \mfp$ und man sieht leicht, dass dieses $f$ das Gewünschte liefert.
			\item Quasikompakt heißt, dass jede offene Überdeckung $(V_i)_{i\in I}$ von $D(f)$ eine endliche Teilüberdeckung hat, das heißt es gibt ein endloiches $I_0 \subseteq I$ mit $\bigcup_{i\in I_0} V_i \supseteq D(f)$. Nach ii) bilden die Mengen $D(f)$ eine Basis, also sei ohne Beschränkung der Allgemeinheit $V_i = D(f_i)$ für ein $f_i \in A$. Nun gilt:
			\begin{align*}
				&D(f) \subseteq \bigcup_{i \in I} D(f_i)\\
				\overset{\text{i)}}{\Longleftrightarrow} \quad & f \in \sqrt{\langle \{f_i \mid i \in I\}\rangle}\\
				\Longleftrightarrow \quad & \exists\; m \ge 1,\; f^m = \sum_{i\in I_0} a_i f_i \in \langle \{f_i \mid i \in I\}\rangle \text{ für ein endliches }I_0 \subseteq I, a_i \in A\\
				\Longleftrightarrow \quad & \exists \text{ endliches } I_0 \subseteq I,\; f \in \sqrt{\langle\{f_i\mid i \in I_0\}\rangle}\\
				\Longleftrightarrow \quad & D(f) \subseteq \bigcup_{i \in I_0} D(f_i)
			\end{align*}
			Also ist $D(f)$ quasikompakt. Insbesondere ist $D(1) = \Spec(A)$ quasikompakt.
		\end{enumerate}
	\end{proof}
\end{prop}

\begin{prop}
	Sei $\varphi\colon A \to B$ ein Ringhomomorphismus. Dann gilt:
	\begin{enumerate}[i)]
		\item Die Abbildung $f\colon X = \Spec(B) \to Y = \Spec(A),\; \mfp \mapsto \varphi^{-1}(\mfp)$ ist stetig. Oft bezeichnen wir $f$ auch mit $\Spec(\varphi) = f$. Weiter gelten folgende Identitäten:
		\begin{alignat*}{2}
			f^{-1}(V(M)) &= V(\varphi(M)) \quad &&\forall \;M \subseteq A\\
			f^{-1}(D(a)) &= D(\varphi(a)) \quad &&\forall \;a \in A\\
			\overline{f(V(\mfb))} &= V(\varphi^{-1}(\mfb)) \quad &&\forall\; \mfb \text{ Ideal in } B
		\end{alignat*}
		\item Ist $\varphi$ surjektiv und $\mfa \coloneqq \ker(\varphi)$, dann definiert $f$ einen Homöomorphismus
		\[
			\Spec(B) \overset{\sim}{\longto} V(\mfa).
		\]
	\end{enumerate}
	
	\begin{proof}
		Dies wird in Aufgabe~4.2 gezeigt.
	\end{proof}
\end{prop}

\begin{eri}
	Sei $S \subseteq A$ abgeschlossen unter Multiplikation und $1 \in S$. Dann ist die Lokalisierung $A_S$ definiert als die Menge der Äquivalenzklassen unter folgender Äquivalenzrelation auf $A \times S$:
	\[
		(a,s)\sim(a',s') \Longleftrightarrow \exists\; t \in S \text{ mit } t(s'a-sa') = 0
	\]
	Die Äquivalenzklasse von $(a,s)$ wird mit $\frac{a}{s}$ bezeichnet. $A_S$ ist ein Ring.
	\begin{enumerate}[i)]
		\item Für $\mfp \in \Spec(A)$ sei $A_{\mfp}\coloneqq A_S$ mit $S=A\setminus \mfp$.
		\item Für $f \in A$ sei $A_f \coloneqq A_S$ mit $S = \{f^m \mid m \in \N \}$.
	\end{enumerate}
\end{eri}

\begin{kons}[Lokal geringter Raum auf $\Spec(A)$]
\label{kons:4.11}
	\begin{enumerate}[i)]
		\item Sei $U$ offen in $\Spec(A)$. Dann bezeichnen wir mit $\mco(U)$ die Menge der Funktionen $s\colon U \to \coprod_{\mfp \in U}A_{\mfp}$, die folgende Eigenschaften erfüllen:
		\begin{enumerate}[a)]
			\item Für alle $\mfp \in U$ gilt $s(p) \in A_{\mfp}$
			\item Für alle $\mfp \in U$ gibt es eine offene Umgebung $V$ von $\mfp$ in $U$ und $a,f\in A$ mit $V \subseteq D(f)$ und $s(\mfq) =  \frac{a}{f} \in A_\mfq$ für alle $\mfq \in V$.
		\end{enumerate}
		\item Aufgrund der lokalen Natur der Axiome a) und b) sieht man, dass $U \mapsto \mco(U)$ eine Garbe von Ringen auf $\Spec(A)$ definiert. Dabei sind die Addition und Multiplikation von solchen Funktionen punktweise unter Benutzung der entsprechenden Operation in den $A_{\mfp}$ definiert.
	\end{enumerate}
\end{kons}

\begin{prop}
\label{prop:4.12}
	\begin{enumerate}[i)]
		\item Für alle $\mfp \in \Spec(A)$ gibt es einen kanonischen Isomorphismus
		\[
			\mco_{\mfp} \overset{\sim}{\longto} A_{\mfp}
		\]
		von Ringen, wobei $\mco_\mfp$ der Halm der Garbe $\mco$ in $\mfp$ ist. Insbesondere ist $\mco_\mfp$ ein lokaler Ring und damit ist $(\Spec(A),\mco)$ ein lokal geringter Raum.
		\item Für alle $f \in A$ gibt es einen kanonischen Isomorphismus
		\[
			A_f \overset{\sim}{\longto}\mco_{(D(f))}
		\]
		von Ringen.
		\item Es gilt $\mco(\Spec(A)) = A$.
	\end{enumerate}
	\begin{proof}
		\begin{enumerate}[i)]
			\item Es gilt
			\[
				\mco_\mfp = \varinjlim_{\substackclap{\mfp \in U\\U \text{ offen}}} \, \mco(U),
			\]
			das heißt ein Element von $\mco_\mfp$ ist repräsentiert durch $(U,s)$, wobei $U$ offen mit $\mfp \in U$ und $s \in \mco(U)$ ist. Weiter gilt $[(U,s)] = [(U',s')] \in \mco_\mfp$ genau dann, wenn es ein offenes $V \subseteq U \cap U'$ gibt mit $\mfp \in V$ und $s\vert_V = s'\vert_V$. Sei $\mfp \in \Spec(A)$ und $U$ eine offene Umgebung von $\mfp$. Für $U$ offen in $\Spec(A)$ betrachten wir
			\[
			 	\mco(U) \to A_\mfp,\; s \mapsto s(\mfp).
			\]
			Nach a) sind dies wohldefinierte Ringhomomorphismen. Weiter sind diese Homomorphismen verträglich mit Einschränkungen auf kleinere Umgebungen $V$. Also wird ein Ringhomomorphismus $\varphi\colon\mco_\mfp \to A_\mfp,\; [(U,s)] \mapsto s(\mfp)$ induziert.
			Sei $\frac{a}{f}\in A_\mfp$, also $f \notin \mfp,\; a \in A$. Dies definiert einen Schnitt
			\[
				s = \frac{a}{f}\colon D(f) \to \coprod_{\mfq \in D(f)} A_\mfq,\; \mfq \mapsto s(\mfq) \coloneqq \frac{a}{f} \in A_\mfq.
			\]
			Offenbar gilt $\varphi(s) = \frac{a}{f}\in A_\mfp$, also ist $\varphi$ surjektiv.

			Sei $U$ eine Umgebung von $\mfp$ und seien $s,t \in \mco(U)$ mit $s(\mfp) = t(\mfp)$, das heißt
			\[
				\varphi([(U,s)]) = \varphi([(U,t)]).
			\]
			Wir wollen nun $s=t \in \mco_\mfp$ zeigen. Indem wir die Umgebung von $\mfp$ gegebenenfalls verkleinern, dürfen wir nach b) annehmen, dass es $a,b,f,g \in A$ mit $f,g \notin \mfp$ gibt mit $s = \frac{a}{f} \in \mco(U)$ und $t=\frac{b}{g}\in \mco(U)$ gibt. Dann gilt:
			\begin{align*}
				& s(\mfp) = \varphi(s) = \varphi(t) = t(\mfp)\\
				\Longrightarrow \quad & \frac{a}{f} = \frac{b}{g} \in A_\mfp\\
				\Longrightarrow \quad & \exists\; h \in A \setminus \mfp \text{ mit } h(ga-fb)=0
			\end{align*}
			Weiter gilt $\frac{a}{f}=\frac{b}{g} \in A_\mfq$ für alle $\mfq \in \Spec(A)$ mit $f,g,h \notin \mfq$. Die Menge dieser $\mfq$ ist gleich der offenen Menge $W \coloneqq D(f)\cap D(g) \cap D(h) \ni \mfp$. Also gilt $s\vert_{W\cap U} = t\vert_{W\cap U}$ und damit $s=t \in \mco_\mfp$ nach der Definition von $\mco_\mfp$. Also ist $\varphi$ injektiv.

			Damit ist $\varphi$ ein kanonischer Isomorphismus.

			\item Wir definieren
			\begin{align*}
				\psi_f\colon A_f &\to \mco(D(f))\\
				\frac{a}{f^n} & \mapsto \left(s\colon D(f) \to \coprod_{\mfp \in D(f)}A_\mfp,\; \mfp \mapsto s(\mfp)\coloneqq \frac{a}{f^n}\in A_\mfp\right).
			\end{align*}
			Offenbar ist $\psi_f$ ein wohldefinierter Ringhomomorphismus.

			Sei $\psi_f\left(\frac{a}{f^n}\right) = \psi_f \left(\frac{b}{f^m}\right)$. Dann gilt $\frac{a}{f^n}=\frac{b}{f^m} \in A_\mfp$ für alle $\mfp \in D(f)$. Sei
			\[
				\mfa\coloneqq \Ann_A(f^m a-f^n b) = \{g \in A \mid g \cdot (f^m a - f^n b) = 0\}.
			\]
			Beachte, dass $\mfa$ ein Ideal in $A$ ist. Nun gibt es ein $h \in A \setminus \mfp$ mit $h(f^m a -f^n b) = 0$, also $h \in \mfa \setminus \mfp$. Also gilt $\mfa \not \subseteq \mfp$, das heißt $\mfp \notin V(\mfa)$. Dies gilt für alle $\mfp \in D(f)$, also gilt $D(f) \cap V(\mfa) = \emptyset$. Damit gilt $V(\mfa) \subseteq V(f)$. Mit Proposition~\ref{prop:4.4} folgt wegen $f \in \sqrt{\mfa}$, dass es ein $k \ge 1$ mit $f^k \in \mfa$ gibt. Dann gilt
			$f^k(f^m a -f^n b) = 0$ und damit $\frac{a}{f^n} = \frac{b}{g^m} \in A_f$. Also ist $\psi_f$ injektiv.

			Sei $s \in \mco(D(f))$. Nach Proposition~\ref{prop:4.8} ii) und der Definition von $\mco$ in Konstruktion~\ref{kons:4.11} gibt es eine Familie $(h_i)_{i\in I}$ in $A$ mit
			\begin{equation*}
			\label{eq:4.12.1}
				D(f) = \bigcup_{i\in I} D(h_i) \tag{$\star$}
			\end{equation*}
			und $a_i,g_i \in A$ mit $D(h_i) \subseteq D(g_i)$, sodass für alle $i \in I$
			\[
				s\vert_{D(h_i)} = \frac{a_i}{g_i} \in \mco(D(h_i))
			\]
			gilt. Aus $D(h_i) \subseteq D(g_i)$ folgt mit Proposition~\ref{prop:4.8}, dass es $n_i \ge 1$ und $c_i \in A$ mit $h_i^{n_i} = c_i g_i$ gibt. Dann gilt
			\[
				\frac{a_i}{g_i} = \frac{c_i a_i}{h_i^{n_i}} \in \mco(D(h_i)) = \mco(D(h_i^{n_i})).
			\]
			Wir ersetzen $h_i$ durch $h_i^{n_i}$ und $a_i$ durch $c_i a_i$ und erhalten
			\[
				s\vert_{D(h_i)} = \frac{a_i}{h_i} \in \mco(D(h_i)).
			\]
			Da $D(f)$ nach Proposition~\ref{prop:4.8} quasikompakt ist, gibt es $h_1,\ldots,h_r$ mit
			\[
				D(f) \overset{\eqref{eq:4.12.1}}{=}  D(h_1)\cup \cdots \cup D(h_r).
			\]
			Wir fassen $\frac{a_i}{h_i}$ und $\frac{a_j}{h_j}$ als Elemente von $A_{h_ih_j}$ auf. Wegen $D(h_i)\cap D(h_j) = D(h_ih_j)$ gilt 
			\[
				\psi_{h_ih_j}\left(\frac{a_i}{h_i}\right) = \psi_{h_ih_j}\left(\frac{a_j}{h_j}\right) = s\vert_{D(h_ih_j)} \in \mco(D(h_ih_j)).
			\]
			Aus der Injetivität von $\psi_{h_ih_j}$ folgt $\frac{a_i}{h_i} = \frac{a_j}{h_j} \in A_{h_ih_j}$. Deswegen gibt es $n_{ij}\ge 1$ mit
			\begin{equation*}
			\label{eq:4.12.2}
				(h_ih_j)^{n_{ij}}(h_ja_i-h_ia_j) = 0 \quad \forall \;1 \le i,\; j \le r. \tag{$\star\star$}
			\end{equation*}
			Sei $n \coloneqq \max\{n_{ij}\mid 1\; \le i, j \le r\}$, $\tilde{a}_i\coloneqq h_i^na_i$, $\tilde{h}_i \coloneqq h_i^{n+1}$. Wieder gilt $D(\tilde{h}_i) = D(h_i)$. Aus \eqref{eq:4.12.2} folgt
			\[
				h_j^{n+1}(h_ia_i) - h_i^{n+1}(h_j^na_j) = 0 \quad \forall \;1 \le i,\; j\le r.
			\]
			Deswegen gilt
			\[
				s\vert_{D(\tilde{h}_i)} = \frac{\tilde{a}_i}{\tilde{h}_i} \text{ und } \tilde{h}_i\tilde{a}_j = \tilde{a}_i\tilde{h}_j \quad \forall \;1 \leq i,\; j \le r.
			\]
			Aus \eqref{eq:4.12.1} folgt mit Proposition~\ref{prop:4.8} i), dass es ein $m \ge 1$ mit $f^m = \sum_{i=1}^r b_i \tilde{h}_i$ für gewisse $b_i \in A$ gibt. Wir setzen $a \coloneqq \sum_{i=1}^r b_i \tilde{a}_i$. Für $1 \le j \le r$ gilt
			\[
				\tilde{h}_j a = \sum_{i=1}^r\tilde{h}_jb_i\tilde{a}_i = \sum_{i=1}^r\tilde{h}_ib_i \tilde{a}_j = f^m \tilde{a}_j
			\]
			und damit folgt
			\[
				\frac{\tilde{a}_j}{\tilde{h}_j} = \frac{a}{f^m} \in A_{\tilde{h}_j} \overset{\psi_{\tilde{h}_j}}{\hookrightarrow} \mco(D(\tilde{h}_j))
			\]
			und damit $\psi_f\left(\frac{a}{f}\right) = s$ auf jedem $D(\tilde{h}_j)$. Da $\mco$ eine Garbe ist, gilt dies auch auf $D(f)$. Damit ist $\psi_f$ surjektiv.

			Insgesamt ist $\psi_f$ ein Isomorphismus.
			\item Dies folgt aus ii) mit $f=1$.
		\end{enumerate}
	\end{proof}
\end{prop}

\begin{bem*}
	Wir nennen $\mco$ die Garbe der regulären Funktionen auf $\Spec(A)$.
\end{bem*}

\begin{defn}
	Der lokal geringte Raum $(\Spec(A), \mco)$ heißt \textbf{Spektrum} von $A$.
\end{defn}

\begin{prop}
	Seien $A, B$ Ringe, $X \coloneqq \Spec(B)$ und $Y \coloneqq \Spec(A)$.
	\begin{enumerate}[i)]
		\item Ein Ringhomomorphismus $\varphi\colon A \to B$ induziert eine stetige Abbildung
		\[
			(f = \Spec(\varphi))\colon X \to Y,\; \mfp \mapsto \varphi^{-1}(\mfp),
		\]
		einen Garbenhomomorphismus
		\[
			f^{\#}\colon \mco_X \to f_*(\mco_Y)
		\]
		und einen Morphismus
		\[
			(f,f^{\#})\colon (X,\mco_X)\to (Y,\mco_Y)
		\]
		lokal geringter Räume.
		\item Ein Morphismus lokal geringter Räume $(f,f^{\#})\colon (X,\mco_X)\to(Y,\mco_Y)$ induziert einen Ringhomomorphismus
		\[
			(\varphi = \Gamma(Y,f^{\#}))\colon (A = \mco(Y)) \to (B=\mco(X)).
		\]
		\item Es gibt eine bijektive Korrespondenz
		\begin{align*}
			\phi\colon \Hom_{\Ring}(A,B) &\to \Hom_{\lgR}((\Spec(B),\mco_{\Spec(B}),(\Spec(A),\mco_{\Spec(A}))\\
			\varphi & \mapsto ((f=\Spec(\varphi)),f^{\#})
		\end{align*}
		und die Umkehrabbildung $\Psi$ ist gegeben durch
		\[
			(f,f^{\#})\mapsto \Gamma(\Spec(A),f^{\#}).
		\]
	\end{enumerate}
	\begin{proof}
		Dies wird in Aufgabe~5.1 gezeigt.
	\end{proof}
\end{prop}

\begin{defn}
	Ein \textbf{affines Schema} ist ein lokal geringter Raum $(X,\mco_X)$, der für einen Ring $A$ zu $(\Spec(A),\mco_{\Spec(A)})$ isomorph ist.
\end{defn}

\begin{lem}
\label{lem:4.16}
	Sei $f \in A$. Dann existiert ein kanonischer Isomorphismus
	\[
		(\Spec(A_f),\mco_{\Spec(A_f)}) \overset{\sim}{\longto} (D(f),\mco_{\Spec(A)}\vert_{D(f)})
	\]
	von lokal geringten Räumen. Insbesondere ist $(D(f),\mco_{\Spec(A)}\vert_{D(f)})$ ein affines Schema.
	\begin{proof}
		Dies folgt leicht aus Proposition~\ref{prop:4.12} und wird in Aufgabe~5.2 gezeigt.
	\end{proof}
\end{lem}

\begin{defn}
	Sei $(X,\mco_X)$ ein lokal geringter Raum und $p \in X$. Dann ist $\mco_{X,p}$ ein lokaler Ring mit eindeutigem Maximalideal $\mfm_p$. Also ist
	\[
		\kappa(p) \coloneqq \mco_{X,p}/\mfm_p
	\]
	ein Körper, den wir \textbf{Restklassenkörper} von $(X,\mco_X)$ in $p$ nennen.
\end{defn}

%!TEX root = algebraische_geometrie_2.tex
% vim: tw=0 noet sts=8 sw=8

%%% Local Variables: 
%%% mode: latex
%%% TeX-master: "algebraische_geometrie_2"
%%% End: 

\chapter{Schemata}

Schemata sind die Hauptobjekte der algebraischen Geometrie. Sie verallgemeinern die (quasiprojektiven) Varietäten aus der algebraischen Geometrie I.

\nextmark{Schema, Morphismus von Schemata}
\begin{defn}
\label{defn:5.1}
	\begin{enumerate}[i)]
		\item Ein \textbf{Schema} ist ein lokal geringter Raum $(X,\mco_X)$, wobei $X$ eine offene Überdeckung $(U_i)_{i\in I}$ hat, die die Eigneschaft besitzt, dass $(U_i,\mco_{X}\vert_{U_i})$ für alle $i \in I$ ein affines Schema ist. Wir nennen $X$ den \textbf{unterliegenden topologischen Raum} und $\mco_X$ die \textbf{Strukturgarbe}. Oft wird das Schema $(X,\mco_X)$ einfach mit $X$ abgekürzt, obwohl der topologische Raum das Schema nicht bestimmt.
		\item Ein \textbf{Morphismus} von Schemata ist ein Morphismus von lokal geringten Räumen.
	\end{enumerate}
	Wir erhalten die Katergorie $\Sch$ der Schemata.
\end{defn}

\nextmark{offenes Unterschema, offene Unterschemata bilden Basis}
\begin{prop}
	Sei $X = (X,\mco_X)$ ein Schema.
	\begin{enumerate}[i)]
		\item Für $U$ offen in $X$ ist $(U,(\mco_U \coloneqq \mco_X\vert_U))$ ein Schema. Weiter hat man einen kanonischen Morphismus
		\[
			j\colon (U,\mco_U) \to (X,\mco_X).
		\]
		Wir nennen dies die \textbf{induzierte Schemastruktur} auf $U$ und sagen, dass $(U,\mco_U)$ ein \textbf{offenes Unterschema} von $X$ ist.
		\item Die affinen offenen Unterschemata von $X$ bilden eine Basis der Topologie von $X$.
	\end{enumerate}
	\begin{proof}
		Dies folgt aus Proposition~\ref{prop:4.8} und Lemma~\ref{lem:4.16}.
	\end{proof}
\end{prop}

\begin{bem}
	Ein offenes Unterschema eines affinen Schemas muss im Allgemeinen nicht wieder affin sein.
\end{bem}

\nextmark{Trennungsaxiome für top. Räume}
\begin{bem*}
	Sei $X$ ein topologischer Raum.
	\begin{enumerate}[i)]
		\item $X$ erfüllt das Axiom $T_0$, wenn es zu je zwei verschiedenen Punkten $x,y \in X$ eine offene Menge $U$ mit $x \in U$, $y\notin U$ oder $x \notin U$, $y \in U$ gibt. $X$ heißt dann $T_0$-Raum.
		\item $X$ erfüllt das Axiom $T_1$, wenn es zu je zwei verschiedenen Punkten $x,y \in X$ offene Mengen $U_x,U_y$ gibt mit $x \in U_x$, $y \notin U_x$ und $x \notin U_y$, $y \in U_y$. $X$ heißt dann $T_1$-Raum.
		\item $X$ erfüllt das Axiom $T_2$, wenn es zu je zwei verschiedenen Punkten $x,y \in X$ offene Mengen $U_x,U_y$ gibt mit $x \in U_x$, $y \in U_y$ und $U_x \cap U_y = \emptyset$. $X$ heißt dann $T_2$-Raum oder \textbf{Hausdorffraum}.
	\end{enumerate}
	Man sieht leicht, dass das Axiom $T_2$ das Axiom $T_1$ impliziert und dass das Axiom $T_1$ das Axiom $T_0$ impliziert.
\end{bem*}

\nextmark{Schemata sind T0, Existenz und Eindeutigkeit generischer Punkte}
\begin{prop}
\label{prop:5.4}
	Sei $(X, \mco_X)$ ein Schema.
	\begin{enumerate}[i)]
		\item Der topologische Raum $X$ ist ein $T_0$-Raum.
		\item Jede abgeschlossene irreduzible Teilmenge des topologischen Raums $X$ besitzt genau einen generischen Punkt.
	\end{enumerate}
	\begin{proof}
		\begin{enumerate}[i)]
			\item Seien $x\neq y \in X$. Nach Definition~\ref{defn:5.1} gibt es eine affine offene Umgebung $U \cong \Spec(A)$ von $x$. Falls $y \notin U$, dann sind wir fertig, sei also ohne Beschränkung der Allgemeinheit $y \in U$. Die Punkte $x,y$ sind durch Primideale $\mfp,\mfq$ von $A$ gegeben. Es gilt $\mfp \neq \mfq$, sei also ohne Beschränkung der Allgemeinheit $\mfp \not\subseteq \mfq$. Dann gibt es ein $f \in \mfp \setminus \mfq$, und somit ist $D(f)$ offen in $X$ mit $x \notin D(f)$, aber $y\in D(f)$.
			\item Wir zeigen zunächst die Existenz: Sei $V\subseteq X$ irreduzibel und abgeschlossen. Nach Definition~\ref{defn:5.1} gibt es ein offenes affines Unterschema $U \cong \Spec(A)$ mit $U \cap V \neq \emptyset$. Da $V$ abgeschlossen in $X$ ist, ist $U \cap V$ abgeschlossen in $U$ und weil $U$ offen in $X$ ist, ist $U \cap V$ offen in $V$. In Aufgabe~1.5 zur algebraischen Geometrie I haben wir gesehen, dass jede nicht leere, offene Teilmenge eines irreduziblen topologischen Raumes irreduzibel und dicht ist, also ist $U \cap V$ irreduzibel und dicht in $V$. Sei $\mfp \coloneqq I(U \cap V) \in \Spec(A)$. Nach Proposition~\ref{prop:4.7} iii) ist $\mfp$ ein generischer Punkt von $\Spec(A)$. Dieser entspricht einem Punkt $\eta$ in $U \subseteq X$. Es gilt $\overline{\eta}=U\cap V$ und damit ist $\eta$ dicht in $V$, also ein generischer Punkt von $V$.

			Nun zeigen wir die Eindeutigkeit: Seien $\eta_1,\eta_2$ generische Punkte von $V$. Wir nehmen an, dass $\eta_1 \neq \eta_1$. Dann gibt es ohne Beschränkung der Allgemeinheit ein offenes $U$ mit $\eta_1 \in U$ und $\eta_2 \neq U$. Dann gilt $\eta_2 \in V\setminus U \neq V$ und damit folgt $\overline{\eta_2} \subseteq V\setminus U \neq V$ im Widerspruch dazu, dass $\eta_2$ ein generischer Punkt von $V$ ist.
		\end{enumerate}
	\end{proof}
\end{prop}

\begin{bem*}
	Wir haben inbesondere gezeigt, dass jedes irreduzible affine Schema ein $T_0$-Raum ist. Weiter haben wir gesehen, dass jedes irreduzible affine Schema $X$ genau einen generischen Punkt $\eta$ hat.

	Ist $(X,\mco_X)$ ein Schema, das mindestens zwei Punkte enthält. Dann ist $X$ jedoch kein $T_1$-Raum.
	\begin{proof}
		Wir wählen $y = \eta,\; x \neq y$. Falls $U$ eine offene Umgebung von $x$ ist, so gilt $y \in U$, da $y$ ein generischer Punkt ist.		
	\end{proof}
\end{bem*}

\nextmark{Verkleben von Schema-Morphismen}
\begin{lem}
\label{lem:5.5}
	Seien $X,Y$ Schemata.
	\begin{enumerate}[i)]
		\item Sei $(U_i)_{i\in I}$ eine offene Überdeckung von $X$ und seien $f_i\colon U_i\to Y$ Morphismen mit
		\[
			f_i\vert_{U_i\cap U_j} = f_j\vert_{U_i\cap U_j} \quad \forall\; i,j \in I.
		\]
		Dann gibt es genau einen Morphismus $f \colon X \to Y$ mit $f\vert_{U_i} = f_i$ für alle $i \in I$.
		\item Die durch $U \mapsto \Hom_{\Sch}(U,Y)$ mit den Restriktionsabbildungen
		\[
			\Hom_{\Sch}(U,Y) \to \Hom_{\Sch}(V,Y)
		\]
		definierte Prägarbe von Mengen auf $X$ ist eine Garbe.
	\end{enumerate}
	\begin{proof}
		\begin{enumerate}[i)]
			\item Genauer ist ein Morphismus $(f,f^{\#})\colon (X,\mco_X) \to (Y,\mco_Y)$ von lokal geringten Räumen zu konstruieren und die Eindeutigkeit zu zeigen. Gegeben sind Morphismen
			\[
				(f_i,f_i^{\#})\colon (U_i,\mco_{U_i}) \to (Y,\mco_Y),
			\]
			die auf den Überlappungen übereinstimmen. Für alle $x \in X$ gibt es ein $i \in I$ mit $x \in U_i$. Nun definieren wir die stetige Abbildung
			\[
				f\colon X \to Y,\; x \mapsto f(x) \coloneqq f_i(x).
			\]
			Dies ist wohldefiniert, da die $f_i$ auf den Überlappungen der $U_i$ übereinstimmen. Wir konstruieren nun $f^{\#}\colon \mco_Y \to f_*\mco_X$. Für $V$ offen in $Y$ müssen wir einen Homomorphismus
			\[
				f^{\#}_V\colon \mco_Y(V) \to ((f_*\mco_X)(V) = \mco_X(f^{-1}V))
			\]
			konstruieren. Sei $s \in \mco_Y(V)$. Beachte, dass $f^{-1}V$ von den offenen Teilmegenen $U_i \cap f^{-1}V$ überdeckt wird. Wir betrachten die Abbildungen
			\begin{align*}
				f_{i,V}^{\#}\colon \mco_Y(V)&\to (({f_i}_*\mco_{U_i})(V) = \mco_X(U_i \cap f^{-1}V))\\
				s &\mapsto f^{\#}_{i,V}(s).
			\end{align*}
			Wegen der Vorraussetzung stimmen die $f_{i,V}^{\#}(s)$ auf den Überlappungen der Mengen $U_i\cap f^{-1}V$ überein. Wegen der Garbeneigenschaft gibt es ein eindeutiges $t \in \mco_X(f^{-1}V)$ mit $t\vert_{U_i\cap f^{-1}V} = f_{i,V}^{\#}(s)$ für alle $i \in I$. Wir definieren $f^{\#}_V(s) \coloneqq t$. Man sieht sofort, dass $f_V^{\#}$ das Gewünschte liefert und eindeutig ist.
			\item Dies folgt aus i), indem man $X \coloneqq U$ setzt.
		\end{enumerate}
	\end{proof}
\end{lem}

\nextmark{Morphismen X --> Spec(A) vs. Morphismen A --> OX(X)}
\begin{prop}
\label{prop:5.6}
	Sei $X$ ein Schema und $A$ ein Ring. Dann ist die Abbildung
	\begin{align*}
		\Hom_{\Sch}(X,\Spec(A)) &\to \Hom_{\Ring}(A,\mco_X(X))\\
		(f,f^{\#}) & \mapsto f^{\#}_{\Spec(A)}
	\end{align*}
	eine kanonische Bijektion.
	\begin{proof}
		Sei $\varphi\in \Hom_{\Ring}(A,\mco_X(X))$. Wir müssen nun zeigen, dass es genau ein $(f,f^{\#}) \in \Hom_{\Sch}(X,\Spec(A))$ mit $f^{\#}_{\Spec(A)} = \varphi$ gibt. Wir wählen eine offene Überdeckung $(U_i)_{i\in I}$ von $X$, wobei $U_i$ für alle $i \in I$ ein offenes Unterschema von $X$ ist. Nach Proposition~\ref{prop:4.14} gibt es genau ein $(f_i,f_i^{\#}) \in \Hom_{\Sch}(U_i,\Spec(A))$ mit $f^{\#}_{i,\Spec(A)} = \varphi_i$. Für jedes affine offene Unterschema $V$ von $U_i \cap U_j$ gilt $f_i\vert_V = f_j\vert_V$ für alle $i,j \in I$. Dies folgt aus der Eindeutigkeit in Proposition~\ref{prop:4.14}. Weil die Prägarbe der Morphismen in der Kategorie $\Sch$ schon eine Garbe ist, folgt
		\[
			f_i\vert_{U_i\cap U_j} = f_j\vert_{U_i\cap U_j},
		\]
		da es nach Lemma~\ref{lem:5.5} genau ein $(f,f^{\#}) \in \Hom_{\Sch}(X,\Spec(A))$ mit $f\vert_{U_i} = f_i$ für alle $i \in I$ gibt. Dies ergibt sofort die Existenz von $(f,f^{\#})$ mit Hilfe der Garbeneigenschaft der Morphismen. Die Eindeutigkeit ergibt sich aus der Konstruktion.
	\end{proof}
\end{prop}

\nextmark{S-Schema, Morphismen von S-Schemata}
\begin{defn}
\label{defn:5.7}
	\begin{enumerate}[i)]
		\item Sei $S$ ein Schema. Ein $S$\textbf{-Schema} ist ein Morphismus $f\colon X \to S$ von Schemata. Sei $g\colon X \to S$ ein weiteres $S$-Schema, dann setzen wir
		\[
			\Hom_{S}(X,Y)=\{h\colon X\to Y\mid h \in \Hom_{\Sch}(X,Y) \text{ mit } g \circ h = f\},
		\]
		das heißt $\Hom_{S}(X,Y)$ ist die Menge aller Morphismen $h \in \Hom_{\Sch}(X,Y)$, für die folgendes Diagramm kommutiert:
		\begin{center}
			\begin{tikzcd}
				X\arrow{rr}{h}\arrow{dr}[swap]{f} && Y\arrow{dl}{g}\\
				& S
			\end{tikzcd}
		\end{center}
		\item Sei $S=\Spec(A)$ für einen Ring $A$. Dann sagen wir $A$\textbf{-Schema} statt $S$-Schema.
	\end{enumerate}
\end{defn}

\begin{bem*}
    Alternativ zur Notation in Definition~\ref{defn:5.7} verwenden wir manchmal auch folgende Sprechweisen:
    Für ein $S$-Schema~$X$ sagen wir auch: $X$ ist ein \textbf{Schema über}~$S$. Ist $S=\Spec(A)$, so sagen
    wir für ein $A$-Schema~$X$ auch: $X$ ist ein \textbf{Schema über}~$A$.
\end{bem*}

\begin{bem*}
    Nach Proposition~\ref{prop:5.6} haben wir ein $A$-Schema $f \colon X \to \Spec(A)$ genau dann, wenn wir einen Ringhomomorphismus $f^{\#}_{\Spec(A)}\colon A \to \mco_X(X)$ haben, also wenn es eine $A$-Alge\-brastruktur auf $\mco_X(X)$ gibt.
\end{bem*}

\begin{bsp}
\label{bsp:5.8}
	\begin{enumerate}[i)]
		\item Setzte $A \coloneqq \Z$. Dann existiert für jedes Schema $X$ genau ein Morphismus $X \to \Spec(\Z)$ und somit ist $X$ ein kanonisches $\Z$-Schema, da genau ein Ringhomomorphismus $\Z \to \mco_X(X)$ existiert. Damit ist $\Spec(\Z)$ ein finales Objekt in der Kategorie der Schemata.
		\item Sei $X$ ein Schema, dann setzen wir $A \coloneqq \mco_X(X)$. Dann gibt es einen kanonischen Homomorphismus $A \to \mco_X(X)$, nämlich die Identität. Damit existiert nach Proposition~\ref{prop:5.6} ein kanonischer Morphismus $X \to \Spec(A)$, das heißt $X$ ist kanonisch ein $(A=\mco_X(X))$-Schema.
	\end{enumerate}
\end{bsp}

\nextmark{Verklebedatum}
\begin{defn}
\label{defn:5.9}
	Ein \textbf{Verklebedatum} von Schemata ist eine Familie von Tripeln der Form $(X_i,U_{ij},\varphi_{ij})_{i,j\in I}$, wobei $X_i$ für alle $i \in I$ ein Schema ist, $U_{ij}$ für alle $i,j \in I$ eine offene Teilmenge von $X_i$ ist und $\varphi_{ij}$ für alle $i,j \in I$ ein Isomorphismus
	\[
		\varphi_{ij} \colon (U_{ij}=(U_{ij},\mco_{X_i}\vert_{U_{ij}})) \simto (U_{ji}=(U_{ji},\mco_{X_j}\vert_{U_{ji}}))
	\]
	mit folgenden Eigenschaften ist:
	\begin{enumerate}[i)]
		\item Für alle $i,j \in I$ gilt $\varphi_{ij} = \varphi^{-1}_{ji}$, $U_{ii} = X_i$ und $\varphi_{ii}= \id$.
		\item Für alle $i,j,k \in I$ gilt $\varphi_{ij}(U_{ij}\cap U_{ik}) = U_{ji} \cap U_{jk}$.
		\item Für alle $i,j,k \in I$ gilt $\varphi_{ik} = \varphi_{jk}\circ \varphi_{ij}$ auf $U_{ij} \cap U_{ik}$.
	\end{enumerate}
\end{defn}

\begin{lem}[Verkleben von Schemata]
\label{lem:5.10}
	Sei ein Verklebedatum $(X_i,U_{ij},\varphi_{ij})_{i,j\in I}$ gegeben. Dann gibt es ein bis auf Isomorphie eindeutig bestimmtes Schema $X$ zusammen mit Morphismen $\psi_i \colon X_i \to X$, die für alle $i,j \in I$ folgende Eigenschaften erfüllen:
	\begin{enumerate}[i)]
		\item $\psi_i$ induziert einen Isomorphismus $\psi_i \colon X_i \to \psi_i(X_i)$ von $X_i$ auf das offene Unterschema $\psi_i(X_i)$ von $X$.
		\item Es gilt $\bigcup_{i \in I} \psi_i(X_i) = X$.
		\item Es gilt $\psi_i(U_{ij}) = \psi_i(X_i) \cap \psi_j(X_j)$.
		\item Es gilt $\psi_i = \psi_j \circ \varphi_{ij}$ auf $U_{ij}$.
	\end{enumerate}
	\begin{proof}
		Sei $\widetilde{X} \coloneqq \coprod_{i\in I}X_i$ versehen mit der disjunkten Vereinigungstopologie, das heißt die offenen Mengen in $\widetilde{X}$ haben die Form $\coprod_{i\in I} U_i$, wobei $U_i$ für alle $i \in I$ offen in $X_i$ ist. Wir definieren eine Äquivalenzrelation auf $\widetilde{X}$ wie folgt:
		\begin{align*}
			a \sim b \quad:\Longleftrightarrow\quad \exists \;i,j \in I \text{ mit } a \in U_i,\, b \in U_j \text{ und } \varphi_{ij}(a) = b .
		\end{align*}
                Sei $X \coloneqq \widetilde{X}/{\sim}$ der Raum der Äquivalenzklassen. Dann erhalten wir die Klassenabbildung
		\[
			\pi\colon \widetilde{X} \twoheadrightarrow X,\; a \mapsto [a].
		\]
		Wir versehen $X$ mit der \textbf{Quotiententopologie}, das heißt $U$ ist genau dann offen in $X$, wenn $\pi^{-1}(U)$ offen in $\widetilde{X}$ ist. $\pi$ wird damit eine stetige und offene Abbildung. Nach Definition ist somit $U$ genau dann offen in $X$, wenn $\psi_i^{-1}(U)$ für alle $i \in I$ offen in $X_i$ ist, wobei $\psi_i$ durch
                % FIXME/TODO: Wieso ist $\pi$ offen? (Nicht jede Quotientenabbildung ist offen ...)
		\[
			\psi_i\colon X_i \to X,\; a \mapsto [a]
		\]
		gegeben ist. Beachte, dass $\psi_i$ für alle $i \in I$ stetig und injektiv ist. Nach Konstruktion gelten nun ii), iii), iv). Ebenso folgt, dass $\psi_i$ einen Homöomorphismus $X_i \to \psi_i(X_i)$ induziert. Weiter sehen wir aufgrund der Charakterisierung offener Mengen in $X$ ein, dass $U_i\coloneqq \psi_i(X_i)$ offen in $X$ ist. Wir definieren die Garbe
		\[
			\mco_{U_i} \coloneqq {\psi_i}_*(\mco_{X_i})
		\]
		auf $U_i$. Wir erhalten einen Isomorphismus
		\[
			\psi_i\colon(X_i,\mco_{X_i})\simto (U_i,\mco_{U_i})	
		\]
		von lokal geringten Räumen. Also ist $(U_i,\mco_{U_i})$ ein Schema. Die Garbenisomorphismen
		\[
			\varphi^{\#}_{ij}\colon \mco_{X_j}\vert_{U_{ji}} \simto (\varphi_{ij})_*(\mco_{X_i}\vert_{U_{ij}})
		\]
		auf dem Verklebedatum liefern Garbenisomorphismen 
		\[
			\widetilde{\varphi}^{\#}_{ij}\colon \mco_{U_i}\vert_{U_i \cap U_j} \simto \mco_{U_j}\vert_{U_i \cap U_j}
		\]
		mit $\widetilde{\varphi}^{\#}_{ii} = \id$, die die sogenannte \textbf{Kozykelbedingung}
		\[
			\widetilde{\varphi}_{ik} = \widetilde{\varphi}_{jk} \circ \widetilde{\varphi}_{ij} \text{ auf } U_i \cap U_j \cap U_k
		\]
		erfüllen. Durch Verkleben der Garben $(U_i,\mco_{U_i})$ entlang der $\widetilde{\varphi}_{ij}$ erhält man eine Garbe $\mco_X$ auf~$X$ (siehe Lemma~\ref{lem:5.11}). Nach Konstruktion ist $(X,\mco_X)$ ein lokal geringter Raum. Da $(X,\mco_X)$ eine offene Überdeckung durch die affinen Schemata
		\[
			(U_i,\mco_{U_i}) \simto (X_i, \mco_{X_i})
		\]
		bestitzt, folgt die Existenz. Die Eindeutigkeit ergibt sich aus der Konstruktion.
	\end{proof}
\end{lem}

\begin{lem}[Verkleben von Garben]
\label{lem:5.11}
	Sein $X$ ein topologischer Raum mit einer offenen Überdeckung $(U_i)_{i\in I}$. Für alle $i \in I$ sei $\mcf_i$ eine Garbe auf $U_i$ und für alle $i,j \in I$ sei
	\[
		\varphi_{ij}\colon \mcf_i\vert_{U_i\cap U_j} \simto \mcf_j\vert_{U_i\cap U_j}
	\]
	ein Isomorphismus von Garben mit folgenden Eigenschaften:
	\begin{enumerate}[i)]
		\item für alle $i \in I$ gilt $\varphi_{ii} = \id_{\mcf_i}$.
		\item Für alle $i,j,k \in I$ gilt $\varphi_{ik} = \varphi_{jk} \circ \varphi_{ij}$ auf $U_i \cap U_j \cap U_k$.
	\end{enumerate}
	Dann existiert eine bis auf Isomorphie eindeutige Garbe $\mcf$ auf $X$ zusammen mit Isomorphismen $\psi_i \colon \mcf\vert_{U_i} \to \mcf_i$ derart, dass für alle $i,j \in I$
	\[
                \psi_j\vert_{U_i \cap U_j} = \varphi_{ij} \circ \psi_i\vert_{U_i\cap U_j}
	\]
	gilt. Wir sagen, dass $\mcf$ durch \textbf{Verkleben der $\mcf_i$ entlang der Isomorphismen $\varphi_{ij}$} entsteht.
\end{lem}

\nextmark{disjunkte Vereinigung von Schemata}
\begin{bsp}
\label{bsp:5.12}
	Sei $(X_i)_{i\in I}$ eine Familie von Schemata. Wir setzen $U_{ij}\coloneqq \emptyset$ und wählen als $\varphi_{ij}$ die trivialen Abbildungen. Damit erhalten wir ein Verklebedatum $(X_i,U_{ij},\varphi_{ij})_{i,j\in I}$ und mit Verkleben nach Lemma~\ref{lem:5.10} damit ein Schema $X$. Es ist $X$ die \textbf{disjunkte Vereinigung} $\coprod_{i\in I}X_i$ der $X_i$ als Schemata.
\end{bsp}

\begin{bsp}
\label{bsp:5.13}
	Sei $K$ ein Körper. Für $i=1,2$ sei $X_i\coloneqq \A^1_K = \Spec(K[T])$. Wir setzen $U_{12} \coloneqq X_1 \setminus V(T)$ und $U_{21}\coloneqq X_2 \setminus V(T)$. Als $\varphi_{12}$ und $\varphi_{21}$ wählen wir die identische Abbildung auf $U_{12} = U_{21} = U = \Spec(K[T]) \setminus V(T)$. Verkleben nach Lemma~\ref{lem:5.10} liefert ein Schema $X=$\enquote{$\A^1_K$ mit verdoppeltem Nullpunkt}. Es gilt $X= X_1 \cup X_2$ und $X_1 \cap X_2 = U$. Nach dem Garbenaxiom gilt:
	\begin{align*}
		\mco_X(X) &= \{(f,g) \in \mco_X(X_1)\times \mco_X(X_2) \mid f\vert_U = g\vert_U\}\\
		&=\{(f,g) \in K[T] \times K[T] \mid \frac{f}{1} = \frac{g}{1} \in K[T]_T\}\\
		&= K[T]\text{, wobei wir } \mco(U) = \mco(D(T)) = K[T]_T \text{ benutzt haben.}
	\end{align*}
	\textbf{Behauptung:} $X$ ist kein affines Schema.
	\begin{proof}
		Wäre $X$ ein affines Schema, so wäre
		\[
			X = \Spec(\mco_X(X)) = \Spec(K[T])
		\]
		und
		\[
			(\Gamma(X,\mco_X) = K[T]) \to (\Gamma(X_1,\mco_{X_1}) = K[T])
		\]
		wäre die Identität. Also erhalten wir den Widerspruch $X = X_1$.
	\end{proof}
\end{bsp}

Wir konstruieren jetzt projektive Schemata. Dabei gehen wir analog wie bei affinen Schemata vor, aber \enquote{homogenisieren alles} (vgl. auch Algebraische Geometrie I, projektive Varietäten).

\nextmark{Proj(S), projektives \glqq Nullstellengebilde\grqq}
\begin{defn}
\label{defn:5.14}
	Sei $S = \bigoplus_{d\in \N} S_d$ ein graduierter Ring. Sei $S_+ \coloneqq \bigoplus_{d > 0}S_d$ das homogene Maximalideal, $S^{\text{hom}}\coloneqq \bigcup_{d\in \N}S_d$ die Menge der homogenen Elemente von $S$ und
	\[
		\Proj(S) \coloneqq \{\mfp \mid \mfp \text{ homogenes Primideal in } S \text{ mit } S_+ \not \subseteq \mfp\}.
	\]
	Für ein homogenes Ideal $\mfa$ von $S$ setzen wir
	\[
		V_+(\mfa) = \{\mfp \in \Proj(S) \mid \mfa \subseteq \mfp\}.
	\]
\end{defn}

\begin{lem}
\label{lem:5.15}
	\begin{enumerate}[i)]
		\item Es gilt $V_+(S_+) = \emptyset$ und $V_+(\langle 0 \rangle) = \Proj(S)$.
		\item Für homogene Ideale $\mfa,\mfb$ in $S$ gilt $V_+(\mfa) \cup V_+(\mfb) = V_+(\mfa \cdot \mfb)$.
		\item Für homogene Ideale $(\mfa_i)_{i\in I}$ in $S$ gilt $\bigcap_{i\in I}V_+(\mfa_i) = V_+\left(\sum_{i\in I}\mfa_i\right)$.
	\end{enumerate}
	\begin{proof}
		i) ist trivial. Die Eigenschaften ii) und iii) folgen aus
		\[
			V_+(\mfa) = V(\mfa) \cap \Proj(S) \subseteq \Spec(S)
		\]
		und den entsprechenden Aussagen für $V(\mfa)$ in $\Spec(S)$.
	\end{proof}
\end{lem}

\nextmark{projektives Spektrum als Schema}
\begin{kons}[des Schemas $(\Proj(S),\mco)$]
\label{kons:5.16}
	Das Lemma~\ref{lem:5.15} liefert sofort, dass $\Proj(S)$ ein topologischer Raum ist, bei dem abgeschlossenen Teilmengen gerade Mengen der Form $V_+(\mfa)$ für ein homogenes Ideal $\mfa$ von $S$ sind. Damit erhalten wir die Zariski-Topologie auf $\Proj(S)$.

	Wir definieren die Garbe $\mco$ auf $\Proj(S)$ folgendermaßen: Sei $U$ offen in $\Proj(S)$. Dann ist $\mco(U)$ die Menge der Funktionen $s\colon U \to \coprod_{\mfp \in U}S_{(\mfp)}$, die folgende Eigenschaften erfüllen:
	\begin{enumerate}[a)]
		\item Für alle $\mfp \in U$ gilt $s(\mfp) \in S_{(\mfp)}$.
		\item Für alle $\mfp \in U$ gibt es eine offene Umgebung $V$ von $\mfp$ in $U$ und $a,f\in S^{\text{hom}}$ mit folgenden Eigenschaften:
                    $\deg(a)=\deg(f)$ und für alle $\mfq \in V$ gilt $f \notin \mfq$ und $s(\mfq) =  \frac{a}{f} \in S_{(\mfq)}$.
	\end{enumerate}
	Hier ist
	\[
		S_{(\mfp)} \coloneqq \left\lbrace\text{Elemente vom Grad }0 \text{ in } S_{S^{\text{hom}}\setminus\mfp} \right\rbrace = \left\lbrace \frac{a}{s} \mid a \in S^{\text{hom}},\, s \in S^{\text{hom}}\setminus \mfp,\, \deg(a) = \deg(s)\right\rbrace.
	\]
	Analog zum affinen Fall ist $\mco$ eine Garbe auf $\Proj(S)$. Wir nennen $(\Proj(S),\mco)$ das \textbf{projektive Spektrum} von $S$.
\end{kons}

\nextmark{Eig. der Strukturgarbe auf Proj(S)}
\begin{prop}
\label{prop:5.17}
	\begin{enumerate}[i)]
		\item Für alle $\mfp \in \Proj(S)$ gibt es einen kanonischen Isomorphismus
		\[
			\mco_\mfp \simto S_{(\mfp)}.
		\]
		\item Für $f \in S^{\text{hom}}$ ist $D_+(f) \coloneqq \{\mfp \in \Proj(S) \mid f \notin \mfp\}$ offen in $\Proj(S)$. Für eine Familie $(f_i)_{i\in I}$ in $S^{\text{hom}}$ mit $\langle\{f_i\mid i\in I\}\rangle = S_+$ gilt $\Proj(S) = \bigcup_{i\in I} D_+(f_i)$.
		\item Für $d \in \N$, $d>0$, $f \in S_d$ und $S_{(f)} \coloneqq \left\lbrace \frac{a}{f^m}\mid a \in S^{\text{hom}},\, \deg(a) = md\right\rbrace$ gibt es einen kanonischen Isomorphismus
		\[
			(\varphi, \varphi^{\#})\colon\Big(D_+(f),\mco\vert_{D_+(f)}\Big)\simto \Big(\Spec(S_{(f)}),\mco_{\Spec(S_{(f)})}\Big).
		\]
		\item $(\Proj(S),\mco)$ ist in kanonischer Weise ein $S_0$-Schema.
	\end{enumerate}
	\begin{proof}
		Dies wird analog zum affinen Fall in Aufgabe 6.2 gezeigt.
	\end{proof}
\end{prop}

\nextmark{affiner/projektiver Raum über einem Ring}
\begin{defn}
\label{defn:5.18}
	Sei $A$ ein Ring und $n \in \N$. Dann definieren wir
	\[
		\A^n_A \coloneqq \Spec(A[T_1,\ldots,T_n])
	\]
	als den $n$-dimensionalen \textbf{affinen Raum} über $A$ und
	\[
		\P^n_A \coloneqq \Proj(A[T_1,\ldots,T_n])
	\]
	als den $n$-dimensionalen \textbf{projektiven Raum} über $A$.
\end{defn}
\begin{bsp}
\label{bsp:5.19}
	Sei $S \coloneqq A[T_0,\ldots,T_n]$. Für $i \in \{0,\ldots,n\}$ ergibt sich ein Isomorphismus
	\begin{align*}
		A[T_0,\ldots,T_{i-1},\widehat{T_i},T_{i+1},\ldots,T_n] & \simto S_{(T_i)}\\
		f(T_0,\ldots,T_{i-1},\widehat{T_i},T_{i+1},\ldots,T_n) & \longmapsto f\left(\frac{T_0}{T_i},\ldots,\frac{T_n}{T_i}\right),
	\end{align*}
	wobei mit $\widehat{T_i}$ gemeint ist, dass wir $T_i$ weglassen. Damit folgt $D_+(T_i) \cong \A^n_A$ und wir erhalten aus Proposition~\ref{prop:5.17}, dass $\P^n_A = \bigcup_{i=0}^n D_+(T_i)$ eine offene Überdeckung durch affine Räume ist (vergleiche Standardüberdeckung von $\P^n_k$ in der algebraischen Geometrie I).
\end{bsp}

%!TEX root = algebraische_geometrie_2.tex
% vim: tw=0 noet sts=8 sw=8

\chapter{Erste Eigenschaften von Schemata}
\label{chap:6}

Nachdem wir Schemata definiert haben, untersuchen wir hier erste Eigenschaften, wie \enquote{Endlichkeit} von Morphismen.

\nextmark{quasikompakt, Nilradikal, reduzierter Ring}
\begin{eri}
\label{eri:6.1}
	\begin{enumerate}[i)]
		\item Ein topologischer Raum heißt \textbf{quasikompakt}, wenn jede offene Überdeckung eine endliche Teilüberdeckung hat.
		\item Ein Ring $A$ heißt \textbf{reduziert}, wenn $\nil(A) = \{0\}$ gilt, wobei $\nil(A) \coloneqq \sqrt{\{0\}} = \bigcap_{\mfp \in \Spec(A)} \mfp$ das \textbf{Nilradikal} von $A$ ist.
	\end{enumerate}
\end{eri}

\nextmark{Grundlegende Eigenschaften von Schemata}
\begin{defn}
\label{defn:6.2}
	\begin{enumerate}[i)]
		\item Ein Schema heißt
		\multitext{%
		zusammenhängend\\
		irreduzibel\\
		quasikompakt},
		wenn der unterliegene topologische Raum
		\multitext{%
		zusammenhängend\\
		irreduzibel\\
		quasikompakt}
		ist.
		\item Ein Schema $(X, \mco_X)$ heißt
		\multitext{%
		integer\\
		reduziert},
		falls für jedes offene $U$ in $X$ auch schon $\mco_X(U)$
		\multitext{%
		integer\\
		reduziert}
		ist.
	\end{enumerate}
\end{defn}

\nextmark{Irreduzibilitätskriterum für Spec(A)}
\begin{lem}
\label{lem:6.3}
	$X=\Spec(A)$ ist genau dann irreduzibel, wenn $\nil(A)$ ein Primideal ist. In diesem Fall ist $\nil(A)$ der generische Punkt von $X$.
	\begin{proof}
		Es gilt
		\begin{align*}
			&X \text{ irreduzibel}\\
			\underset{\mathclap{\text{und Lemma~\ref{lem:4.6} ii)}}}{\overset{\mathclap{\text{Proposition~\ref{prop:5.4} ii)}}}{\Longleftrightarrow}} \qquad\quad & X \text{ hat generischen Punkt}\\
			 \Longleftrightarrow \qquad\quad & \exists\; \mfp \in \Spec(A) \text{ mit } X = \overline{\{\mfp\}} = V(\mfp)\\
			 \Longleftrightarrow \qquad\quad & \exists \text{ kleinstes Primideal in } A\\
			 \Longleftrightarrow \qquad\quad &  \bigcap_{\mfp \in X} \mfp \text{ ist Primideal} .
		\end{align*}
	\end{proof}
\end{lem}

\nextmark{Def./Eig. von V(f) für globale Schnitte~f}
\begin{lem}
\label{lem:6.4}
	Sei $X$ ein Schema und $f \in \mco_X(X)$. Wir betrachten die zu $f$ assoziierte Funktion
	\[
		F\colon X \to \coprod_{p \in X}\kappa(p),\; p \mapsto F(p)\coloneqq f_p+\mfm_{X,p}\in \kappa(p).
	\]
	Die Menge 
	\[
		V(f) \coloneqq \{p\in X\mid F(p) = 0\}
	\]
	ist abgeschlossen in $X$ und $X_f\coloneqq X \setminus V(f)$ ist offen in~$X$.
	\begin{proof}
		$X$ wird durch offene affine Unterschemata $U=\Spec(A)$ überdeckt. Um zu zeigen, dass $V(f)$ abgeschlossen ist, genügt es zu zeigen, dass für eine fixierte solche Überdeckung für alle Unterschemata~$U$ der Überdeckung schon $V(f)\cap U$ abgeschlossen in $U$ ist. Setze
		\[
			\widetilde{f}\coloneqq f\vert_U\in \mco_X(U) \cong A.
		\]
		Für $\mfp\in U = \Spec(A)$ gilt:
		\begin{align*}
			&\mfp \in V(f) \cap U\\
			\Longleftrightarrow \quad & F(\mfp)=0\\
			\Longleftrightarrow \quad & f_{\mfp}+\mfm_{X,\mfp} = 0 \in \kappa(\mfp) \cong A_\mfp/\mfp A_\mfp\\
			\Longleftrightarrow \quad & \widetilde{f}_\mfp \in \mfp A_\mfp\\
			\Longleftrightarrow \quad & \widetilde{f} \in \mfp\\
			\Longleftrightarrow \quad & \mfp \in V(\langle \widetilde{f} \rangle) \subset \Spec(A) .
		\end{align*}
		Somit ist $V(f)\cap U$ abgeschlossen in $U = \Spec(A)$. Dies zeigt auch, dass diese Definition von $V(f)$ konsistent mit der Defininition von $V(f)$ ist, im Fall, dass $X$ ein affines Schema ist.
	\end{proof}
\end{lem}

\nextmark{Eig. reduzierter und irreduzibler Schemata}
\begin{prop}\label{prop:6.5}
	Sei $X$ ein Schema.
	\begin{enumerate}[i)]
		\item $X$ ist genau dann reduziert, wenn $\mco_{X,p}$ für alle $p \in X$ ein reduzierter Ring ist.
		\item Ist $X$ integer, so ist $\mco_{X,p}$ für alle $p \in X$ ein Integritätsbereich.
		\item $X$ ist integer genau dann, wenn $X$ reduziert und irreduzibel ist.
	\end{enumerate}
	\begin{proof}
		\begin{enumerate}[i)]
			\item
			\enquote{$\Longrightarrow$}: Sei $X$ also reduziert und $p \in X$. Falls es ein $f \in \mco_{X,p}\setminus\{0\}$ mit $f^n=0 \in \mco_{X,p}$ gibt für ein $n \in \N$, so gibt es ein $\widetilde{f}\in \mco_{X}(U)$ für eine offene Umgebung $U$ von $p$ mit $\widetilde{f}_p=f \in \mco_{X,p}$ und $\widetilde{f}^n = 0$. Dies ist jedoch ein Widerspruch zu der Tastsache, dass $\mco_{X}(U)$ ein reduzierter Ring ist.\\
			\enquote{$\Longleftarrow$}: Sei $U$ offen in $X$. Falls $\mco_X(U)$ nicht reduziert ist, dann gibt es ein $f \in \mco_X(U)\setminus\{0\}$ mit $f^n = 0$ für ein $n \in \N$. Dann gibt es ein $p \in U$ mit $f_p \neq 0 \in \mco_{X,p}$. Mit
			\[
				(f^n)_p = (f_p)^n=0 \in \mco_{X,p}
			\]
			folgt dann aber ein Widerspruch.
			\item Sei $X$ integer und sei $p \in X$. Wir wählen eine affine offene Umgebung $U = \Spec(A)$ von $p$ in $X$. Dann ist $p$ durch ein $\mfp \in \Spec(A)$ gegeben. Weil $X$ integer ist, ist $A = \mco_X(U)$ ein Integritätsbereich. Es gilt $\mco_{X,p} = A_\mfp$. Da die Lokalisierung eines Integritätsbereiches wieder integer ist, folgt die Behauptung.
			\item \enquote{$\Longrightarrow$}: Sei $X$ ein integres Schema. Dann ist $X$ trivialerweise reduziert. Wir zeigen wieder indirekt, dass $X$ irreduzibel ist. Ist $X$ nicht irreduzibel, so gibt es nicht-leere disjunkte offene Teilmengen $U_1,U_2$ von $X$. Da $U_1$ und $U_2$ disjunkt sind, gilt
			\[
				\mco_X(U_1 \dcup U_2) = \mco_{X}(U_1) \times \mco_{X}(U_2).
			\]
			Solch ein Produkt hat jedoch immer Nullteiler der Form $(f,0) \cdot (0,g) = (0,0)$.\\
			\enquote{$\Longleftarrow$}: Sei nun $X$ reduziert und irreduzibel. Sei weiter $U$ offen in $X$. Wir müssen zeigen, dass $\mco_X(U)$ ein Integritätsbereich ist. Seien $f,g \in \mco_X(U)$ mit $f\cdot g = 0$. Es gilt
			\[
				X = V(0) = V(fg) = V(f) \cup V(g).
			\]
			Da $U$ offen im irreduziblem Raum $X$ ist, ist auch $U$ irreduzibel. Deswegen können wir ohne Beschränkung der Allgemeinheit $V(f) = U$ setzen. Beachte, dass $V(f)$ wie in Lemma~\ref{lem:6.4} und damit abgeschlossen ist. Wir zeigen nun für jedes affine offene Unterschema $V=\Spec(A)$ von $U$, dass $f\vert_V = 0$ gilt, woraus mit der Garbeneigenschaft $f=0$ folgt, dass $U$ integer ist. Für $V=\Spec(A) \subset U = V(f)$ offen gilt nach Definition $f\vert_V \in \mfp$ für alle $\mfp \in \Spec(A)$. Also gilt
			\[
				f\vert_V \in \bigcap_{\mfp \in \Spec(A)} \mfp = \nil(A) = \{0\},
			\]
			da $A = \mco_X(V)$ reduziert ist.
		\end{enumerate}
	\end{proof}
\end{prop}

\nextmark{(lokal) noethersches Schema}
\begin{defn}
\label{defn:6.6}
	\begin{enumerate}[i)]
		\item Ein Schema $X$ heißt \textbf{lokal noethersch}, wenn $\mco_{X}(U)$ für alle affinen offenen Unterschemata $U = \Spec(A)$ noethersch ist.
		\item Ein Schema $X$ heißt \textbf{noethersch}, wenn $X$ lokal noethersch und quasikompakt ist.
	\end{enumerate}
\end{defn}

\nextmark{Grundlegende Eig. noetherscher Schemata}
\begin{prop}
\label{prop:6.7}
	Sei $X$ ein Schema.
	\begin{enumerate}[i)]
		\item $X$ ist genau dann lokal noethersch (bzw. noethersch), wenn es eine Überdeckung (bzw. endliche Überdeckung) $(U_i)_{i\in I}$ durch affine offene Unterschemata $U_i = \Spec(A_i)$ mit noetherschen Ringen $A_i$ gibt.
		\item Ein affines Schema $\Spec(A)$ ist genau dann noethersch, wenn $A$ ein noetherscher Ring ist.
		\item Ist $X$ ein noethersches Schema, so ist der unterliegende topologische Raum noethersch und hat damit eine Zerlegung in endlich viele irreduzible Komponenten.
		\item Ist $X$ ein noethersches Schema, so ist $\mco_{X,p}$ für alle $p \in X$ ein noetherscher lokaler Ring.
	\end{enumerate}
	\begin{proof}
		\cite[Proposition II.3.2]{hartshorne1977algebraic}.
	\end{proof}
\end{prop}

\nextmark{Eigenschaften von Schema-Morphismen (u.\,a. quasikompakt, (lokal) von endlichem Typ, endlich)}
\begin{defn}
\label{defn:6.8}
	Sei $f\colon X \to Y$ ein Morphismus von Schemata.
	\begin{enumerate}[i)]
		\item $f$ heißt \textbf{injektiv} beziehungsweise \textbf{surjektiv} beziehungsweise \textbf{bijektiv}, wenn die unterliegende Abbildung topologischer Räume injektiv beziehungsweise surjektiv beziehungsweise bijektiv ist.
		\item $f$ heißt \textbf{offen} beziehungsweise \textbf{abgeschlossen} beziehungsweise \textbf{Homöomorphismus}, wenn die unterliegende Abbildung topologischer Räume  offen beziehungsweise abgeschlossen beziehungsweise ein Homöomorphismus ist.
		\item $f$ heißt \textbf{quasikompakt}, wenn $f^{-1}(V)$ für alle $V$ quasikompakt und offen in $Y$ quasikompakt ist.
		\item\label{defn:6.8:iv} $f$ heißt \textbf{lokal von endlichem Typ}, wenn für alle affinen offenen Unterschemata $V=\Spec(A)$ von $Y$ und jedes affine offene Unterschema $U = \Spec(B)$ in $f^{-1}(V)$ schon $B$ eine endliche $A$-Algebra ist vermöge der Abbildung
		\[
			(A = \mco_{Y}(V))\overset{f^{\#}}{\longto}(\Gamma(V,f_*\mco_X)=\Gamma(f^{-1}V,\mco_X))\overset{\text{Einschränkung}}{\longto}(\Gamma(U,\mco_X) = B).
		\]
		\item $f$ heißt \textbf{von endlichem Typ}, wenn $f$ lokal von endlichem Typ und quasikompakt ist.
		\item $f$ heißt \textbf{endlich}, wenn für alle affinen offenen Unterschemata $V = \Spec(A)$ von $Y$ schon $f^{-1}(V)$ ein affines offenes Unterschema von $X$ und $B = \Gamma(f^{-1}(V),\mco_X)$ ein endlich erzeugter $A$-Modul ist vermöge derselben Abbildung, wie in \ref{defn:6.8:iv}.
	\end{enumerate}
\end{defn}

\begin{bem}
\label{bem:6.9}
	Nach Proposition~\ref{prop:4.7} ist jedes affine Schema quasikompakt. Also ist ein Schema $X$ genau dann quasikompakt, wenn es eine endliche Vereinigung von affinen offenen Unterschemata ist.
\end{bem}

\nextmark{Äquivalente Formulierung für quasikompakt, lokal von endlichem Typ und endlich}
\begin{prop}
\label{prop:6.10}
	Sei $f\colon X \to Y$ ein Morphismus von Schemata.
	\begin{enumerate}[i)]
		\item $f$ ist genau dann quasikompakt, wenn es eine Überdeckung $(V_i)_{i\in I}$ von $Y$ durch affine offene Unterschemata $V_i = \Spec(B_i)$ mit quasikompakten Urbildern $f^{-1}(V_i)$ gibt.
		\item $f$ ist genau dann lokal von endlichem Typ, wenn es eine Überdeckung $(V_i)_{i \in I}$ von $Y$ durch affine offene Unterschemata $V_i = \Spec(B_i)$ mit
		\[
			f^{-1}(V_i) = \bigcup_{j \in J_i}U_{ij}
		\]
		gibt für gewisse offene affine Unterschemata $U_{ij} = \Spec(B_{ij})$, wobei $B_{ij}$ für alle $i,j \in I$  eine endlich erzeugte $B_i$-Algebra ist.
		\item $f$ ist genau dann endlich, wenn es eine Überdeckung $(V_i)_{i \in I}$ von $Y$ durch affine offene Unterschemata $V_i = \Spec(A_i)$ gibt, mit der Eigenschaft, dass $f^{-1}(V_i) = \Spec(B_i)$ für alle $i \in I$ ein affines offenes Unterschema von $X$ ist und $B_i$ für alle $i \in I$ ein endlich erzeugter $A_i$-Modul ist.
	\end{enumerate}
	\begin{proof}
		Dies wird in den Aufgaben 7.1 bis 7.3 bewiesen.
	\end{proof}
\end{prop}

\begin{prop}
\label{prop:6.11}
	\begin{enumerate}[i)]
		\item\label{prop:6.11:i} Sei $Y$ ein lokal noethersches Schema und sei $f\colon X \to Y$ ein Morphismus lokal von endlichem Typ. Dann ist auch $X$ ein lokal noethersches Schema.
		\item Sei $Y$ ein noethersches Schema und sei $f \colon X \to Y$ ein Morphismus von endlichem Typ. Dann ist auch $X$ ein noethersches Schema.
	\end{enumerate}
	\begin{proof}
		Nach dem Hilbert'schen Basissatz ist eine endlich erzeugte Algebra über einem noetherschen Ring wieder noethersch.
		\begin{enumerate}[i)]
			\item Es gibt eine Überdeckung $(V_i)_{i \in I}$ von $Y$ durch offene affine Unterschemata $V_i=\Spec(B_i)$ mit der Eigenschaft, dass $f^{-1}(V_i) = \bigcup_{j \in J_i} U_{ij}$ gilt, wobei $U_{ij} = \Spec(B_{ij})$ ist und $B_{ij}$ für alle $i,j \in I$ eine endlich erzeugte $B_i$-Algebra ist. Da $Y$ lokal noethersch ist, ist $B_i$ noethersch und damit nach dem Hilbert'schen Basissatz auch $B_{ij}$. Nach Proposition~\ref{prop:6.7} ist $X$ lokal noethersch.
			\item Dies folgt aus \ref{prop:6.11:i}, weil $f$ quasikompakt ist.
		\end{enumerate}
	\end{proof}
\end{prop}

\begin{bsp}
\label{bsp:6.12}
	Sei $A$ ein Ring. Dann sind die kanonischen Morphismen $\A^n_A \to \Spec(A)$ und $\P^n_A\to\Spec(A)$ quasikompakt (benutze die endliche Standardüberdeckung durch affine Räume). Sie sind auch von endlichem Typ. Weiter sind sie genau dann endlich, wenn $n = 0$ ist. Falls $A$ ein noetherscher Ring ist, sind $\A_A^n$ und $\P_A^n$ noethersche Schemata.
\end{bsp}

\nextmark{Dimension eines Schemas, Kodimension abgeschlossener Teilmengen}
\begin{defn}
\label{defn:6.13}
	Sei $X$ ein Schema.
	\begin{enumerate}[i)]
		\item Die \textbf{Dimension} von $X$ ist definiert als die Dimension des unterliegenden topologischen Raumes und wird mit $\dim(X)$ bezeichnet. Es gilt also
		\begin{align*}
			\dim(X) = \sup\{&n \in \N \mid \text{ es gibt eine Kette }Z_0 \subsetneq Z_1 \subsetneq \ldots \subsetneq Z_n\\
			&\text{von abgeschlossenen irreduziblen Teilmengen in }X\}.
		\end{align*}
		\item Für $Z \subseteq X$ abgeschlossen und irreduzibel heißt
		\begin{align*}
			\codim(Z,X) \coloneqq \sup\{&n \in \N \mid \text{ es gibt eine Kette } Z = Z_0 \subsetneq Z_1 \subsetneq \ldots \subsetneq Z_n\\
			&\text{von abgeschlossenen irreduziblen Teilmengen in }X\}.
		\end{align*}
		die \textbf{Kodimension} von $Z$ in $X$.
		\item Für $Y\subseteq X$ abgeschlossen (aber nicht notwendigerweise irreduzibel) definieren wir
		\[
			\codim(Y,X) \coloneqq \inf\{\codim(Z,X) \mid Z \text{ abgeschlossen und irreduzibel in } Y\}.
		\]
		Konvention: $\codim(\emptyset,X) = +\infty$.
	\end{enumerate}	
\end{defn}

\begin{bem}
\label{bem:6.14}
	\begin{enumerate}[i)]
		\item Ist $X = \Spec(A)$, so gilt $\dim(X) = \dim(A)$, wobei $\dim(A)$ die Krulldimension von $A$ ist.
		\item Ist $X$ ein affines integres Schema über dem Körper $K$ mit der Eigenschaft, dass der Morphismus $X \to \Spec(K)$ von endlichem Typ ist, dann gilt
		\[
			\dim(Y) + \codim(Y,X) = \dim(X)
		\]
		für alle abgeschlossenen irreduziblen Teilmengen~$Y$ von $X$. Dies folgt aus dem entsprechenden Satz für die Krulldimension (siehe auch \cite[Chapter 5, §14]{matsumura1970commutative}). Für beliebige Schemata ohne gewisse Endlichkeitsvorraussetzungen ist dies im Allgemeinen falsch.
	\end{enumerate}
\end{bem}

%%% Local Variables: 
%%% mode: latex
%%% TeX-master: "algebraische_geometrie_2"
%%% End: 

%!TEX root = algebraische_geometrie_2.tex
% vim: tw=0 noet sts=8 sw=8

\chapter{Abgeschlossene Immersionen und Unterschemata}
Wir untersuchen die Schemastrukturen von einer abgeschlossenen Teilmenge eines gegebenen Schemas $X$. Im affinen Fall werden sie durch die Ideale in $\mco(X)$ bestimmt.

\nextmark{offene Immersion}
\begin{defn}
\label{defn:7.1}
	Eine \textbf{offene Immersion} ist ein injektiver offener Morphismus $j\colon U \to X$ von Schemata, der einen Isomorphismus von $U$ auf das offene Unterschema $(j(U),\mco_X\vert_U)$ von $X$ induziert.
\end{defn}

\nextmark{abgeschlossene Immersion, abgeschlossenes Unterschema}
\begin{defn}
\label{defn:7.2}
	\begin{enumerate}[i)]
		\item Ein Morphismus $i \colon Z \to X$ von Schemata heißt \textbf{abgeschlossene Immersion} oder \textbf{abgeschlossene Einbettung}, falls folgende Eigenschaften gelten:
		\begin{enumerate}[a)]
			\item\label{defn:7.2:a} i(Z) ist eine abgeschlossene Teilmenge von $X$.
			\item\label{defn:7.2:b} Die stetige Abbildung $i\colon Z \to i(Z)$ ist ein Homöomorphismus.
			\item\label{defn:7.2:c} Der Homomorphismus $i^{\#}\colon \mco_X \to i_*\mco_Z$ von Ringgarben ist surjektiv, d.\,h.
			    der induzierte Homomorphismus $i^\#_p$ auf den Halmen ist für alle $p\in X$ surjektiv.
		\end{enumerate}
		\item Zwei abgeschlossene (beziehungsweise offene) Immersionen $i\colon Z \to X$ und $i'\colon Z' \to X$ heißen \textbf{isomorph}, wenn es einen Isomorphismus $h\colon Z \to Z'$ mit der Eigenschaft, dass das Diagramm
		\begin{center}
			\begin{tikzcd}
				Z\arrow{dr}[']{i} \arrow{rr}{h} & & Z' \arrow{dl}{i'} \\
				& X &
			\end{tikzcd}
		\end{center}
		kommutiert, gibt.
		\item Ein \textbf{abgeschlossenes Unterschema} ist eine Isomorphieklasse von abgeschlossenen Immersionen $i \colon Z \to X$.
	\end{enumerate}	
\end{defn}

\nextmark{abgeschlossenes Unterschema von Spec(A) zu einem Ideal von A}
\begin{kons}
\label{kons:7.3}
	Sei $A$ ein Ring und $X = \Spec(A)$. Wir betrachten ein Ideal $\mfa$ von $A$ und den Morphismus
	\[
		(i\coloneqq\Spec(\varphi))\colon (Z\coloneqq \Spec(A/\mfa)) \to (X=\Spec(A)),
	\]
	der vom kanonischen Homomorphismus $A \overset{\varphi}{\longto}A/\mfa$ induziert wird. Wir wollen zeigen, dass $i$ eine abgeschlossene Immersion ist.

	Nach Übungsaufgabe~4.2\,b) ist klar, dass $i$ ein Homöomorphismus
	\[
		i \colon \Spec(A/\mfa) \to (V(\mfa)=\{\mfp \in \Spec(A) \mid \mfp \supseteq \mfa\})
	\]
	ist. Somit gelten \ref{defn:7.2:a} und \ref{defn:7.2:b}. Es bleibt \ref{defn:7.2:c} zu zeigen: Sei zuerst $\mfp \in (\Spec(A/\mfa) = Z)$ und $\mfq = i(\mfp) = \varphi^{-1}(\mfp)\in (V(\mfa) \subseteq \Spec(A))$. Da Lokalisieren exakt ist, erhalten wir
	\begin{equation*}
	\label{eq:7.3.1}\tag{$\star$}
		A_\mfq/\mfa_\mfq \cong (A/\mfa)_{\varphi(\mfq)}.
	\end{equation*}
	Wir betrachten
	\[
		i_\mfq^{\#}\colon (\mco_{X,\mfq} = A_\mfq) \to ((i_*\mco_Z)_\mfq = \mco_{Z,\mfp} = (A/\mfa)_\mfp \overset{\eqref{eq:7.3.1}}{=} A_\mfq/\mfa_\mfq).
	\]
	Wir sehen also, dass $i_\mfq^{\#}$ surjektiv ist. Es bleibt $\mfq \in X \setminus V(\mfa)$ zu betrachten. Dann gilt
	\[
		(i_*\mco_Z)_\mfq = \{0\}.
	\]
	Also ist $i_\mfq^{\#}$ trivialerweise surjektiv.

	$i$ ist also eine abgeschlossene Immersion und das dadurch definierte abgeschlossene Unterschema von~$X$ bezeichnen wir mit $V(\mfa)$.
\end{kons}

\begin{bem}
\label{bem:7.4}
	\begin{enumerate}[i)]
		\item Sei $(X,\mco_X)$ ein geringter Raum. Eine \textbf{Idealgarbe} $\mcj$ in $\mco_X$ ist eine Vorschrift, die jeder offenen Teilmenge $U$ von $X$ ein Ideal $\mcj(U)$ in $\mco_X(U)$ zuordnet, wobei die $J(U)$ für alle $V$ offen in $U$ durch die Restriktionsabbildung $\rho_{UV}$ von $\mco_X$ auf $J(V)$ abgebildet wird.
		\item Sei $\mcj$ eine Idealgarbe, dann haben wir die Quotientengarbe $\mco_X/\mcj$, die als zur Prägarbe
		\[
			U \mapsto \mco_X(U)/\mcj(U)
		\]
		assoziierte Garbe definiert ist. Beachte, dass der Quotientenhomomorphismus $\mco_X\to\mco_X/\mcj$ im Sinne der Garben surjektiv ist, das heißt er ist auf den Halmen surjektiv.
		\item Sei nun $i \colon Z \to X$ eine abgeschlossene Immersion von Schemata. Nach \ref{defn:7.2:c} haben wir einen surjektiven Homomorphismus $i^{\#}\colon \mco_X \to i_*\mco_Z$, dessen Kern eine Idealgarbe $\mcj$ ist. Wir haben einen kanonischen Isomorphismus
		\[
			\mco_X/\mcj \simto i_*\mco_Z.
		\]
		Da isomorphe abgeschlossene Immersionen dieselbe Idealgarbe induzieren, ist $\mcj$ eine Invariante des von $i$ induzierten abgeschlossenen Unterschemas.
	\end{enumerate}
\end{bem}

\nextmark{Ideale in A vs. abg. Unterschemata von Spec(A)}
\begin{thm}
\label{thm:7.5}
	Sei $A$ ein Ring und $X=\Spec(A)$. Dann definiert die Abbildung
	\begin{align*}
		\{\text{Ideale in }A\}&\to \{\text{abgeschlossene Unterschemata von }X\}\\
		\mfa &\mapsto i_\mfa\colon V(\mfa) \to X
	\end{align*}
	eine Bijektion. Insbesondere ist jedes abgeschlossene Unterschema von $X$ wieder affin.
	\begin{proof}
		Die Abbildung ist injektiv, denn wegen
		\[
			\mfa = \ker(i^{\#}_\mfa\colon (A = \Gamma(X,\mco_X) \to (\Gamma(X,{i_\mfa}_*\mco_{V(\mfa)}) = \Gamma(V(\mfa),\mco_{V(\mfa)}) = A/\mfa)
		\]
		kann man das Ideal $\mfa$ aus dem abgeschlossenen Unterschema $i_\mfa \colon V(\mfa) \to X$ zurückgewinnen. Dabei bezeichnet $\Gamma(U,\mcf) = \mcf(U)$ für alle Garben $\mcf$ und alle offenen Mengen $U$ von $\mcf$. Wir wollen nun die Surjektivität zeigen. Sei also ein abgeschlossenes Unterschema von $X$ durch die abgeschlossene Immersion $i \colon Z \to X$ gegeben. Betrachte
		\[
			i_X^{\#}\colon \underbrace{\Gamma(X,\mco_X)}_{=A} \to (\Gamma(X,i_*\mco_Z) = \Gamma(Z,\mco_Z)).
		\]
		Sei $\mfa = \ker(i_X^{\#}$, da heißt $\mfa$ ist ein Ideal in $A$. Nach dem Homomorphiesatz erhalten wir einen induzierten Homomorphismus
		\[
			\psi \colon A/\mfa \to \Gamma(Z,\mco_Z).
		\]
		Nach Proposition~\ref{prop:5.6} gibt es genau einen Morphismus $i'\colon Z \to (V(\mfa) = \Spec(A/\mfa))$ mit der Eigenschaft, dass ${i'}^{\#}_{V(\mfa)} = \psi$.
		\begin{center}
			\begin{tikzcd}
				{} & \Gamma(Z,\mco_Z) & {} & {} & {} & Z\arrow{dl}[swap]{i'} \arrow{dr}{i} & {}\\
				A/\mfa \arrow{ur} & {} & A \arrow{ll} \arrow{ul} & {} & V(\mfa)\arrow{rr}[swap]{i_\mfa} & {} & X
			\end{tikzcd}
		\end{center}
		Mir müssen zeigen, dass $i'$ eine Isomorphismus ist, da dann $i$ isomorph zu $i_\mfa$ und damit die Abbildung in der Behauptung surjektiv ist.

		\textbf{1. Schritt}: Sei $i \colon Z \to (X=\Spec(A))$ eine abgeschlossene Immersion und es sei $i_X^{\#}\colon (A=\Gamma(X,\mco_X)) \to \Gamma(Z,\mco_Z)$ injektiv. Dann ist $i$ ein Isomorphismus.
		\begin{proof}[Beweis des 1. Schrittes]
			Wir zeigen zuerst, dass $i$ ein Homöomorphismus ist. Da $i$ eine abgeschlossene Immersion ist, muss $i$ ein Homöomorphismus auf die abgeschlossene Teilmenge $i(Z)$ von $X$ sein. Also genügt es zu zeigen, dass $i$ surjektiv ist. Da $i(Z)$ abgeschlossen in $X$ ist und die offenen Mengen $D(a)\subseteq X \setminus Z$ eine Basis bilden, genügt es zu zeigen, dass $D(a) \subseteq X \setminus Z$ für alle $a \in A$ leer ist. Sei $U$ ein offenes affines Unterschema von $Z$, das heißt $U=\Spec(B)$. Sei $f\coloneqq i^{\#}(a)\vert_U \in B$. Es gilt
			\[
				U \overset{i(Z) \cap D(a) = \emptyset}{=} i^{-1}(V(a))\cap U \overset{\ref{prop:4.9}}{=} V(f),
			\]
			also ist $f \in \nil(B)$. Also gibt es ein $n \in \N$ mit $f^n = 0$ und damit folgt $i^{\#}(a^n)\vert_U = 0$. Da $i$ ein Homöomorphismus auf $i(Z)$ ist und $i(Z)$ eine abgeschlossene Teilmenge des quasikompakten Raumes $X$ ist, ist auch $Z$ quasikompakt. Also gibt es eine endliche Überdeckung von $Z$ durch offene affine Unterschemata $U$, wie oben. Nehmen wir das größte $n$, dann gilt $i^{\#}(a^n)\vert_U=0$ für diese endlich vielen $U$ und mit dem Garbenaxiom folgt $i^{\#}(a^n)=0$. Da $i^{\#}_x$ injektiv ist, folgt $a^n=0$. Dann gilt $a \in \nil(A)$ und somit $V(a) = X$, das heißt $D(a) = \emptyset$. Dies zeigt, dass $i$ ein Homöomorphismus ist. Es bleibt zu zeigen, dass $i^{\#}\colon\mco_X\to i_*\mco_Z$ ein Isomorphismus von Garben ist. Nach Definition einer abgeschlossenen Immersion ist $i^{\#}$ surjektiv. Es gnügt die Injektivität auf den Halmen zu zeigen. Sei also $\mfp \in (X=\Spec(A))$. Es gilt $\mco_{X,\mfp} = A_\mfp$. Weil $i$ bijektiv ist, gibt es genau ein $z \in Z$ mit $i(z) = \mfp$. Es gilt wie zuvor $(i_*\mco_Z)_{\mfp} = \mco_{Z,z}$, da $i$ ein Homöomorphismus ist. Die Halmabbildung durch
			\[
				(A_\mfp = \mco_{X,\mfp}) \overset{i^{\#}_z}{\longto}\mco_{Z,z}
			\]
			gegeben. Sei $\frac{a}{s}$ im Kern von $i^{\#}_z$ mit $a,s \in A,\; s \notin \mfp$. Dann liegt auch $a$ im Kern von $i^{\#}_z$. Wir wollen $\frac{a}{s}=0$ zeigen, sei also ohne Beschränkung der Allgemeinheit $s=1$. Es gibt eine offene Umgebung $U_0$ von $z$ in $Z$ mit
			\begin{equation*}
			\label{eq:7.4.1}\tag{$\star\star$}
				i^{\#}(a)\vert_{U_0} = 0
			\end{equation*}
			und wir dürfen annehmen, dass $U_0$ wieder affin ist. Da $Z$ quasikompakt ist, gilt
			\[
				Z = U_0 \cup U_1 \cup \cdots \cup U_n
			\]
			für affine offene Unterschemata $(U_k,\mco\vert_{U_k}),\; k \in \{1,\ldots,n\}$. Da die $D_X(s)$ mit $s \in A$ eine Basis von $X = \Spec(A)$ bilden, gibt es ein $s_0\in A$ mit $\mfp \in D_X(s_0) \subseteq i(U_0)$. Wir zeigen
			\begin{equation*}
			\label{eq:7.4.2}\tag{$\star{\star}\star$}
				\exists\; m \in \N \text{ mit } i^{\#}(s_0^ma) = 0.
			\end{equation*}
			Die Injektivität von $i^{\#}_X$ zeigt dann, dass $s_0^ma =$ gilt. Wegen $s_0\in \mco_{X,\mfp}^\times$ folgt dann, dass $a = 0$ gilt, wie gewünscht. Um \eqref{eq:7.4.2} zu zeigen, genügt es für jedes $j \in \{0,\ldots,n\}$ ein $m_j$ mit
			\begin{equation*}
			\label{eq:7.4.3}\tag{$\star{\star}{\star}\star$}
				i^{\#}(s_0^{m_j}a)\vert_{U_j} = 0
			\end{equation*}
			zu finden (wegen dem Garbenaxiom). Für $j=0$ folgt mit \eqref{eq:7.4.1} $m_0=0$. Aus
			\[
				D_{U_j}(i^{\#}(s_0)\vert_{U_j}) \overset{Proposition~\ref{prop:4.9}}{=} i^{-1}(D(s_0)) \cap U_j \subseteq U_0 \cap U_j
			\]
			folgt mit \eqref{eq:7.4.1} und mit $f_j \coloneqq i^{\#}(s_0)\vert_{U_j}$, dass
			\[
				i^{\#}(a)\vert_{D_{U_j}(f_j)} = 0 \in (\mco_Z(D_{U_j}(f_j)) = \mco_Z(U_j)_{f_j})
			\]
			gilt. Dies zeigt \eqref{eq:7.4.3} und mit Hilfe der Definition der Lokalisierung folgt der 1. Schritt.
		\end{proof}
		\textbf{2. Schritt}: Wende den 1. Schritt auf $i'\colon Z \to (V(\mfa) = \Spec(A/\mfa))$ an.
	\end{proof}
\end{thm}


%!TEX root = algebraische_geometrie_2.tex
% vim: tw=0 noet sts=8 sw=8

\chapter{Gefaserte Produkte und Basiswechsel}
\label{chap:8}
Das kartesische Produkt von Mengen, Vektorräumen oder Ringen ist wohlbekannt. In diesem Abschnitt wollen wir etwas ähnliches in der Kategorie der Schemata betrachten. In der modernen algebraischen Geometrie betrachtet man alles relativ bezüglich eines Schemas $S$ als Basis, das heißt wir wollen ein gefasertes Produkt $X \times_S Y$ für $S$-Schemata $f\colon X \to S$ und $g \colon Y \to S$ definieren. Das gefaserte Produkt wird ein $S$-Schema sein, dessen unterliegende Menge im Allgemeinen verschieden vom kartesischen Produkt der Mengen sein wird. $X \times_S Y$ ist (im Allgemeinen) ebenfalls verschieden von $\{(x,y) \in X \times Y\mid f(x) = g(y)\}$.

\begin{defn}
\label{defn:8.1}
	Seien $X,Y,S$ Schemata und seien $f\colon X \to S$ und $g \colon Y \to S$ Morphismen. Ein \textbf{gefasertes Produkt} der $S$-Schemata $X$ und $Y$ ist ein Schema $X\times_S Y$ zusammen mit Morphismen, die das Diagramm
	\begin{center}
		\begin{tikzcd}
			{} & X \times_S Y \arrow{dl}[swap]{p_1} \arrow{dr}{p_2} & {}\\
			X\arrow{dr}[swap]{f} & {} & Y\arrow{dl}{g}\\
			{} & S & {}
		\end{tikzcd}
	\end{center}
	kommutativ machen. Wir verlangen weiter, dass folgende universelle Eigenschaft gilt: Für alle $S$-Schemata $T$ und alle Morphismen $h_1\colon T \to X$ und $h_2\colon T \to Y$ mit der Eigenschaft, dass das Diagramm
	\begin{center}
		\begin{tikzcd}
			{} & T \arrow{dl}[swap]{h_1} \arrow{dr}{h_2} & {}\\
			X\arrow{dr}[swap]{f} & {} & Y\arrow{dl}{g}\\
			{} & S & {}
		\end{tikzcd}
	\end{center}
	kommutiert, gibt es einen eindeutigen Morphismus $\theta\colon T \to X \times_S Y$, der das Diagramm
	\begin{center}
		\begin{tikzcd}
			T \arrow[bend left]{drr}{h_2}\arrow[bend right]{ddr}[swap]{h_1}\arrow[dashed]{dr}{\exists ! \;\theta}\\
			{} & X \times_S Y \arrow{d}[swap]{p_1}\arrow{r}{p_2} & Y \arrow{d}{g}\\
			{} & X \arrow{r}[swap]{f} & S
		\end{tikzcd}
	\end{center}
	kommutativ macht. Man nennt $p_1$ und $p_2$ die Projektionsabbildungen.
\end{defn}

\begin{thm}
\label{thm:8.2}
	Für zwei $S$-Schemata $X$ und $Y$ existiert das gefaserte Produkt $X \times_S Y$, es ist ein $S$-Schema und eindeutig bis auf eindeutige Isomorphie.
\end{thm}

\begin{bem}[Vorbemerkung zum Beweis]
\label{bem:8.3}
	Für Schemata $X,Y$ und für ein offenes Unterschema $U$ von $Y$ gibt es eine natürliche Bijektion zwischen den Morphismen $h\colon X \to U$ und den Morphismen $h \colon X \to Y$ mit $h(X) \subseteq U$.
\end{bem}

\begin{proof}[Beweis von Theorem~\ref{thm:8.2}]
	Mit den bekannten Standardargumenten bei universellen Eigenschaften folgt die Eindeutigkeit bis auf eindeutige Isomorphie.

	Wir konstruieren $X \times_S Y$ zuerst im affinen Fall mit Hilfe des Tensorprodukts der unterliegenden Ringe. Der allgemeine Fall folgt durch \enquote{Verkleben}.

	\textbf{1. Schritt (affiner Fall):} Seien $X = \Spec(A), Y = \Spec(B)$ und $S=\Spec(R)$. Dann existiert $X\times_S Y$ und \enquote{ist} $\Spec(A \otimes_R B)$.
	\begin{proof}
		Da $X$ und $Y$ beide $S$-Schemata sind, sind $A$ und $B$ nach Proposition~\ref{prop:5.6} $R$-Algebren. Dann ist $A \otimes_R B$ eine $R$-Algebra mit dem Produkt
		\[
			(a_1 \otimes b_1) \cdot (a_2 \otimes b_2) = (a_1a_2)\otimes (b_1b_2).
		\]
		Die Projektionen $p_1\colon X\times_S Y \to X$ und $p_2\colon X \times_S Y \to Y$ entsprechen nach Proposition~\ref{prop:5.6} den Morphismen
		\[
			\Spec(p_1)\colon A \to A \otimes_R B,\; a \mapsto a \otimes 1
		\]
		und
		\[
			\Spec(p_2)\colon B \to A \otimes_R B,\; b \mapsto 1 \otimes b.
		\]
		Allgemeiner gilt nach Proposition~\ref{prop:5.6}: $\Hom_{\Sch}(T, \Spec(D)) \cong \Hom_{\Ring}(D,\Gamma(T,\mco_T))$. Wir benutzen die universelle Eigenschaft des Tensorproduktes, die impliziert dass es für jeden Ring $C$ und für alle Homomorphismen $\varphi_1\colon A \to C$ und $\varphi_2\colon B \to C$ mit $\varphi_1 \circ \Spec(f) = \varphi_2 \circ \Spec(g)$ genau einen Morphismus $\varphi\colon A \otimes_R B \to C$ gibt, der das Diagramm
		\begin{center}
			\begin{tikzcd}
				C\\
				{} & A \otimes_R B\arrow[dashed]{ul}[swap]{\exists! \; \varphi} & B\arrow[bend right, start anchor=north]{llu}[swap]{\varphi_2}\arrow{l}[swap]{\Spec(p_2)}\\
				{} & A\arrow[bend left=40]{uul}{\varphi_1}\arrow{u}{\Spec(p_1)} & R\arrow{l}{\Spec(f)}\arrow{u}[swap]{\Spec(g)}
			\end{tikzcd}
		\end{center}
		kommutativ macht. Benutzen wir die \enquote{Dualität} aus Proposition~\ref{prop:5.6}, dann folgt die universelle Eigenschaft des gefaserten Produkts und damit der 1. Schritt.
	\end{proof}
	\textbf{2. Schritt:} Seien $X,Y$ zwei $S$-Schemata, für die $X\times_S Y$ existiert und sei $U$ ein offenes Unterschema von $X$. Dann existiert $U \times_S Y = p_1^{-1}(U)$.
	\begin{proof}
		Das folgt aus der Vorbemerkung~\ref{bem:8.3} für das offene Unterschema $p_1^{-1}(U)$, das dann offensichtlich die universelle Eigenschaft erfüllt.
	\end{proof}
	\textbf{3. Schritt:} Seien $X,Y$ zwei $S$-Schemata und sei $(X_i)_{i\in I}$ eine offene Überdeckung von $X$ mit der Eigenschaft, dass $X_i\times_S Y$ für alle $i \in I$ existiert. Dann existiert auch $X \times_S Y$.
	\begin{proof}
		Für alle $i,j \in I$ sei
		\[
			X_{ij} \coloneqq X_i \cap X_j
		\]
		und
		\[
			U_{ij} \coloneqq (p^i_1)^{-1}(X_{ij}) = X_{ij}\times_S Y \subseteq X_i \times_S Y,
		\]
		wobei $p^i_1 \colon X_i \times_S Y \to X$ die kanonische Projektion ist. Wegen $X_{ij} = X_{ji}$ gilt $\varphi_{ij}\colon U_{ij} \simto U_{ji}$ kanonisch. Dies ergibt ein Verklebedatum und nach Lemma~\ref{lem:5.10} erhalten wir ein Schema $X\times_S Y$ durch Verklebung der $X_i\times_S Y$ entlang der $\varphi_{ij}$. Es ist leicht zu sehen, dass $X \times_S Y$ die universelle Eigenschaft des gefaserten Produkts erfüllt.
	\end{proof}
	\textbf{4. Schritt:}
	\begin{itemize}
		\item Nach dem 1. Schritt existiert $X \times_S Y$ für $X$, $Y$ und $S$ affin.
		\item Nach dem 3. Schritt existiert $X \times_S Y$ für $X$ beliebig und $Y$ und $S$ affin ($X$ ist überdeckt durch affine offene $U$).
		\item Nach dem 3. Schritt existiert $X \times_S Y$ für $X$ und $Y$ beliebig und $S$ affin.
	\end{itemize}
	\textbf{5. Schritt:} Sei nun $S$ ein beliebiges Schema. Dann gibt es eine affine offene Überdeckung $(S_i)_{i \in I}$ von $S$. Sei $X_i \coloneqq f^{-1}(S_i)$ und $Y_i\coloneqq g^{-1}(S_i)$. Das sind offene Unterschemata von $X$ beziehungsweise $Y$. Nach dem 4. Schritt existiert $X_i \times_{S_i} Y_i$. Dann existiert aber auch $X_i\times_S Y = X_i \times_{S_i} Y_i$, da für jedes Paar von Morphismen, für die das Diagramm
	\begin{center}
		\begin{tikzcd}
			T \arrow{r}\arrow{d} & Y\arrow{d}{g}\\
			X_i \arrow{r}[swap]{f\vert_{X_i}} & S
		\end{tikzcd}
	\end{center}
	kommutiert, $T\to Y$ über $Y_i$ faktorisiert (da das Bild von $T$ in $Y$ in $g^{-1}(f\vert_{X_i}(X_i)) \subset g^{-1}(S_i)$ enthalten ist). Nach dem 3. Schritt existiert dann $X\times_S Y$ (da $X$ von $(X_i)_{i \in I}$ überdeckt wird).
\end{proof}

\begin{bsp}
\label{bsp:8.4}
	\begin{enumerate}[i)]
		\item Sei $X$ ein Schema und seien $U,V$ offene Unterschemata von $X$. Dann gilt $U\times_X V = U \cap V$ (hier tragen $U$ bzw. $V$ eine $X$-Schema-Struktur bezüglich der offenen Immersion $U \to X$ bzw. $V\to X$). Insbesondere ist im affinen Fall $U = \Spec(B)$, $V=\Spec(C)$ offen in $X=\Spec(A)$
		\[
			U \cap V = U\times_X V = \Spec(B \otimes_A C) 	
		\]
		und damit ist $U \cap V$ affin.
		\item Für einen Morphismus $f \colon X \to Y$ von Schemata und ein offenes Unterschema $U$ von $Y$ gilt
		\[
			U \times_Y X = f^{-1}(U)
		\]
		als offenes Unterschema von $X$.
	\end{enumerate}
	\begin{proof}
		Mit Vorbemerkung~\ref{bem:8.3} folgt leicht, dass $U\cap V$ (beziehungsweise $f^{-1}(U)$) die universelle Eigenschaft des gefaserten Produkts erfüllt.
	\end{proof}
\end{bsp}

\begin{bem}
\label{bem:8.5}
	Sei $X$ ein Schema, $p \in X$ und $\kappa(p)$ der Restklassenkörper von $p$. Für eine offene Umgebung $U$ von $p$ in $X$ haben wir kanonische Homomorphismen
	\[
		\mco_X(U) \to \mco_{X,p} \to \kappa(p), \; s \mapsto s_p + \mfm_p
	\]
	(wobei $\mfm_p$ das Maximalideal von $\mco_{X,p}$ ist).
	Wir erhalten weiter einen kanonischen Morphismus
	\[
		\iota_p\colon \Spec(\kappa(p))\to X,
	\]
	indem wir den einzigen Punkt von $\Spec(\kappa(p))$ auf $p$ abbilden und für $U$ offen in $X$ den Homomorphismus
	\[
		\iota_{p,U}^{\#} \colon \mco_X(U)\to ({\iota_{p}}_*\mco_{Spec(\kappa(p))})(U) = \begin{cases}\kappa(p) & \text{falls } p \in U\\0 & \text{falls } p \notin U\end{cases}
	\]
	von zuvor bzw. den Nullmorphismus benutzen.
\end{bem}

\begin{defn}
\label{defn:8.6}
	Sei $f\colon X \to Y$ ein Morphismus von Schemata und $p \in Y$. Wir betrachten $\Spec(\kappa(p))$ als Schema über $Y$ mit Hilfe von $\iota_p$ aus Bemerkung~\ref{bem:8.5}. Dann heißt das $Y$-Schema $X\times_Y\Spec(\kappa(p))$ \textbf{Faser} von $f$ in $p$ und wird mit $X_p$ bezeichnet.
	\begin{center}
		\begin{tikzcd}
			(X_p=X\times_Y \Spec(\kappa(p))) \arrow{r}{p_2}\arrow{d}[swap]{p_1} & \Spec(\kappa(p))\arrow{d}{\iota_p}\\
			X \arrow{r}[swap]{f} & Y
		\end{tikzcd}
	\end{center}
\end{defn}

\begin{bem}
\label{bem:8.7}
	Man kann zeigen, dass der $X_p$ zugrunde liegende topologische Raum homöomorph zu $f^{-1}(\{p\})$ mit der induzierten Topologie (siehe \cite[{}I.3.4.6]{grothendieck1971elements} oder \cite[Ex.II.3.10]{hartshorne1977algebraic} oder \cite[Prop. 4.20 und (4.8)]{goertz2010algebraic}).
\end{bem}

\begin{bsp}
\label{bsp:8.8}
	Sei $K$ ein Körper, $X=\Spec(K[T_1,T_2,T_3]/\langle T_2T_3 - T_1^2 \rangle)$. Wir betrachten den Morphismus
	\[
		f \colon X \to \A^1_K \coloneqq \Spec(K[T_3]),
	\]
	der durch den Ringhomomorphismus
	\[
		K[T_3] \to K[T_1,T_2,T_3]/\langle T_2T_3 - T_1^2 \rangle, T_3 \mapsto T_3 + \langle T_2T_3 - T_1^2 \rangle.
	\]
	induziert wird. Der Punkt $p$ von $\A^1_K$ sei gegeben als Bild des Morphismus $\Spec(K) \to \A^1_K$, der durch den Ringhomomorphismus $K[T_3]\to K,\; T_3 \mapsto a$ mit $a \in K$ induziert wird. Das heißt $a$ ist die Koordinate des Punktes $p$. Es gilt:
	\begin{align*}
		X_p &= X \times_{\A^1_K}\Spec(\kappa(p))\\
		&= \Spec(K[T_1,T_2,T_3]/\langle T_2T_3 - T_1^2 \rangle) \times_{\A^1_K}\Spec(K)\\
		&= \Spec\bigl( (K[T_1,T_2,T_3]/\langle T_2T_3 - T_1^2 \rangle) \otimes_{K[T_3]} K \bigr)\\
		&= \Spec(K[T_1,T_2]/\langle aT_2-T_1^2 \rangle)
	\end{align*}
	Für $a\neq 0$ gilt
	\[
		X_p \overset{T_2= \frac{1}{a}T_1^2}{\cong} \Spec(K[T_1]) \cong \A^1_K
	\]
	und damit ist $X_p$ ein integres affines Schema. Für $a = 0$ gilt
	\[
		X_p = \Spec(K[T_1,T_2]/ \langle T_1^2 \rangle)
	\]
	und damit ist diese Faser nicht reduziert und damit auch nicht integer.
\end{bsp}

\begin{defn}[Basiswechsel]
\label{defn:8.9}
	Wir betrachten ein $Y$-Schema $X$, das heißt einen Morphismus $f\colon X \to Y$. Weiter sei ein Morphismus $g\colon Y' \to Y$ von Schemata gegeben. Wir betrachten das gefaserte Produkt
	\begin{center}
		\begin{tikzcd}
			(X'\coloneqq X \times_Y Y') \arrow{d}{g'} \arrow{r}{f'} & Y' \arrow{d}{g}\\ X \arrow{r}{f} & Y,
		\end{tikzcd}
	\end{center}
	wobei wir die Projektionen mit $f'$ und $g'$ bezeichnen. Wir nennen dies ein \textbf{kartesisches Diagramm}. Wir sagen, dass das $Y'$-Schema $X'$ \textbf{durch den Basiswechsel $Y'\to Y$ aus dem $Y$-Schema $X$ entsteht}. Äquivalent sagen wir, dass $f'$ Der Basiswechsel von $f$ bezüglich $g$ ist. Analog ist $g'$ der Basiswechsel von $g$ bezüglich $f$.
\end{defn}

\begin{defn}
\label{defn:8.10}
	Sei $\mce$ eine Eigenschaft von Morphismen.
	\begin{enumerate}[i)]
		\item Die Eigenschaft $\mce$ heißt \textbf{abgeschlossen unter Komposition}, wenn für Morphismen $X \overset{f}{\longto} Y \overset{g}{\longto}Z$, die die Eigenschaft $\mce$ haben schon gilt, dass $g \circ f$ die Eigenschaft $\mce$ hat.
		\item Die Eigenschaft $\mce$ heißt \textbf{stabil unter Basiswechsel}, wenn für Morphismen $f\colon X \to Y$ und $g \colon Y' \to Y$, wobei $f$ die Eigenschaft $\mce$ hat, schon gilt, dass der Basiswechsel $f'$ von $f$ bezüglich $g$ die Eigenschaft $\mce$ hat.
	\end{enumerate}
\end{defn}

\begin{prop}
\label{prop:8.11}
	Die Eigenschaft \enquote{lokal von endlichem Typ} ist abgeschlossen unter Komposition und stabil unter Basiswechsel.
	\begin{proof}
		\enquote{Komposition}: Seien $f\colon X \to Y$ und $g \colon Y \to Z$ lokal von endlichem Typ. Sei $W = \Spec(A)$ ein offenes affines Unterschema von $Z$. Sei weiter $p \in (g \circ f)^{-1}(W)$. Sei $V=\Spec(B)$ ein offenes affines Unterschema von $g^{-1}(W)$ mit $f(p) \in V$. Da $g$ lokal von endlichem Typ ist, folgt, dass $B$ eine endlich erzeugte $A$-Algebra ist. Sei $U = \Spec(C)$ ein offenes affines Unterschema von $f^{-1}(V)$ mit $p \in U$. Weil $f$ lokal von endlichem Typ ist, folgt, dass $C$ eine endlich erzeugte $B$-Algebra ist. Insgesamt ist also $C$ eine endlich ereugte $A$-Algebra. Wenn $p$ über $X$ läuft, erhalten wir entsprechend obige Mengen $W$, $V$ und $U$, wobei $X$ durch die Mengen $U$ überdeckte wird. Aus Proposition~\ref{prop:6.10} folgt, dass $g\circ f$ lokal von endlichem Typ ist.

		\enquote{Basiswechsel}: Wir haben folgendes kartesisches Diagramm:
		\begin{center}
			\begin{tikzcd}
				(X'\coloneqq X \times_Y Y') \arrow{d}{g'} \arrow{r}{f'} & Y' \arrow{d}{g}\\ X \arrow{r}{f} & Y,
			\end{tikzcd}
		\end{center}
		Wir nehmen an, dass $f$ lokal von endlichem Typ ist und müssen zeigen, dass auch $f'$ lokal von endlichem Typ ist. Nach Proposition~\ref{prop:6.10} ii) existieren offene affine Überdeckungen $(V_i)_{i\in I}$ von~$Y$ und $(U_{ij})_{j\in J_i}$ von $f^{-1}(V_i)$ mit der Eigenschaft, dass $\mco_X(U_{ij})$ für alle $i\in I$ und alle $j \in J_i$ eine endlich erzeugte $\mco_Y(V_i)$-Algebra ist. Sei $(V_{ik})_{k \in K_i}$ eine offene affine Überdeckung von $g^{-1}(V_i)$.
		\begin{center}
			\begin{tikzcd}
				U_{ijk} & \arrow[phantom]{l}{\coloneqq} U_{ij} \times_{V_i} V_{ik}' \arrow{r} \arrow{d} & V_{ik}'\arrow{d} \arrow[phantom]{r}{\subseteq} & Y'\\
				X' & U_{ij} \arrow[phantom]{l}{\supseteq\mkern15mu} \arrow{r} & V_i \arrow[phantom]{r}{\subseteq} & X
			\end{tikzcd}
		\end{center}
		Für dieses gefaserte Produkt $U_{ijk}$ gilt
		\[
			U_{ijk} = \Spec(\mco(U_{ij}) \otimes_{\mco(V_i)} \mco(V_{ik}'))
		\]
		nach dem ersten Schritt im Beweis von Theorem~\ref{thm:8.2}. Also ist $\mco(U_{ijk}) = \mco(U_{ij}) \otimes_{\mco(V_i)} \mco(V_{ik}')$ eine endlich erzeugte $\mco(V_{ik}')$-Algebra. Wir haben im Beweis von Theorem~\ref{thm:8.2} gesehen, dass $(U_{ijk})_{i\in I,\,j\in J_i,\,k\in K_i}$ eine offene affine Überdeckung von $X'$ ist. Nach Proposition~\ref{prop:6.10} ist dann $f'$ lokal von endlichem Typ.
	\end{proof}
\end{prop}


\printbibliography
\end{document}
